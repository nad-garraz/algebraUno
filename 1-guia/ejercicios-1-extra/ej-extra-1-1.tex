\begin{enunciado}{\ejExtra}
  Probar la propiedad distributiva: $X \inter (Y \union Z) = (X \inter Y) \union (X \inter Z)$
\end{enunciado}

Tengo que hacer una doble inclusión:

\begin{enumerate}[label=\magenta{\arabic*)}]
  \item  $X \inter (Y \union Z) \subseteq (X \inter Y) \union (X \inter Z)$
  \item  $X \inter Y) \union (X \inter Z) \subseteq X \inter (Y \union Z)$
\end{enumerate}

\begin{enumerate}[label=\magenta{\arabic*)}]
  \item
        $x \en X \inter (Y \union Z)$ quiere decir que $x \en X$ y
        $\llaves{c}{
            x \en Y \\
            \text{o bien}      \\
            x \en Z
          } $.
        Por lo tanto
        $\to
          \llaves{c}{
            x \en X \inter Y\\
            \text{o bien}      \\
            x \en X \inter Z
          }$, lo que equivale a $x \en (X \inter Y) \union (X \inter Z)$ \Tilde.

  \item
        Ahora hay que probar la vuelta. Uso razonamiento análogo:\par

        $$
          x \en (X \inter Y) \union (X \inter Z)
          \entonces x \en X
          \quad y \quad
          \llave{c}{
            x \en X \inter Y \\
            \otext           \\
            x \en X \inter Z
          }$$
        Pero teniendo en cuenta que:
        $$
          \llave{c}{
            Y \subseteq Y \union Z \\
            \text{ y que }         \\
            Z \subseteq Z \union Y,
          }
          \Entonces{\red{!!}}
          \llave{c}{
            x \en X \inter (Y \union Z) \\
            \text{ o bien }             \\
            x \en X \inter (Z \union Y)

          }\entonces x \en X \inter (Y \union Z)
        $$
        En \red{!!} uso algo "obvio" pero que me sirve para seguir bien donde está $x$: Resalto que si un elemento está en
        $Y$ seguro va a estar en la unión de $Y$ con lo que sea.

\end{enumerate}

% Contribuciones
\begin{aportes}
  %% iconos : \github, \instagram, \tiktok, \linkedin
  %\aporte{url}{nombre icono}
  \item \aporte{https://github.com/nad-garraz}{Nad Garraz \github}
\end{aportes}
