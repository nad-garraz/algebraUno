\begin{enunciado}{\ejExtra}\fechaEjercicio{recuperatorio 25/11/25}

  Sea $\F = \set{f: \set{1,2,3,4,5,6,7} \to \set{1,2,\ldots,18}: f ~ \text{es una función} }$. Se define en $\F$ la relación dada por
  $$
    f \relacion g \sisolosi \im(f) \triangle \im(g) \distinto \im(f) \union \im(g).
  $$
  \begin{enumerate}[label=\alph*)]
    \item Determinar si la relación es reflexiva, simétrica, transitiva y/o antisimétrica.
    \item Sean $f,\, g \en \F$ definidas por $f(n) = 2n$ y $g(n) = 18 - n$. ¿Cuántas funciones $h \en \F$ satisfacen
          que $f\relacion h$ pero $g \norelacion h$?
  \end{enumerate}
\end{enunciado}

El lado izquierdo de la relación me cae mejor escrito así:
$$
  \boxed{
    f \relacion g \sisolosi \im(f) \inter \im(g) \distinto \vacio.
  }
$$
Eso se puede ver fácil en un \textit{diagrama de Venn} o también restando $\blue{ \im(f) \union \im(g) }$ en la definición original escribiendo
que
$$
  (\im(f) \triangle \im(g))
  -
  (\im(f) \union \im(g))
  =
  \im(f) \inter \im(g)
$$

En fin.

\bigskip

\begin{enumerate}[label=\alph*)]
  \item

        \textit{Reflexividad:} ¿$f \relacion f$?

        Sí. Sale trivial, porque la intersección es reflexiva.

        \bigskip

        \textit{Simetría:} ¿$f \relacion g \entonces g \relacion f$?

        Sí, por el mismo motivo de antes.

        \bigskip

        \textit{Transitividad:} ¿$f \relacion g \land g \relacion h \entonces f \relacion h$?
        $$
          \ub{
            \llave{rcl}{
              \rowcolor{blue!10}
              f(1) & = & 1 \\
              f(2) & = & 1 \\
              f(3) & = & 1 \\
              f(4) & = & 1 \\
              f(5) & = & 1 \\
              f(6) & = & 1 \\
              f(7) & = & 1
            }
            \quad \relacion \quad
            \llave{rcl}{
              \rowcolor{blue!10}
              g(1) & = & 1 \\
              g(2) & = & 2 \\
              g(3) & = & 2 \\
              g(4) & = & 2 \\
              g(5) & = & 2 \\
              g(6) & = & 2 \\
              g(7) & = & 2
            }
          }{
            f \relacion g
          }
          ~ \ytext~
          \ub{
            \llave{rcl}{
              g(1) & = & 1 \\
              g(2) & = & 2 \\
              g(3) & = & 2 \\
              \rowcolor{blue!10}
              g(4) & = & 2 \\
              g(5) & = & 2 \\
              g(6) & = & 2 \\
              g(7) & = & 2
            }
            \quad \relacion \quad
            \llave{rcl}{
              h(1) & = & 3 \\
              h(2) & = & 3 \\
              h(3) & = & 3 \\
              h(4) & = & 3 \\
              h(5) & = & 3 \\
              h(6) & = & 3 \\
              \rowcolor{blue!10}
              h(7) & = & 2
            }
          }{
            g \relacion h
          }
        $$
        Peeeeero, $f\norelacion g$, porque sus conjuntos imagen no tienen nada en común. No es transitiva.

        \bigskip

        \textit{Antisimetría:} ¿$f \relacion g \land g \relacion f \entonces f = g$?

        Noup, agarrando las funciones $f$ y $g$ que usé antes tengo un \textit{contraejemplo}. $f \distinto g$, sin embargo $f \relacion g$ y $g \relacion f$.

  \item
        Nos dan 2 funciones bien definidas $f$ y $g$:
        $$
          \llave{rcl}{
            f(1) & = & 2 \\
            f(2) & = & 4 \\
            f(3) & = & 6 \\
            f(4) & = & 8 \\
            f(5) & = & 10 \\
            f(6) & = & 12 \\
            f(7) & = & 14
          }
          \quad \ytext \quad
          \llave{rcl}{
            g(1) & = & 17 \\
            g(2) & = & 16 \\
            g(3) & = & 15 \\
            g(4) & = & 14 \\
            g(5) & = & 13 \\
            g(6) & = & 12 \\
            g(7) & = & 11
          }
        $$

        Para cumplir lo que pide el enunciado observo que necesito una función cuyo conjunto $\im(h)$ tenga algún número par de $\im(f)$,
        peeero, que ese número par
        no aparezca en el conjunto $\im(g)$. Es decir que necesito que:
        $$
          \im(h) \inter \set{2,4,6,8,10}  \distinto \vacio
          ~ \ytext ~
          \im(h) \inter \set{17, 16, 15, 14, 13, 12, 11} = \vacio
        $$

        Por lo tanto puedo formarme las funciones $h$ con el siguiente conjunto de números:
        $$
          A = \set{1,2,3,4,5,6,7,8,9,10,\magenta{18}}
          \ytext
          \#A = 11
        $$

        Podemos reducir lo que queda a ¿Cuántas funciones $h(n)$ me puedo formar con los elementos de $A$, si debe aparecer por lo menos
        un \textit{número par de $A$} distinto de \magenta{18}?

        En total (sin pensar en las restricciones) con 11 números me puedo formar un total de $11^7 \llamada1$ funciones. Ahora quiero agarrar de todas
        esas aquellas que tengan algún número par ($\distinto 18$).

        O dicho de otra manera quiero las funciones (restringidas a $A$) que \underline{\textit{no sean todos números impares o \magenta{18}}}.
        Voy a ir por acá, entonces calculo cuántas funciones me puedo armar con:
        $$
          B = \set{1,3,5,7,9,18} \ytext \#B = 6
        $$
        De donde saco que puedo armar restringido a $B$ un total de $6^7\llamada2$ funciones.

        Concluyo entonces que en total satisfaciendo lo que pide el enunciado voy a tener un total de ($\llamada1 \ytext \llamada2$):
        $$
          \cajaResultado{
            11^7 - 6^7 ~ \text{funciones}~ h(n)
          }
        $$
        funciones que cumplen que por lo menos tienen un número par de $A$ distinto de \magenta{18}.

\end{enumerate}

\begin{aportes}
  \item \aporte{\dirRepo}{naD GarRaz \github}
\end{aportes}
