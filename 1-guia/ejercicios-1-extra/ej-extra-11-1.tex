\begin{enunciado}{\ejExtra[parcial 18/2/26]}\fechaEjercicio{parcial 18/2/26}

  Sea $A = \naturales \times \naturales$. Se define en $A$ la relación $\relacion$ de la siguiente manera:
  $$
    (a,b) \relacion (c,d) \sisolosi 5^a \cdot 7^b \leq 5^c \cdot 7^d
  $$
  Probar que $\relacion$ es una relación de orden en $A$.
\end{enunciado}

Hay que ver que la relación sea \textit{reflexiva, antisimétrica y transitiva}.

\bigskip

\textit{Reflexividad:}
\parrafoDestacado{
  \it
  ¿Se cumple que:
  $
    5^a \cdot 7^b \leq 5^a \cdot 7^b
    \paratodo (a,b) \en A$ ?
}

Sí, se cumple de forma inmediata.

\bigskip

\textit{Antisimetría:}
\parrafoDestacado{
  \it
  ¿Se cumple que si
  $
    5^a \cdot 7^b \leq 5^c \cdot 7^d
    \entonces
    5^c \cdot 7^d \not\leq 5^a \cdot 7^b
  $
  para todos los pares $(a,b),\, (c,d) \en A$
  ?
}
