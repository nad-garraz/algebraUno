\begin{enunciado}{\ejExtra}
  \textbf{Este no es de parcial, pero está por razones históricas {\small\faIcon[regular]{grin-beam-sweat}}:}

  Probar la propiedad $(A \inter B)^c = A^c \union B^c$.
\end{enunciado}

Tengo que hacer una doble inclusión
$\to \begin{cases}
    1) & (A \inter B)^c \subseteq A^c \union B^c \\
    2) & A^c \union B^c \subseteq (A \inter B)^c
  \end{cases}
$
\begin{enumerate}[label=\arabic*)]
  \item Prueba directa: Si $x \en (A \inter B)^c \entonces x \en A^c \union B^c $

        Por hipótesis $x \en (A \inter B)^c  \Sii{def}  x \notin A \otext x \notin B
          \entonces x \en A^c \otext x \en B^c \entonces x \en A^c \union B^c$
        $$\begin{array}{|c|c|c|c|}
            \hline
            A & B & A^c \union B^c & (A \inter B)^c \\ \hline\rowcolor{Cerulean!10}
            V & V & \orange{F}     & \orange{F}     \\
            V & F & \orange{V}     & \orange{V}     \\\rowcolor{Cerulean!10}
            F & V & \orange{V}     & \orange{V}     \\
            F & F & \orange{V}     & \orange{V}     \\ \hline
          \end{array}
        $$

        Uso la tabla para ver la definición $x \en (A \inter B)^c  \Sii{def}  x \notin A \otext x \notin B$

  \item Pruebo por absurdo. Si $\paratodo x \en A^c \union B^c \entonces x \en (A \inter B)^c$\\
        \green{Supongo} que $ x \notin (A \inter B)^c  \Sii{def}  x \en (A \inter B) \flecha{por}[hipótesis] x \en A^c \union B^c \to
          \llaves{c}{
            x \notin A\\
            \otext \\
            x \notin B\\
          }$, por lo que $x \notin A \union B \entonces x \notin A \inter B$ contradiciendo el \green{supuesto}, absurdo. Debe ocurrir que $x \en (A \inter B)^c   $

        $$\begin{array}{|c|c|c|c|c|}
            \hline
            A & B & A \inter B & (A \union B) & (A \inter B) \subseteq (A \union B) \\ \hline\rowcolor{Cerulean!10}
            V & V & V          & V            & V                                   \\
            V & F & F          & V            & V                                   \\\rowcolor{Cerulean!10}
            F & V & F          & V            & V                                   \\
            F & F & F          & F            & V                                   \\ \hline
          \end{array}
        $$
\end{enumerate}

\begin{aportes}
  \item \aporte{\dirRepo}{naD GarRaz \github}
\end{aportes}
