\begin{enunciado}{\ejExtra}
  Sean $X = \set{n \en \naturales : n \leq 200}$ e $Y = \set{n\en \naturales : n \leq 100}$.\par
  En $\partes(X)$ se define la relación $\relacion$ de la forma:
  $$
    A \relacion B \sisolosi B - A \subseteq Y.
  $$
  \begin{enumerate}[label=\alph*)]
    \item Determinar si $\relacion$ es una relación reflexiva, simétrica, antisimétrica y/o transitiva.
    \item Sea $B = \set{n \en X : n \text{ es par} }$. ¿Cuántos conjuntos $A \en \partes(X)$ satisfacen simultáneamente
          $A \relacion B$ y $\#(A\inter B) = 80$?
  \end{enumerate}
\end{enunciado}

\begin{enumerate}[label=\alph*)]
  \item Para ver esto de qué cosa son \hyperlink{teoria-1:prop-relaciones}{las propiedades de reflexión y eso acá está la teoría}.

        \textit{¿Es $\relacion$ reflexiva?:}
        $$
          A \relacion A \sisolosi A - A = \vacio \subseteq Y
        $$
        Lo cual es cierto, dado que el conjunto vacío, $\vacio$, está  en todo conjunto.

        \medskip

        \textit{¿Es $\relacion$ simétrica?:}\par
        Por hipótesis:
        $$
          A \relacion B \sisolosi B - A \subseteq Y \Tilde
        $$
        Quiero ver que pasa con $B\relacion A$:
        $$
          B \relacion A \Sii{\red{??}} A - B \subseteq Y
        $$
        Propongo que $A = \set{101,200}$ y que $B = \set{1,200}$. Con estos conjuntos se tiene:
        $$
          B - A = \set{1} \subseteq Y
        $$
        peeeeero,
        $$
          A - B = \set{101} \nsubseteq Y
        $$
        Por lo tanto la relación \ul{no} es simétrica.

        \medskip

        \textit{¿Es $\relacion$ antisimétrica?:}\par
        Por hipótesis:
        $$
          A \relacion B \sisolosi B - A \subseteq Y \Tilde
        $$
        De ser antisimétrica debería ocurrir que
        $$
          A \relacion B \entonces B \norelacion A \text{  para  } B \distinto A.
        $$
        Veamos por ejemplo qué pasa con $A = \set{1}$ y $B = \set{2}$, donde $A \distinto B$:
        $$
          B - A = \set{2} \subseteq Y \entonces A \relacion B
        $$
        y también,
        $$
          A - B = \set{1} \subseteq Y \entonces B \relacion A
        $$
        Por lo tanto la relación \ul{no} es antisimétrica.

        \medskip

        \textit{¿Es $\relacion$ transitiva?:}\par
        Por hipótesis:
        $$
          \begin{array}{c}
            \llamada1 A \relacion B \sisolosi B - A \subseteq Y \Tilde \\
            \llamada2 B \relacion C \sisolosi C - B \subseteq Y \Tilde
          \end{array}
        $$
        En diagramas de Venn:
        $$
          \llamada1
          \begin{venndiagram3sets}[shade=blue!30!white, showframe = false,hgap=0, vgap=0, overlap = 1.1cm]
            \fillBCapCNotA
            \fillOnlyB
          \end{venndiagram3sets}
          \ytext
          \llamada2
          \begin{venndiagram3sets}[shade=orange!30!white, showframe = false,hgap=0, vgap=0, overlap = 1.1cm]
            \fillCCapANotB
            \fillOnlyC
          \end{venndiagram3sets}
        $$
        Quiero ver si $A \relacion C$ es decir si $C-A \subseteq Y$:
        $$
          \llamada3
          \begin{venndiagram3sets}[shade=yellow!30!white, showframe = false,hgap=0, vgap=0, overlap = 1.1cm]
            \fillCCapBNotA
            \fillOnlyC
          \end{venndiagram3sets}
        $$
        Esto está lindo porque $\llamada1$ y $\llamada2$ están en $Y$, lo cual equivale a decir que su unión también está en $Y$:
        $$
          \begin{venndiagram3sets}[shade=blue!30!white, showframe = false,hgap=0, vgap=0, overlap = 1.1cm]
            \fillBCapCNotA
            \fillOnlyB
          \end{venndiagram3sets}
          \quad
          \union
          \quad
          \begin{venndiagram3sets}[shade=orange!30!white, showframe = false,hgap=0, vgap=0, overlap = 1.1cm]
            \fillCCapANotB
            \fillOnlyC
          \end{venndiagram3sets}
          \quad
          =
          \quad
          \begin{venndiagram3sets}[shade=magenta!30!white, showframe = false,hgap=0, vgap=0, overlap = 1.1cm]
            \fillBNotA
            \fillCNotB
          \end{venndiagram3sets}
        $$
        En los diagramas se puede ver que $\llamada3$ es un conjunto que está en la unión de $\llamada1$ y $\llamada2$,
        y como, por hipótesis, esa unión está en $Y$, la relación es \textit{transitiva}.

  \item
        $$
          \#B = \#\set{n \en X : n \text{ es par} } = \#\set{2,4,6,\dots,198,200} = 100
        $$
        Para que $A \relacion B$ necesito que los \underline{todos} los conjuntos $A$ \textit{le saquen} a $B$
        los elementos del $102$ al $200$. Por lo tanto:
        $$
          \set{102, 104, \dots, 200} \subseteq A.
        $$
        De forma tal que $B - A \subseteq Y$. Teniendo en cuenta que $\#\set{102, 104, \dots, 200} = 50$, ahora tengo que jugar con el
        conjunto $C$ que es como llamo al resto de números que le queda a $B$ después de restar, es decir:
        $$
          C =
          B - \set{102,\ldots,200} =
          \set{2, 4, \dots, 100}
          \flecha{con cardinal}
          \#C = \#\set{2, 4, \dots, 100} = 50
        $$
        y tomar \ul{solo} 30 elementos de $C$ para darselos a $A$ y así satisfacer $\#(A \inter B) = 80$.

        Formas de tomar esos números:
        $$
          \binom{50}{30} = \frac{50!}{30! \cdot 20!}
        $$
        Ese resultado cumple las condiciones, peeeeero no hay que olvidar que $A \en \partes(X)$,
        por lo tanto $A$, podría tener números impares entre $1 \ytext 200$, esto me va a agregar posibles conjuntos $A$. Dado
        que si agrego números impares a los conjuntos de arriba \underline{no rompo ninguna condición}.

        \medskip

        ¿Cuántos $A$ me agrega?
        Empiezo por agregar \underline{ningún} número impar de un conjunto de 100 para elegir:
        $$
          \binom{100}{0}
        $$
        Puedo formar así \underline{un} $A$ que cumple todo lo pedido y no tiene \underline{ningún} número impar.

        Ahora agarro 1 número entre 100:
        $$
          \binom{100}{1}
        $$
        Puedo formar así \underline{100} $A$ que cumplen todo lo pedido y tienen \underline{un} número impar.

        Ahora agarro 2 números entre 100:
        $$
          \binom{100}{2}
        $$
        Puedo formar así \underline{5445} $A$ que cumplen todo lo pedido y tienen \underline{2} números impar.

        Ahora agarro 3 números entre 100:
        $$
          \binom{100}{3}
        $$

        \bigskip

        \begin{center}
          ...
          bueh, creo que se ve a donde estamos yendo.
          ...
        \end{center}

        \bigskip

        Sigo así hasta agarrar a todos en algún momento llego a que:
        $$
          \binom{100}{100} = 1
        $$

        Juntando todo eso con el resultado de la parte \textit{par} quedaría que el total de conjuntos $A$ es:
        $$
          \cajaResultado{
            \binom{50}{30} \cdot \sumatoria{\blue{i} = 0 }{100} \binom{100}{\blue{i}}
          }
        $$
        Y bueh para mí ahí está un posible resultado final. Lo que viene ahora no creo que haya sido necesario, y de haberlo
        sido, me hubiese puesto a llorar en el examen:
        $$
          (x+y)^n = \sumatoria{i = 0}{n} \binom{n}{i}x^{n-i}y^k
          \quad
          \Entonces{$x = y = 1$}[$n = 100$]
          \quad
          2^{100} = \sumatoria{\blue{i} = 0}{100} \binom{100}{\blue{i}}
        $$

        Por lo tanto y nuevamente la cantidad de conjuntos $A$ que cumplen lo pedido pero escrito en una
        forma más \textit{elegante} es:
        $$
          \cajaResultado{
            \binom{50}{30} \cdot 2^{100}
          }
        $$

\end{enumerate}

\begin{aportes}
  \item \aporte{\dirRepo}{Nad Garraz \github}
  \item \aporte{https://github.com/fionamclou}{Fiona M L \github}
\end{aportes}
