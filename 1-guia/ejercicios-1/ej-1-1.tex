\begin{enunciado}{\ejercicio}
  Dado el conjunto $A = \set{1,2,3}$, determinar cuáles de las siguientes afirmaciones son verdaderas
  \begin{enumerate}[label=(\roman*)]
    \begin{multicols}{5}
      \item $1 \en A $
      \item $\set{1} \subseteq A $
      \item $\set{2,1} \subseteq A $
      \item $\set{1,3} \en A $
      \item $\set{2} \en A $
    \end{multicols}
  \end{enumerate}
\end{enunciado}

El símbolo \textit{pertenece}:
'$\en$' se usa para decir si un \magenta{elemento} cualquiera está en un dado \magenta{conjunto}. Es un
operador que toma 2 cosas un \magenta{elemento} y un \magenta{conjunto} y te devuelve un valor
\textit{booleano} ya sea \textit{\red{verdadero}} o \textit{\red{falso}}.
$$
  \magenta{elemento} \en \magenta{conjunto}
$$

El símbolo \textit{subconjunto o inclusión}: '$\subseteq$', es como el anterior pero toma 2 conjuntos
$$
  \magenta{conjunto_1} \en \magenta{conjunto_2}
$$

Por ejemplo:
$$
  \begin{array}{c}
    C_1 = \set{1, 2, \set{1,2,3}} \ytext C_2 = \set{1, 2, \set{1,2}}                                                     \\
    \llave{rl}{
    1 \en C_1           & \text{ y también }\set{1} \subseteq C_1 \text{ pero } \set{\set{1}} \not\subseteq C_1          \\
    \set{1,2,3} \en C_1 & \text{ y también } \set{\set{1,2,3}} \subseteq C_1 \text{ pero } \set{1,2,3} \not\subseteq C_1 \\
    \set{1,2} \en C_2   & \text{ y también } \set{1,2} \subseteq C_2 \text{ y todavía } \set{\set{1,2}} \subseteq C_2
    }
  \end{array}
$$

Hagamos el ejercicio. Jugamos con el conjunto:
$$
  A = \set{1,2,3}
$$
\begin{enumerate}[label=(\roman*)]
  \item $1 \en A$. Es verdadero, porque 1 es un elemento que pertenece al conjunto $A$.

  \item $\set{1} \subseteq A$. Es verdadero, porque el conjunto, lo bautizo, $B = \set{1}$,
        es un conjunto cuyos elementos están todos (en este caso particular solo el 1) en $A$. Se dice que $B$ es un subconjunto de $A$.

  \item $\set{2,1} \subseteq A$. Es verdadero, porque el conjunto, lo bautizo, $C = \set{2,1}$,
        es un conjunto cuyos elementos están todos en $A$. Se dice que $C$ es un subconjunto de $A$.

  \item\label{ej-1:item4} $\set{1,3} \en A$. Es falso, porque el \ul{elemento} \textit{conjunto que tiene al 1 y a 3}: $\ub{\set{1,3}}{\text{Sí, esto es \red{un}}\\ \text{\red{solo} elemento}}$ no está en el conjunto $A$.
        Peeero ojo que $\set{1,3} \subseteq A$, ¿Comprás?

  \item $\set{2} \en A$ Es falso, por lo mismo que el ítem \ref{ej-1:item4}. El \ul{elemento} \textit{conjunto que tiene al 2} : $\set{2}$, no es un elemento de $A$,
        peeero como antes $\set{2} \subseteq A$.
\end{enumerate}

\begin{aportes}
  \item \aporte{\dirRepo}{naD GarRaz \github}
\end{aportes}
