\begin{enunciado}{\ejercicio}
  Sean $p,\, q$ proposiciones. Verificar que las siguientes expresiones tienen la misma tabla de verdad para
  concluir que son equivalentes:

  \begin{enumerate}[label=\roman*)]
    \item\label{10-1-i} $p \entonces q,
            \qquad
            \neg q \entonces \neg p, \qquad \neg p \otext q \qquad \ytext \qquad  \neg (p \y \neg q)$.

          Esto nos dice que podemos demostrar una afirmación de la forma $p \entonces q$ probando en su lugar
          $\neg q \entonces \neg p$ (es decir \textit{demostrando el contrarrecíproco}), o probando $\neg (p \y \neg q)$ (esto
          es una \textit{demostración por reducción al absurdo}).

    \item $\neg(p \entonces q)\qquad \ytext \qquad \neg q$.
  \end{enumerate}
\end{enunciado}

\begin{enumerate}[label=\roman*)]
  \item
        Sean p, q proposiciones. Verificar que las siguientes expresiones tienen la misma tabla de verdad para
        concluir que son equivalentes:
        $$
          \begin{array}{|c|c|c|c|c|c|c|c|}
            \hline
            p & q & \neg p & \neg q & p \entonces q & \neg q \entonces \neg p & \neg p \lor q & \neg(p \land \neg q) \\ \hline  \hline\rowcolor{Cerulean!10}
            V & V & F      & F      & V             & V                       & V             & V                    \\
            V & F & F      & V      & F             & F                       & F             & F                    \\\rowcolor{Cerulean!10}
            F & V & V      & F      & V             & V                       & V             & V                    \\
            F & F & V      & V      & V             & V                       & V             & V                    \\
            \hline
          \end{array}
        $$
  \item
        $$
          \begin{array}{|c|c|c|c|c|c|}
            \hline
            p & q & \neg q & p \entonces q & \neg (p \entonces q) & p\, \y \neg q \\ \hline  \hline\rowcolor{Cerulean!10}
            V & V & F      & V             & F                    & F             \\
            V & F & V      & F             & V                    & V             \\\rowcolor{Cerulean!10}
            F & V & F      & V             & F                    & F             \\
            F & F & V      & V             & F                    & F             \\ \hline
          \end{array}
        $$
\end{enumerate}

\begin{aportes}
  \item \aporte{\dirRepo}{naD GarRaz \github}
\end{aportes}
