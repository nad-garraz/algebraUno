\begin{enunciado}{\ejercicio}
  Hallar contraejemplos para mostrar que las siguientes proposiciones son falsas:
  \begin{enumerate}[label=\roman*)]
    \item $\paratodo a \en \naturales,\, \frac{a - 1}{a}$ no es un número entero.
    \item $\paratodo x,\, y \en \reales$ con $x,\, y$ positivos, $\sqrt{x + y} = \sqrt{x} + \sqrt{y}$.
    \item $\paratodo x \en \reales,\, x^2 > 4 \entonces x > 2$.
  \end{enumerate}
\end{enunciado}

\begin{enumerate}[label=\roman*)]
  \item La proposición es falsa, dado que si:
        $a = \green{1}
          \entonces
          \frac{\green{1} - 1}{\green{1}} =
          \frac{0}{1} =
          0 \en \enteros$

  \item La proposición es falsa, dado que si:
        $$
          x = 2
          ~ \land ~
          y = 2
          \entonces
          \sqrt{2 + 2} = \sqrt{4} = 2 \distinto \sqrt{2} + \sqrt{2} = 2 \cdot \sqrt{2}
        $$

  \item La  proposición es falsa, dado que si:
        $$
          x = -3 \entonces 9 > 4 \Entonces{\red{\skull}} -3 > 2
        $$
        Y eso no es verdad.
\end{enumerate}

\begin{aportes}
  \item \aporte{\dirRepo}{naD GarRaz \github}
\end{aportes}
