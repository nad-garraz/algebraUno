\begin{enunciado}{\ejercicio}

  Determinar cuáles de las siguientes afirmaciones son verdaderas y
  cualesquiera sean los subconjuntos $A$, $B$ y $C$ de un conjunto
  referencial $\universo$ y cuáles no.
  Para las que sean verdaderas, dar una demostración, para las otras dar un contraejemplo.
  \begin{multicols}{2}
    \begin{enumerate}[label=\roman*)]
      \item $(A \triangle B) - C = (A-C) \triangle (B - C)$.
      \item $(A \inter B) \triangle C = (A \triangle C) \inter (B \triangle C)$
      \item $C \subseteq A \entonces B \inter C \subseteq (A \triangle B)^c$
      \item $A \triangle B = \vacio \sisolosi A = B$
    \end{enumerate}
  \end{multicols}

\end{enunciado}

\begin{enumerate}[label=\roman*)]
  \item $(A \triangle B) - C = (A-C) \triangle (B - C)$. Es verdadera. Pruebo con tabla de verdad.\\
        $
          \begin{array}{|c|c|c|c|c|c|c|c|c|}
            \hline
            A & B & C & C^c & A - C & B - C & A \triangle B & (A \triangle B) - C & (A - C) \triangle (B - C) \\
            \hline  \hline
            V & V & V & F   & F     & F     & F             & F                   & F                         \\
            V & V & F & V   & V     & V     & F             & F                   & F                         \\
            \hline
            V & F & V & F   & F     & F     & V             & F                   & F                         \\
            V & F & F & V   & V     & F     & V             & V                   & V                         \\
            \hline
            F & V & V & F   & F     & F     & V             & F                   & F                         \\
            F & V & F & V   & F     & V     & V             & V                   & V                         \\
            \hline
            F & F & V & F   & F     & F     & F             & F                   & F                         \\
            F & F & F & V   & F     & F     & F             & F                   & F                         \\
            \hline
          \end{array}
        $\\
        Hay distribución entre la resta y una diferencias simétrica.

\item \hacer
\item \red{hacer}
\item \red{hacer}
\end{enumerate}
