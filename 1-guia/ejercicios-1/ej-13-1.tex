\begin{enunciado}{\ejercicio}
  Determinar cuáles de las siguientes afirmaciones son verdaderas y
  cualesquiera sean los subconjuntos $A$, $B$ y $C$ de un conjunto
  referencial $\universo$ y cuáles no.
  Para las que sean verdaderas, dar una demostración, para las otras dar un contraejemplo.

  \begin{multicols}{2}
    \begin{enumerate}[label=\roman*)]
      \item $(A \triangle B) - C = (A-C) \triangle (B - C)$.
      \item $(A \inter B) \triangle C = (A \triangle C) \inter (B \triangle C)$
      \item $C \subseteq A \entonces B \inter C \subseteq (A \triangle B)^c$
      \item $A \triangle B = \vacio \sisolosi A = B$
    \end{enumerate}
  \end{multicols}

\end{enunciado}

\begin{enumerate}[label=\roman*)]
  \item $(A \triangle B) - C = (A-C) \triangle (B - C)$. Es verdadera.
        La demo sale fácil con tabla de verdad.

        $$
          \begin{array}{|c|c|c|c|c|c|c|c|c|}
            \hline
            A & B & C & C^c & A - C & B - C & A \triangle B & (A \triangle B) - C & (A - C) \triangle (B - C) \\\hline \rowcolor{Cerulean!10}
            V & V & V & F   & F     & F     & F             & \orange{F}          & \orange{F}                \\
            V & V & F & V   & V     & V     & F             & \orange{F}          & \orange{F}                \\\rowcolor{Cerulean!10}
            V & F & V & F   & F     & F     & V             & \orange{F}          & \orange{F}                \\
            V & F & F & V   & V     & F     & V             & \orange{V}          & \orange{V}                \\\rowcolor{Cerulean!10}
            F & V & V & F   & F     & F     & V             & \orange{F}          & \orange{F}                \\
            F & V & F & V   & F     & V     & V             & \orange{V}          & \orange{V}                \\\rowcolor{Cerulean!10}
            F & F & V & F   & F     & F     & F             & \orange{F}          & \orange{F}                \\
            F & F & F & V   & F     & F     & F             & \orange{F}          & \orange{F}                \\ \hline
          \end{array}
        $$
        Del resultado de la tabla se concluye que hay distribución entre la resta y una diferencia simétrica.

  \item ¡Es falsa!
        Lo demuestro por \textit{contraejemplo}.
        Sean:
        $$
          A = C = \set{1},\, B = \vacio,
        $$
        luego,
        $$
          (A \inter B) \triangle C = \vacio \triangle A = A
        $$
        peeeero,
        $$
          (A \triangle C) \inter (B \triangle C) =
          (A \triangle A) \inter (B \triangle A) =
          \vacio \inter A = \vacio \neq A
        $$
        $$
          \cajaResultado{
            \therefore (A \inter B) \triangle C \neq (A \triangle C) \inter (B \triangle C)
          }
        $$

  \item ¡Es verdadera! Supongo que:
        $$
          C \subseteq A
        $$
        quiero probar que
        $$
          B \inter C \subseteq (A \triangle B)^c
        $$
        Tenemos que:
        $$
          (A \triangle B)^c \igual{def} ((A \union B) \inter (A \inter B)^c)^c
          \igual{\red{!!}} (A^c \inter B^c) \union (A \inter B) \llamada1
        $$
        y también:
        $$
          B \inter C \taa{$C \subseteq A$}[\red{!}]{\subseteq}
          B \inter A = A \inter B \subseteq (A \inter B) \union (A^c \inter B^c)
          \igual{$\llamada1$}
          (A \triangle B)^c
        $$
        Es así que:
        $$
          \cajaResultado{
            B \inter C \subseteq (A \triangle B)^c
          }
        $$

  \item\label{ej-13-1:itemiv} ¡Es verdadera!
        \begin{itemize}
          \item[$\entonces)$]
                $$
                  A \triangle B = \vacio
                  \Entonces{def}
                  (A - B) \union (B - A) = \vacio
                  \Entonces{\red{!}}
                  A - B = \vacio \ytext B - A = \vacio
                  \entonces
                  A = B
                $$

          \item[$\Leftarrow)$]
                $$
                  A = B \entonces A \triangle B = A \triangle A = \vacio
                $$
                Probada la ida y vuelta, queda demostrado que:
                $$
                  \cajaResultado{
                    A \triangle B = \vacio \sisolosi A = B
                  }
                $$
        \end{itemize}
\end{enumerate}

\begin{aportes}
  \item \aporte{https://github.com/MateCon}{Mateo Z \github}
  \item \aporte{\dirRepo}{naD GarRaz \github}
\end{aportes}

