\begin{enunciado}{\ejercicio}

  Determinar cuáles de las siguientes afirmaciones son verdaderas y
  cualesquiera sean los subconjuntos $A$, $B$ y $C$ de un conjunto
  referencial $\universo$ y cuáles no.
  Para las que sean verdaderas, dar una demostración, para las otras dar un contraejemplo.
  \begin{multicols}{2}
    \begin{enumerate}[label=\roman*)]
      \item $(A \triangle B) - C = (A-C) \triangle (B - C)$.
      \item $(A \inter B) \triangle C = (A \triangle C) \inter (B \triangle C)$
      \item $C \subseteq A \entonces B \inter C \subseteq (A \triangle B)^c$
      \item $A \triangle B = \vacio \sisolosi A = B$
    \end{enumerate}
  \end{multicols}

\end{enunciado}

\begin{enumerate}[label=\roman*)]
  \item $(A \triangle B) - C = (A-C) \triangle (B - C)$. Es verdadera. Pruebo con tabla de verdad.\\
        $
          \begin{array}{|c|c|c|c|c|c|c|c|c|}
            \hline
            A & B & C & C^c & A - C & B - C & A \triangle B & (A \triangle B) - C & (A - C) \triangle (B - C) \\
            \hline  \hline
            V & V & V & F   & F     & F     & F             & F                   & F                         \\
            V & V & F & V   & V     & V     & F             & F                   & F                         \\
            \hline
            V & F & V & F   & F     & F     & V             & F                   & F                         \\
            V & F & F & V   & V     & F     & V             & V                   & V                         \\
            \hline
            F & V & V & F   & F     & F     & V             & F                   & F                         \\
            F & V & F & V   & F     & V     & V             & V                   & V                         \\
            \hline
            F & F & V & F   & F     & F     & F             & F                   & F                         \\
            F & F & F & V   & F     & F     & F             & F                   & F                         \\
            \hline
          \end{array}
        $\par
        Hay distribución entre la resta y una diferencias simétrica.

\item Es falsa, lo demuestro por contraejemplo. Sean $A = C = \{1\}, B = \emptyset$, luego \\
$(A \inter B) \triangle C = \emptyset \triangle A = A$ pero \\
$(A \triangle C) \inter (B \triangle C) = (A \triangle A) \inter (B \triangle A) = \emptyset \inter A = \emptyset \neq A$ \\
$\therefore (A \inter B) \triangle C \neq (A \triangle C) \inter (B \triangle C)$
\item Es verdadera. Supongo que $C \subseteq A$, Q.P.Q $B \inter C \subseteq (A \triangle B)^c$ \\
$(A \triangle B)^c = ((A \union B) \inter (A \inter B)^c)^c = (A^c \inter B^c) \union (A \inter B)$ \\
$B \inter C \stackrel{C \subseteq A}{\subseteq} B \inter A = A \inter B \subseteq (A \inter B) \union (A^c \inter B^c) = (A \triangle B)^c$
\item $\text{$\Rightarrow$)} A \triangle B = \emptyset \entonces (A - B) \union (B - A) = \emptyset \entonces A - B = \emptyset \y B - A = \emptyset$ 
$\entonces A \subseteq B \y B \subseteq A \entonces A = B$ \\
$\text{$\Leftarrow$)} A = B \entonces A \triangle B = A \triangle A = \emptyset$
\end{enumerate}
