\begin{enunciado}{\ejercicio}
  Determinar cuáles de las siguientes afirmaciones son verdaderas y
  cualesquiera sean los subconjuntos $A$, $B$ y $C$ de un conjunto
  referencial $\universo$ y cuáles no.
  Para las que sean verdaderas, dar una demostración, para las otras dar un contraejemplo.

  \begin{multicols}{2}
    \begin{enumerate}[label=\roman*)]
      \item $(A \triangle B) - C = (A-C) \triangle (B - C)$.
      \item $(A \inter B) \triangle C = (A \triangle C) \inter (B \triangle C)$
      \item $C \subseteq A \entonces B \inter C \subseteq (A \triangle B)^c$
      \item $A \triangle B = \vacio \sisolosi A = B$
    \end{enumerate}
  \end{multicols}

\end{enunciado}

\begin{enumerate}[label=\roman*)]
  \item $(A \triangle B) - C = (A-C) \triangle (B - C)$. Es verdadera.
        La demo sale fácil con tabla de verdad.

        $$
          \begin{array}{|c|c|c|c|c|c|c|c|c|}
            \hline
            A & B & C & C^c & A - C & B - C & A \triangle B & (A \triangle B) - C & (A - C) \triangle (B - C) \\\hline \rowcolor{Cerulean!10}
            V & V & V & F   & F     & F     & F             & \orange{F}          & \orange{F}                \\
            V & V & F & V   & V     & V     & F             & \orange{F}          & \orange{F}                \\\rowcolor{Cerulean!10}
            V & F & V & F   & F     & F     & V             & \orange{F}          & \orange{F}                \\
            V & F & F & V   & V     & F     & V             & \orange{V}          & \orange{V}                \\\rowcolor{Cerulean!10}
            F & V & V & F   & F     & F     & V             & \orange{F}          & \orange{F}                \\
            F & V & F & V   & F     & V     & V             & \orange{V}          & \orange{V}                \\\rowcolor{Cerulean!10}
            F & F & V & F   & F     & F     & F             & \orange{F}          & \orange{F}                \\
            F & F & F & V   & F     & F     & F             & \orange{F}          & \orange{F}                \\ \hline
          \end{array}
        $$
        Del resultado de la tabla se concluye que hay distribución entre la resta y una diferencia simétrica.

  \item ¡Es falsa!
        Lo demuestro por \textit{contraejemplo}.
        Sean:
        $$
          A = C = \set{1},\, B = \vacio,
        $$
        luego,
        $$
          (A \inter B) \triangle C = \vacio \triangle A = A
        $$
        peeeero,
        $$
          (A \triangle C) \inter (B \triangle C) =
          (A \triangle A) \inter (B \triangle A) =
          \vacio \inter A = \vacio \neq A
        $$
        $$
          \cajaResultado{
            \therefore (A \inter B) \triangle C \neq (A \triangle C) \inter (B \triangle C)
          }
        $$

  \item Este ejercicio sale en 2 patadas haciendo unos \textit{diagramas de Venn}, o casi de cualquier forma. Pero tenía ganas de hacer esta versión,
        porque en las materias de \textit{algoritmos} se usa más esta forma de atacar los problemas.
        \parrafoDestacado{
          \textit{Sugerencia: Amigate con las implicaciones. Ayer.}
        }
        $$
          \ob{
            \ub{
              C \subseteq A
            }{
              \text{\purple{hipótesis}}
            }
            \entonces
            \ub{
              B \inter C \subseteq (A \triangle B)^c
            }{
              \text{\purple{tesis}}
            }
          }{
            \textit{razonamiento}
          }
        $$
        Para que ese \textit{razonamiento} sea \green{válido}, la premisa, la \purple{hipótesis} tiene que ser \green{verdadera} y luego
        la \purple{tesis} también tiene que ser \green{verdadera}.
        Es decir que por \purple{hipótesis} laburo con valores, elementos $x$ tales que:
        $$
          C \subseteq A
          \Sii{\red{!}}
          x \en C \entonces x \en A
        $$
        o equivalentemente:
        $$
          \paratodo x \en \universo : (x \notin C) \lor (x \en A) \quad \llamada1
        $$
        Esa proposición tiene un valor \green{verdadero} \ul{excepto para los $x \en C \land x \notin A$},
        por lo que laburo con los valores de $x$ para los cuales eso \ul{no pasa}.
        Esto es lo que me dice la \purple{hipótesis} $\llamada1$ y \ul{voy a usar más adelante}.

        \bigskip

        Estudio la \purple{tesis}

        La proposición $B \inter C \subseteq (A \triangle B)^c$ en formato lógico:
        $$
          \big(
          \forall x \en \universo
          \big):
          \big(
          (x \en B) \land (x \en C)
          )
          \entonces
          \big(
          (x \en (A \land B^c) ) \lor  (x \en ( A^c \land B))
          \big)^c
          \big) \llamada2
        $$
        Esa proposición es \underline{inmediatamente \purple{verdadera}} si $x \notin C$ (también lo es $\llamada1$), así que el caso que me interesa estudiar,
        que cumple que la \purple{hipótesis} es \textit{verdadera} es,
        $$
          x \en C \land x \en A.
        $$

        Al igual que antes para que el razonamiento
        de $\llamada2$ sea válido, necesito partir de algo \green{verdadero} y llegar a algo \green{verdadero}:
        $$
          \ob{
            \ub{x \en B}{\text{debe ser}\\ \text{\green{verdadera}}} ~\land~ \ub{(x \en C)}{\text{por \purple{hipótesis}}\\\llamada1 \text{\green{verdadera}}}
          }{
            \text{\green{verdadero}}
          }
          \entonces
          \ob{
            \Big(
            \big(
            \ub{(x \en (A \land B^c) )}{\text{\red{falso}}\\\text{porque } x \en B}
            \lor
            \ub{(x \en (A^c \land B) )}{\text{\red{falso}}\\\text{por \purple{hipótesis} } x \en A}
            \big)^c
            \Big)
          }{
            \text{\green{verdadero}}
          }
        $$
        \parrafoDestacado{
          Si te cuestionás el ¿Pero que pasa si $x \notin B$? Bueno, la premisa sería falsa y la implicación $\llamada2$ sería verdadera, porque en la
          implicación $p \entonces q$ con $p$ \red{falso} no importa $q$, la implicación es \green{verdadera} por default.
        }

        Por lo tanto el razonamiento del enunciado resulta válido.
        $$
          \cajaResultado{
            C \subseteq A
            \entonces
            B \inter C \subseteq (A \triangle B)^c
            \quad
            \text{es \green{verdadera}.}
          }
        $$

  \item\label{ej-13-1:itemiv} ¡Es verdadera!
        \begin{itemize}
          \item[\red{$(\Rightarrow)$}]
                $$
                  \begin{array}{rcl}
                    A \triangle B  =  \vacio
                     & \Entonces{def} &
                    (A - B) \union (B - A) = \vacio            \\
                     & \Sii{\red{!}}  &
                    (A - B = \vacio) ~\land~  (B - A = \vacio) \\
                     & \sii           &
                    A = B
                  \end{array}
                $$

          \item[\red{$(\Leftarrow)$}]
                $$
                  A = B \entonces A \triangle B = A \triangle A = \vacio
                $$
                Probada la ida y vuelta, queda demostrado que:
                $$
                  \cajaResultado{
                    A \triangle B = \vacio \sisolosi A = B
                  }
                $$
        \end{itemize}
\end{enumerate}

\begin{aportes}
  \item \aporte{https://github.com/MateCon}{Mateo Z \github}
  \item \aporte{\dirRepo}{naD GarRaz \github}
\end{aportes}

