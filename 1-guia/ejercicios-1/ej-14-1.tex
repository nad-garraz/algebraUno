\begin{enunciado}{\ejercicio}

  Sean $A$, $B$ y $C$ subconjuntos de un conjunto referencial $\universo$. Probar que:
  \begin{multicols}{2}
    \begin{enumerate}[label=\roman*)]
      \item $A \inter (B \triangle C) = (A \inter B) \triangle (A \inter C)$.
      \item $A - (B - C) = (A-B) \union (A \inter C)$.
      \item $A \triangle B \subseteq (A \triangle C) \union (B \triangle C)$.
      \item $(A \inter B)^c \union (C \inter D)^c = (A \inter C)^c \union (B \inter D)^c$.
      \item $A \subseteq B \entonces A \triangle B = B \inter A^c$.
      \item $A \inter C = \vacio \entonces A \inter (B \triangle C) = A \inter B$.
      \item $(A \inter C ) - B = (A - B) \inter C$.
      \item $A \subseteq B \sisolosi B^c \subseteq A^c$.
    \end{enumerate}
  \end{multicols}

\end{enunciado}

\begin{enumerate}[label=\roman*)]
  \item\label{ej-14-1:itemi}
        Voy a usar tablas con los resultados que hay \hyperlink{teoria-1:tablasDeVerdad}{en las tablas de verdad acá.}
        $$
          \begin{array}{|c|c|c|c|c|c|c|c|}
            \hline
            A & B & C & B \triangle C & A \inter B & A \inter C & A \inter (B \triangle C) & (A \inter B) \triangle (A \inter C) \\ \rowcolor{Cerulean!10}\hline
            V & V & V & F             & V          & V          & \magenta{F}              & \magenta{F}                         \\
            V & V & F & V             & V          & F          & \magenta{V}              & \magenta{V}                         \\ \rowcolor{Cerulean!10}
            V & F & V & V             & F          & V          & \magenta{V}              & \magenta{V}                         \\
            V & F & F & F             & F          & F          & \magenta{F}              & \magenta{F}                         \\ \rowcolor{Cerulean!10}
            F & V & V & F             & F          & F          & \magenta{F}              & \magenta{F}                         \\
            F & V & F & V             & F          & F          & \magenta{F}              & \magenta{F}                         \\ \rowcolor{Cerulean!10}
            F & F & V & V             & F          & F          & \magenta{F}              & \magenta{F}                         \\
            F & F & F & F             & F          & F          & \magenta{F}              & \magenta{F}                         \\  \hline
          \end{array}
        $$

  \item Este sale sin tablas:
        Tratá de hacerlo con estas propiedades, \hyperlink{teoria-1:basicos-conjuntos}{(notas teóricas acá)}:
        \begin{enumerate}[label=\arabic*)]
          \item Notación de diferencia
          \item Distributivas
          \item DeMorgan
        \end{enumerate}

        $$
          \scriptstyle
          (A - B) \union (A \inter C)
          \igual{\red!}[]
          \cyan{[(A \inter B^c) \union A ]} \inter [ (A \inter B^c) \union C]
          \igual{\red{!!}}[]
          \cyan{A} \inter (A \union C) \inter (B^c \union C)
          \igual{\red{!!!}}[]
          A \inter (\magenta{B \inter C^c})^c =
          A \inter (\magenta{B - C})^c
          \igual{\red{!}}
          A - (B - C) \Tilde
        $$

  \item\label{ej-14-1:itemiii} Opción 1, con diagramas de Venn:\par
        \begin{center}
          $A \triangle B \subseteq (A \triangle C) \union (B \triangle C)$:\par
          \begin{venndiagram3sets}[shade=blue!30!white, showframe = false,hgap=0, vgap=0, overlap = 1.1cm]
            \fillANotB
            \fillBNotA
          \end{venndiagram3sets}
          $\taa{\Tilde}{\subseteq}$
          \begin{venndiagram3sets}[shade=orange!30!white, showframe = false,hgap=0, vgap=0, overlap = 1.1cm]
            \fillANotB
            \fillBNotC
            \fillCNotA
          \end{venndiagram3sets}
        \end{center}

        Opción 2, para probar que un conjunto es subconjunto de otro,
        me alcanza con probar que para cualquier elemento de $\universo$, si pertenece al primero entonces pertenece al segundo.

        Luego, quiero probar que
        $$
          x \en A \triangle B
          \entonces
          x \en (A \triangle C) \union (B \triangle C), \paratodo x \en \universo
        $$
        Hay que acomdar las expresiones para hacer el seguimiento del elmento $x$:
        $$
          x \en A \triangle B
          \Sii{def}
          \ub{(x \en A \y x \notin B)}{I}
          ~\lor~
          \ub{(x \notin A \y x \en B)}{II},
        $$
        la otra parte:
        $$
          x \en (A \triangle C) \union (B \triangle C)
          \Sii{def}
          ((x \en A \y x \notin C)
          ~\lor~
          (x \notin A \y x \en C))
          ~\lor~
          ((x \en B \y x \notin C) ~\lor~ (x \notin B \y x \en C))
        $$

        $$
          \sii
          (x \en A \y x \notin C)
          ~\lor~
          (x \notin A \y x \en C)
          ~\lor~
          (x \en B \y x \notin C)
          ~\lor~
          (x \notin B \y x \en C).
        $$

        Se que $x \in A \triangle B \entonces I ~\lor~ II$. Separo en casos,
        $$
          \text{Si }I\text{ es Verdadero, }I \Entonces{$\llamada1$} (x \in A \y x \notin C) ~\lor~ (x \notin B \y x \in C) \entonces x \in (A \triangle C) \union (B \triangle C)   \\
        $$
        $$
          \text{Si }II\text{ es Verdadero, }II \Entonces{$\llamada1$} (x \notin A \y x \in C) ~\lor~ (x \in B \y x \notin C) \entonces x \in (A \triangle C) \union (B \triangle C) \\
        $$
        $$
          \text{Si }I \y II\text{ es Verdadero, }I \y II \entonces I
          \Entonces{idem}
          x \in (A \triangle C) \union (B \triangle C)
        $$
        $$
          \cajaResultado{
            \therefore x \in A \triangle B \entonces x \in (A \triangle C) \union (B \triangle C),
          }
        $$
        como quería probar.

        $\llamada1$ Observo que $(\text{Verdadero} \y p) ~\lor~ (\text{Verdadero } \y \neg p)$ es una tautología.

  \item $(A \inter B)^c \union (C \inter D)^c = (A \inter C)^c \union (B \inter D)^c$
        Las intrucciones: Intentalo y después mirá la solución.
        \begin{enumerate}[label=\arabic*)]
          \item De Morgan
          \item Conmutatividad, asociatividad de la unión
          \item De Morgan nuevamente
        \end{enumerate}
        $$
          \begin{array}{rcl}
            (A \inter B)^c \union (C \inter D)^c
             & = &
            (A^c \union B^c) \union (C^c \union D^c) \\
             & = &
            A^c\union B^c \union C^c  \union D^c     \\
             & = &
            A^c \union C^c \union B^c \union D^c     \\
             & = &
            (A^c \union C^c) \union (B^c \union D^c) \\
             & = &
            (A \inter C)^c \union (B \inter D)^c
          \end{array}
        $$

  \item Para probar la igualdad, hay que probar la ida y la vuelta:
        \begin{itemize}
          \item[\red{($\Rightarrow$)}]
                Por hipótesis del ejercicio:
                $$
                  A \subseteq B \llamada1
                $$
                $$
                  A \triangle B
                  \igual{def}
                  A - B \union B - A
                  \igual{\red{!!}}[$\llamada1$]
                  B - A = B \inter A^c
                $$

          \item[$\red{(\Leftarrow)}$]
                La vuelta es similar:
                $$
                  B \inter A^c
                  \igual{def}
                  B - A
                  =
                  B - A \union \blue{\vacio}
                  \igual{\red{!!}}[$\llamada1$]
                  B - A \union \blue{A - B}
                  \igual{def}
                  A \triangle B
                $$

                ¿Había qué hacer la ida y la vuelta? \faIcon[regular]{grin-beam-sweat}
        \end{itemize}
  \item Sale casi en forma directa:
        $$
          (A \inter C) - B
          \igual{def}
          (A \inter C) \inter B^c
          \igual{\red{!}}
          (A \inter B^c) \inter C
          \igual{def}
          (A - B) \inter C
        $$
        Queda así demostrada la igualdad.

  \item Mirando el ítem \ref{ej-14-1:itemi} del ejercicio sale en dos patadas.
        Tené en cuenta que:
        $$
          X \triangle \vacio
          \igual{\red{!}}
          X
        $$
        Todo tuyo!

  \item Este lo resuelvo \textit{pulenta} para los puristas que subestiman
        las \textit{tablas de verdad} y los \textit{diagramas de Venn}. En otras
        palabras para la gente que le gusta complicar las cosas innecesariamente \faIcon[regular]{grin-tongue}.
        \begin{itemize}

          \item[\red{($\Rightarrow$)}]
                Quiero probar que:
                $$
                  A \subseteq B
                  \entonces
                  B^c \subseteq A^c
                $$
                $A$ subconjunto de $B$ se puede escribir en lenguaje lógico como:
                $$
                  \big[
                    A \subseteq B
                    \big]
                  \taa{def}{\equivalente}
                  \big[
                    A \entonces B
                    \big]
                  \taa{$\llamada1$}{\equivalente}
                  \big[
                    \paratodo x \en \universo : (x \notin A) \lor (x \in B)
                    \big]
                $$
                Esos corchetes son equivalentes ¡Tienen la misma información! Tomate
                el tiempo hasta que lo veas. Laburo con el último corchete:
                $$
                  \begin{array}{rcl}
                    \paratodo x \en \universo : (x \notin A) \lor (x \in B)
                     & \Sii{def}                                 &
                    \paratodo x \en \universo : (x \en A^c) \lor (x \notin B^c)  \\
                     & \Sii{conmutatividad}[del $\red{\lor}$]    &
                    \paratodo x \en \universo :  (x \notin B^c) \lor (x \en A^c) \\
                     & \taa{$\llamada1$}[\red{!!}]{\equivalente} &
                    B^c \entonces  A^c                                           \\
                     & \sii                                      &
                    B^c \subseteq  A^c                                           \\
                  \end{array}
                $$

          \item[\red{($\Leftarrow$)}]
                La vuelta es lo meeeeeesmo. Toda tuya.
        \end{itemize}

\end{enumerate}

% Contribuciones
\begin{aportes}
  \item \aporte{https://github.com/nad-garraz}{naD GarRaz \github}
  \item \aporte{https://github.com/MateCon}{Mateo Z \github}
\end{aportes}
