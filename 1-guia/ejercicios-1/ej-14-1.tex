\begin{enunciado}{\ejercicio}

  Sean $A$, $B$ y $C$ subconjuntos de un conjunto referencial $\universo$. Probar que:
  \begin{multicols}{2}
    \begin{enumerate}[label=\roman*)]
      \item $A \inter (B \triangle C) = (A \inter B) \triangle (A \inter C)$
      \item $A - (B - C) = (A-B) \union (A \inter C)$
      \item $A \triangle B \subseteq (A \triangle C) \union (B \triangle C)$
      \item $(A \inter C ) - B = (A - B) \inter C$
      \item $A \subseteq B \entonces A \triangle B = B \inter A^c$
      \item $A \subseteq C \sisolosi B^c \subseteq A^c$
      \item $A \inter C = \vacio \entonces A \inter (B \triangle C) = A \inter B$
    \end{enumerate}
  \end{multicols}

\end{enunciado}

\begin{enumerate}[label=\roman*)]
  \item\label{ej-14-1:itemi}
        Voy a usar tablas con los resultados que hay \hyperlink{teoria-1:tablasDeVerdad}{en las tablas de verdad acá.}
        $$
          \begin{array}{|c|c|c|c|c|c|c|c|}
            \hline
            A & B & C & B \triangle C & A \inter B & A \inter C & A \inter (B \triangle C) & (A \inter B) \triangle (A \inter C) \\
            \hline  \hline
            V & V & V & F             & V          & V          & \magenta{F}              & \magenta{F}                         \\
            V & V & F & V             & V          & F          & \magenta{V}              & \magenta{V}                         \\
            \hline
            V & F & V & V             & F          & V          & \magenta{V}              & \magenta{V}                         \\
            V & F & F & F             & F          & F          & \magenta{F}              & \magenta{F}                         \\
            \hline
            F & V & V & F             & F          & F          & \magenta{F}              & \magenta{F}                         \\
            F & V & F & V             & F          & F          & \magenta{F}              & \magenta{F}                         \\
            \hline
            F & F & V & V             & F          & F          & \magenta{F}              & \magenta{F}                         \\
            F & F & F & F             & F          & F          & \magenta{F}              & \magenta{F}                         \\
            \hline
          \end{array}
        $$

  \item Este sale sin tablas:
        Tratá de hacerlo con estas propiedades, \hyperlink{teoria-1:basicos-conjuntos}{(notas teóricas acá)}:
        \begin{enumerate}[label=\arabic*)]
          \item Notación de diferencia
          \item Distributivas
          \item DeMorgan
        \end{enumerate}

        $$
          \scriptstyle
          (A - B) \union (A \inter C)
          \igual{\red!}[]
          \cyan{[(A \inter B^c) \union A ]} \inter [ (A \inter B^c) \union C]
          \igual{\red{!!}}[]
          \cyan{A} \inter (A \union C) \inter (B^c \union C)
          \igual{\red{!!!}}[]
          A \inter (\magenta{B \inter C^c})^c =
          A \inter (\magenta{B - C})^c
          \igual{\red{!}}
          A - (B - C) \Tilde
        $$

  \item\label{ej-14-1:itemiii} Opción 1, con diagramas de Venn:\par
    \begin{center}
      $A \triangle B \subseteq (A \triangle C) \union (B \triangle C)$:\par
      \begin{venndiagram3sets}[shade=blue!30!white, showframe = false,hgap=0, vgap=0, overlap = 1.1cm]
        \fillANotB
        \fillBNotA
      \end{venndiagram3sets}
      $\taa{\Tilde}{}{\subseteq}$
      \begin{venndiagram3sets}[shade=orange!30!white, showframe = false,hgap=0, vgap=0, overlap = 1.1cm]
        \fillANotB
        \fillBNotC
        \fillCNotA
      \end{venndiagram3sets}
    \end{center}
  
    Opción 2, para probar que un conjunto es subconjunto de otro me alcanza con probar que para cualquier elemento de $\universo$, si pertenece al primero entonces pertenece al segundo. \\
    Luego, Q.P.Q. $x \in A \triangle B \entonces x \in (A \triangle C) \union (B \triangle C), \paratodo x \in \universo$ \\
    $x \in A \triangle B \stackrel{\text{def}}{\sii} \underbrace{(x \in A \y x \notin B)}_I \o \underbrace{(x \notin A \y x \in B)}_{II}$ \\
    $x \in (A \triangle C) \union (B \triangle C) \stackrel{\text{def}}{\sii} ((x \in A \y x \notin C) \o (x \notin A \y x \in C)) \o ((x \in B \y x \notin C) \o (x \notin B \y x \in C))$\\
    $\sii (x \in A \y x \notin C) \o (x \notin A \y x \in C) \o (x \in B \y x \notin C) \o (x \notin B \y x \in C)$\\

    Se que $x \in A \triangle B \entonces I \o II$. Separo en casos, \\
    $\begin{cases}
      \text{Si }I\text{ es Verdadero, }I \stackrel{\textcolor{orange}*}{\entonces} (x \in A \y x \notin C) \o (x \notin B \y x \in C) \entonces x \in (A \triangle C) \union (B \triangle C) \\
      \text{Si }II\text{ es Verdadero, }II \stackrel{\textcolor{orange}*}{\entonces} (x \notin A \y x \in C) \o (x \in B \y x \notin C) \entonces x \in (A \triangle C) \union (B \triangle C) \\
      \text{Si }I \y II\text{ es Verdadero, }I \y II \entonces I \stackrel{\text{idem}}{\entonces} x \in (A \triangle C) \union (B \triangle C)
    \end{cases}$ \\
    $\therefore x \in A \triangle B \entonces x \in (A \triangle C) \union (B \triangle C)$, como quería probar. \\
    {\textcolor{orange}*} Observo que $(\text{Verdadero} \y p) \o (\text{Verdadero } \y \neg p)$ es una tautología.


  \item  \Hacer
  \item  \Hacer
  \item  \Hacer
  \item  Mirando el item \ref{ej-14-1:itemi} sale solo. Dado que $X \triangle \vacio \igual{\red{!}} X$
\end{enumerate}
