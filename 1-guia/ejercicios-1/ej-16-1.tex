\begin{enunciado}{\ejercicio}
  Sean $A, B$ y $C$ conjuntos. Probar que:
  \begin{enumerate}[label=\roman*)]
    \item $(A \union B) \times C = (A \times C) \union (B \times C)$
    \item $(A \inter B) \times C = (A \times C) \inter (B \times C)$
    \item $(A - B) \times C = (A \times C) - (B \times C)$
    \item $(A \triangle B) \times C = (A \times C) \triangle (B \times C)$
  \end{enumerate}
\end{enunciado}

\begin{enumerate}[label=\roman*)]
  \item
        Para demostrar igualdad de conjuntos habría que probar la doble inclusión, es decir:

        $$
          \begin{array}{l}
            (A \union B) \times C \magenta{\subseteq} (A \times C) \union (B \times C) \\
            (A \times C) \union (B \times C) \magenta{\subseteq} (A \union B) \times C
          \end{array}
        $$
        O bien si podemos conectar los pasos con "$\sisolosi$". En este caso se usa el de los "$\sisolosi$" y mucho
        de las \hyperlink{teoria-1:basicos-conjuntos}{definiciones que podés ver acá en las notas teóricas}:

        Sea el par $(x,y)$
        $$
          \begin{array}{c}
            (x,y) \en (A \union B) \times C)
            \Sii{def prod.}[Cartesiano]
            x \en (A \union B) \ytext y \en C
            \Sii{def}[$\union$]
            (x \en A \otext x \en  B  )\ytext x \en C
          \end{array}
        $$
        Si está en $A$ o en $B$ y seguro está en $C$, entonces $x$ tiene que estar en $A \inter C$ o bien en $B \inter C$,
        que no es otra cosa que distribuir el "y" con el "o":
        $$
          \Sii{distribución}
          (x \en A  \ytext x \en C) \otext (x \en  B \ytext x \en C)
          \Sii{\red{!}}
          (x,y) \en (A \times C) \otext (x,y) \en (B\times C)
        $$
        Ese paso del \red{!} es la definición de producto cartesiano como al principio y se concluye que:
        $$
          (A \union B) \times C \igual{\magenta{\Tilde}}
          (A \times C) \union (B\times C)
        $$
  \item $(A \inter B) \times C = (A \times C) \inter (B \times C)$
        \begin{align*}
           & (x,y) \en (A \inter B) \times C \sii x \en A \inter B \land y \en C \sii x \en A \land x \en B \land y \en C    \\
           & \sii (x \en A \land y \en C) \land (x \en B \land y \en C) \sii (x,y) \en A \times C \land (x,y) \en B \times C \\
           & \sii (x,y) \en (A \times C) \inter (B \times C)
        \end{align*}
  \item $(A - B) \times C = (A \times C) - (B \times C)$
        \begin{align*}
           & (x,y) \en (A \times C) - (B \times C) \sii (x,y) \en (A \times C) \land (x,y) \notin (B \times C)  \\
           & \sii (x \en A \land y \en C) \land (x \notin B \lor y \notin C)                                    \\
           & \Sii{Aux.1} (x \en A \land y \en C \land x \notin B) \lor (x \en A \land y \en C \land y \notin C) \\
           & \Sii{Aux.2}  x \en A \land y \en C \land x \notin B  \sii x \en (A-B) \land y \en C
          \sii (x,y) \en (A-B) \times C
        \end{align*}
        \paragraph{Auxiliar 1}{
          Sean las proposiciones $p$, $q$ y $r$. Podemos distribuir el $\land$ con respecto a $\lor$
          \begin{align*}
            p \land (q \lor r) \sii (p \land q) \lor (p \land r)
          \end{align*}
          Tomemos
          \begin{align*}
             & p: x \en A \land y \en C \\
             & q: x \notin B            \\
             & r: y \notin C
          \end{align*}
          Entonces
          \begin{align*}
            (x \en A \land y \en C) \land (x \notin B \lor y \notin C) \sii
            (x \en A \land y \en C \land x \notin B) \lor (x \en A \land y \en C \land y \notin C)
          \end{align*}
        }
        \paragraph{Auxiliar 2}{
          Sean las proposiciones $p$ y $q$, donde $q$ es falsa. Entonces
          \begin{align*}
            p \lor q \sii p
          \end{align*}
          Tomemos
          \begin{align*}
             & p: x \en A \land y \en C \land x \notin B \\
             & q: x \en A \land y \en C \land y \notin C
          \end{align*}
          Deberiamos ver que valor de verdad de $q$ es falso. En $q$ tenemos como condición que $y \en C$ y que $y \notin C$,
          y esto no puede ser posible, por lo tanto $q$ es falsa. Entonces
          \begin{align*}
            (x \en A \land y \en C \land x \notin B) \lor (x \en A \land y \en C \land y \notin C)
            \sii x \en A \land y \en C \land x \notin B
          \end{align*}
        }
  \item $(A \triangle B) \times C = (A \times C) \triangle (B \times C)$ \\
        Veamos dos formas de demostrar esto, una es con igualdad de conjuntos, la que venimos usando en las demostraciones
        anteriores y la otra es usando los puntos (i) y (iii).
        \begin{enumerate}
          \item Por igualdad de conjuntos
                \begin{align*}
                   & (x,y) \en (A \times C) \triangle (B \times C)
                  \Sii{Aux.1} (x,y) \en ((A \times C) - (B \times C)) \union ((B \times C) - (A \times C))       \\
                   & \sii (x,y) \en ((A \times C) - (B \times C)) \lor (x,y) \en ((B \times C) - (A \times C))   \\
                   & \sii ((x,y) \en A \times C \land (x,y) \notin B \times C)
                  \lor ((x,y) \en B \times C \land (x,y) \notin A \times C)                                      \\
                   & \sii (x \en A \land y \en C \land (x \notin B \lor y \notin C))
                  \lor (x \en B \land y \en C \land (x \notin A \lor y \notin C))                                \\
                   & \sii (x \en A \land y \en C \land x \notin B) \lor (x \en A \land y \en C \land y \notin C) \\
                   & \phantom{\hspace{22px}} \lor (x \en B \land y \en C \land x \notin A)
                  \lor (x \en B \land y \en C \land y \notin C)                                                  \\
                   & \Sii{Aux.2}   (x \en A \land y \en C \land x \notin B)
                  \lor (x \en B \land y \en C \land x \notin A)                                                  \\
                   & \sii (x \en (A - B) \land y \en C) \lor (x \en (B - A) \land y \en C)                       \\
                   & \sii (x \en (A - B) \lor x \en (B - A)) \land y \en C
                  \sii x \en (A - B) \union (B - A) \land y \en C                                                \\
                   & \sii x \en A \triangle B \land y \en C \sii (x,y) \en (A \triangle B) \times C
                \end{align*}
          \item Usando los puntos (i) y (iii)
                \begin{align*}
                                  & (A \times C) \triangle (B \times C)
                  \igual{Aux}[1] ((A \times C) - (B \times C)) \union ((B \times C) - (A \times C)) \overset{(iii)}{=} \\
                  \overset{(iii)} & {=} ((A - B) \times C) \union ((B-A) \times C)
                  \igual{$(i)$} ((A - B) \union (B - A)) \times C
                  \igual{Aux}[1] (A \triangle B) \times C                                                              \\
                                  & \entonces (A \times C) \triangle (B \times C) = (A \triangle B) \times C
                \end{align*}
        \end{enumerate}
        \paragraph{Auxiliar 1}{Sean los conjuntos $A,B \subseteq V$
          \begin{align*}
            A \triangle B = (A - B) \union (B - A)
          \end{align*}
        }
        \paragraph{Auxiliar 2}{
          Sean las proposiciones $p$ y $q$, donde $q$ es falsa. Entonces
          \begin{align*}
            p \lor q \sii p
          \end{align*}
          Podemos tomar
          \begin{align*}
            p: x \en A \land y \en C \land x \notin B \\
            q: x \en A \land y \en C \land y \notin C
          \end{align*}
          donde claramente $q$ es falso pues $y \en C$ e $y \notin C$. Entonces
          \begin{align*}
            (x \en A \land y \en C \land x \notin B) \lor (x \en A \land y \en C \land y \notin C)
            \sii x \en A \land y \en C \land x \notin B
          \end{align*}
          Para este otro caso diferente al anterior, podemos tomar
          \begin{align*}
            p: x \en B \land y \en C \land x \notin A \\
            q: x \en B \land y \en C \land y \notin C
          \end{align*}
          donde $q$ es falso pues $y \en C$ e $y \notin C$. Entonces
          \begin{align*}
            (x \en B \land y \en C \land x \notin A) \lor (x \en B \land y \en C \land y \notin C)
            \sii x \en B \land y \en C \land x \notin A
          \end{align*}
        }
\end{enumerate}

% Contribuciones
\begin{aportes}
  \item \aporte{\dirRepo}{naD GarRaz \github}
  \item \aporte{https://github.com/koopardo}{Marcos Zea \github}
\end{aportes}
