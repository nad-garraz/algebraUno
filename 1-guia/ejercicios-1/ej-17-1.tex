\def\nodosDiecisiete{
  \node[nodo=black, label=left:$1$] (1) {};
  \node[nodo=black, label=left:$2$, below of=1] (2) {};
  \node[nodo=black, label=left:$3$, below of=2] (3) {};

  \draw[thick, rounded corners = 10pt]
  ([xshift=5pt,yshift=-5pt]current bounding box.south east)
  rectangle
  ([xshift=-5pt,yshift=5pt]current bounding box.north west) node [left] {$A$};

  \node[nodo=black, label=right:$1$, right of=1] (1b) {};
  \node[nodo=black, label=right:$3$, below of=1b] (3b) {};
  \node[nodo=black, label=right:$5$, below of=3b] (5) {};
  \node[nodo=black, label=right:$7$, below of=5] (7) {};

  \draw[rectangle, draw, fit={(1b) (3b) (5) (7)}, thick, rounded corners = 10pt]
  ([xshift=-5pt, yshift=-5pt]7.south west)
  rectangle
  ([xshift=20pt, yshift=5pt]1b.north east) node [left] {$B$};
}

\separador

\underline{\textit{Relaciones}}

\begin{enunciado}{\ejercicio}
  Sean $A = \set{1, 2, 3}$ y $B = \set{1, 3, 5, 7}$. Verificar las siguientes
  relaciones de $A$ y $B$ y en caso afirmativo graficarlas por medio de un diagrama
  con flechas de $A$ en $B$ y por medio de puntos en el producto cartesiano $A \times B$.
  \begin{enumerate}[label=\roman*)]
    \begin{multicols}{2}
      \item  $\relacion = \set{(1,1), (1,3), (1,7), (3,1), (3,5)}$
      \item $\relacion = \set{(1,1), (1,3), (2,7), (3,2), (3,5)}$
      \item $\relacion = \set{(1,1), (2,7), (3,7)}$
      \item  $\relacion = \set{(1,3), (2,1), (3,7)}$
    \end{multicols}
  \end{enumerate}
\end{enunciado}

Es útil traer la \textit{definición de Relación, $\relacion$:}
\parrafoDestacado{
  Sean $A$ y $B$ conjuntos. Una \textit{relación $\relacion$ de $A$ en $B$}
  es un subconjunto cualquiera $\relacion$ del producto cartesiano $A \times B$. Es decir $\relacion$
  de $A$ en $B$ si $\relacion \en \partes(A \times B)$.
}
\begin{enumerate}[label=\roman*)]
  \item Esta es una $\relacion$:
        $$
          \begin{tikzpicture}
            [
            scale = 0.7,
            node distance=1.2cm,
            nodo/.style={circle, fill=black, color={#1}, minimum size = 5pt, inner sep = 2pt},
            arista/.style={-{Latex[length=3pt]}, ultra thin, bend left=25, color={#1}},
            ]
            \nodosDiecisiete
            \draw[arista=Cerulean] (1) to (1b);
            \draw[arista=Cerulean] (1) to (3b);
            \draw[arista=Cerulean, bend right=10] (1) to (7);
            \draw[arista=Cerulean] (3) to (3b);
            \draw[arista=Cerulean, bend right] (3) to (5);
          \end{tikzpicture}
        $$

  \item No es una relación dado que $3 \relacion 2 \notin \partes(A \times B) $

  \item Es una relación:
        $$
          \begin{tikzpicture}
            [
            scale = 0.7,
            node distance=1.2cm,
            nodo/.style={circle, fill=black, color={#1}, minimum size = 5pt, inner sep = 2pt},
            arista/.style={-{Latex[length=3pt]}, ultra thin, bend left=25, color={#1}},
            ]
            \nodosDiecisiete
            \draw[arista=violet] (1) to (1b);
            \draw[arista=violet, bend left=10] (2) to (7);
            \draw[arista=violet, bend right] (3) to (7);
          \end{tikzpicture}
        $$

  \item Es una relación:
        $$
          \begin{tikzpicture}
            [
            scale = 0.7,
            node distance=1.2cm,
            nodo/.style={circle, fill=black, color={#1}, minimum size = 5pt, inner sep = 2pt},
            arista/.style={-{Latex[length=3pt]}, ultra thin, bend left=25, color={#1}},
            ]
            \nodosDiecisiete
            \draw[arista=orange] (1) to (3b);
            \draw[arista=orange, bend left=10] (2) to (1b);
            \draw[arista=orange, bend right] (3) to (7);
          \end{tikzpicture}
        $$
\end{enumerate}

\begin{aportes}
  \item \aporte{\dirRepo}{naD GarRaz \github}
\end{aportes}
