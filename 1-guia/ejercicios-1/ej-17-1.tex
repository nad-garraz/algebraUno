\def\diecisietei{
  \begin{tikzpicture}[scale=0.7, >=Latex, draw=Aquamarine, transform shape]
    %A vértices
    \node (1a) {$\bullet$};
    \node[] at (1a.west) {1};
    \node[below=of 1a] (2a) {$\bullet$};
    \node[] at (2a.west) {2};
    \node[below=of 2a] (3a) {$\bullet$};
    \node[] at (3a.west) {3};

    %B vértices
    \node[right=2cm of 1a] (1b) {$\bullet$};
    \node[] at (1b.east) {1};
    \node[below right=of 1b] (3b) {$\bullet$};
    \node[] at (3b.east) {3};
    \node[below left=of 3b] (5b) {$\bullet$};
    \node[] at (5b.east) {5};
    \node[below right=of 5b] (7b) {$\bullet$};
    \node[] at (7b.east) {7};

    % Elipses
    \node[shape=ellipse, draw, black, minimum size=2cm,fit={(1a) (3a)}] {};
    \node[shape=ellipse, draw, black, minimum size=2cm,fit={(1b) (3b) (7b)}] {};
    \node[below=1cm of 3a] {$A$};
    \node[below=1.2cm of 7b] {$B$};

    % Aristas
    \draw[->, bend left] (1a) to (1b);
    \draw[->, bend right] (1a) to (3b);
    \draw[->, bend right] (1a) to (7b);
    \draw[->, bend right] (3a) to (1b);
    \draw[->, bend right] (3a) to (5b);
  \end{tikzpicture}
}

\separador
\underline{\textit{Relaciones}}
Definición de Relación, $\relacion$:
\begin{center}
  \fbox{
    \parbox{.9\textwidth}{
      Sean $A$ y $B$ conjuntos. Una \textit{relación $\relacion$ de $A$ en $B$}
      es un subconjunto cualquiera $\relacion$ del producto cartesiano $A \times B$. Es decir $\relacion$
      de $A$ en $B$ si $\relacion \en \partes(A \times B)$.
    }
  }
\end{center}

\begin{enunciado}{\ejercicio}
  Sean $A = \set{1, 2, 3}$ y $B = \set{1, 3, 5, 7}$. Verificar las siguientes
  relaciones de $A$ y $B$ y en caso afirmativo graficarlas por medio de un diagrama
  con flechas de $A$ en $B$ y por medio de puntos en el producto cartesiano $A \times B$.
\end{enunciado}
\begin{enumerate}[label=\roman*)]
  \item
        \begin{minipage}{1\textwidth}
          \begin{minipage}{0.5\textwidth}
            $\relacion = \set{(1,1), (1,3), (1,7), (3,1), (3,5)}$
          \end{minipage}
          \begin{minipage}{0.25\textwidth}
            \diecisietei
          \end{minipage}
        \end{minipage}

  \item $\relacion =
          \set{(1,1), (1,3), (2,7), (3,2), (3,5)}
          \to
          3 \relacion 2 \notin \partes(A \times B) $

  \item $\relacion = \set{(1,1), (2,7), (3,7)}$
        \Hacer

  \item $\relacion = \set{(1,3), (2,1), (3,7)}$
        \Hacer
\end{enumerate}
