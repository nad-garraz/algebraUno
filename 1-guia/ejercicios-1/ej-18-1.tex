\begin{enunciado}{\ejercicio}
  Sean $A = \set{1, 2, 3}$ y $B = \set{1, 3, 5, 7}$. Describir por extensión cada una de las
  siguientes relaciones de $A$ en $B$:

  \begin{multicols}{2}
    \begin{enumerate}[label=\roman*)]
      \item $(a,b) \en \relacion \sisolosi a \leq b $

      \item $(a,b) \en \relacion \sisolosi a > b$

      \item $(a,b) \en \relacion \sisolosi a \cdot b$ es par

      \item $(a,b) \en \relacion \sisolosi a + b > 6$
    \end{enumerate}
  \end{multicols}
\end{enunciado}

\begin{enumerate}[label=\roman*)]
  \item $(a,b) \en \relacion \sisolosi a\leq b \to   (a,b) \en \relacion \sii   \set{(1,1), (1,3), (1,5), (1,7), (2,3), (2,5), (2,7), (3,3), (3,5), (3,7)}$

  \item $(a,b) \en \relacion \sisolosi a > b \to     (a,b) \en \relacion \sii   \set{(2,1), (3, 1)}$

  \item $(a,b) \en \relacion \sisolosi a \cdot b \to (a,b) \en \relacion \sii    \set{(2,1), (2,3), (2,5), (2,7)}$

  \item $(a,b) \en \relacion \sisolosi a + b > 6 \to (a,b) \en \relacion \sii\set{(1,7), (2,5), (2, 7), (3, 5), (3, 7)}$
\end{enumerate}
