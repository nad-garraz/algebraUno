% Gráficos

\def\diecinuevei{
  \begin{tikzpicture}[scale=0.5, baseline=0, >=Latex, draw=Aquamarine]

    \node[] (a) {$\bullet$};
    \node[] at (a.west) {$a$};

    \node[above right= of a] (b) {$\bullet$};
    \node[] at (b.east) {$b$};

    \node[below right= of b] (c) {$\bullet$};
    \node[] at (c.east) {$c$};

    \node[below= of a] (d) {$\bullet$};
    \node[] at (d.west) {$d$};

    \node[below= of c] (e) {$\bullet$};
    \node[] at (e.west) {$e$};

    \node[right= of d] (f) {$\bullet$};
    \node[] at (f.west) {$f$};

    \node[right= of c] (g) {$\bullet$};
    \node[] at (g.north) {$g$};

    \node[below= of g] (h) {$\bullet$};
    \node[] at (h.south) {$h$};

    % Universo
    \node[shape=ellipse, draw, black, fit={ (b) (d) (g) (e)}] (universo) {};
    \node[above left = 0.1cm of universo] {$A$};

    % Aristas
    \draw[->, bend left] (a.center) to (b.center);
    \draw[->, bend left] (b.center) to (a.center);
    \draw[->, bend right] (c.center) to (d.center);
    \draw[->, loop above] (c) to (c);
    \draw[->, loop below ] (f) to (f);
    \draw[->, bend right] (c.center) to (h.center);
    \draw[->, bend left] (e.center) to (c.center);
    \draw[->, bend right] (h.center) to (g.center);
  \end{tikzpicture}
}

% 19 ii
\def\diecinueveiv{
  \begin{tikzpicture}[scale=0.5, baseline=0, >=Latex, draw=Aquamarine]

    \node[] (a) {$\bullet$};
    \node[] at (a.west) {$a$};

    \node[above right= of a] (b) {$\bullet$};
    \node[] at (b.east) {$b$};

    \node[below right= of b] (c) {$\bullet$};
    \node[] at (c.east) {$c$};

    \node[below= of a] (d) {$\bullet$};
    \node[] at (d.west) {$d$};

    \node[below= of c] (e) {$\bullet$};
    \node[] at (e.west) {$e$};

    \node[right= of d] (f) {$\bullet$};
    \node[] at (f.west) {$f$};

    \node[right= of c] (g) {$\bullet$};
    \node[] at (g.east) {$g$};

    \node[below= of g] (h) {$\bullet$};
    \node[] at (h.east) {$h$};

    % Universo
    \node[shape=ellipse, draw, black, fit={ (b) (d) (g) (e)}] (universo) {};
    \node[above left = 0.1cm of universo] {$A$};

    % Aristas
    \draw[->, loop below] (a) to (a);
    \draw[->, loop above ] (b) to (b);
    \draw[->, loop above] (c) to (c);
    \draw[->, loop below ] (d) to (d);
    \draw[->, loop below] (e) to (e);
    \draw[->, loop below] (f) to (f);
    \draw[->, loop above] (g) to (g);
    \draw[->, loop below ] (h) to (h);

    \draw[->, bend left] (a.center) to (b);
    \draw[->, bend left] (b.center) to (a);

    \draw[->, bend right] (e.center) to (h);
    \draw[->, bend right] (e.center) to (g);
    \draw[->, bend right] (h.center) to (g);
    \draw[->, bend right] (h.center) to (e);
    \draw[->, bend right] (g.center) to (h);
    \draw[->, bend right] (g.center) to (e);
  \end{tikzpicture}
}

% fin gráficos


\begin{enunciado}{\ejercicio}
  Sea $A = \set{a,b,c,d,e,f,g,h}$. Para cada uno de los siguientes gráficos describit por
  extensión la relación en $A$ que representa y determinar si es \textit{reflexiva, simétrica, antisimétrica o transitiva}.
\end{enunciado}

Un poco de la teoría de esto:
\begin{enumerate}
  \item  \textit{Reflexiva}: $(x, x) \en \relacion \paratodo x \en A$ o $x \relacion x.\, \paratodo x \en A$. Gráficamente,
        cada elemento tiene que tener un bucle.
        \begin{tikzpicture}[baseline=0, >=Latex, draw=Aquamarine]
          \node[](a) {$\bullet$} edge [in=0,out=90,loop] ();
          \node[] at (a.west) {$x$};
        \end{tikzpicture}

  \item  \textit{Simétrica}: $(x, y) \en \relacion$, entonces el par $(y, x) \en \relacion$,
        también si $\paratodo x, y \en A, x\relacion y \entonces y \relacion x$.
        Gráficamente tiene que haber un ida y vuelta en cada elemento de la relación.
        \begin{tikzpicture}[baseline=-10, >=Latex, draw=Aquamarine]
          \node[] (x) {$\bullet$};
          \node[] at (x.west) {$x$};
          \node[below right =0.2 of x] (y) {$\bullet$};
          \node[] at (y.south east) {$y$};
          \draw[->, bend left=1cm] (x.center) to (y);
          \draw[->, bend left=1cm] (y.center) to (x);
        \end{tikzpicture}

  \item  \textit{Antisimétrica}: $(x, y) \en \relacion$, con $x \distinto y$ entonces el par $(y, x) \notin \relacion$, también se puede pensar
        como $\paratodo x,y \en A, x\relacion y$ e $ y \relacion x \entonces x = y$.
        Gráficamente \textbf{no} tiene que haber ningún ida y vuelta en el gráfico. Solo en una dirección.
        \begin{tikzpicture}[baseline=-10, >=Latex, draw=Aquamarine]
          \node[] (x) {$\bullet$};
          \node[] at (x.west) {$x$};
          \node[below right =0.2 of x] (y) {$\bullet$};
          \node[] at (y.south east) {$y$};
          \draw[->] (x.center) to (y.center);
        \end{tikzpicture}

  \item  \textit{Transitiva}: Para toda terna $x, y, z \en A$ tales que $(x, y) \en \relacion$ e $(y,z) \en \relacion$,
        se tiene que $(x, z) \en \relacion$.
        Otra manera sería si $\paratodo x, y, z \en A$, $x \relacion y$ e $y \relacion z \entonces x\relacion z$.
        Gráficamente tiene que haber flecha directa entre las puntas de cualquier camino que vaya por más de dos nodos.
        \begin{tikzpicture}[baseline=0, >=Latex, draw=Aquamarine]
          \node[] (x) {$\bullet$};
          \node[] at (x.north east) {$x$};
          \node[below right = 0.6cm of x] (y) {$\bullet$};
          \node[] at (y.north east) {$y$};
          \node[below right = 0.6cm of y] (z) {$\bullet$};
          \node[] at (z.north east) {$z$};

          \draw[->, bend left] (x.center) to (y.center);
          \draw[->, bend left] (y.center) to (z.center);
          \draw[->,magenta, bend right]
          (x.center)
          to node[midway,below left] {atajo}
          (z.center);
        \end{tikzpicture}
\end{enumerate}

\begin{enumerate}[label=\roman*)]

  \item
        \begin{minipage}{0.37\textwidth}
          \diecinuevei
        \end{minipage}
        \begin{minipage}{0.5\textwidth}
          \begin{itemize}
            \item No es reflexiva, porque no hay bucles en todos los vértices, en particular $a \norelacion a$.
            \item No es simétrica, porque $d \norelacion c$.
            \item No es antisimétrica, porque $a \relacion b$ y $b \relacion a$ con $a \neq b$.
            \item No es transitiva, porque $c \relacion h$ y $h \relacion g$, pero $c \norelacion h$.
          \end{itemize}
        \end{minipage}

  \item \hacer
  \item \hacer
  \item
        \begin{minipage}{0.37\textwidth}
          \diecinueveiv
        \end{minipage}
        \begin{minipage}{0.5\textwidth}
          \begin{itemize}
            \item Reflexiva, porque hay bucles en todos los elementos de $A$.
            \item Es simétrica, porque hay ida y vuelta en todos los pares de vértices.
            \item No es antisimétrica, porque $a \relacion b$ y $b \relacion a$ con $a \neq b$.
            \item Es transitiva, porque hay \textit{atajos} en todas las relaciones de ternas.
          \end{itemize}
        \end{minipage}
\end{enumerate}
