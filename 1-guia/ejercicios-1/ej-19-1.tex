% Gráficos
\def\nodosDiecinueve{
  \node[nodo=black, label=above:$a$] (a) {};
  \node[nodo=black, label=above:$b$, above right of=a] (b) {};
  \node[nodo=black, label=right:$c$, below right of=b] (c) {};
  \node[nodo=black, label=left:$d$, below of=a] (d) {};
  \node[nodo=black, label=below:$e$, below=1.5cm of c] (e) {};
  \node[nodo=black, label=left:$f$, below right of=d] (f) {};
  \node[nodo=black, label=above left:$g$, right of=c] (g) {};
  \node[nodo=black, label=below right:$h$, below of=g] (h) {};
}
\def\universoDiecinueve{
  \draw[thick, rounded corners=5pt]
  ([xshift=5pt,yshift=-5pt]current bounding box.south east)
  rectangle
  ([xshift=-5pt,yshift=5pt]current bounding box.north west) node [above left] {$A$};
}
\def\diecinuevei{
  \begin{tikzpicture}
    [
    node distance=1.2cm,
    nodo/.style={circle, fill=black, color={##1}, minimum size = 5pt, inner sep = 2pt},
    arista/.style={-{Latex[length=3pt]}, ultra thin, bend left=25, color={##1}},
    rulo/.style 2 args = {-{Latex[length=3pt]}, out=##1, in=##1+80, looseness=12, color={##2}}
    ]
    \nodosDiecinueve

    \draw[arista=Cerulean] (a) to (b);
    \draw[arista=Cerulean] (b) to (a);
    \draw[arista=Cerulean, bend right] (c) to (d);
    \draw[arista=Cerulean, bend right] (c) to (h);
    \draw[arista=Cerulean] (e) to (c);
    \draw[arista=Cerulean, bend right] (h) to (g);

    \draw[rulo={60}{Cerulean}] (c) to (c);
    \draw[rulo={240}{Cerulean}] (f) to (f);

    % Al final para que agarre todo lo graficado
    \universoDiecinueve
  \end{tikzpicture}
}

% 19 ii
\def\diecinueveii{
  \begin{tikzpicture} [
    node distance=1.2cm,
    nodo/.style={circle, fill=black, color={##1}, minimum size = 5pt, inner sep = 2pt},
    arista/.style={-{Latex[length=3pt]}, ultra thin, bend left=25, color={##1}},
    rulo/.style 2 args = {-{Latex[length=3pt]}, out=##1, in=##1+80, looseness=12, color={##2}}
    ]
    \nodosDiecinueve

    % Aristas
    \draw[rulo={150}{Cerulean}] (a) to (a);
    \draw[rulo={-30}{Cerulean}] (b) to (b);
    \draw[rulo={60}{Cerulean}] (c) to (c);
    \draw[rulo={60}{Cerulean}] (f) to (f);

    \draw[arista=Cerulean] (a) to (b);
    \draw[arista=Cerulean] (b) to (a);
    \draw[arista=Cerulean, bend right] (c) to (g);
    \draw[arista=Cerulean, bend right] (c) to (h);
    \draw[arista=Cerulean, bend right] (c) to (e);
    \draw[arista=Cerulean, bend right] (h) to (g);

    % Al final para que agarre todo lo graficado
    \universoDiecinueve
  \end{tikzpicture}
}

% 19 iii
\def\diecinueveiii{
  \begin{tikzpicture}
    [
    node distance=1.2cm,
    nodo/.style={circle, fill=black, color={##1}, minimum size = 5pt, inner sep = 2pt},
    arista/.style={-{Latex[length=3pt]}, ultra thin, bend left=25, color={##1}},
    rulo/.style 2 args = {-{Latex[length=3pt]}, out=##1, in=##1+80, looseness=12, color={##2}}
    ]

    \nodosDiecinueve

    % Aristas
    \draw[rulo={150}{Cerulean}] (a) to (a);
    \draw[rulo={-30}{Cerulean}] (b) to (b);
    \draw[rulo={60}{Cerulean}] (c) to (c);
    \draw[rulo={240}{Cerulean}] (d) to (d);
    \draw[rulo={-50}{Cerulean}] (e) to (e);
    \draw[rulo={200}{Cerulean}] (f) to (f);
    \draw[rulo={40}{Cerulean}] (g) to (g);
    \draw[rulo={210}{Cerulean}] (h) to (h);

    \draw[arista=Cerulean] (a) to (b);
    \draw[arista=Cerulean] (b) to (a);

    \draw[arista=Cerulean] (c) to (d);
    \draw[arista=Cerulean] (c) to (e);
    \draw[arista=Cerulean] (c) to (h);
    \draw[arista=Cerulean] (d) to (c);
    \draw[arista=Cerulean] (h) to (g);

    % Al final para que agarre todo lo graficado
    \universoDiecinueve
  \end{tikzpicture}
}

% 19 iV
\def\diecinueveiv{
  \begin{tikzpicture}
    [
    node distance=1.2cm,
    nodo/.style={circle, fill=black, color={##1}, minimum size = 5pt, inner sep = 2pt},
    arista/.style={-{Latex[length=4pt]}, ultra thin, bend left=25, color={##1}},
    rulo/.style 2 args = {-{Latex[length=5pt]}, out=##1, in=##1+80, looseness=12, color={##2}}
    ]

    \nodosDiecinueve

    % Aristas
    \draw[rulo={150}{Cerulean}] (a) to (a);
    \draw[rulo={-30}{Cerulean}] (b) to (b);
    \draw[rulo={150}{Cerulean}] (c) to (c);
    \draw[rulo={30}{Cerulean}] (d) to (d);
    \draw[rulo={150}{Cerulean}] (e) to (e);
    \draw[rulo={60}{Cerulean}] (f) to (f);
    \draw[rulo={20}{Cerulean}] (g) to (g);
    \draw[rulo={-50}{Cerulean}] (h) to (h);

    \draw[arista=Cerulean] (a) to (b);
    \draw[arista=Cerulean] (b) to (a);
    \draw[arista=Cerulean] (e) to (h);
    \draw[arista=Cerulean] (e) to (g);
    \draw[arista=Cerulean, bend left = 10] (h) to (g);
    \draw[arista=Cerulean, bend left = 10] (h) to (e);
    \draw[arista=Cerulean, bend left = 10] (g) to (h);
    \draw[arista=Cerulean, bend left = 10] (g) to (e);

    % Al final para que agarre todo lo graficado
    \universoDiecinueve
  \end{tikzpicture}
}
% fin gráficos

\begin{enunciado}{\ejercicio}
  Sea $A = \set{a,b,c,d,e,f,g,h}$. Para cada uno de los siguientes gráficos describir por
  extensión la relación en $A$ que representa y determinar si es \textit{reflexiva, simétrica, antisimétrica o transitiva}.
\end{enunciado}
Podés ver el resumen de la \hyperlink{teoria-1:relaciones}{teoría acá y así entender, o corregir \faIcon[regular]{grimace}, lo que se hizo.}

\begin{enumerate}[label=\roman*)]

  \item
        \begin{minipage}{0.3\textwidth}
          \diecinuevei
        \end{minipage}
        \begin{minipage}{0.7\textwidth}
          \textit{Por extensión:}
          $$
            \set{(a,b),(b,a),(c,d),(c,h),(e,c),(f,f),(h,g)}
          $$
          \begin{itemize}
            \item \textit{Reflexiva}: Noup, porque no hay bucles en todos los vértices, en particular $a \norelacion a$.
            \item \textit{Simétrica:} Noup, porque $d \norelacion c$.
            \item \textit{Transitiva}: No, falta atajo, $c \relacion h$ y $h \relacion g$, pero $c \norelacion g$.
            \item \textit{Antisimétrica:} No, porque $a \relacion b$ y $b \relacion a$ con $a \neq b$.
          \end{itemize}
        \end{minipage}

        \bigskip

  \item
        \begin{minipage}{0.25\textwidth}
          \diecinueveii
        \end{minipage}
        \begin{minipage}{0.7\textwidth}
          \textit{Por extensión:}
          $$
            \set{(a,a),(a,b),(b,a),(b,b),(c,c),(c,e),(c,g),
              (c,h),(f,f),(h,g)}
          $$
          \begin{itemize}
            \item \textit{Reflexiva}: No, faltan bucles en algunos vértices.
            \item \textit{Simétrica}: No, $c \relacion e$, pero $e \norelacion c$.
            \item \textit{Transitiva}: Sí. está el \textit{atajo} en la única terna:
                  $$
                    c \relacion h,\, h \relacion g \entonces c \relacion g.
                  $$
            \item \textit{Antisimétrica:} No, porque $a \relacion b$ y $b \relacion a$ con $a \neq b$.
          \end{itemize}
        \end{minipage}

        \bigskip

  \item
        \begin{minipage}{0.3\textwidth}
          \diecinueveiii
        \end{minipage}
        \begin{minipage}{0.7\textwidth}
          \textit{Por extensión:}
          $$
            \scriptstyle
            \set{(a,a),(a,b),(b,a),(b,b),
              (c,c),(c,d),(c,e),(c,h)
              (d,d),(d,c), (e,e), (f,f), (g,g), (h,h),(h,g) }.
          $$
          \begin{itemize}
            \item \textit{Reflexiva}: Sí, están todos los bucles.
            \item \textit{Simétrica}: No, $c \relacion h$, pero $h \norelacion c$.
            \item \textit{Transitiva}: No, falta atajo, $c \relacion h$ y $h \relacion g$, pero $c \norelacion g$.
            \item \textit{Antisimétrica:} No, porque $a \relacion b$ y $b \relacion a$ con $a \neq b$.
          \end{itemize}
        \end{minipage}

        \bigskip

  \item
        \begin{minipage}{0.3\textwidth}
          \diecinueveiv
        \end{minipage}
        \begin{minipage}{0.7\textwidth}
          \textit{Por extensión:}
          {\tiny
            $$
              \set{(a,a),(a,b),(b,a),(b,b),(c,c),(d,d),
                (e,e),(e,g),(e,h),
                (f,f),
                (g,g),(g,e),(g,h),
                (h,h),(h,e),(h,g)
              }.
            $$
          }
          \begin{itemize}
            \item Reflexiva, porque hay bucles en todos los elementos de $A$.
            \item Es simétrica, porque hay ida y vuelta en todos los pares de vértices.
            \item No es antisimétrica, porque $a \relacion b$ y $b \relacion a$ con $a \neq b$.
            \item Es transitiva, porque hay \textit{atajos} en todas las relaciones de ternas.
          \end{itemize}
        \end{minipage}
\end{enumerate}

\begin{aportes}
  \item \aporte{\dirRepo}{naD GarRaz \github}
\end{aportes}
