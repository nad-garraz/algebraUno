%Graficos
\def\veinte{
  \begin{tikzpicture}[
    node distance=1.2cm,
    nodo/.style={circle, draw, color={##1}, inner sep=1pt, outer sep=1pt},
    arista/.style={-{Latex[length=2pt]}, ultra thin, bend left=15, color={##1}},
    rulo/.style 2 args = {-{Latex[length=2pt]}, out=##1, in=##1+45, looseness=4, color={##2}}
    ]

    \node[nodo=Cerulean] (1) {$1$};
    \node[nodo=Cerulean, right of=1] (3) {$3$};

    \node[nodo=Cerulean, below of=1] (4) {$4$};
    \node[nodo=Cerulean, right of=4] (6) {$6$};

    \node[nodo=Cerulean, above left of=1] (2) {$2$};
    \node[nodo=Cerulean, above right of=1] (5) {$5$};

    \draw[arista=Cerulean] (1) to (3);
    \draw[arista=Cerulean] (3) to (1);
    \draw[arista=Cerulean] (4) to (6);
    \draw[arista=Cerulean] (6) to (4);

    \draw[rulo={150}{Cerulean}] (1) to (1);
    \draw[rulo={-30}{Cerulean}] (3) to (3);
    \draw[rulo={150}{Cerulean}] (4) to (4);
    \draw[rulo={-30}{Cerulean}] (6) to (6);

    % Al final para que agarre todo lo graficado
    \draw[thick, rounded corners=5pt]
    ([xshift=-5pt,yshift=-5pt]current bounding box.south west)
    rectangle
    ([xshift=5pt,yshift=5pt]current bounding box.north east) node [above right] {$A$};
  \end{tikzpicture}
}
% fin gráficos

\begin{enunciado}{\ejercicio}
  Sea $A = \set{1, 2, 3, 4, 5, 6}$. Graficar la relación,
  $$
    \relacion = \set{(1,1), (1,3), (3,1), (3,3), (6,4), (4,6), (4,4), (6,6)}
  $$
  como está hecho en el ejercicio anterior y determinar si es reflexiva, simétrica, antisimétrica o transitiva.
\end{enunciado}

\begin{minipage}{0.25\textwidth}
  \veinte
\end{minipage}
\begin{minipage}{0.7\textwidth}
  \begin{itemize}
    \item No es reflexiva porque no hay bucles ni en 2 ni en 5.
    \item Es simétrica, porque hay ida y vuelta en todos los pares de vértices.
    \item No es antisimétrica, porque $1 \relacion 3$ y $3 \relacion 1$ con $1 \neq 3$.
    \item Es transitiva. \\
          \red{Chequear. Caso particula donde no hay ternas de $x,y,z$ distintos}.
          \blue{Sí, el que $2$ esté ahí solo ni cumple la hipótesis de transitividad.}
  \end{itemize}
\end{minipage}
