\ejercicio Sea $A = \set{1, 2, 3, 4, 5, 6}$. Graficar la relación, $\relacion= {(1,1), (1,3), (3,1), (3,3), (6,4), (4,6), (4,4), (6,6)}$
\begin{minipage}{0.25\textwidth}
	\veinte
\end{minipage}
\begin{minipage}{0.7\textwidth}
	\begin{itemize}
		\item No es reflexiva porque no hay bucles ni en 2 ni en 5.
		\item Es simétrica, porque hay ida y vuelta en todos los pares de vértices.
		\item No es antisimétrica, porque $1 \relacion 3$ y $3 \relacion 1$ con $1 \neq 3$.
		\item Es transitiva. \\
		      \red{Chequear. Caso particula donde no hay ternas de $x,y,z$ distintos}.
		      \blue{Sí, el que $2$ esté ahí solo ni cumple la hipótesis de transitividad.}
	\end{itemize}
\end{minipage}
