\begin{enunciado}{\ejercicio}
  Sea $A = \set{a,b,c,d,e,f}$ y sea $\relacion$ la relación en $A$ representada por el gráfico.
  $$
    \begin{tikzpicture}[
        node distance=1.5cm,
        nodos/.style={ circle, fill=black, outer sep = 4pt, inner sep=0pt, minimum size=5pt},
        arista/.style={-Latex, bend left = 30, very thick},
        rulo/.style={-Latex, very thick},
        universo/.style={rectangle, draw,  thick, scale=1.4 }
      ]

      \node[nodos, label=left:{a}] (a) at (0,0) {};
      \node[nodos, right of=a, label=above right:{b}, yshift=10pt] (b) {};
      \node[nodos, right of=c, label=above right:{c}] (c) {};
      \node[nodos, below of=a, label=below left:{d}, xshift = -10pt] (d) {};
      \node[nodos, right=1.7cm of d, label=right:{e}, yshift=-10pt] (e) {};
      \node[nodos, below right of=c, label= below:{f}] (f) {};

      \draw[rulo] (a) to [out=120, in=220, looseness=12]  (a);
      \draw[rulo] (e) to [out=60, in=-30, looseness=12]  (e);

      \draw[arista] (b) to[bend right] (c);
      \draw[arista] (d) to (e);
      \draw[arista] (e) to (d);

      %Universo
      \draw[thick, rounded corners=5pt]
      ([xshift=5pt,yshift=-5pt]current bounding box.south east)
      rectangle
      ([xshift=-5pt,yshift=5pt]current bounding box.north west) node [above left] {$A$};
    \end{tikzpicture}
  $$
  Hallar la minima cantidad de pares que se deben agregar a $\relacion$ de manera que la nueva relación
  obtenida sea
  \begin{multicols}{3}
    \begin{enumerate}[label=\roman*)]
      \item reflexiva
      \item simétrica
      \item transitiva
      \item reflexiva y simétrica
      \item simétrica y transitiva
      \item de equivalencia
    \end{enumerate}
  \end{multicols}
\end{enunciado}

\begin{enumerate}[label=\roman*)]
  \item Para que la relación sea reflexiva se debe cumplir que $\paratodo a \en A, a \relacion a$, visto como grafo,
        cada ítem tiene que tener un bucle (es decir una flecha hacia sí mismo). Deberíamos agregar 4 pares : $\{(b,b),(c,c),(d,d),(f,f)\}$

  \item Para que sea simétrica necesitamos que $a \relacion b \entonces b \relacion a, \quad a,b \en A$. En este caso el único par que está
        conectado pero no tiene la vuelta es el par $(b,c)$, así que agregando el par $(c,b)$ ya hacemos que sea simétrica.

  \item Para que sea transitiva queremos que $a \relacion b \, \land \, b \relacion c \entonces a \relacion c, \quad a,b,c \en A$. Vemos que
        $d \relacion e \land e \relacion d$ pero $d \norelacion d$, por ende lo que falta sería agregar el par $(d,d)$, en todos los
        demas se cumple la transitividad.

  \item Para hacerla reflexiva teniamos que agregar $\{(b,b),(c,c),(d,d),(f,f)\}$ y para que sea simétrica $(c,b)$

  \item Para hacerla simétrica agregabamos $(c,b)$ y para transitiva $(d,d)$, pero también hay que agregar bucles en $b$ y $c$ ya que al haber
        hecho simétrico ese par queremos que cumpla la transitividad también. Así que agregamos también $\{(b,b),(c,c)\}$

  \item Para que sea de equivalencia tiene que ser, reflexiva, simétrica y transitiva, combinando todo lo que hicimos arriba queda
        que tenemos que agregar $\{(b,b),(c,c),(d,d),(f,f),(c,b)\}$
\end{enumerate}

\begin{aportes}
  \item \aporte{https://github.com/sigfripro}{sigfripro \github}
\end{aportes}
