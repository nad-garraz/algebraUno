\begin{enunciado}{\ejercicio}
  Sea $A$ un conjunto. Describir todas las relaciones en $A$ que son a la vez
  \begin{enumerate}[label=\roman*)]
    \begin{multicols}{2}
      \item simétricas y antisimétricas
      \item de equivalencia y de orden
    \end{multicols}
  \end{enumerate}
\end{enunciado}

\begin{enumerate}[label=\roman*)]
  \item
        Una relación en $A$ es simétrica si $a \relacion b \entonces b \relacion a \quad a,b \in A$.

        Una relación en $A$ es antisimétrica si $a \relacion b \, \land \, b \relacion a \entonces a = b$

        Juntando estas dos restricciones surge que los unicos elementos que sirven para construir tal relación son los de la forma
        $(a,a)$, ya que cumplen ambas condiciones, cualquier otro tipo de elemento no cumpliria ambas condiciones simultáneamente.
        Es decir $\relacion \subseteq \set{(a,a) | a \in A}$. O visto como un cuadro, todos los elementos de la diagonal, la cantidad
        de relaciones de este tipo que se pueden construir son $2^N$, siendo $N$ el cardinal del conjunto $A$. Notése que hay $N$ pares
        $(a,a) | a \in A$, luego para constuir la relación podemos elegir si usar o si no ese par, por cada par, el argumento es idéntico a porque
        el cardinal de las partes de un conjunto es $2^n$.

  \item de equivalencia y de orden
        Para que sea de equivalencia y de orden tienen que cumplir las cuatro propiedades que vimos: \textit{Reflexividad, Simetria, Antisimetria, Transitividad}.

        Por el ítem anterior ya sabemos que solo nos sirven los elementos de la diagonal,
        pero al tener que ser reflexiva, es decir que $\paratodo a \en A, a \relacion a$,
        si o si tenemos que usar todos los elementos a la vez, ya que si dejamos un elemento sin usar, no sería reflexiva la relacion.
        Por el lado de la transitividad, la relación va a ser transitiva siempre pues al no haber dos elementos distintos
        relacionados para aplicar la hipótesis, la hipótesis no se aplica nunca, entonces es transitiva.

        Entonces tenemos $\relacion = \set{(a,a) | a \en A}$, por lo tanto solo se puede construir una relación en $A$ que sea
        de equivalencia y de orden

\end{enumerate}

\begin{aportes}
  \item \aporte{https://github.com/sigfripro}{sigfripro \github}
\end{aportes}
