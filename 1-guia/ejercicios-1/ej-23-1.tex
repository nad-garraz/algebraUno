\begin{enunciado}{\ejercicio}
  Sea $A$ un conjunto. Describir todas las relaciones en $A$ que son a la vez
  \begin{enumerate}[label=\roman*)]
    \begin{multicols}{2}
      \item simétricas y antisimétricas
      \item de equivalencia y de orden
    \end{multicols}
  \end{enumerate}
\end{enunciado}

\begin{enumerate}[label=\roman*)]
  \item 
  Una relacion en $A$ es simetrica si $a \relacion b \entonces b \relacion a \quad a,b \in A$.
  
  Una relacion en $A$ es antisimetrica si $a \relacion b \, \land \, b \relacion a \entonces a = b$
  
  Juntando estas dos restricciones surge que los unicos elementos que sirven para construir tal relacion son los de la forma
  $(a,a)$, ya que cumplen ambas condiciones, cualquier otro tipo de elemento no cumpliria ambas condiciones simultaneamente. 
  Es decir $\relacion \subseteq \{(a,a) | a \in A\}$. O visto como un cuadro, todos los elementos de la diagonal, la cantidad 
  de relaciones de este tipo que se pueden construir son $2^N$, siendo $N$ el cardinal del conjunto $A$. Notése que hay $N$ pares
  $(a,a) | a \in A$, luego para constuir la relacion podemos elegir si usar o si no ese par, por cada par, el argumento es identico a porque 
  el cardinal de las partes de un conjunto es $2^n$.

  \item de equivalencia y de orden
  Para que sea de equivalencia y de orden tienen que cumplir las cuatro propiedades que vimos: Reflexividad, Simetria, Antisimetria, Transitividad. 
  
  Por el item anterior ya sabemos que solo nos sirven los elementos de la diagonal, pero al tener que ser reflexiva, es decir que $\forall a \in A, a \relacion a$, 
  si o si tenemos que usar todos los elementos a la vez, ya que si dejamos un elemento sin usar, no seria reflexiva la relacion. 
  Por el lado de la transitividad, la relacion va a ser transitiva siempre pues al no haber dos elementos distintos relacionados para aplicar la hipotesis, la hipotesis no se aplica nunca, entonces es transitiva.

  Entonces tenemos $\relacion = \{(a,a) | a \in A\}$, por lo tanto solo se puede construir una relacion en $A$ que sea 
  de equivalencia y de orden

\end{enumerate}

\begin{aportes}
 \item \aporte{https://github.com/sigfripro}{sigfripro \github}
\end{aportes}

