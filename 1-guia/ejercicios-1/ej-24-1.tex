\def\veinticuatro{
  \begin{tikzpicture}[
      baseline=0,
      aristas/.style={-Latex, thin, bend left=14, color={##1}},
      nodo/.style={circle, draw, color={##1}, minimum size=4pt, inner sep=2pt, outer sep=3pt},
      rulo/.style={-Latex, color={##1}},
      node distance = 1.5cm,
      every nodes/.style={font=\small},
    ]
    \node[nodo=magenta] (a) {$a$};
    \node[nodo=magenta, below left of=a] (b) {$b$};
    \node[nodo=magenta, below right of=a] (f) {$f$};

    \node[nodo=OliveGreen, right of=a] (d) {$d$};

    \node[nodo=Cerulean, above left of=a] (c) {$c$};
    \node[nodo=Cerulean, right of=c] (e) {$e$};

    \draw[rulo=magenta, loop above] (a) to (a);
    \draw[rulo=magenta, loop left ] (b) to (b);
    \draw[rulo=Cerulean, loop left] (c) to (c);
    \draw[rulo=Cerulean, loop right] (e) to (e);
    \draw[rulo=magenta, loop right] (f) to (f);
    \draw[rulo=OliveGreen, loop above ] (d) to (d);
    \draw[aristas=magenta] (a) to (b);
    \draw[aristas=magenta] (b) to (a);
    \draw[aristas=magenta] (a) to (f);
    \draw[aristas=magenta] (f) to (a);
    \draw[aristas=magenta] (b) to (f);
    \draw[aristas=magenta] (f) to (b);
    \draw[aristas=Cerulean] (c) to (e);
    \draw[aristas=Cerulean] (e) to (c);

    % Al final para que agarre todo lo graficado
    \draw[thick, rounded corners=5pt]
    ([xshift=-5pt,yshift=-5pt]current bounding box.south west)
    rectangle
    ([xshift=5pt,yshift=5pt]current bounding box.north east) node [above right] {$A$};
  \end{tikzpicture}
}

\begin{enunciado}{\ejercicio}

  Sea $A = \set{a, b, c, d, e, f}$. Dada la relación de equivalencia en $A$:
  \begin{align*}
    \relacion = \set{(a, a), (b, b), (c, c), (d, d),
      (e, e), (f, f), (a, b), (b, a), (a, f),
      (f, a), (b, f), (f, b), (c, e), (e, c)}
  \end{align*}
  Hallar la clase $\clase{a}$ de $a$,
  la clase $\clase{b}$ de $b$,
  la clase $\clase{c}$ de $c$,
  la clase $\clase{d}$ de $d$,
  y la partición asociada a $\relacion$
\end{enunciado}

Gráficamente la relación $\relacion$:
$$
  \veinticuatro
$$
Las clases son los conjuntos de elementos que están relacionados entre sí:
$$
  \cajaResultado{
    \llave{rclclcl}{
      \magenta{\clase{a}} & = & \set{a, b, f} & = & \clase{b} & = & \clase{f} \\
      \blue{\clase{c}}    & = & \set{c, e}                    & = & \clase{e} &   &           \\
      \green{\clase{d}} & = & \set{d}        &   &         &   &
    }
  }
$$
La partición asociada a $\relacion$:
$$
  \cajaResultado{
    \relacion:\,
    \set{
      \set{d}, \set{c,e}, \set{a, b, f}
    } =
    \set{
      \clase{d}, \clase{c}, \clase{a}
    }.
  }
$$

\begin{aportes}
  \item \aporte{\dirRepo}{naD GarRaz \github}
\end{aportes}
