%Graficos
\def\veinticuatro{
  \begin{tikzpicture}[scale=0.7, baseline=0, >=Latex, draw=Aquamarine]

    \node[] (a) {$\bullet$};
    \node[] at (a.north west) {$a$};

    \node[below left=1cm of a] (b) {$\bullet$};
    \node[] at (b.south) {$b$};

    \node[below right=1cm of a] (f) {$\bullet$};
    \node[] at (f.south) {$f$};

    \node[above right=1cm of a] (d) {$\bullet$};
    \node[] at (d.west) {$d$};

    \node[right=1cm of a] (c) {$\bullet$};
    \node[] at (c.south) {$c$};

    \node[right=1cm of c] (e) {$\bullet$};
    \node[] at (e.south) {$e$};

    % Universo
    \node[shape=ellipse, draw, black, fit={ (a) (b) (d) (f) (e)}] (universo) {};
    \node[above left = 0.1cm of universo] {$A$};

    % Aristas
    \draw[magenta, ->, loop above] (a) to (a);
    \draw[magenta, ->, loop left ] (b) to (b);
    \draw[->, loop left] (c) to (c);
    \draw[->, loop right ] (e) to (e);
    \draw[magenta, ->, loop right ] (f) to (f);

    \draw[OliveGreen, ->, loop above ] (d) to (d);

    \draw[magenta, ->, bend left] (a.center) to (b);
    \draw[magenta, ->, bend left] (b.center) to (a);
    \draw[magenta, ->, bend left] (a.center) to (f);
    \draw[magenta, ->, bend left] (f.center) to (a);
    \draw[magenta, ->, bend left] (b.center) to (f);
    \draw[magenta, ->, bend left] (f.center) to (b);

    \draw[->, bend right] (c.center) to (e);
    \draw[->, bend right] (e.center) to (c);
  \end{tikzpicture}
}

% fin gráficos

\begin{enunciado}{\ejercicio}

  Sea $A = \set{a, b, c, d, e, f}$. Dada la relación de equivalencia en $A$:
  \begin{align*}
    \relacion = \set{(a, a), (b, b), (c, c), (d, d),
      (e, e), (f, f), (a, b), (b, a), (a, f),
      (f, a), (b, f), (f, b), (c, e), (e, c)}
  \end{align*}
  Hallar la clase $\clase{a}$ de $a$,
  la clase $\clase{b}$ de $b$,
  la clase $\clase{c}$ de $c$,
  la clase $\clase{d}$ de $d$,
  y la partición asociada a $\relacion$
\end{enunciado}

\veinticuatro
$\to
  \llave{rclclcl}{
    \magenta{\clase{a}} & = & \set{a, b, f} & = & \clase{b} & = & \clase{f} \\
    \blue{\clase{c}}    & = & \set{c, e}                    & = & \clase{e} &   &           \\
    {\textcolor{OliveGreen}{\clase{d}}} & = & \set{d}        &   &         &   & \\
  }$\par
La partición asociada a
$\relacion:\,
  \set{\set{d},
    \set{c,e},
    \set{a, b, f}} = \set{\clase{d}, \clase{b}, \clase{a}}$.
