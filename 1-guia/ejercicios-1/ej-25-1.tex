\usetikzlibrary{bending}
\begin{enunciado}{\ejercicio}
  Sea $A = \set{1,2,3,4,5,6,7,8,9,10}$.
  Hallar y graficar la relación de equivalencia en $A$ asociada a la partición
  $\set{ \set{1,3}, \set{2,6,7}, \set{4,8,9,10}, \set{5} }$.
  ¿Cuántas clases de equivalencia distintas tiene? Hallar un representante para cada clase.
\end{enunciado}

\begin{center}
\begin{tikzpicture}[carrow/.style={to path={%
    ($(\tikztotarget)+(#1:\pgfkeysvalueof{/tikz/cradius})+(#1+170:\pgfkeysvalueof{/tikz/cradius})$)
    arc[start angle=#1+170,end angle=#1-170,radius=\pgfkeysvalueof{/tikz/cradius}]},
    ->},dot/.style={circle,inner sep=1.2pt,fill,outer sep=0.5ex},
    cradius/.initial=0.4,
    >={Latex[bend]},bend angle=12]
 \path node[dot,label=225:{$1$}](a){}
  ++ (45:1.5) node[dot,label=45:{$3$}](b){}
  ++ (-30:1.5) node[dot,label=90:{$5$}](c){}
  ($(a)+(0,-1.8)$) node[dot,label=135:{$d$}](d){}
  ($(d)+(1.25,0)$) node[dot,label=45:{$f$}](f){}
  ++ (1.5,-0.3) node[dot,label=-120:{$2$}](e){}
  ++ (0:1.8) node[dot,label=-60:{$6$}](h){}
  ++ (120:1.8) node[dot,label=90:{$7$}](g){}
  ($(d)+(0,-1.5)$) node[dot,label=225:{$8$}](i){}
  ($(f)+(0,-1.5)$) node[dot,label=315:{$9$}](j){};
 \path[shorten >=1ex, shorten <=0.5ex,semithick] 
 (a) edge[->,bend left] (b) edge[carrow=45] (b)
 (b) edge[->,bend left] (a) edge[carrow=-135,<-] (a)
 edge[carrow=135,<-] (d)
 edge[carrow=45,<-] (f)
 edge[carrow=-120] (e)
 edge[carrow=-60,<-] (h)
 edge[carrow=90,<-] (g)
 edge[carrow=90,<-] (c)
 edge[carrow=-135,<-] (i)
 edge[carrow=315,<-] (j)
 (e) edge[->,bend left] (g)
 (g) edge[->,bend left] (e)
 (h) edge[->,bend left] (g)
 (g) edge[->,bend left] (h)
 (e) edge[->,bend left] (h)
 (h) edge[->,bend left] (e)
 (i) edge[->,bend left] (j)
 (j) edge[->,bend left] (i)
 (d) edge[->,bend left] (f)
 (f) edge[->,bend left] (d)
 (i) edge[->,bend left] (d)
 (d) edge[->,bend left] (i)
 (j) edge[->,bend left] (d)
 (d) edge[->,bend left] (j)
 (i) edge[->,bend left] (f)
 (f) edge[->,bend left] (i)
 (j) edge[->,bend left] (f)
 (f) edge[->,bend left] (j);    
\end{tikzpicture}
\end{center}

Ese seria el grafo de la relacion, es de equivalencia pues cumple que es reflexiva, simetrica y transitiva. 

Hay $4$ clases de equivalencia ya que son $4$ las particiones disjuntas del conjunto, y como representante podemos elegir cualquiera de cada clase,
como convencion digamos que elegimos el primero que aparece, luego los representantes son : $\cajaResultado{1, 2, 4, 5.}$

\begin{aportes}
 \item \aporte{https://github.com/sigfripro}{sigfripro \github}
\end{aportes}