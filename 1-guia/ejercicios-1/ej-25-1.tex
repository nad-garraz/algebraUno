\begin{enunciado}{\ejercicio}
  Sea $A = \set{1,2,3,4,5,6,7,8,9,10}$.
  Hallar y graficar la relación de equivalencia en $A$ asociada a la partición
  $\set{ \set{1,3}, \set{2,6,7}, \set{4,8,9,10}, \set{5} }$.
  ¿Cuántas clases de equivalencia distintas tiene? Hallar un representante para cada clase.
\end{enunciado}
$$
  \begin{tikzpicture}[
    node distance=1.4cm,
    nodo/.style={circle, draw, color={#1}, minimum size=4pt, inner sep=2pt, outer sep=3pt},
    arista/.style={-{Latex[length=3pt]}, thin, bend left=10, color={#1}},
    rulo/.style 2 args = {-{Latex[length=3pt]}, out=#1, in=#1+45, looseness=4, color={#2}}
    ]
    \node[nodo=Cerulean] (1) {$1$};
    \node[nodo=Cerulean, above right of=1] (3) {$3$};
    \node[nodo=OliveGreen, below right of=3] (2) {$2$};
    \node[nodo=OliveGreen, above right of=2] (6) {$6$};
    \node[nodo=OliveGreen, below right of=6] (7) {$7$};
    \node[nodo=Orange, below of=7] (5) {$5$};
    \node[nodo=Purple, below right of=1] (4) {$4$};
    \node[nodo=Purple, right of=4] (8) {$8$};
    \node[nodo=Purple, below of=4] (9) {$9$};
    \node[nodo=Purple, right of=9] (10) {$10$};

    \foreach \c/\nodes in {Cerulean/{1,3}, OliveGreen/{2,6,7}, Purple/{4,8,9,10}} {
    \foreach \i in \nodes {
      \foreach \j in \nodes {
        \ifnum\i<\j
          \draw[arista=\c] (\i) to (\j);
          \draw[arista=\c] (\j) to (\i);
        \fi
      }
    }
    }

    \draw[rulo={200}{Cerulean}] (1) to (1);
    \draw[rulo={20}{Cerulean}] (3) to (3);

    \draw[rulo={200}{OliveGreen}] (2) to (2);
    \draw[rulo={65}{OliveGreen}] (6) to (6);
    \draw[rulo={290}{OliveGreen}] (7) to (7);

    \draw[rulo={20}{Orange}] (5) to (5);

    \draw[rulo={110}{Purple}] (4) to (4);
    \draw[rulo={20}{Purple}] (8) to (8);
    \draw[rulo={200}{Purple}] (9) to (9);
    \draw[rulo={290}{Purple}] (10) to (10);

    %Universo
    \draw[thick, rounded corners=5pt]
    ([xshift=-5pt,yshift=-5pt]current bounding box.south west)
    rectangle
    ([xshift=5pt,yshift=5pt]current bounding box.north east) node [above right] {$a$};
  \end{tikzpicture}
$$
Ese sería el grafo de la relación, es de equivalencia pues cumple que es reflexiva, simétrica y transitiva.

Hay $4$ clases de equivalencia ya que son $4$ las particiones disjuntas del conjunto, y como representante podemos elegir cualquiera de cada clase,
como convención digamos que elegimos el primero que aparece, luego los representantes son :
$$
  \cajaResultado{1, 2, 4, 5}.
$$

\begin{aportes}
  \item \aporte{https://github.com/sigfripro}{sigfripro \github}
\end{aportes}
