\begin{enunciado}{\ejercicio}
  Sean $P = \partes(\set{1,2,3,4,5,6,7,8,9,10})$
  el conjunto de partes de $\set{1,\dots, 10}$ y $\relacion$ la relación en $P$ definida por:
  $$
    A \relacion B \sisolosi (A \triangle B) \inter \set{1,2,3} = \vacio
  $$
  \begin{enumerate}[label=\roman*)]

    \item  Probar que $\relacion$ es una relación de equivalencia y decidir si es antisimétrica
          (\textit{\underline{Sugerencia:}} usar adecuadamente el ejercicio \textbf{14\ref{ej-14-1:itemiii}})).
    \item Hallar la clase de equivalencia de $A = \set{1,2,3}$.
  \end{enumerate}
\end{enunciado}

\begin{enumerate}[label=\roman*)]

  \item
        Para probar que es una relación de equivalencias hay que probar que sea \textit{reflexiva, simétrica y transitiva}.
        La sugerencia que nos dan es:
        $$
          A \triangle B \subseteq (A \triangle C) \union (B \triangle C)
        $$
        \textit{Reflexiva: ¿$A \relacion A$? }
        $$
          A \relacion A \sisolosi (A \triangle A) = \vacio \inter \set{1,2,3} = \vacio \Tilde
        $$
        Por lo tanto la realción $\relacion$ es reflexiva.

        \textit{Simétrica: ¿$A \relacion B \entonces B \relacion A$?}
        $$
          A \relacion B \sisolosi \ub{(A \triangle B)}{ = B \triangle A} \inter \set{1,2,3} = \vacio
        $$
        Como la diferencia simétrica es conmutativa,
        $A \triangle B = B \triangle A$ se tiene que la relación $\relacion$ es simétrica también.

        \textit{Transitiva: ¿$A \relacion B \ytext B \relacion C \entonces A \relacion C$?}
        $$
          \llave{l}{
            A \relacion B \sisolosi (A \triangle B) \inter \set{1,2,3} = \vacio \Tilde \\
            B \relacion C \sisolosi (B \triangle C) \inter \set{1,2,3} = \vacio \Tilde
          }
        $$

        Acá uso la \magenta{sugerencia}.\par

        Si el conjunto $\set{1,2,3}$ no está ni en $A \triangle B$ ni en $B \triangle C$, en particular
        tampoco está en $(A \triangle B) \union (B \triangle C)$.\par
        Sabemos que $\magenta{(A \triangle C) \subseteq (A \triangle B) \union (B \triangle C)}$, es decir que $(A \triangle C)$
        es un subconjunto de un conjunto que \underline{no} tiene al conjunto $\set{1,2,3}$. Se concluye que
        $$
          (A \triangle C) \inter \set{1,2,3} = \vacio.
        $$
        La relación $\relacion$ es transitiva.\par

        Como la relación es \textit{reflexiva, simétrica  y transitiva} es de equivalencia \Tilde.\par\bigskip

        \textit{Antisimétrica: $\paratodo A, B \en P si A \relacion B y B \relacion A \entonces A = B$}\par
        Se podría encontrar un contraejemplo: Ya dijimos que $A \triangle B = B \triangle A$. No debería ser muy complicado encontrar un $A$ y un $B$
        distintos que cumplan
        $$
          A \relacion B \ytext B\relacion A
        $$

  \item
        La clase de equivalencia de $A = \set{1,2,3}$ va a estar formada por $A$ y por todos los conjuntos $X \en P$
        que cumplan
        $$
          (\set{1,2,3} \triangle X) \inter \set{1,2,3} = \vacio
        $$
        Resulta que cerca de la sugerencia dada del \refEjercicio{ej:14}\ref{ej-14-1:itemiii}, está el ejercicio
        \refEjercicio{ej:14}\ref{ej-14-1:itemi}, donde se muestra que la intersección ($\inter$) es distributiva
        con la diferencia simétrica ($\triangle$). Con eso puedo reescribir la condición de más arriba como:
        $$
          (\set{1,2,3} \triangle X) \inter \set{1,2,3}
          \igual{\red{!}}
          \set{1,2,3} \triangle (X \inter \set{1,2,3}).
        $$
        Si te perdiste en el \red{!}, \textit{escribilo y miralo fuerte}. La condición para que $X \relacion \set{1,2,3}$ queda:
        $$
          \set{1,2,3} \triangle (X \inter \set{1,2,3}) = \vacio,
        $$
        que, en mi opinión, está más fácil de leer. Para que una diferencia simétrica entre 2 conjuntos
        resulte en vacío, necesito que los conjuntos sean iguales. Por lo tanto quiero los conjuntos $X$ tales que:
        $$
          X \inter \set{1,2,3} = \set{1,2,3}.
        $$
        La clase $\clase{A}$:
        $$
          \clase{A} = \set{X \en P \big/ \set{1,2,3} \subseteq X }
          \text{ o también }
          \clase{A} = \set{ \set{1,2,3} \union X \text{ con } X \en \partes\set{4,5,6,7,8,9,10} }\Tilde
        $$
\end{enumerate}

% Contribuciones
\begin{aportes}
  %% iconos : \github, \instagram, \tiktok, \linkedin
  %\aporte{url}{nombre icono}

  \item \aporte{https://github.com/nad-garraz}{Nad Garraz \github}
  \item \aporte{https://github.com/gusvianadev}{Gus Viana \github}
\end{aportes}
