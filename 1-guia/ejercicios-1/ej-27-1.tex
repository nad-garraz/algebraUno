%Graficos
\def\veintisiete{
  \begin{tikzpicture}[
    node distance=1.2cm,
    nodo/.style={circle, draw, color={##1}, inner sep=1pt, outer sep=1pt},
    arista/.style={-{Latex[length=2pt]}, ultra thin, bend left=15, color={##1}},
    rulo/.style 2 args = {-{Latex[length=2pt]}, out=##1, in=##1+45, looseness=4, color={##2}}
    ]

    \node[nodo=Aquamarine] (1) {$1$};
    \node[nodo=Aquamarine, right of=1] (92) {$92$};
    \draw[arista=Cerulean] (1) to (92);
    \draw[arista=Cerulean] (92) to (1);
    \draw[rulo={150}{Cerulean}] (1) to (1);
    \draw[rulo={-30}{Cerulean}] (92) to (92);

    \node[nodo=Aquamarine, below of=1] (2) {$2$};
    \node[nodo=Aquamarine, right of=2] (91) {$91$};
    \draw[arista=Cerulean] (2) to (91);
    \draw[arista=Cerulean] (91) to (2);
    \draw[rulo={150}{Cerulean}] (2) to (2);
    \draw[rulo={-30}{Cerulean}] (91) to (91);

    \node[nodo=Aquamarine, below of=2] (3) {$3$};
    \node[nodo=Aquamarine, right of=3] (90) {$90$};
    \draw[arista=Cerulean] (3) to (90);
    \draw[arista=Cerulean] (90) to (3);
    \draw[rulo={150}{Cerulean}] (3) to (3);
    \draw[rulo={-30}{Cerulean}] (90) to (90);

    \node[below right=0.2cm of 3] (puntos) {$\vdots$};

    \node[nodo=Aquamarine, below left=0.1cm of puntos] (46) {$46$};
    \node[nodo=Aquamarine, right of=46] (47) {$47$};
    \draw[arista=Cerulean] (46) to (47);
    \draw[arista=Cerulean] (47) to (46);
    \draw[rulo={150}{Cerulean}] (46) to (46);
    \draw[rulo={-30}{Cerulean}] (47) to (47);

    %Universo
    \draw[thick, rounded corners=5pt]
    ([xshift=-5pt,yshift=-5pt]current bounding box.south west)
    rectangle
    ([xshift=5pt,yshift=5pt]current bounding box.north east) node [above right] {$a$};

  \end{tikzpicture}
}

\begin{enunciado}{\ejercicio}

  Sean $A = \set{n \en \naturales \talque n \leq 92}$ y
  $\relacion$ la relación en $A$ definida por
  $x \relacion y \sisolosi x^2 - y^2 = 93x - 93y$
  \begin{enumerate}[label=\alph*)]
    \item Probar que $\relacion$ es una relación de equivalencia. ¿Es antisimétrica?
    \item Hallar la clase de equivalencia de cada $x \en A$.
          Deducir cuántas clases de equivalencia \textbf{distintas} determina la relación $\relacion$.
  \end{enumerate}

\end{enunciado}

\begin{enumerate}[label=\alph*)]
  \item Primero acomodo la condición de la relación:
        $$x^2 - y^2 = 93x - 93y
          \Sii{\red{!!!}}[\magic]
          \llave{c}{
            x      \igual{$\llamada{1}$}  y \\
            \text{ o bien }                            \\
            x + y  \igual{$\llamada{2}$}  93
          }
        $$
        Hacer este ejercicio sin avivarse de lo que pasa en \red{!!!} es horrible.\par
        Para ser relación de equivalencia es necesario que sea \textit{reflexiva, simétrica} y \textit{transitiva}:\par
        \textit{Reflexiva: }
        $$
          x \relacion x \sisolosi x \igual{$\llamada{1}$} x  \Tilde
        $$
        \textit{Simétrica: }
        $$
          \llave{l}{
            x \relacion y \sisolosi x + y \igual{$\llamada{2}$} 93 \\
            y \relacion x \sisolosi y + x \igual{$\llamada{2}$} 93
          }\Tilde
        $$
        \textit{Transitiva: }
        $$
          \llave{l}{
            x \relacion y \sisolosi x \igual{$\llamada{2}$} 93 - y  \\
            y \relacion z \sisolosi y \igual{$\llamada{2}$}  93 - z \\
          }
          \Entonces{resto}[M.A.M] x - y = -y + z \to x \igual{$\llamada{1}$} z \sisolosi x \relacion z \Tilde
        $$

        \textit{Antisimétrica: }\par
        La $\relacion$ no es antisimétrica, como contraejemplo se ve que
        $1 \relacion 92$ y $92 \relacion 1$ con $1 \distinto 92\quad$ \faIcon{skull}.

  \item
        A priori no sé como encontrar las clases de equivalencia, pero solo buscando la relación del $1$
        con algún número (excepto el mismo) veo que únicamente se puede relacionar con el $92$
        por la condición $\llamada2$, dado que $1 + 92 \igual{$\llamada2$} 93$.
        De ahí se pueden inferir que todas las clases van a ser conjuntos \textit{chiquitos}, con los números que sumen
        93.

        \begin{minipage}{0.7\textwidth}
          Las clases de equivalencia :\par
          $\llave{ccccc}{
              \clase{1}  & = & \clase{92} & = & \set{1, 92}  \\
              \clase{2}  & = & \clase{91} & = & \set{2, 91}  \\
              \vdots     &   & \vdots     &   & \vdots       \\
              \clase{46} & = & \clase{47} & = & \set{46, 47} \\
            }$\par
          Hay entonces 46 clases. $A = \set{\clase{1},\, \clase{2},\, \dots,\, \clase{45},\, \clase{46}}$
        \end{minipage}
        \begin{minipage}{0.2\textwidth}
          \veintisiete
        \end{minipage}
\end{enumerate}

% Contribuciones
\begin{aportes}
  \item \aporte{\dirRepo}{naD GarRaz \github}
\end{aportes}
