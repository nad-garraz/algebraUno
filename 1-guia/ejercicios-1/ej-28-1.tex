\begin{enunciado}{\ejercicio}

  \begin{enumerate}[label=\roman*)]
    \item
          Sea $A = \set{1, 2, 3, 4, 5, 6, 7, 8, 9, 10}$. Consideremos en $\partes(A)$ la relación de equivalencia dada
          por el cardinal (es decir, la cantidad de elementos): Dos subconjuntos de $A$ están relacionados si y solo si
          tienen la misma cantidad de elementos ¿Cuántas clases de equivalencia \textbf{distintas}
          determina la relación? Hallar un representante para cada clase.

    \item
          En el conjunto de todos los subconjuntos finitos de $\naturales$, consideremos nuevamente la relación
          de equivalencia dada por el cardinal: Dos subconjuntos finitos de $\naturales$ están relacionados
          si y solo si tienen la misma cantidad de elementos ¿Cuántas clases de equivalencia
          \textbf{distintas} determina la relación?
          Hallar un representante para cada clase.
  \end{enumerate}
\end{enunciado}

\begin{enumerate}[label=\roman*)]
  \item $\partes(A) = \set{\vacio, \set{1}, \set{1,2}, \cdots, \set{1, 2, 3, 4, 5, 6, 7, 8, 9, 10}}$, el conjunto $\partes(A)$ tiene un total de
        $2^{10} = 1024$ elementos. La relación determina 11 \textit{clases de equivalencia} distintas.
        $$
          \begin{array}{cccc}
            \rowcolor{Cerulean!10}
            \text{Característica de la clase} & \text{clase}                       & \text{elementos ejemplo}                  & \text{Cantidad de conjuntos} \\ \hline
            \text{Conjuntos con \# = 0: }     & \clase{\vacio}                     & \vacio                                    & \binom{10}{0} = 1            \\
            \text{Conjuntos con \# = 1: }     & \clase{\set{1}}                    & \set{3}, \set{7}, \set{10}                & \binom{10}{1} = 10           \\
            \text{Conjuntos con \# = 2: }     & \clase{\set{1,2}}                  & \set{5,2}, \set{1,10}, \set{1,2}          & \binom{10}{2} = 45           \\
            \text{Conjuntos con \# = 3: }     & \clase{\set{1,2,3}}                & \set{1,6, 3}, \set{5,6,7}, \set{8, 9, 10} & \binom{10}{3} = 120          \\
            \text{Conjuntos con \# = 4: }     & \clase{\set{1,2,3,4}}              & \set{1,8, 10,4}, \set{2,4,6,8}            & \binom{10}{4} = 210          \\
            \text{Conjuntos con \# = 5: }     & \clase{\set{1,2,3,4,5}}            & \set{1,3, 5, 7, 9}, \set{2,4,6,8, 10}     & \binom{10}{5} = 252          \\
            \vdots                            & \vdots                             & \vdots                                    & \vdots                       \\
            \text{Conjuntos con \# = 10: }    & \clase{\set{1,2,3,4,5,6,7,8,9,10}} & A                                         & \binom{10}{10} = 1           \\
          \end{array}
        $$
        Entonces las clases de equivalencia del subconjuntos de $\partes(A)$ con cardinal = 0 es $\clase{\vacio}$ y
        de cardinal = 10 es $\clase{A}$.

        Mirá que \textit{bonito} queda el gráfico de la clase de equivalencia de elementos de $\partes(A)$ que tienen un cardinal igual 1,
        es decir \ul{un solo} elemento, hay un total de 10 de esos elementos de $\partes(A)$:
        $$
          \begin{tikzpicture}[
            nodo/.style={color={#1}, text={black}, font={\small}, minimum size=4pt, inner sep=1pt, outer sep=1pt},
            arista/.style={-{Latex[length=3pt, color = red]}, bend left=12, color={#1}, line width=0.06},
            rulo/.style 2 args = {-{Latex[length=3pt]}, out=#1, in=#1+45, looseness=4, color={#2}, line width=0.06}
            ]
            \foreach \i in {1,...,10} {
                \node[nodo=magenta] at ({36 * (\i-1)}:2.5cm) (\i) {$\set{\i}$};
              }

            \foreach \i in {1,...,10} {
                \foreach \j in {1,...,10}{
                    \ifnum\i<\j
                      \draw[arista=magenta] (\i) to (\j);
                      \draw[arista=magenta] (\j) to (\i);
                    \fi
                  }
              }
            \foreach \fase in {1,...,10}{
                \pgfmathsetmacro{\angulo}{-25 + (\fase-1)*36}
                \draw[rulo={\angulo}{magenta}] (\fase) to (\fase);
              }
          \end{tikzpicture}
        $$

        El grafo para los conjuntos que tienen un cardinal igual a 2 tiene $\binom{10}{2} = 45$ nodos.
        Graficaría el resto de los conjuntos, pero no tengo interés en invocar a \red{Satanás} en el proceso...
        % $$
        %   \begin{tikzpicture}
        %     [
        %     node distance=1cm,
        %     nodo/.style={color=#1, minimum size=4pt, inner sep=1pt, outer sep=1pt},
        %     arista/.style={-{Latex[length=2pt]}, bend left=12, color={#1},line width=0.06},
        %     rulo/.style 2 args = {-{Latex[length=2pt]}, out=#1, in=#1+45, looseness=4, color={#2}, line width=0.06}
        %     ]
        %     \foreach \i in {1,...,45} {
        %         \node[nodo=magenta, text=black] at ({360/45 * (\i-1)}:5cm) (\i) {$\bullet$};
        %       }
        %
        %     \foreach \i in {1,...,45} {
        %         \foreach \j in {1,...,45}{
        %             \ifnum\i<\j
        %               \draw[arista=magenta] (\i) to (\j);
        %               \draw[arista=magenta] (\j) to (\i);
        %             \fi
        %           }
        %       }
        %     \foreach \fase in {1,...,45}{
        %         \pgfmathsetmacro{\angulo}{-25 + (\fase-1)*360/45}
        %         \draw[rulo={\angulo}{magenta}] (\fase) to (\fase);
        %       }
        %   \end{tikzpicture}
        % $$

  \item
        Es parecido al inciso anterior, donde ahora $A = \set{1,2,3, \cdots, N-1, N}$, donde $\partes(\naturales_N)$ tiene $2^N$ elementos.\par
        La relación determina $N+1$ \textit{clases de equivalencia} distintas.
        $$
          \begin{array}{cccc}
            \rowcolor{Cerulean!10}
            \text{Característica de la clase} & \text{clase}                   & \text{elemento ejemplo}      & \text{Conjuntos en clase} \\ \hline
            \text{Conjuntos con \#0 }         & \clase{\vacio}                 & \vacio                       & \binom{N}{0}              \\
            \text{Conjuntos con \#1}          & \clase{\set{1}}                & \set{3}                      & \binom{N}{1}              \\
            \text{Conjuntos con \#2 }         & \clase{\set{1,2}}              & \set{5,2}                    & \binom{N}{2}              \\
            \text{Conjuntos con \#3 }         & \clase{\set{1,2,3}}            & \set{1,6, 3}                 & \binom{N}{3}              \\
            \vdots                            & \vdots                         & \vdots                       &                           \\
            \text{Conjuntos con \#$N-1$: }    & \clase{\set{1,2,\ldots,N-1}}   & \set{1,2,3, \ldots,N-2, N-1} & \binom{N}{N-1}            \\
            \text{Conjuntos con \#$N$: }      & \clase{\set{1,2,\ldots,N-1,N}} & A                            & \binom{N}{N}              \\
          \end{array}
        $$
\end{enumerate}

\begin{aportes}
  \item \aporte{\dirRepo}{naD GarRaz \github}
  \item \aporte{https://github.com/franramosfx}{Fran Ramos \github}
\end{aportes}
