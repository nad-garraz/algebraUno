\def\nodosVeintinueve{
  \node[nodo=black, label=left:$1$] (1) {};
  \node[nodo=black, label=left:$2$, below of=1] (2) {};
  \node[nodo=black, label=left:$3$, below of=2] (3) {};
  \node[nodo=black, label=left:$4$, below of=3] (4) {};
  \node[nodo=black, label=left:$5$, below of=4] (5) {};

  \node[nodo=black, label=right:$a$, right=2cm of 1] (a) {};
  \node[nodo=black, label=right:$b$, below of=a] (b) {};
  \node[nodo=black, label=right:$c$, below of=b] (c) {};
  \node[nodo=black, label=right:$d$, below of=c] (d) {};

  \node[container={$A$}, fit={(1) (2) (3) (4) (5)}]  {};
  \node[container={$B$}, fit={(a) (b) (c) (d)}]  {};
}

\begin{enunciado}{\ejercicio}
  Determinar si $\relacion$ es una función de $A$ en $B$ en los casos
  \begin{enumerate}[label=\roman*)]
    \item $A = \set{1,2,3,4,5},\, B = \set{a,b,c,d},\, \relacion = \set{(1,a), (2,a),(3,a),(4,b),(5,c),(3,d)} $
    \item $A = \set{1,2,3,4,5},\, B = \set{a,b,c,d},\, \relacion = \set{(1,a), (2,a),(3,d),(4,b)}$
    \item $A = \set{1,2,3,4,5},\, B = \set{a,b,c,d},\, \relacion = \set{(1,a), (2,a),(3,d),(4,b),(5,c)} $
    \item $A = \naturales,\, B = \reales, \, \relacion = \set{(a,b) \en \naturales \times \reales \talque a = 2b - 3} $
    \item $A = \reales,\, B = \naturales, \, \relacion = \set{(a,b) \en \reales \times \naturales \talque a = 2b - 3} $
    \item $A = \enteros,\, B = \enteros, \, \relacion = \set{(a,b) \en \enteros \times \enteros \talque a +b \text{ es divisible por }5} $
  \end{enumerate}

\end{enunciado}
\begin{enumerate}[label=\roman*)]
  \item No es función, dado que $3 \relacion a$, $3 \relacion d$ y $a \distinto d$
        $$
          \begin{tikzpicture}
            [
            scale = 0.4,
            node distance=0.8cm,
            nodo/.style={circle, fill=black, color={#1}, minimum size = 2pt, inner sep = 1pt},
            arista/.style={-{Latex[length=3pt]}, ultra thin, bend left=25, color={#1}},
            container/.style={draw, rectangle, rounded corners = 10pt,label=above:#1, inner sep=15pt}
            ]
            \nodosVeintinueve
            \draw[arista=Cerulean] (1) to (a);
            \draw[arista=Cerulean] (2) to (a);
            \draw[arista=Cerulean, bend right=10] (3) to (a);
            \draw[arista=red, bend right=10] (3) to (d);
            \draw[arista=Cerulean] (4) to (b);
            \draw[arista=Cerulean, bend right] (5) to (c);
          \end{tikzpicture}
        $$
        Onda, si salen dos flechas del mismo valor ya no es función.

  \item No es función, dado que \ul{todo} elemento de $A$ tiene que estar relacionado a algún elemento de $B,\, 5 \norelacion y$ para ningún $ y \en B$
        $$
          \begin{tikzpicture}
            [
            scale = 0.4,
            node distance=0.8cm,
            nodo/.style={circle, fill=black, color={#1}, minimum size = 2pt, inner sep = 1pt},
            arista/.style={-{Latex[length=3pt]}, ultra thin, bend left=25, color={#1}},
            container/.style={draw, rectangle, rounded corners = 10pt,label=above:#1, inner sep=15pt}
            ]
            \nodosVeintinueve
            \draw[arista=Cerulean] (1) to (a);
            \draw[arista=Cerulean] (2) to (a);
            \draw[arista=Cerulean, bend right=10] (3) to (d);
            \draw[arista=Cerulean] (4) to (b);
          \end{tikzpicture}
        $$

  \item Es función.
        $$
          \begin{tikzpicture}
            [
            scale = 0.4,
            node distance=0.8cm,
            nodo/.style={circle, fill=black, color={#1}, minimum size = 2pt, inner sep = 1pt},
            arista/.style={-{Latex[length=3pt]}, ultra thin, bend left=25, color={#1}},
            container/.style={draw, rectangle, rounded corners = 10pt,label=above:#1, inner sep=15pt}
            ]
            \nodosVeintinueve
            \draw[arista=Cerulean] (1) to (a);
            \draw[arista=Cerulean] (2) to (a);
            \draw[arista=Cerulean, bend right=10] (3) to (d);
            \draw[arista=Cerulean] (4) to (b);
            \draw[arista=Cerulean] (5) to (c);
          \end{tikzpicture}
        $$

  \item Es función.

  \item No es función, $\sqrt{2} \norelacion b$ para ningún $b \en \naturales$ tal que $\sqrt{2} = 2b - 3$. Por lo tanto hay infinitos
        elementos en el conjunto $A$ que no se relacionan con el conjunto $B$.

  \item No es función, porque $0 \relacion 5$ y $0 \relacion 10$ y necesito que si dos elementos $b,\, b' \en B ~\land~ a \relacion b \entonces a \norelacion b'$.
        Un elemento del conjunto de partida no puede estar relacionado con más de un elmentos del conjunto de llegada.
\end{enumerate}
