\begin{enunciado}{\ejercicio}
  Determinar si $A \subseteq B$ en cada uno de los siguientes casos:
  \begin{enumerate}[label=\roman*)]
    \item $ A = \set{1, 2, 3}, B = \set{5,4,3,2,1}$
    \item $ A = \set{1, 2, 3}, B = \set{1,2,\set{3},-3}$
    \item $ A = \set{x \en \reales \talque 2< |x| <3}, B = \set{x \en \reales \talque x^2 < 3 }$
    \item $ A = \set{\vacio}, B = \vacio $
  \end{enumerate}
\end{enunciado}

Inclusión, subconjunto. $A$ es un subconjunto de $B$:
$$
  \cajaResultado{
    A \subseteq B
    \sii
    \paratodo x \en \universo,\, x \en A \entonces x \en B
  }
$$
\begin{enumerate}[label=(\roman*)]
  \item Se tiene que
        $$
          \llave{l}{
            A = \set{1, 2, 3} \\
            B = \set{5,4,3,2,1}
          }
          \flecha{sale con la}[definición]
          A \subseteq B
        $$

  \item $$
          \llave{l}{
            A = \set{1, 2, 3} \\
            B = \set{1,2,\set{3},-3}
          }
          \flecha{respuesta} \quad
          A \nsubseteq B
          \flecha{dado}[que]
          \ub{
            3 \en A\text{ pero } 3 \notin B
          }{
            \text{contra ejemplo}
          }
        $$

  \item
        \def\tresiiiUno{
          \begin{tikzpicture}[
            scale=0.8,
            baseline=0,
            nodo vacio/.style={draw,circle,inner sep = 2pt, fill=white, label=below: {$##1$}},
            ejes/.style={thick, Latex-Latex},
            intervalo/.style={ultra thick, color=magenta}
            ]
            \draw[ejes] (-3.5,0) -- (3.5,0);

            \node[nodo vacio={2}] (2) at (2,0) {};
            \node[nodo vacio={3}] (3) at (3,0) {};
            \node[nodo vacio={-2}] (-2) at (-2,0) {};
            \node[nodo vacio={-3}] (-3) at (-3,0) {};

            \draw[intervalo] (-3) to (-2);
            \draw[intervalo] (2) to (3);
          \end{tikzpicture}
        }

        \def\tresiiiDos{
          \begin{tikzpicture}[
            scale=0.8,
            baseline=0,
            nodo vacio/.style={draw,circle,inner sep = 2pt, fill=white, label=below: {$##1$}},
            ejes/.style={thick, Latex-Latex},
            intervalo/.style={ultra thick, color=Cerulean},
            ]
            \draw[ejes] (-3.5,0) -- (3.5,0);

            \node[nodo vacio={\sqrt{3}}] (sqrt3) at (1.732,0) {};
            \node[nodo vacio={-\sqrt{3}}] (nsqrt3) at (-1.732,0) {};

            \draw[intervalo] (sqrt3) to (nsqrt3);
          \end{tikzpicture}
        }
        $$
          \llaves{ll}{
            A = \set{x \en \reales \talque 2<|x|<3} & \tresiiiUno \\
            B = \set{x \en \reales \talque x^2 < 3 } & \tresiiiDos
          }
        $$
        En este caso:
        $$
          A \nsubseteq B \text{ dado que } 2.5 \en A \text{ y } 2.5 \not\en B
        $$

  \item
        $
          \llave{l}{
            A = \set{\vacio} \\
            B = \vacio
          }
          \flecha{respuesta}
          A \nsubseteq B \flecha{dado}[que] \text{$B$ no tiene ningún elemento, sin embargo $A$ tiene un elemento: $\vacio$.}
        $
\end{enumerate}

\begin{aportes}
  \item \aporte{\dirRepo}{naD GarRaz \github}
  \item \aporte{https://github.com/MateCon}{Mateo Z \github}
\end{aportes}
