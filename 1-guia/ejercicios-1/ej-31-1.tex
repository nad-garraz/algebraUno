\begin{enunciado}{\ejercicio}
  \begin{enumerate}[label=\roman*)]
    \item Dadas las funciones
          $$
            \textstyle
            f : \naturales \to \naturales, \,
            f(n)=
            \llave{cc}{
              \frac{n^2}{2} & \text{si $n$ es divisible por 6}  \\
              3n + 1 & \text{en los otros casos}
            }
            \ytext
            g : \naturales \times \naturales \to \naturales, \, g(n,m) = n(m + 1),
          $$
          calcular, de ser posible, $(f \circ g)(3, 4)$, $(f \circ g)(2, 5)$ y $(f \circ g)(3, 2)$.

    \item Dadas las funciones
          $$
            \textstyle
            f : \reales \to \reales,\,
            f(x)=
            \llave{ccl}{
              x^2 & \text{si} & x \leq 7 \\
              2x - 1 &\text{si} & x > 7
            }
            \ytext
            g : \naturales \to \reales,\,
            g(n) = \sqrt{n},
          $$
          hallar, si existen, todos los $n \en \naturales$ tales que $(f \circ g)(n) = 13$ y todos los
          $m \en \naturales$ tales que $(f \circ g)(m) = 15$.
  \end{enumerate}
\end{enunciado}

\begin{enumerate}[label=\roman*)]
  \item  Es cuestión de evaluar con los numeritos nada más:
        $$
          h(n, m) = (f \circ g)(n, m) = f(g(n,m))
        $$
        \begin{itemize}
          \item $g(3, 4) =\blue{15}, \, f(\blue{15}) = 46$
          \item $g(2,5) = \blue{12}, \, f(\blue{12}) = 72$
          \item $g(3,2) = \blue{9}, \,   f(\blue{9}) = 28$
        \end{itemize}

  \item
        Como $g : \naturales \to \reales$ y $f :\reales \to \reales$ podemos hacer una composición
        $h = (f \circ g): \naturales \to \reales$
        $$
          h(n) =
          \llave{ccl}{
            (\sqrt{n})^2 &\text{si}& \sqrt{n} \leq 7 \\
            2\sqrt{n} - 1& \text{si} &\sqrt{n} > 7
          }
          \quad
          \flecha{o}[más pulido]
          \quad
          \cajaResultado{
            h(n) =
            \llave{ccl}{
              n & \text{si} & n \leq 49 \\
              2\sqrt{n} - 1 & \text{si} & n > 49
            }
          }
        $$
        De esta forma ya vemos directamente que $n = 13$ y $m = 15$ son opciones válidas para lo que pide el enunciado.

        Ya tenemos la primera rama de la función cubierta con esos dos valores, así que buscamos en la otra rama.
        $$
          2\sqrt{n} - 1 = 13 \sii \sqrt{n} = 7 \sii n = 49,
        $$
        pero justo $n = 49$ cae en la primera rama de $h(n)$, así que \ul{no sirve}.
        $$
          2\sqrt{m} - 1 = 15 \sii \sqrt{m} = 8 \sii m =64,
        $$
        es válido pues $64 \geq 49$.

        Conclusión:
        $$
          \cajaResultado{
            (f \circ g)(n) = 13 \sii n = 13
            \ytext
            (f \circ g)(n) = 15 \sii  (n = 15 ~\lor~ n = 64)
          }
        $$
\end{enumerate}

\begin{aportes}
  \item \aporte{https://github.com/sigfripro}{sigfripro \github}
\end{aportes}
