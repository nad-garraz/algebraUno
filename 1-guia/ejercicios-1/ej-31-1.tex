\begin{enunciado}{\ejercicio}
\begin{enumerate}[label=\roman*)]
 \item Dadas las funciones 
 
 $f : \naturales \to \naturales, \, f(n)= $
 $
 \begin{cases}
 \frac{n^2}{2} \text{ si } n \text{ es divisible por } 6 \\
 3n + 1 \text{ en los otros casos}
 \end{cases}
 $
 y
 $g : \naturales \times \naturales \to \naturales, \, g(n,m) = n(m + 1)$,
 calcular, de ser posible, $(f \circ g)(3, 4)$, $(f \circ g)(2, 5)$ y $(f \circ g)(3, 2)$

 \item Dadas las funciones 
 $f : \reales \to \reales, \, f(x)= $
 $
 \begin{cases}
 x^2 \text{ si } x \leq 7 \\
 2x - 1 \text{ si } x > 7
 \end{cases}
 $
 y
 $g : \naturales \to \reales, \, g(n) = \sqrt{n}$, 

 hallar, si existen, todos los $n \en \naturales$ tales que $(f \circ g)(n) = 13$ y todos los
 $m \in \naturales$ tales que $(f \circ g)(m) = 15$.
\end{enumerate}
\end{enunciado}

\begin{enumerate}[label=\roman*)]
\item \begin{enumerate}
        \item $g(3, 4) = 15, \, f(15) = 46$
        \item $g(2,5) = 12, \, f(12) = 72$
        \item $g(3,2) = 9, \, f(9) = 28$
      \end{enumerate}

\item \begin{enumerate}
       Como $g$ va de $\naturales$ a $\reales$ y $f$ toma inputs de $\reales$ podemos hacer una composicion
       $(f \circ g): \naturales \to \reales$
       $$
       (f \circ g)(n) =
       \begin{cases}
       \sqrt{n}^2 \text{ si } \sqrt{n} \leq 7 \\
       2\sqrt{n} - 1 \text{ si } \sqrt{n} > 7 
       \end{cases}
       =
       \begin{cases}
       n \text{ si } n \leq 49 \\
       2\sqrt{n} - 1 \text{ si } n > 49 
       \end{cases}
       $$
       De esta forma ya vemos directamente que $n = 13$ y $m = 15$ son opciones validas para lo q pide el enunciado, veamos
       si hay mas. 
       Ya tenemos la primera rama de la funcion cubierta con esos dos valores, asique buscamos en la otra rama.  
       $2\sqrt{n} - 1 = 13 \sii \sqrt{n} = 7 \sii n = 49$, pero $n = 49$ cae en la primera rama de la funcion, asi que no nos sirve. 
       $2\sqrt{m} - 1 = 15 \sii \sqrt{m} = 8 \sii m =64$, valido pues $64 \geq 49$. 
       Luego los $n \en \naturales$ tales que $(f \circ g)(n) = 13$ es $n = 13$. 

       Y los  $m \in \naturales$ tales que $(f \circ g)(m) = 15$ son $m = 15$ y $m = 64$
      \end{enumerate}
\end{enumerate}

\begin{aportes}
 \item \aporte{https://github.com/sigfripro}{sigfripro \github}
\end{aportes}