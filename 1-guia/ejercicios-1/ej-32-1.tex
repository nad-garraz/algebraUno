\begin{enunciado}{\ejercicio}
  Hallar $f \circ g$ y $g \circ f$ (cuando sea posible) en los casos
  \begin{enumerate}[label=\roman*)]
    \item $f : \reales \to \reales,\, f(x) = 2x^2 - 18 \ytext g: \reales \to \reales,\, g(x) = x + 3.$
    \item $f : \naturales \to \naturales,\, f(n) =
            \llave{ll}{
              n - 2 & \text{si $n$ es divisible por 4}\\
              n + 1 & \text{si $n$ no es divisible por 4}
            }
            \ytext g: \naturales \to \naturales,\, g(n) = 4n.
          $

  \item $f : \reales \to \reales \times \reales, \, f(x) = (x + 5, 3x) \ytext g: \naturales \to \reales,\, g(n) = \sqrt{n}$.
  \end{enumerate}
\end{enunciado}

\begin{enumerate}[label=\roman*)]
 \item $f \circ g = f(x + 3) = 2(x+3)^2 - 18 = 2x^2 + 12x$
 
 $g \circ f = g(2x^2 - 18) = 2x^2 - 18 + 3 = 2x^2 - 15$
 \item $f \circ g = f(4n) = 4n - 2$ 
 
 $g \circ f =$
 $
 \begin{cases}
 g(n - 2) = 4n - 8 \text{ si $n$ es divisible por 4}\\
 g(n + 1) = 4n + 4 \text{ si $n$ no es divisible por 4}
 \end{cases}
 $
 \item En este caso $f \circ g$ se puede componer, pero al revés no (no coinciden las salidas con las entradas). 
 
 $f \circ g = f(\sqrt{n}) = (\sqrt{n} + 5, 3\sqrt{n})$
\end{enumerate}

\begin{aportes}
 \item \aporte{https://github.com/sigfripro}{sigfripro \github}
\end{aportes}
