% MACRO LOCAL
%=======
\def\idn{\text{id}_{\naturales}}
\tikzset{
plot style/.style={
baseline=0,
oval/.style={ellipse, draw, minimum width=1cm, minimum height=1.5cm, align=center, label={##1}:{$\naturales$}},
arrow/.style={-{Latex[length=5pt]}, bend left=20},
comp/.style={-{Latex[length=5pt]}, color=##1, bend right=25, font={\tiny}}
}
}
%=======
\begin{enunciado}{\ejercicio}
  Hallar dos funciones $f : \naturales \to \naturales$ \ytext $g : \naturales \to \naturales$ tales que $f \circ g = \idn
    \ytext g \circ f \distinto \idn$, donde $\idn: \naturales \to \naturales$ denota la función identidad del conjunto $\naturales$.
\end{enunciado}

La identidad en naturales es la función: $\idn = n$.

Quiero funciones que no sean biyectivas, porque de serlo:
$$
  \text{Si } (f \circ g)(n) = \idn \text{ con } f,\, g \text{ biyectivas }
  \entonces
  (g \circ f)(n) = \idn
  \qquad
  \begin{tikzpicture}[
    plot style,
    arrow/.style={{Latex[length=5pt]}-{Latex[length=5pt]}, bend left=20},
    ]
    \node[oval={above left}] (A) {};
    \node[oval={above}, right of=A] (B) {};
    \node[oval={above right}, right of=B] (C) {};

    \draw[arrow] (A.north east) to node[above] {$g$} (B.north west);
    \draw[arrow] (B.north east) to node[above] {$f$} (C.north west);

    \draw[comp={Cerulean}] (A.south east) to node[below] {$f \circ g$} (C.south west);
    \draw[comp={magenta}, bend left=40] (C.south) to node[below] {$g \circ f$} (A.south);
  \end{tikzpicture}
$$
Algo que no quiero porque por enunciado necesito que:
$$
  g \circ f \distinto \idn
$$

\bigskip
Por otro lado sí quiero algo así:
\begin{tikzpicture}[
    plot style,
    node distance=1.7cm,
    imagen/.style={ellipse, draw, dashed, minimum width=0.3cm, minimum height=0.8cm, align=center, color=Cerulean, font={\tiny}},
  ]
  \node[oval={above left}] (A) {};
  \node[oval={above}, right of=A] (B) {};
  \node[oval={above right}, right of=B] (C) {};
  \node[imagen] at (B) (gimagen) {Im};

  \draw[arrow] (A.north east) to node[above] {$g$} (gimagen.north west);
  \draw[arrow] (gimagen.north east) to node[above] {$f$} (C.north west);

  \draw[comp=Cerulean] (A.south east) to node[below] {$f \circ g$} (C.south west);

\end{tikzpicture}
Es decir que $g$ no sea \textit{sobreyectiva}, pero $f$ sí. Buscando inspiración en el ejercicio \refEjercicio{ej:30}:
$$
  g(m) = 2m
  \quad
  \ytext
  \quad
  f = \llave{cl}{
    \frac{n+1}{2} & \text{si $n$ impar}  \\
    \frac{n}{2} & \text{si $n$ par}
  }
$$
La composición quedaría
$$
  (f \circ g)(m)=
  f(g(m)) =
  \llave{cl}{
    \frac{2m+1}{2} & \text{si $2m$ impar}  \\
    m & \text{si $2m$ par}
  }
  \sisolosi
  f \circ g(m) = m = \idn
$$
No obstante
$$
  (g \circ f)(n)=
  g(f(n)) =
  \llave{cl}{
    n + 1 & \text{si $n$ impar}  \\
    n & \text{si $n$ par}
  }
$$
Donde enseguida comprobamos que $g(f(1)) = 2$ por lo tanto, $g \circ f \distinto \idn$.

\begin{aportes}
  \item \aporte{\dirRepo}{naD GarRaz \github}
  \item \aporte{https://github.com/sigfripro}{sigfripro \github}
\end{aportes}
