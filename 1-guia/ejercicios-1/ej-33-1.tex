\begin{enunciado}{\ejercicio}
  Hallar dos funciones $f : \naturales \to \naturales$ \ytext $g : \naturales \to \naturales$ tales que $f \circ g = \text{id}_{\naturales}
    \ytext g \circ f \distinto \text{id}_{\naturales}$, donde $\text{id}_{\naturales}: \naturales \to \naturales$ denota la función identidad del
  conjunto $\naturales$.
\end{enunciado}

Antes que nada, ganemos un poco de intuición de lo que quiere decir el problema, exploremos
como puede ser $g$.

$g$ no puede ser sobreyectiva, ya que implicaría que $\existe x_1, x_2 \en \naturales \text{ tal que } g(x_1) = g(x_2)$,
al componerla con $f$, tendríamos que $f(g(x_1)) = f(g(x_2))$ pero $f$ solo puede dar un valor, no dos, por lo tanto
estamos perdiendo algún valor, y no va a ser igual a la identidad nunca.

$g$ no puede ser biyectiva, ya que implicaría $\paratodo y \en \naturales \existe x \en \naturales \text{ tal que } g(x) = y$, esto
hasta ahora no es un problema, pero vamos a tener $f(y) = \text{id}_{\naturales}$ si y solo si $f$ es biyectiva, y si $f$ es biyectiva entonces
$g \circ f = \text{id}_{\naturales}$, pero por enunciado no queremos eso.

La única opción que queda es que $g$ sea inyectiva así que vamos con eso. Ahora nos fijamos
en $f$, $f$ no puede ser inyectiva tampoco, pues como $g$ es inyectiva existe algun $y_0 \en \naturales$ que no
es alcanzado por la función, es decir que no forma parte de la imagen de $f$, esto fuerza a $f$ a asignar a dos \textit{inputs} diferentes
el mismo valor para poder construir la función identidad, es decir $f$ es sobreyectiva (Ahora con el ejemplo se va a ver mejor esto).
$$
  g = \llave{rcl}{
    1 & \text{si} & n = 1 \\
    2n & \text{si} & n > 1
  }
  \quad
  f = \llave{rcl}{
    1 & \text{si} & n = 1 \\
    \lfloor \frac{n}{2} \rfloor & \text{si} & n > 1
  }
$$
Donde $\lfloor x \rfloor$ denota la función piso, es decir el valor entero de $x$.

Vemos que $f \circ g = f(g(n)) = \lfloor \frac{2n}{2} \rfloor = n$, la función identidad, sin embargo
$g \circ f \neq \text{id}_{\naturales}$, consideré $n = 3$, $f(3) = 1$, $g(1) = 2$, $2 \neq 3$.
De este tipo de funciones se pueden construir infinitas cambiando el $2$ multiplicando por $k \en {\naturales}_{\geq 2}$.

\textit{Nota final:} Los $1$ en la función están para que no haya problemas al hacer la cuenta
$\lfloor \frac{1}{2} \rfloor = 0$ ya que consideramos que $0 \not\en \naturales$

\begin{aportes}
  \item \aporte{https://github.com/sigfripro}{sigfripro \github}
\end{aportes}
