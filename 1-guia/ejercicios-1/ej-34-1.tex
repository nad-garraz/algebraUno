\begin{enunciado}{\ejercicio}
  Sean $A, B \ytext C$ conjuntos. Probar que si $f: B \to C$ y $g: A \to B$ son funciones entonces valen
  \begin{enumerate}[label=\roman*), itemsep = -5pt]
    \item si $f \circ g$ es inyectiva entonces $g$ es inyectiva.
    \item si $f \circ g$ es sobreyectiva entonces $f$ es sobreyectiva.
    \item si $f \ytext g$ son inyectivas entonces $f \circ g$ es inyectiva.
    \item si $f \ytext g$ son sobreyectivas entonces $f \circ g$ es sobreyectiva.
    \item si $f \ytext g$ son biyectivas entonces $f \circ g$ es biyectiva.
  \end{enumerate}
\end{enunciado}

Primero refresquemos las definiciones para el problema:

\underline{Inyectividad:} Sea $f:A \to B$ inyectiva,  $\forall x_1, x_2 \in A | f(x_1) = f(x_2) \entonces x_1 = x_2$ \vspace{0.25em}

\underline{Sobreyectividad:} Sea $f:A \to B$ sobreyectiva, $\forall y \in B, \exists x \en A | f(x) = y$ \vspace{0.25em}

\underline{Biyectividad:} Sea $f:A \to B$ biyectiva, si es inyectiva y sobreyectiva a la vez, o $\forall y \in B, \exists! x \en A | f(x) = y$
\begin{enumerate}[label=\roman*)]
 \item por hipotesis $f \circ g$ es inyectiva, es decir $\forall x_1, x_2 \en A, \, (f \circ g)(x_1) = (f \circ g)(x_2) \entonces 
 x_1 = x_2$, bien, ahora supongamos que para algun $x_1, x_2$, se cumple que $g(x_1) = g(x_2)$, aplicamos la funcion $f$ y tenemos 
 $(f \circ g)(x_1) = (f \circ g)(x_2)$, por hipotesis esto implica que $x_1 = x_2$, quedando que $g(x_1) = g(x_2) \entonces x_1 = x_2$, como se
 queria demostrar. 
 \item por hipotesis $f \circ g$ es sobreyectiva, es decir $\forall z \en C, \exists x \en A | (f \circ g)(x) = z$, queremos demostrar que f
 es sobreyectiva, es decir que $\forall z \en C, \exists y \en B | f(y) = z$. Notemos que $g(x) \en B$, Sea $y := g(x) \llamada1$, luego $f(g(x)) = f(y) = z$, es decir 
 para todo $z$ encontramos un $y$ tal que $f(y) = z$, como se queria demostrar. 
 \item hipotesis: $\forall y_1, y_2 \en B, f(y_1) = f(y_2) \entonces y_1 = y_2$  y  
 $\forall x_1, x_2 \en A, g(x_1) = g(x_2) \entonces x_1 = x_2$.

 Queremos demostrar que $f(g(x_1)) = f(g(x_2)) \entonces x_1 = x_2$.
 Supongamos que $f(g(x_1)) = f(g(x_2))$, por inyectividad de f tenemos que $g(x_1) = g(x_2)$, luego por inyectividad de g
 tenemos que $x_1 = x_2$, como se queria demostrar. 
 \item hipotesis: $\forall z \en C, \exists y \en B | f(y) = z$  y  
 $\forall y \en B, \exists x \en A | g(x) = y$. 

 Queremos demostrar que $\forall z \en C, \exists x \en A | f(g(x)) = z$.
 Por hipotesis sabemos que $f$ y $g$ son sobreyectivas, para el caso de g podemos decir que existe $x$ tal que $g(x) = y$, reemplazamos en f y tenemos 
 $f(g(x)) = z$, es decir para todo $z$ encontramos un $x$ tal que $f(g(x)) = z$, como se queria demostrar.
 
 \item Por hipotesis $f$ y $g$ son biyectivas entonces son inyectivas y sobreyectivas, por ende $f \circ g$ es inyectiva y sobreyectiva (por las dos proposiciones probadas
 arriba), luego $f \circ g$ es biyectiva
\end{enumerate}

$\llamada1$ quiero hacer una salvedad acá, cuando decimos sea $y := g(x)$, estamos tomando un $y \en B$ especifico construido a partir de un $x \en A$. 
No estamos eligiendo un $y$ arbitrario ya que eso implicaria que \textbf{todo} $y$ se puede escribir como $g(x)$ lo que 
implicaría que $g$ es sobreyectiva, lo cual no necesariamente sea cierto.

\begin{aportes}
 \item \aporte{https://github.com/sigfripro}{sigfripro \github}
\end{aportes}