\begin{enunciado}{\ejercicio}
  Sea $f: \set{1,2,3,4} \to \set{1,2,3,4}$ una función.
  Consideremos el conjunto de \textbf{todas} las funciones de $\set{1,2,3,4}$ en $\set{1,2,3,4,5,6,7,8}$, es decir,
  $$
    \F = \set{f: \set{1,2,3,4} \to \set{1,2,3,4,5,6,7,8}}
  $$
  y definimos sobre $\F$ la relación dada por
  $$
    g \relacion h \sisolosi g \circ f = h\circ f.
  $$

  \begin{enumerate}[label=\roman*)]
    \item Probar  que $\relacion$ es una relación de equivalencia. ¿Es siempre antisimétrica (sin importar cómo sea $f$)?

    \item Asumiendo que $f$ es sobreyectiva, calcular la clase de equivalencia de cada $g \en \F$.
  \end{enumerate}
\end{enunciado}

\hyperlink{teoria-1:relaciones}{La teoría de estas importantes propiedades de relaciones está acá}.

Seguramente vos no te confundís, porque sos un {\color{lightgray}{\textit{insoportable sabelotodo}}}, a diferencia mía pero:

\parrafoDestacado[\atencion]{
  Notar que la $f$ en la definición de la relación $\relacion$
  es una \textit{función específica}, siempre te devuelve algo en $\set{1,2,3,4}$. Es por eso que las composiciones que aparecen en la definición
  de $\relacion$ no explotan \faIcon{bomb} por los aires. Lo aclaro porque la \underline{$f$ que está en el conjunto $\F$ esa sí es una $f$ genérica}.

  La $f$ específica la pinto: $\blue{f}$
}

\begin{enumerate}[label=\roman*)]
  \item Para ver si una relación es de equivalencia hay que probar que sea \textit{reflexiva, simétrica y transitiva}.
        \begin{itemize}
          \item[\textit{Reflexiva}:] Quiero ver que:
                $$
                  \paratodo g \en \F,\quad g \relacion g.
                $$
                Lo cual se cumple de forma trivial:
                $$
                  g \relacion g \Sii{def} g \circ \blue{f} = g \circ \blue{f}.
                $$
                Por lo que la relación $\relacion$ es reflexiva.

          \item[\textit{Simétrica}:] Quiero ver que:
                $$
                  \paratodo g, h \en \F,\  \text{si } g \relacion h \entonces h \relacion g.
                $$
                También se cumple de forma trivial:
                $$
                  \ub{g \relacion h}{\text{\purple{hipótesis}}} \Sii{def} g \circ \blue{f} = h \circ \blue{f}
                  \ytext
                  h \relacion g \Sii{def} \ub{h \circ \blue{f} = g \circ \blue{f}}{\text{igual a la \purple{hipótesis}}}.
                $$
                Por lo que la relación $\relacion$ es simétrica.

          \item[\textit{Transitiva}:] Quiero ver que:
                $$
                  \paratodo g,\, h,\, i \en \F,\quad  \text{si } g \relacion h \ytext h \relacion i \entonces g \relacion i.
                $$
                También se cumple de forma trivial:
                $$
                  \ub{g \relacion h \ytext h \relacion i}{\text{\purple{hipótesis}}}
                  \Sii{def}
                  \llave{c}{
                    g \circ \blue{f} = \ub{h \circ \blue{f}}{\text{Hola}} \\
                    \ytext                                                                         \\
                    \ub{h \circ \blue{f}}{\text{qué tal?}} = i \circ \blue{f}
                  }
                  \entonces
                  g \circ \blue{f} = i \circ \blue{f} \entonces g \relacion i.
                $$
                Por lo que la relación $\relacion$ es transitiva.

                \bigskip

                Dado que $\relacion$ resultó ser \textit{reflexiva, simétrica y transitiva} es una \underline{relación de
                  equivalencia}.

                \bigskip

          \item[\textit{Antisimétrica}:] Quiero ver que:
                $$
                  \begin{array}{c}
                    \paratodo g, h \en \F,\quad  \text{si } g \relacion h \ytext g \distinto h \entonces h \norelacion g
                  \end{array}
                $$

                Proponemos dos funciones y las definimos completas:

                $$
                  \begin{array}{rcl}
                    g(1) & = & 5 \\
                    g(2) & = & 5 \\
                    g(3) & = & 5 \\
                    g(4) & = & 5
                  \end{array}
                  \ytext
                  \begin{array}{rcl}
                    h(1) & = & 5 \\
                    h(2) & = & 6 \\
                    h(3) & = & 6 \\
                    h(4) & = & 6
                  \end{array}
                $$
                Claramente $g \neq h$, la idea ahora es proponer que ¡$\blue{f}$ sea una función que me las relacione!
                propongo:
                $$
                  \begin{array}{rcl}
                    \blue{f}(1) & = & 1  \\
                    \blue{f}(2) & = & 1  \\
                    \blue{f}(3) & = & 1  \\
                    \blue{f}(4) & = & 1.
                  \end{array}
                $$
                Por lo tanto:
                $$
                  \begin{array}{rclcl}
                    g \circ \blue{f}(1) & = & g(1) & = & 5  \\
                    g \circ \blue{f}(2) & = & g(1) & = & 5  \\
                    g \circ \blue{f}(3) & = & g(1) & = & 5  \\
                    g \circ \blue{f}(4) & = & g(1) & = & 5.
                  \end{array}
                  \ytext
                  \begin{array}{rclcl}
                    h \circ \blue{f}(1) & = & h(1) & = & 5  \\
                    h \circ \blue{f}(2) & = & h(1) & = & 5  \\
                    h \circ \blue{f}(3) & = & h(1) & = & 5  \\
                    h \circ \blue{f}(4) & = & h(1) & = & 5.
                  \end{array}
                $$

                Y así llegamos a que la función no es antisimétrica, porque tenemos dos funciones \textit{\magenta{distintas}} que
                cumplen ser simétricas, es decir:
                $$
                  g \relacion h = h \relacion g
                $$

                Por lo que la relación $\relacion$ no es antisimétrica.

        \end{itemize}

  \item
        Si $\blue{f}$ es \textit{sobreyectiva}:
        $$
          \blue{f} \text{ sobreyectiva }
          \Sii{def}
          \im(\blue{f}) = \ub{\set{1,2,3,4}}{\purple{\text{codominio}}}
          \Sii{def}
          \paratodo y \en \set{1,2,3,4} \existe x \en \set{1,2,3,4} \text{ tal que } \blue{f(}x\blue{)} = y
        $$
        En particular si el dominio y el codominio tienen la misma cantidad de elementos,
        podemos decir que $\blue{f}$ tiene que ser \textit{biyectiva}.
        Así que $\blue{f}$ es \textit{sobreyectiva}, distintos $x_i$ van a parar a distinto $y_i$
        $$
          \set{x_1, x_2, x_3, x_4}
          \flecha{$\blue{f}$}
          \set{y_1, y_2, y_3, y_4}.
        $$

        $$
          \begin{array}{rcl}
            g \circ \blue{f}(x_1) & = & g(y_1)  \\
            g \circ \blue{f}(x_2) & = & g(y_2)  \\
            g \circ \blue{f}(x_3) & = & g(y_3)  \\
            g \circ \blue{f}(x_4) & = & g(y_4).
          \end{array}
          \ytext
          \begin{array}{rcl}
            h \circ \blue{f}(x_1) & = & h(y_1)  \\
            h \circ \blue{f}(x_2) & = & h(y_2)  \\
            h \circ \blue{f}(x_3) & = & h(y_3)  \\
            h \circ \blue{f}(x_4) & = & h(y_4).
          \end{array}
        $$
        Entonces para que una función $g$
        esté relacionada con otra función $h$:
        $$
          g \relacion h
          \sisolosi
          (g \circ \blue{f})(x_i) = (h \circ \blue{f})(x_i)
          \sisolosi
          g(y_i) = h(y_i)
          \sisolosi
          g = h
          \quad \text{ con } i \en \set{1,2,3,4}
        $$
        Por lo que cada función $g \en \F$ es una clase de un solo elemento:
        $$
          \clase{g} = \set{g}
        $$
\end{enumerate}

\begin{aportes}
  \item \aporte{\dirRepo}{naD GarRaz \github}
  \item \aporte{https://github.com/APNieto/}{Ale Nieto \github}
\end{aportes}
