\textit{Básicos sobre conjuntos y coso: }
\begin{itemize}[label={\tiny\faIcon{smile}}]
  \item \textit{ Las uniones e intersecciones de conjuntos conmutan:}
        $$
          \begin{array}{c}
            A \union B = B \union A \\
            A \inter B = B \inter A
          \end{array}
        $$

  \item
        \textit{De Morgan Law's: }
        $$
          \begin{array}{c}
            (A \union B)^c = A^c \inter B^c \to \text{De Morgan 1} \\
            (A \inter B)^c = A^c \union B^c \to \text{De Morgan 2}
          \end{array}
        $$

  \item \textit{Distribución de la intersección en una unión y alverre: }
        $$
          \begin{array}{c}
            A \yellow{\inter} (B \union C) = (A \yellow{\inter} B) \union (A \yellow{\inter} C) \\
            A \cyan{\union} (B \inter C) = (A \cyan{\union} B) \inter (A \cyan{\union} C)
          \end{array}
        $$
        \begin{center}
          \begin{venndiagram3sets}[shade=orange!30!white, showframe = false,hgap=0, vgap=0, overlap = 1.1cm]
            \fillACapB
            \fillACapC
          \end{venndiagram3sets}
          \begin{venndiagram3sets}[shade=cyan, showframe = false,hgap=0, vgap=0, overlap = 1.1cm]
            \fillA
            \fillBCapC
          \end{venndiagram3sets}
        \end{center}

  \item \textit{Diferencias en sus varios colores, sabores y notaciones: }
        $$
          \begin{array}{c}
            A - B
            \Sii{idem}[notación]
            A \diferencia B
            \Sii{idem}[notación]
            A \inter B^c
          \end{array}
          \
        $$
        \begin{center}
          \begin{venndiagram2sets}[shade=gray!20!white, showframe = false,hgap=0, vgap=0, overlap = 1.1cm]
            \fillANotB
          \end{venndiagram2sets}
        \end{center}

  \item \textit{Diferencia simétrica: }\par
        $$
          A \triangle B =
          \llave{lcl}{
            (A - B)        & \union      & (B - A)                                                     \\
            (A \union B)   & \inter      & (A \inter B)^c                                              \\
            (A \union B)   & \diferencia & (A \inter B)  \to \text{mi favorita \faIcon{meh}} \\
            (A \inter B^c) & \union      & (B \inter A^c)
          }
        $$

        \begin{center}
          \begin{venndiagram2sets}[shade=gray!20!white, showframe = false,hgap=0, vgap=0, overlap = 1.1cm]
            \fillANotB
            \fillBNotA
          \end{venndiagram2sets}
        \end{center}

  \item \textit{Complemento:}\par
        $$
          A^c = \set{x \en \universo \talque x \notin A}
        $$

\item \hypertarget{teoria-1:tablasDeVerdad}{\textit{Tablas de verdad: }}
        %%%MACRO
        \def\subconjuntoYequivalente{
          \begin{array}{|c|}
            A \subseteq B \\
            \hline
            A^c \union B
          \end{array}
        }
        %%%END MACRO

En las tablas de verdad que un elemento esté en un conjunto, $x \en A$ es equivalente a decir que la proposición $A$ es verdadera.
        En mi cabeza es más fácil recordar las tablas en conjuntos que en ... lo otro.
        \[
          \begin{array}{|c|c|c|c|c|c|c|c|}
            \hline
            x \en A & x \en B & x \en A^c & x \en A \inter B & x \en A \union B & x \en \subconjuntoYequivalente & x \en A \triangle B & A - B \\
            \hline
            V       & V       & F         & V                & V                & V                              & F                   & F     \\
            V       & F       & F         & F                & V                & F                              & V                   & V     \\
            F       & V       & V         & F                & V                & V                              & V                   & F     \\
            F       & F       & V         & F                & F                & V                              & F                   & F     \\
            \hline
          \end{array}
        \]

        Cuando para probar $p \entonces q$ se prueba en su lugar $\neg q \entonces \neg p$ se dice que es
        una \textit{demostración
          por contrarrecíproco}.\par
        Cuando para probar $p \entonces q$ se prueba en su lugar $p \land \neg q$ para llegar así
        a una contradicción, se dice que es una demostración por reducción al absurdo.

\end{itemize}
