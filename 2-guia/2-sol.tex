% Preambulo donde están los paquetes que tienen
% los comandos que se usan en el código.
\documentclass[12pt, a4paper, spanish, twoside]{article}
% Sacar draft para que aparezcan las imagenes.
% Opciones: 12pt, 10pt, 11pt, landscape, twocolumn, fleqn, leqno...
% Opciones de clase: article, report, letter, beamer...

% Paquetes:
% =========
\usepackage[headheight=110pt, top = 2cm, bottom = 2cm, left=1cm, right=1cm]{geometry} %modifico márgenes
\usepackage[T1]{fontenc} % tildes
\usepackage[utf8]{inputenc} % Para poder escribir con tildes en el editor.
\usepackage[english]{babel} % Para cortar las palabras en silabas, creo.
\usepackage[ddmmyy]{datetime}
\usepackage{amsmath} % Soporte de mathmatics
\usepackage{mathtools} % Más herramientas para matemáctica
\usepackage{amssymb} % fuentes de mathmatics
\usepackage{array} % Para tablas y eso
\usepackage[dvipsnames]{xcolor} % Para colorear el texto: black, blue, brown, cyan, darkgray, gray, green, lightgray, lime, magenta, olive, orange, pink, purple, red, teal, violet, white, yellow.
\usepackage{enumitem} % Cambiar labels y más flexibilidad para el enumerate
\usepackage{multicol} 
\usepackage{tikz} % para graficar
\usepackage{cancel} % cancelar fórmulas
\usepackage{titlesec} % para editar titulos y hacer secciones con formato a medida
\usepackage{ulem}
\usepackage{centernot} % tacha cosas
\usepackage{bbding} % símbolos de donde uso FiveStar
\usepackage{skull} % símbolos de donde uso Skull
\usepackage{soul} % Para tachar texto en text y math mode
\usepackage{polynom} % para división de polinomios y mcd
\usepackage{fontawesome5} % fuentes "extras"
\usepackage{venndiagram} % Para los diagramas de Venn
\usepackage{qrcode} % genera código qr
%\usepackage{listings} % Escribir código

%\usepackage{algorithm}
%\usepackage{algpseudocode}
%\usepackage{algorithmicx}

\usepackage{fancyhdr} % Encabezados y pie de páginas
% \usepackage{lipsum} % dummy text
% \usepackage{caption} % Configuracion de figuras y tablas



% para hacer los graficos tipo grafos
\usetikzlibrary{shapes,arrows.meta, chains, matrix, calc, trees, positioning, fit}
\usetikzlibrary{external,decorations.pathreplacing,angles,quotes}

% En general quiero que este paquete sea el último en importarse
\usepackage{hyperref} % para que haya links navegables en el PDF
\hypersetup{
    colorlinks=true,
    linkcolor=blue,
    %filecolor=magenta,
    urlcolor=OliveGreen!90!black,
    pdftitle={Álgebra I - Resuelta, sueltísima},
    pdfauthor={Por los alumnos y exalumnos de Álgebra I}
    }
\urlstyle{same}

\setlength{\parindent}{0pt} % Para que no haya indentación en las nuevas líneas.

%% Info SOCIAL
\def\dirRepo{https://github.com/nad-garraz/algebraUno}
\def\dirTelegram{https://t.me/+1znt2GV1i8cwMTNh}
\newcommand{\dirGuia}[1]{\dirRepo/blob/main/#1-guia/#1-sol.pdf}


% Algunos paquetes quizás exclusivos de este archivo que no
% quiero poner en el preamble-general
% el día de mañana podría integrarse
\usetikzlibrary{external,angles,quotes}

% Local acá que se usa mucha inducción
\def\V{\text{ verdadera }}
\def\eq?{\stackrel{\text{?}}}
\def\eqHI{\stackrel{\text{HI}}}
\def\eqDef{\stackrel{\text{def}}}

\begin{document}


% Info para armar título.
\title{Práctica 2 de álgebra 1} % título
\author{Comunidad algebraica} % autor
\date{última compilacion: \today} % Cambiar de ser necesario

% \maketitle  % Para que aprezca el título en el documento

\begin{enumerate}
  \item\label{2-teoria:suma-prod} Propiedades de la sumatoria y productoria:

        \begin{multicols}{2}
          \begin{itemize}
            \item
                  $(\sumatoria{k=1}{n} a_k) + (\sumatoria{k=1}{n} b_k) = \sumatoria{k=1}{n} (a_k + b_k)$

            \item Sea $c$ un número dado:\par
                  $\sumatoria{k=1}{n} (c \cdot a_k) = c \cdot \sumatoria{k=1}{n} a_k$

            \item
                  $(\productoria{k=1}{n} a_k) \cdot (\productoria{k=1}{n} b_k) = \productoria{k=1}{n} (a_k b_k)$

            \item Sea $c$ un número dado:\par
                  $\productoria{k=1}{n} (c \cdot a_k) = (\productoria{k=1}{n} c) \cdot (\productoria{k=1}{n} a_k) = c^n \cdot \productoria{k=1}{n} a_k$
          \end{itemize}
        \end{multicols}

  \item\label{2-teoria:suma-gauss} $\paratodo n \en \naturales: \sumatoria{i = 1}{n} i =  1 + 2 + \cdots + (n-1) + n = \frac{n(n+1)}{2}$

  \item\label{2-teoria:geometrica} $\paratodo n \en \naturales: \sumatoria{i = 0}{n} q^i =
          1 + q + q^2 + \cdots  + q^{n-1} + q^n =
          \llave{lll}{
            n+1 & \text{si} & q = 1\\
            \frac{q^{n+1}-1}{q-1} & \text{si} & q \distinto 1\\
          }$

  \item Inducción: Sea $H \subseteq \reales$ un conjunto. Se dice que $H$ es un conjunto \textit{inductivo} si se cumplen las dos condiciones siguiente:
        \begin{itemize}
          \item $1 \in H$
          \item $\paratodo x , x \in H \entonces x+1 \en H$
        \end{itemize}

  \item Principio de inducción: Sea $p(n), n \in \naturales$ , una afirmación sobre los números naturales.
        Si $p$ satisface
        \begin{itemize}
          \item (Caso Base) $p(1)$ es Verdadera.
          \item (Paso inductivo) $\paratodo h \en \naturales,\, p(h)$ \textit{Verdadera}
                $\entonces p(h+1)$ \textit{Verdadera, entonces $p(n)$ es Verdadera} $\paratodo n \en \naturales$.
        \end{itemize}

  \item Principio de inducción \textit{corrido}: Sea $n_0 \en \enteros$ y sea $p(n),\, n\geq n_0,\,$ una afirmación sobre $\enteros_{\geq n_0}$. Si $p$
        satisface:
        \begin{itemize}
          \item (Caso Base) $p(n_0)$ es Verdadera.
          \item (Paso inductivo) $\paratodo h \geq n_0,\, p(h)$ \textit{Verdadera}
                $\entonces p(h+1)$ \textit{Verdadera, entonces $p(n)$ es Verdadera} $\paratodo n \en \naturales$.
        \end{itemize}
\end{enumerate}

\begin{enumerate}
  \item explicación de las torres de Hanoi.
        \begin{enumerate}[label=\arabic*)]
          \item $a_1 = 1$
          \item $a_3 = 7$
          \item $a_4 = 15$
          \item $a_9 = a_9 +1+a_9 = 2 a_9 +1$
        \end{enumerate}
        $\to$ \boxed{a{_n+1} = 2a_n + 1}

  \item Una sucesión $(a_n)_{n \en \naturales}$ como las torres de Hanoi $a_1 = 1 \y a_{n+1}= 2a_n + 1, \paratodo n \en \naturales$, es una
        sucesión definida por recurrencia.

  \item El patrón de las torres de Hanoi parece ser $\underbrace{a_n = 2^n -1 }_{\text{término general}} \paratodo n \en \naturales$.
        Esto puedo probarse por inducción.
        $ \llave{l}{
            \text{Proposición:} p(n): a_n = 2^n -1\\
            \text{Caso Base: } p(1) \text{ es verdadero?} a_1 = 2^1 -1 =1 \Tilde\\
            \text{Paso inductivo: } p(h) \text{ es verdadero}  \entonces p(h+1) V?\\

            \llave{l}{
              \text{HI}:  a_h = 2^h -1\\
              \text{QPQ}: a_{h+1} = 2^{h+1}
            } \to \text{cuentas y queda que }  \boxed{ p(n)\ es\ V, \paratodo n \en \naturales}
          } $

  \item $\sum$ es una def por recurrencia 
          $\to \sumatoria{k=1}{1} a_k = a_1 \y \sumatoria{k=1}{n+1} a_k =\dots \text{fácil} $
\end{enumerate}


\textit{Principio de inducción III: } Sea $p(n)$ una proposición sobre $\naturales$. Si se cumple:
\begin{enumerate}
  \item  $p(1) \y p(2) \ V$
  \item $\paratodo h \en \naturales,\, p(h) \y p(h+1),\, V \entonces p(h+2)\ V\ (\text{paso inductivo})$,
        entonces $p(n)$ es verdadera.
\end{enumerate}

$p(n): a_n = 3^n$\par

$\llave{l}{
    \text{caso base: } a_1 = 3, a_2 =9 \Tilde\\
    \text{Paso inductivo: } \paratodo h \en \naturales, p(h) \y p(h+1) \ V \entonces p(h+2)\ V\\

    \llave{l}{
      \text{HI: }  a_h = 3^h \y a_{h+1} = 3^{h+1}\\
      \text{Quiero probar que: } a_{h+2}= 3^{h+2}\\
      \text{Usando la fórmula de recurrencia sale enseguida}
    }
  }
$

\textit{Principio de inducción IV } Sea $p(n)$ una proposición sobre $\enteros_{\geq n_0}$. Si se cumple:
\begin{enumerate}
  \item  $p(n_0) \y p(n_0 + 1) \ V$
  \item $\paratodo h \en \enteros_{\geq n_0},\, p(h+1) \y p(h+2)\, V \entonces p(h+2)\ V\ (\text{paso inductivo})$,
        entonces $p(n)$ es verdadera. $\paratodo n \geq n_0$\par
\end{enumerate}


\textit{Sucesión de Fibonacci}: $F_0 = 0, F_1 = 1, F_{n+2} = F_{n+1} + F_n, \paratodo n \geq 0$\par
Truco para sacar fórmulas a partir de Fibo.\\
$F_{n+2} - F_{n+1} - F_n = 0 \to x^2 - x -1 = 0 =
  \llaves{ l }{
    \Phi = \frac{1+\sqrt{5}}{2}\\
    \tilde\Phi = \frac{1-\sqrt{5}}{2}\\
  } \to \Phi^2 = \Phi + 1 \y \tilde\Phi^2  =\tilde\Phi + 1 $
\begin{itemize}
  \item  defino sucesiones $\Phi^n$ que satisfacen la recurrencia de la sucesión de Fibonacci pero no sus condiciones iniciales.
  \item puedo formar una combineta lineal talque: $(c_n)_{n\en \naturales_0} = (a\Phi^n + b\tilde\Phi^n)$ es la sucesión que satisface:
        $\llave{ l }{
            c_o = a+b\\
            c_1 = a\Phi + b\tilde\Phi
          }$ y la recurrencia de Fibonacci.\\
        Resuelvo todo y llego a $\boxed{}$
\end{itemize}

\textit{Sucesione de Lucas}: Generalizaciones de Fibonacci.$(a_n)_{n\in\naturales_0}$\par

$a_0 = \alpha, a_1 = \beta \y a_{n+2} = \gamma a_{n+1} +\delta a_n,\, \paratodo n \geq 0,\, con \alpha, \beta, \gamma, \delta $ dados.\par
Esto lo meto en la ecuación característica: $x^2 - \gamma x -\delta = 0$, necesito raíces distintas.\par
Notar que $r^2 = \gamma r^1 + \delta$, y lo mismo es para $\tilde r$. Las sucesiones ($r^n$) y ($\tilde r^n$) satisfacen la recurrencia de Lucas,
pero no las condiciones iniciales $\alpha$ y $\beta$.
$c_n = (a r^n + b \tilde r^n)$, satisface Lucas, pero las condiciones iniciales son $c_0$ y $c_1$ o
$
  \llave{l}{
    a + b = \alpha\\
    r a +\tilde b = \beta\\
  }\to
  \llave{l}{
    ra +rb = r\alpha\\
    ra + \tilde r b = \beta\\
  }
$ luego hago lo mismo con $\tilde r$
Como resultado: $a = \frac{\beta - \tilde r \alpha}{r - \tilde r}$


\subsubsection*{Ejercicio de la clase del 12/4}
Sea $(a_n)_{n \en \naturales_0}$ con
$\llave{l}{
		a_0 = 1\\
		a_1 = 3\\
		a_n = a_{n-1} - a_{n-2}\ \paratodo n \geq 2
	}$
\begin{enumerate}[label=(\alph*)]
	\item Probar que $a_{n+6} = a_n$\\
	      Por inducción: \boxed{p(n):   a_{n+6} = a_n \paratodo n \geq \naturales_0 \V?}\\
	      $\llave{l}{
		      \textit{ Caso Base:} \text{ Primero notar que,} \\
		      \to
		      \llaves{l}{
			      a_0 = 1\\
			      a_1 = 3\\
			      a_2 \eqDef= 2\\
			      a_3 \eqDef= -1\\
			      a_4 \eqDef= -3\\
			      a_5 \eqDef= -2
		      } \to
		      \llaves{l}{
			      a_6 \eqDef= 1\\
			      a_7 \eqDef= 3\\
			      a_8 \eqDef= 2\\
			      a_9 \eqDef= -1\\
			      a_{10} \eqDef= -3\\
			      a_{11} \eqDef= -2
		      } \to
		      \cdots \text{ Se ve que tiene un período de 6 elementos.}\\

		      p(n=2) \V? \to a_8 \eq?= a_2 \Tilde\\

		      \textit{Paso inductivo: } \text{Supongo } p(k) \V \entonces p(k+1) \V?\\
		      \textit{Hipótesis inductiva: }
		      \text{Supongo } a_{k+6} = a_k \paratodo k \in \naturales_0 \V ,\, \qvq a_{k+7} = a_{k+1}\\
		      \red{a_{k+7}} \eqDef=
		      a_{k+6} - a_{k+5} \stacktext{HI}{=}
		      a_k - a_{k+5} \eqDef=
		      a_k - (\ub{ a_k + a_{k+4}}{a_{k+5}}) = -a_{k+4}\\
		      \to \red{a_{k+7}} = -a_{k+4} \eqDef=
		      -(a_{k+3} - a_{k+2}) \eqDef =
		      - ( a_{k+2} - \red{a_{k+1}} - a_{k+2} ) = \red{a_{k+1}} \Tilde
		      }\\
	      $\\
	      Como $p(0) \y p(1) \y \cdots p(5)$ son verdaderas y $p(k)$ es verdadera así como $p(k+1)$ también lo es, por el principio de inducción $p(n)$ es verdadera $\paratodo n \in \naturales_0$

	\item Calcular $\sumatoria{k=0}{255} a_k$ \\
	      $\sumatoria{k=0}{255} a_k =
		      \underbrace{\textstyle a_0 + a_1 + a_2 + a_3 + a_4 + a_5}_{= 0} +
		      \underbrace{\textstyle a_6 + a_7 + a_8 + a_9 + a_{10} + a_{11} }_{=0} +
		      \cdots +
		      a_{252} + a_{253} + a_{254} + a_{255}
	      $\\
	      En la sumatoria hay \red{256 términos}. $256 = 42 \cdot 6 + 4$ por lo tanto van a haber 42 bloques que dan 0 y sobreviven los últimos 4 términos.
	      $\sumatoria{k=0}{255} a_k = \underbrace{\textstyle 0 + 0 + \dots + 0}_{42 \text{ ceros}} + a_{252} + a_{253} + a_{254} + a_{255} =
		      \cancel{a_{252}} + a_{253} + a_{254} + \cancel{a_{255}} = a_{253} + a_{254} = 5\\
	      $ Donde usé que: $a_n =
		      \llave{rcl}{
			      1 & \text{si} & n\mod6 = 0 \\
			      3 & \text{si} & n\mod6 = 1 \\
			      2 & \text{si} & n\mod6 = 2 \\
			      -1 & \text{si} & n\mod6 = 3 \\
			      -3 & \text{si} & n\mod6 = 4 \\
			      -2 & \text{si} & n\mod6 = 5 \\
		      }\longrightarrow
	      $
	      \boxed{\sumatoria{k=0}{255} a_k = 5} \Tilde
\end{enumerate}

\separador

Sea $(a_n)_{n\en \naturales}$ la sucesión definida por:
$a_1 = 1 \y a_{n+1} = \parentesis{\sqrt{a_n} - (n+1)}^2, \, \paratodo n \en \naturales$.
Voy a encontrar la fórmula general.

$
	\llave{ l }{
		a_1 = 1,\, a_2 = (1 - 2)^2,\, a_3 = 4,\, a_4 = 4,\, a_5 = 9,\, \dots\\
		a_n =
		\llave{l}{
			\parentesis{\frac{n+2}{3}^2} \text{si $n$ es impar}\\
			\parentesis{\frac{n}{3}^2} \text{si $n$ es par}\\
		}\\
		\to \text{ Se muestra por inducción } \hacer\\
	}$


\separador

Sea $(a_n)_{n\en \naturales}$ la sucesión definida por: $a_1 = 3, a_2 = 9 \y a_{n+2} = a+{n+1}+ 3a_n +3^{3+1},\, \paratodo n \en \naturales$\\
Tengo que encontrar el término general de esta  sucesión definida por recurrencia.
$a_1 =3, a_2 = 9. a_3 = a_2 + 3a_1 + 3^2=27 \to$ pinta ser $a_n = 3^n$. \\
Interesante que acá la HI dependería de muchos términos. Así que ahora viene una versión
cambiada del principio de inducción.\\

\section*{Ejercicios de la guía}
\setcounter{ejercicio}{0}

% Para hacer un input de todos los archivos en la
% carpeta de "ejercicio-2"
% ATENCION: Al agregar un ejercicio hay que modificar el loop de manera acorde
\foreach \x in {1,...,17} {
    \input{./ejercicios-2/ej-\x-2}
  }

\end{document}
