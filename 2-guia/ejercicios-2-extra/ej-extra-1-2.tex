\begin{enunciado}{\ejExtra}
  Probar para todo $n \en \naturales$ se cumple la siguiente desigualdad:
  $$
    \frac{(2n)!}{(n!)^2} \leq (n+1)!
  $$
\end{enunciado}

Se prueba usando el principio de inducción $\en \naturales$.\par

\textit{Proposición: }\par
$$
  p(n): \frac{(2n)!}{(n!)^2} \leq (n+1)!
$$

\textit{Caso base: } Evalúo en $n=1$.

$$
  p(\blue{1}):
  \frac{(2 \cdot \blue{1})!}{\blue{1}!^2} = 2 \leq (1+1)! \Tilde
$$
Se concluye que $p(1)$ es verdadera.

\textit{Paso inductivo: }
$$
  p(k): \ub{\frac{(2k)!}{(k!)^2} \leq (k+1)!}{\text{\purple{hipótesis inductiva}}}
$$ la supongo verdadera.

Quiero probar que:
$$
  p(k+1): \frac{(2(k + \magenta{1}))!}{(k + \magenta{1})!^2} \leq (k + \magenta{1} + 1)!
$$ también lo es.\par

$$
  \frac{(2k+2)!}{(k+1)!^2} \leq  (k+2)!
  \Sii{abro}[factorial]
  \frac{(2k + 2) \cdot (2k + 1) \cdot \blue{(2k)!}}{(k + 1)^2 \cdot \blue{(k!)^2}}
  \menorIgual{\purple{HI}}
  \frac{
    \ob{
      (2k + 2) \cdot (2k + 1)
    }{ 4\cdot \cancel{(k+1)} (k + \frac{1}{2}) }
  } {(k + 1)^{\cancel{2}} } \blue{(k + 1)!}
  =
  \frac{4 \cdot (k + \frac{1}{2})}{k + 1} (k+1)!
  \llamada1
$$
En $\llamada1$ fácil probar que:
$$
  \frac{
    4\cdot (k + \frac{1}{2})}{k + 1}
  \menorIgual{$\llamada2$}
  \yellow{(k+2)}
$$
Por lo tanto queda:
$$
  \llamada1 \leq \yellow{(k+2)}(k+1)! = (k+2)!
$$
Y con este último resultado se llega a que:
$$
  \cajaResultado{
    \frac{(2(k+1))!}{(k+1)!^2} \leq  (k+2)!
  }
$$
$\llamada2$ se prueba fácil en 2 cuentas, queda como ejercicio para vos
\purple{\faIcon{hands-wash}}.

Es así que $p(1), p(k), \ytext p(k+1)$ resultaron verdaderas y por el principio de inducción
$p(n)$ también lo será $\paratodo n \en \naturales$.
