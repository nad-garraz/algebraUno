\begin{enunciado}{\ejExtra}
  Probar que $\sumatoria{i=1}{n} \frac{3}{n+i} \geq \frac{37}{20}$ para todo $n \geq 3$.
\end{enunciado}

\textit{Just another problem} de inducción:

Quiero probar que:
$$
  p(n) : \sumatoria{i=1}{n} \frac{3}{n+i} \geq \frac{37}{20}  \paratodo n \geq 3.
$$

\textit{Caso base:}
$$
  p(\blue{3}) : \sumatoria{i=1}{\blue{3}} \frac{3}{\frac{3}+i} = \frac{3}{4} +\frac{3}{5} +\frac{3}{6} = \frac{37}{20}  \geq \frac{37}{20}.
$$
La proposición $p(\blue{3})$ resultó verdadera.

\textit{Paso inductivo:}
Voy a asumir que para algún $\blue{k} \en \naturales_{\geq 3}$:
$$
  p(\blue{k}) :
  \ub{\displaystyle
    \sumatoria{i=1}{\blue{k}} \frac{3}{\blue{k}+i} \geq \frac{37}{20}
  }{\text{\purple{hipótesis inductiva}}}
$$
es verdadera. Entonces quiero probar que:
$$
  p(\blue{k+1}) :
  \sumatoria{i=1}{\blue{k+1}} \frac{3}{\blue{k+1}+i} \geq \frac{37}{20},
$$
también lo sea.

Arranco por el $k+1$, abro la sumatoria porque el término general tiene la $n$ ahí en el medio. A mí me gusta así,
no sé eso de cambiar los índices, estás bienvenido a subir tu solución alternativa con el cambio de índices que
me tiene sin cuidado.
$$
  \begin{array}{rcl}
    \sumatoria{i=1}{\blue{k+1}} \frac{3}{\blue{k+1}+i}
     & = &
    \frac{3}{\blue{k + 1} + 1} +
    \frac{3}{\blue{k + 1} + 2} +
    \cdots +
    \frac{3}{\blue{k + 1} + k - 2} +
    \frac{3}{\blue{k + 1} + k - 1} +
    \frac{3}{\blue{k + 1} + k} +
    \frac{3}{\blue{k + 1} + k + 1} \\
     & = &
    \frac{3}{k + 2} +
    \frac{3}{k + 3} +
    \cdots +
    \frac{3}{2k - 1} +
    \frac{3}{2k} +
    \frac{3}{2k + 1} +
    \frac{3}{2(k+1)}               \\
  \end{array}
$$
Ahora mirá fuerte esa última expresión y encontrá ahí la forma de \textit{armar la expresión de la sumatoria
  de la \purple{hípotesis inductiva} sumando y restando algo}:
$$
  \begin{array}{rcl}
    \sumatoria{i=1}{\blue{k+1}} \frac{3}{\blue{k+1}+i}
     & =                        &
    -\violet{\frac{3}{k+1}}+
    \ub{
      \violet{\frac{3}{k+1}}+
      \frac{3}{k + 2} +
      \frac{3}{k + 3} +
      \cdots +
      \frac{3}{2k - 1} +
      \frac{3}{2k}
    }{
      \text{\purple{¡\grimaceR!}}
    }+
    \frac{3}{2k + 1} +
    \frac{3}{2(k+1)}                                                                                  \\
     & \igual{\red{!}}          &
    \sumatoria{i=1}{\blue{k}} \frac{3}{\blue{k}+i}- \frac{3}{k+1} + \frac{3}{2k+1} + \frac{3}{2(k+1)} \\
     & \mayorIgual{\purple{HI}} &
    \frac{39}{20} - \frac{3}{k+1} + \frac{3}{2k+1} + \frac{3}{2(k+1)}
    \igual{\red{!}}
    \frac{39}{20} + \ub{\frac{3}{2(k+1)(2k+1)}}{\geq 0 \paratodo k \en \naturales}
    \geq \frac{39}{20}
  \end{array}
$$

Por lo tanto
$p(\blue{3}), p(\blue{k}) \ytext p(\blue{k+1})$ resultaron verdaderas entonces por el principio de inducción también
lo es $p(n) \paratodo n \en \naturales_{\geq3}$

\begin{aportes}
  \item \aporte{\dirRepo}{naD GarRaz \github}
\end{aportes}
