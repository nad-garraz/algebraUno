\begin{enunciado}{\ejExtra}
  Sea $(F_k)_{k\en \naturales_0}$ la sucesión de números enteros, conocida como sucesión de Fibonacci,
  definida recursivamente por
  $$
    F_0 = 0,~ F_1= 1
    \ytext
    F_{k+2} = F_k + F_{k+1},~ \paratodo k \geq 0 .
  $$
  Probar que para todo $n \geq 1$ se tiene que $3 \divideA F_{4n}$.
\end{enunciado}

Inducción:

Quiero probar el siguiente \textit{predicado}:
$$
  p(n) : 3 \divideA F_{4n} \paratodo n \en \naturales
$$

\textit{Caso base:}
$$
  p(\blue{1}) : 3 \divideA F_{4 \cdot \blue{1}}
$$
Proposición que resulta verdadera dado que:
$$
  F_4 \igual{def}
  F_2 + F_3 \igual{def}
  2F_0 + 3F_1 = 3
  \sii
  3 \divideA F_4
$$

\textit{Paso inductivo}:

Asumo que para algún $\blue{h} \en \naturales$ la siguiente proposición:
$$
  p(\blue{h}) :
  \ub{
    3 \divideA F_{4\blue{h}}
  }{
    \text{\purple{hipótesis inductiva}}
  }
$$
es verdadera. Entonces quiero probar que la proposición:
$$
  p(\blue{h+1}) : 3 \divideA F_{4(\blue{h+1})},
$$
también lo sea.

\medskip

Parto del paso $\blue{h+1}$, por definición se tiene que:
$$
  F_{4(\blue{h+1})} =
  F_{4h+4} \igual{def}
  F_{4h+2} + F_{4h+3} \igual{def}
  F_{4h} + 2F_{4h+1} + F_{4h+2} \igual{def}
  2F_{4h} + 3F_{4h+1} \conga{3}[\text{\purple{HI}}] 0
$$
Por lo tanto la pr0posición $p(\blue{h+1})$ también resultó verdadera.

Como
$p(\blue{1}), p(\blue{h}) \ytext p(\blue{h + 1})$ resultaron verdaderas, por el principio de inducción también
lo es $p(n) \paratodo n \en \naturales$

\begin{aportes}
  \item \aporte{\dirRepo}{naD GarRaz \github}
\end{aportes}
