\begin{enunciado}{\ejExtra}\fechaEjercicio{final 23/02/2023}
  Sea $(a_n)_{n \en \naturales}$ la sucesión de números enteros definida recursivamente de la forma siguiente:
  $$
    a_1 = 10
    \quad,\quad
    a_{n+1} = 6a_n + 14^{n+2}
  $$
  Probar que para todo $n \en \naturales$ se tiene que $(a_n : 2^{n+3}) = 2^n$.
\end{enunciado}

Ejercicio de inducción, con leves notas de final.

Quiero probar la siguiente proposición:
$$
  p(n) : ~ (a_n : 2^{n+3}) = 2^n \paratodo n \en \naturales
$$

\textit{Caso base}, especificamos al predicado $p(n)$ en $n = \blue{1}$, así obteniendo la siguiente proposición:
$$
  p(\blue{1}) :~
  (a_{\blue{1}} : 2^{\blue{1} + 3}) =
  (10 : 16) = 2
$$
Por lo tanto la proposición $p(\blue{1})$ resultó verdadera.

\textit{Paso inductivo}:
Asumo que para algún valor de $\blue{k} \en \enteros$ la proposción:
$$
  p(\blue{k}) :~
  \ub{
    (a_{\blue{k}} : 2^{\blue{k} + 3}) = 2^{\blue{k}}
  }{
    \text{\purple{hipótesis inductiva}}
  }
$$
es verdadera. Por lo tanto quiero comprobar que la proposición:
$$
  p(\blue{k+1}) :~
  (a_{\blue{k+1}} : 2^{\blue{k+1} + 3}) = 2^{\blue{k+1}},
$$
también lo sea.
$$
  (a_{\blue{k+1}} : 2^{\blue{k+1} + 3}) = 2^{\blue{k+1}}
  \Sii{def}
  (6a_{\blue{k}} + 14^{\blue{k} + 2} : 2^{\blue{k+1} + 3}) = 2^{\blue{k+1}}
  \Sii{\red{!}}
  (2 \cdot 3 \cdot a_{\blue{k}} + 7^{\blue{k} + 2} \cdot 2^{\blue{k} + 2} : 2^{\blue{k+1} + 3}) = 2^{\blue{k} + 1}
$$
Dado que el \textit{máximo común divisor} es un divisor común y por la \purple{hipótesis inductiva} es fácil ver que:
$$
  2^{\blue{k} + 1} \divideA 2^{\blue{k} + 2}
  \Entonces{\red{!}}
  2^{\blue{k} + 1} \divideA 7^{\blue{k} + 2} \cdot 2^{\blue{k} + 2} \quad \llamada1
$$
$$
  2^{\blue{k}} \ua{\divideA}{\text{\purple{HI}}} a_{\blue{k}}
  \sii
  2 \cdot 2^{\blue{k}} \divideA 2 \cdot a_{\blue{k}}
  \Entonces{\red{!!}}
  2^{\blue{k}+1} \divideA 6 \cdot a_{\blue{k}} \quad \llamada2
$$
Sumando $\llamada1$ y $\llamada2$:
$$
  2^{\blue{k} + 1} \divideA 6 \cdot a_{\blue{k}} + 7^{\blue{k} + 2} \cdot 2^{\blue{k} + 2} \llamada3
$$
Ahí vemos que $2^{\blue{k} + 1}$ es un divisor común. Quiero ver que sea el \textit{máximo}.
\parrafoDestacado[\atencion]{
  El \textit{máximo común divisor} entre $\alpha$ y $\beta$ se encuentra multiplicando las \textit{potencias
    con bases comunes elevadas al menor exponente} de las factorizaciones en primos de $\alpha$ y $\beta$
}

La única potencia con base común a ambas expresiones es el 2 ¿Por qué?.
Pruebo ahora que el exponente tiene que ser $\blue{k} + 1$.

Dado que en la \purple{hipótesis inductiva} el \textit{máximo común divisor} es $2^{\blue{k}}$, es decir que:
$$
  2^{\blue{k}} \divideA a_{\blue{k}} ~\land~
  \ub{2^{\blue{k} + 1} \noDivide a_{\blue{k}}}{\text{¿Por qué?}}
  \sii
  a_{\blue{k}} = 2^{\blue{k}} \cdot b_{\blue{k}}
  \quad \text{~con~} b_{\blue{k}} \text{~ impar~}\llamada4
$$
Escribir la sucesión $a_{\blue{k}}$ de esa forma está bueno para mostrar en $\llamada3$ que:
$$
  2^{\blue{k} + 1}
  \divideA
  6 \cdot a_{\blue{k}} + 7^{\blue{k} + 2} \cdot 2^{\blue{k} + 2}
  \Sii{\red{!!}}[$\llamada4$]
  2^{\blue{k} + 1}
  \divideA
  2^{\blue{k} + 1} \cdot
  (
  \ub{
    3 \cdot  b_{\blue{k}} + 7^{\blue{k} + 2} \cdot 2
  }{
    \text{impar}
  }
  )
$$
De forma tal que $2^{\blue{k} + 1}$ no solo es divisor común sino que es el máximo común divisor según la definición.

Es así que la proposición $p(\blue{k} + 1)$ también resultó verdadera.

Dado que $p(1),\, p(\blue{k}), p(\blue{k} + 1)$ resultaron verdaderas por principio de inducción  también lo es $p(n) \paratodo n \en \naturales$.

\begin{aportes}
  \item \aporte{\dirRepo}{naD GarRaz \github}
\end{aportes}
