\begin{enunciado}{\ejExtra}\fechaEjercicio{(7/10/25) primer parcial}

  Probar por inducción que $\sumatoria{k=1}{n} k! \geq \frac{2^n}{n+1}$ para todo $n \en \naturales$.
\end{enunciado}

Ejercicio de \textit{inducción}:

Quiero probar que la proposición $p(n)$:
$$
  p(n) : \sumatoria{k=1}{n} k! \geq \frac{2^n}{n+1} \paratodo n \en \naturales
$$
es verdadera.

\textit{Caso base:}
$$
  p(\blue{1}) : \sumatoria{k=1}{\blue{1}} k!  = 1 \geq \frac{2^{\blue{1}}}{\blue{1} + 1} = 1.
$$
Por lo tanto el caso base es verdadero.

\bigskip

\textit{Paso inductivo:}
Asumo que para algún $\blue{h} \en \naturales$ la proposición
$$
  p(\blue{h}) :
  \ub{
    \sumatoria{k = 1}{\blue{h}} k!  \geq \frac{2^{\blue{h}}}{\blue{h} + 1}
  }{
    \text{\purple{hipótesis inductiva}}
  }
$$
es verdadera. Entonces quiero probar que:
$$
  \textstyle
  p(\blue{h + 1}) : \sumatoria{k=1}{\blue{h + 1}} k!  \geq \frac{2^{\blue{h + 1}}}{\blue{ h + 1 } + 1}
$$
también lo sea.

Arranco por el paso $\blue{h + 1}$:
$$
  \begin{array}{rcl}
    \sumatoria{k = 1}{\blue{h + 1}} k!
     & =                        &
    (\blue{h + 1})! + \sumatoria{k = 1}{\blue{h}} k!                                  \\
     & \mayorIgual{\purple{HI}} &
    (\blue{h + 1})! +  \frac{2^{\blue{h}}}{\blue{h} + 1}                              \\
     & \mayorIgual{\red{!}}     &
    \frac{h+2}{h+2} \cdot  (\blue{h + 1})! +  \frac{2^{\blue{h}}}{\blue{h} + \red{2}} \\
     & =                        &
    \frac{(h+2)! +  2^{\blue{h}}}{h + \red{2}}                                        \\
     & \mayorIgual{\red{!}}     &
    \frac{\red{2^h} +  2^{\blue{h}}}{h + \red{2}}                                     \\
     & =                        &
    \frac{2^{\blue{h + 1}}}{h + \red{2}}                                              \\
  \end{array}
$$
\parrafoDestacado{
  \textit{
    ¿Hace falta demostrar que $(h+2)! \geq 2^h \paratodo h \en \naturales$? No sé. Para mí no es necesario
    pero el corrector tiene la última palabra.
  }
  Lo pruebo porque se me quejan con que hay que demostrarlo y yo que sé:
  $$
    \text{Si} \quad a_h = (h+2)!
    \ytext
    b_h = 2^h
  $$
  La razón a la que crece está sucesión:
  $$
    \frac{a_{n+1}}{a_n} =
    \frac{(h + 3)!}{(h+2)!} = h + 3
    \ytext
    \frac{b_{n+1}}{b_n} =
    \frac{2^{h+1}}{2^h} = 2
  $$
  Ese resultado muestra que la sucesión $a_n$ crece cada vez más rápido (acelera \faIcon{car}) y particularmente siempre crece más que el doble.
  Los primeros valores de cada sucesión:
  $$
    \llave{rclc}{
      a_{\magenta{1}} & = & (\magenta{1} + 2)! &= 6 \\
      b_{\magenta{1}} & = & 2^{\magenta{1}} &= 2
    }
  $$
  Listo, $a_h > b_h \paratodo h \en \naturales$.

  \bigskip

  ¿Por qué hago eso y no hago inducción? Porque en ese \ul{mismo parcial había 2 ejercicios} que salían por inducción, entonces
  me \textit{empaqué} en no hacer 3 veces inducción, y además ahora tenés otra herramienta en tu \textit{toolbox}.

  Pero por inducción sale lo más bien, así que \textit{allá vos}.
}

\bigskip

Por lo tanto $p(\blue{h+1})$ también resultó verdadera.

\bigskip

Dado que $p(1),\, p(\blue{h}), p(\blue{h + 1})$ resultaron verdaderas por principio de inducción
también lo es $p(n) \paratodo n \en \naturales$.
