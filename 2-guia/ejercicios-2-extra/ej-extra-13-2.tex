\begin{enunciado}{\ejExtra}\fechaEjercicio{(7/10/25) primer parcial}

  Probar por inducción que $\sumatoria{k=1}{n} k! \geq \frac{2^n}{n+1}$ para todo $n \en \naturales$.
\end{enunciado}

Ejercicio de \textit{inducción}:

Quiero probar que la proposición $p(n)$:
$$
  p(n) : \sumatoria{k=1}{n} k! \geq \frac{2^n}{n+1} \paratodo n \en \naturales
$$
es verdadera.

\textit{Caso base:}
$$
  p(\blue{1}) : \sumatoria{k=1}{\blue{1}} k!  = 1 \geq \frac{2^{\blue{1}}}{\blue{1} + 1} = 1.
$$
Por lo tanto el caso base es verdadero.

\bigskip

\textit{Paso inductivo:}
Asumo que para algún $\blue{h} \en \naturales$ la proposición
$$
  p(\blue{h}) :
  \ub{
    \sumatoria{k = 1}{\blue{h}} k!  \geq \frac{2^{\blue{h}}}{\blue{h} + 1}
  }{
    \text{\purple{hipótesis inductiva}}
  }
$$
es verdadera. Entonces quiero probar que:
$$
  \textstyle
  p(\blue{h + 1}) : \sumatoria{k=1}{\blue{h + 1}} k!  \geq \frac{2^{\blue{h + 1}}}{\blue{ h + 1 } + 1}
$$
también lo sea.

Arranco por el paso $\blue{h + 1}$:
$$
  \begin{array}{rcl}
    \sumatoria{k = 1}{\blue{h + 1}} k!
     & =                        &
    (\blue{h + 1})! + \sumatoria{k = 1}{\blue{h}} k!                                  \\
     & \mayorIgual{\purple{HI}} &
    (\blue{h + 1})! +  \frac{2^{\blue{h}}}{\blue{h} + 1}                              \\
     & \mayorIgual{\red{!}}     &
    \frac{h+2}{h+2} \cdot  (\blue{h + 1})! +  \frac{2^{\blue{h}}}{\blue{h} + \red{2}} \\
     & =                        &
    \frac{(h+2)! +  2^{\blue{h}}}{h + \red{2}}                                        \\
     & \mayorIgual{\red{!}}     &
    \frac{\red{2^h} +  2^{\blue{h}}}{h + \red{2}}                                     \\
     & =                        &
    \frac{2^{\blue{h + 1}}}{h + \red{2}}                                              \\
  \end{array}
$$
Por lo tanto $p(\blue{h+1})$ también resultó verdadera.

\bigskip

Dado que $p(1),\, p(\blue{h}), p(\blue{h + 1})$ resultaron verdaderas por principio de inducción
también lo es $p(n) \paratodo n \en \naturales$.
