\begin{enunciado}{\ejExtra}
  Probar que, para todo $n \en \naturales$,
  $$
    \sumatoria{k=1}{n+1} \frac{3}{n+k} \leq \frac{5}{2}.
  $$
\end{enunciado}

\textit{Inducción: }
$$
  p(n): \sumatoria{k=1}{n+1} \frac{3}{n+k} \leq \frac{5}{2}
  \paratodo n \en \naturales
$$

\textit{Caso base: }
$$
  p(1):
  \sumatoria{k=1}{\magenta{1}+1} \frac{3}{\magenta{1} + k} =
  \frac{3}{2} + \frac{3}{3} =
  \frac{5}{2}
  \leq
  \frac{5}{2}
$$
Por lo que $p(1)$ resultó verdadera.

\textit{Paso inductivo: }
Asumo como \textit{verdadero} para algún $\magenta{j} \en \enteros$
$$
  p(\magenta{j}): \ub{\sumatoria{k=1}{\magenta{j} + 1} \frac{3}{\magenta{j} + k} \leq \frac{5}{2}}{\text{\purple{hipótesis inductiva}}},
$$
entonces quiero probar que:
$$
  p(\magenta{j + 1}):
  \sumatoria{k=1}{\magenta{j + 1} + 1} \frac{3}{\magenta{j+1} + k} \leq
  \frac{5}{2},
$$
también  sea verdadera.

\bigskip

En los ejercicios donde la $n$ aparece adentro de la sumatoria, conviene abrirla para encontrar la
\purple{hipótesis inductiva}. Arranco abriendo la sumatoria de $p(\magenta{j})$ para que sea más fácil ubicar las cosas:
$$
  \sumatoria{k=1}{\magenta{j} + 1} \frac{3}{\magenta{j} + k} =
  \frac{3}{j+1} +
  \frac{3}{j+2} +
  \frac{3}{j+3} +
  \frac{3}{j+4} +
  \dots +
  \frac{3}{2j} +
  \frac{3}{2j+1}
  \llamada1
$$

Ahora laburon con la expresión de $p(\magenta{j+1})$:
$$
  \begin{array}{rcl}
    \sumatoria{k=1}{\magenta{j + 1} + 1} \frac{3}{\magenta{j+1} + k} =
    \sumatoria{k=1}{j + 2} \frac{3}{j+1 + k}
     & =                        &
    \frac{3}{j+1 + 1} +
    \frac{3}{j+1 + 2} +
    \frac{3}{j+1 + 3} +
    \dots +
    \frac{3}{j+1 + j-1} +
    \frac{3}{j+1 + j} +
    \frac{3}{j+1 + j+1}+
    \frac{3}{j+1 + j+2} =                                                                                                     \\
     & =                        &
    \frac{3}{j+2} +
    \frac{3}{j+3} +
    \frac{3}{j+4} +
    \dots +
    \frac{3}{2j} +
    \frac{3}{2j+1} +
    \frac{3}{2j+2}+
    \frac{3}{2j+3}                                                                                                            \\
     & \igual{\red{!!}}         &
    \yellow{-\frac{3}{j+1}} +
    \ub{
      \yellow{\frac{3}{j+1}  } +
      \frac{3}{j+2} +
      \frac{3}{j+3} +
      \frac{3}{j+4} +
      \dots +
      \frac{3}{2j} +
      \frac{3}{2j+1}
    }{
      \sumatoria{k=1}{j + 1} \frac{3}{j + k}
    } +
    \frac{3}{2j+2}+
    \frac{3}{2j+3}                                                                                                            \\
     & =                        &
    \yellow{-\frac{3}{j+1}} + \frac{3}{2j+2} + \frac{3}{2j+3} + \sumatoria{k=1}{j + 1}\frac{3}{j + k} \igual{\red{!}}
    \ub{- \frac{3}{2j+2} + \frac{3}{2j+3}}{\leq 0}  + \purple{\sumatoria{k=1}{j + 1} \frac{3}{j + k}}\menorIgual{\purple{HI}} \\
     & \menorIgual{\purple{HI}} &
    \ob{- \ub{\frac{3}{(jk+2)(2j+3)}}{> 0 }}{<0} + \purple{\frac{5}{2}}
    \leq
    \frac{5}{2}
    \entonces
    \sumatoria{k=1}{j + 2} \frac{3}{j+1 + k}
    \leq
    \frac{5}{2}
  \end{array}
$$
Mostrando que $p(j+1)$ también es verdadera.

\bigskip

Dado que $p(1),\, p(j), p(j+1)$ resultaron verdaderas por principio de inducción  también lo es $p(n) \paratodo n \en \naturales$.

\begin{aportes}
  \item \aporte{\dirRepo}{naD GarRaz \github}
  \item \aporte{https://github.com/gdcorrea97}{Gustavo Correa \github}
\end{aportes}
