%Macro local
\def\h{\magenta h}

\begin{enunciado}{\ejExtra}
  Probar que
  $$
    \sumatoria{k = 1}{n} (2k-1)^2 \geq \frac{(2n-1)^3}{6} \quad
    \paratodo n \en \naturales.
  $$
\end{enunciado}

Ejercicio de inducción. Voy a probar que la preposición
$$
  p(n) : \sumatoria{k = 1}{n} (2k-1)^2 \geq \frac{(2n-1)^3}{6}
$$
sea verdadera para todos los naturales.

\textit{Caso base: }
$$
  p(\blue1): \sumatoria{k = 1}{\blue1} (2k-1)^2  =
  (2\cdot \blue1 - 1)^2 = 1
  \geq
  \frac{(2 \cdot \blue1 - 1)^3}{6} =
  \frac{1}{6}.
$$
Por lo tanto $p(1)$ es verdadera \Tilde

\textit{Paso inductivo: }

Asumo $p(\h)$ verdadera, entonces quiero probar que $p(\magenta{h+1})$ también lo sea. En este caso:
$$
  p(\h) : \ub{\sumatoria{k = 1}{\h} (2k-1)^2 \geq \frac{(2\h-1)^3}{6}}{\text{\purple{hipótesis inductiva}}}
$$
para algún $\magenta{h} \en \enteros$.

Quiero probar que:
$$
  \sumatoria{k = 1}{\magenta{h + 1}} (2k-1)^2 \geq \frac{(2(\magenta{h+1})-1)^3}{6} = \frac{(2h+1)^3}{6},
$$
sea verdadera para algún $h \en \enteros$.
$$
  \sumatoria{k = 1}{\magenta{h+1}} (2k-1)^2 =
  \sumatoria{k = 1}{\magenta h} (2k-1)^2 + (2(\magenta{h+1}) - 1) =
  \sumatoria{k = 1}{\magenta h} (2k-1)^2 + (2\h +1)^2
$$

\parrafoDestacado{
  \textit{Nota innecesaria pero que quizás aporta: }

  Lo que acabamos de hacer recién nos deja la \purple{HI} regalada.
  Pero atento que esto \underline{suele} funcionar cuando \underline{no} tenemos a la $n$
  en el término principal de la sumatoria. Después de hacerte éste, mirá el \refEjExtra{ejExtra:2} y ya que estás
  también mirá el \refEjExtra{ejExtra:7} donde si bien hay una $n$ conviene encararlo distinto al \refEjExtra{ejExtra:2}.

  \textit{Fin nota innecesaria pero que quizás aporta}.
}

$$
  \sumatoria{k = 1}{\magenta{h+1}} (2k-1)^2 =
  \sumatoria{k = 1}{\h} (2k-1)^2 + (2\h +1)^2
  \mayorIgual{\purple{HI}}
  \ub{
    \frac{(2\h-1)^3}{6} + (2\h + 1)^2
    \geq
    \frac{(2h+1)^3}{6}
  }
  {
    \llamada1
  }
$$
Si ocurre eso en $\llamada1$, entonces $p(h+1)$ será verdadera:
$$
  \begin{array}{rcl}
    \llamada1
     & \Sii{$\times 6$}          &
    (2h-1)^3 + 6(2h + 1)^2
    \geq
    (2h+1)^3                                   \\
     & \Sii{distribuyo}[a morir] &
    \cancel{8 h^3} + \cancel{12 h^2} + 30 h + 5
    \geq
    \cancel{8 h^3} + \cancel{12 h^2} + 6 h + 1 \\
     & \sii                      &
    24h + 4
    \geq
    0\quad \paratodo h \en \naturales
  \end{array}
$$
Quedó entonces que:
$$
  \sumatoria{k = 1}{h+1}(2k-1)^2 \geq \frac{(2h+1)-1)^3}{6},
$$
concluyéndose que $p(h+1)$ también es verdadera.\medskip

Como tanto $p(1), p(h) \ytext p(h+1)$ resultaron verdaderas, por el principio de inducción se tiene que
$p(n)$ es verdadera para todo $n \en \naturales$.

\begin{aportes}
  \item \aporte{\dirRepo}{naD GarRaz \github}
\end{aportes}
