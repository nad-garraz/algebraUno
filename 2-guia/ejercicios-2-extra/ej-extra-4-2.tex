\begin{enunciado}{\ejExtra}
  Hallar una fórmula cerrada para
  $$
    \sumatoria{k=1}{n} \frac{k}{(k+1)!} \,.
  $$
  Probar su validez usando inducción.
\end{enunciado}

Probando números viendo de encontrar un patrón:
$$
  \llave{l}{
    n = \magenta{1} \to \sumatoria{k = 1}{\magenta{1}} \frac{k}{(k+1)!} = \green{\frac{1}{2}}                                                  \\
    n = \magenta{2} \to \sumatoria{k = 1}{\magenta{2}} \frac{k}{(k+1)!} = \green{\frac{1}{2}} + \frac{1}{3} = \blue{\frac{5}{6}}   \\
    n = \magenta{3} \to \sumatoria{k = 1}{\magenta{3}} \frac{k}{(k+1)!} = \blue{\frac{5}{6}} + \frac{1}{8} = \yellow{\frac{23}{24}} \\
    n = \magenta{4} \to \sumatoria{k = 1}{\magenta{4}} \frac{k}{(k+1)!} = \yellow{\frac{23}{24}} + \frac{1}{30} = \frac{119}{120}
  }
$$
Es notable que una potencial fórmula cerrada sería:
$$
  (a_n)_{n \geq 1} = \frac{(n+1)! - 1}{ (n + 1)!}.
$$

Inducción para probar la fórmula.
$$
  p(n):
  \sumatoria{k=1}{n} \frac{k}{(k+1)!} =
  \frac{(n+1)! - 1}{ (n + 1)!}
  \quad \paratodo n \en \naturales
$$

\textit{Caso base:}\par
$$
  p(\magenta{1}):
  \sumatoria{k = 1}{\magenta{1}} \frac{k}{(k+1)!} =
  \green{\frac{1}{2}} = \frac{(\magenta{1}+1)! - 1}{(\magenta{1} + 1)!} =\green{\frac{1}{2}}   \Tilde
$$
p(1) es verdadera.\medskip

\textit{Paso inductivo:}\par
$$
  \text{asumo que }
  p(\blue{j}):
  \ob{\displaystyle
  \sumatoria{k=1}{\blue{j}} \frac{k}{(k+1)!} =
  \frac{(\blue{j}+1)! - 1}{ (\blue{j} + 1)!}
  }{\text{\purple{hipótesis inductiva}}}
  \quad \text{para algún } j  \en \naturales, \text{ es verdadera.}
$$
entonces, quiero ver que
$$
  p(\green{j+1}):
  \sumatoria{k=1}{\green{j+1}} \frac{k}{(k+1)!} =
  \frac{(\green{j+1} + 1)! - 1}{ (\green{j + 1} + 1)!}
  \quad \text{también es verdadera.}
$$
\par\medskip

$$
\textstyle
  \sumatoria{k=1}{\green{j+1}} \frac{k}{(k+1)!}
  \igual{\red{!}}
  \sumatoria{k=1}{\green{j}} \frac{k}{(k+1)!} + \frac{j + 1}{ (j + 2)!}
  \igual{\purple{HI}}
  \purple{\frac{(j+1)! - 1}{ (j + 1)!}} + \frac{j + 1}{ (j + 2)!}
  \igual{\red{!!!}}
  \frac{(j+2)! - (j+2) + j + 1}{(j+2)!} = 
  \frac{(j+2)! - 1}{(j+2)!}.
$$\par

Por lo tanto $p(j+1)$ es verdadera.\par
Si te quedaste pedaleando en el \red{!!!}, multipliqué y dividí por
\textit{algo} en el primer término, para tener mismo denominador y bla, bla, listo \faIcon{hands-wash}.\par\bigskip

Dado que $p(1), p(j) \ytext p(j+1)$ son verdaderas por el principio de inducción, $p(n)$ también lo es para todo $n \en \naturales$.



