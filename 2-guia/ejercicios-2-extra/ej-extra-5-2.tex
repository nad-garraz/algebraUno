\begin{enunciado}{\ejExtra}
  Probar que para todo $n \en \naturales$
  $$
    \productoria{i=1}{n} \frac{i!}{i+n} \geq \frac{1}{10}.
  $$
\end{enunciado}

Ejercicio de inducción que tiene una productoria o sumatoria y una $n$ en el término general.
Si hiciste los ejercicios anteriores, sabés que hay que atacar el problema \textit{abriendo la productoria}.\bigskip

\textit{Proposición:}
$$
  p(n):
  \productoria{i=1}{n} \frac{i!}{i+n} \geq \frac{1}{10} \quad \paratodo n \en \naturales
$$
\textit{Caso base}
$$
  p(1):
  \productoria{i=1}{1} \frac{i!}{i+1}  = \frac{1}{2} \geq \frac{1}{10} \Tilde
$$
El caso $p(1)$ resulta verdadero.\bigskip

\textit{Paso inductivo: }

Asumo que

$$
  p(\blue{k}) :
  \ub{
    \productoria{i=1}{\blue{k}} \frac{i!}{i+\blue{k}} \geq \frac{1}{10}}{\text{\purple{hipótesis inductiva}}} \Tilde
$$

es verdadero para algún $k \en \enteros$. Y quiero probar que:

$$
  p(\blue{k+1}) :
  \productoria{i=1}{\blue{k+1}} \frac{i!}{i+\blue{k+1}} \geq \frac{1}{10}
$$
también lo sea.\par

Empiezo abriendo las productorias para estudiar los factores, uno por uno:
$$
  \begin{array}{l}
    \productoria{i=1}{\blue{k}} \frac{i!}{i+\blue{k}} =
    \frac{1!}{k+1} \cdot
    \frac{2!}{k+2} \cdots
    \frac{(k-2)!}{2k-2} \cdot
    \frac{(k-1)!}{2k-1} \cdot
    \frac{k!}{2k}
  \end{array}
$$

$$
  \begin{array}{l}
    \productoria{i=1}{\blue{k+1}} \frac{i!}{i+\blue{k+1}} =
    \frac{1!}{k+2} \cdot
    \frac{2!}{k+3} \cdots
    \frac{(k-2)!}{2k-1} \cdot
    \frac{(k-1)!}{2k} \cdot
    \frac{k!}{2k+1} \cdot
    \frac{(k+1)!}{2k+2}
  \end{array}
$$

Con esos resultados hay que ver que todos los factores \textit{corridos} de la
productoria hasta $\blue{k}$ están ahí escondidos en la productoria de $\blue{k+1}$.
Tomate un tiempo para acomodar y hacer aparecer la \purple{hipótesis inductiva}.\par
Queda como corriendo los numeradores un lugar a la izquierda, no?

$$
  \begin{array}{rcl}
    \productoria{i=1}{\blue{k+1}} \frac{i!}{i+\blue{k+1}}
     & =               &
    \ob{\magenta{\frac{k+1}{k+1}}}{1} \cdot
    \frac{1!}{k+2} \cdot
    \frac{2!}{k+3} \cdots
    \frac{(k-2)!}{2k-1} \cdot
    \frac{(k-1)!}{2k} \cdot
    \frac{k!}{2k+1} \cdot
    \frac{(k+1)!}{2k+2} = \\
     & \igual{\red{!}} &
    \frac{1!}{\magenta{k+1}} \cdot
    \frac{2!}{k+2} \cdot
    \frac{3!}{k+3} \cdots
    \frac{(k-2)!}{2k-2} \cdot
    \frac{(k-1)!}{2k-1} \cdot
    \frac{k!}{2k} \cdot
    \frac{(k+1)!}{2k+1} \cdot
    \frac{\magenta{k+1}}{2k+2}
  \end{array}
$$

Oka, mismo truco viejo de siempre. Acomodar y multiplicar por un 1 disfrazado de cosas útiles.
Ahora nos sacamos un montón de términos con la \purple{hipótesis inductiva} y luego acotamos.
$$
  \begin{array}{rcl}
    \productoria{i=1}{\blue{k+1}} \frac{i!}{i+\blue{k+1}}
     & \taa{\purple{\text{HI}}}{}{\geq} &
    \frac{1}{10} \cdot \frac{(k+1)!}{2k+1} \cdot \frac{k+1}{2k+2}
    =
    \frac{1}{20} \cdot \frac{(k+1)!}{2k+1}
    \taa{\red{!}}{}\geq
    \frac{1}{20} \cdot \frac{(k+1)!}{2k+\magenta{2}} =
    \frac{1}{40} \cdot k!
    \taa{\red{!}}{}\geq
          \frac{1}{10} \quad \paratodo k \en \naturales_{> 2}
  \end{array}
$$

$$
  \productoria{i=1}{\blue{k+1}} \frac{i!}{i+\blue{k+1}} \geq \frac{1}{10} \quad \paratodo k \en \naturales_{> 2}
$$

Y dado que
$$
  p(2):
  \productoria{i=1}{2} \frac{i!}{i+2}  = \frac{1}{3} \cdot \frac{1}{2} = \frac{1}{6} \geq \frac{1}{10} \Tilde
$$
es verdadera la acotación que hicimos nos sirve.

\bigskip

Es así que $p(1), p(2), p(k), \text{ y } p(k+1)$ resultaron verdaderas y por el principio de inducción
$p(n)$ también lo será $\paratodo n \en \naturales$.

