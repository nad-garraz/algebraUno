\begin{enunciado}{\ejExtra}
  Sea $(a_n)_{n\en\naturales}$ la sucesión de números reales definida por $a_1 = 3, a_2 = 6$, y para $n \geq 1$,
  $$
    a_{n+2} = \frac{2n + 3}{7} (a_{n+1} + 2a_n)
  $$
  Probar que $a_n > 2^n \paratodo n \en \naturales$
\end{enunciado}

Sucesiones definidas por recurrencia e inducción.

\textit{Proposición:}
Quiero probar que
$$
  p(n) : a_n > 2^n \paratodo n \en \naturales
$$

\textit{Casos base:}

$$
  \begin{array}{l}
    p(\blue{1}) : a_1 = 3 > 2^{\blue{1}} = 2 \to a_1 > 2^1 \\
    p(\blue{2}) : a_2 = 6 > 2^{\blue{2}} = 4 \to a_2 > 2^2 \\
  \end{array}
$$

Los casos base $p(1), p(2)$ resultaron verdaderos. \bigskip

\textit{Paso inductivo:}
Asumo como verdadero para algún $k \en \naturales$:
$$
  \begin{array}{c}
    p(\blue{k}) : \ub{ a_{\blue{k}} > 2^{\blue{k}} }{\purple{\text{hipótesis inductiva}}} \\
    p(\blue{k+1}) : \ub{a_{\blue{k+1}} > 2^{\blue{k+1}}}{\purple{\text{hipótesis inductiva}}}
  \end{array}
$$

Entonces quiero probar que:

$$
  p(\blue{k+2}) : a_{\blue{k+2}} > 2^{\blue{k+2}}
$$
también lo sea.

Usando la definición:

$$
  a_{\blue{k+2}}
  \igual{def}
  \frac{2k+3}{7} \cdot (a_{k+1} + 2 \cdot a_k)
  \mayor{\purple{HI}}
  \frac{2k+3}{7} \cdot (2^{k+1} + 2 \cdot 2^k)
  =
  \frac{2k+3}{7} \cdot (2^{k+2})
$$
Por lo tanto se tiene que:
$$
  a_{k+2} > \frac{2k+3}{7} \cdot (2^{k+2}) \geq 2^{k+2} \quad \paratodo k \en \naturales_{\geq 2}
$$

Es así que se cumple $p(k+2) \paratodo k \en \naturales_{\geq 2}$

El caso que faltaría es con $k = 1$

$$
  p(3): a_{1 + 2} = a_3 \igual{def} \frac{5}{7} (6 + 6) = \frac{60}{7}  > 2^3 = 8 \to a_3 > 2^3
$$

también se cumple.

Dado que $p(1), p(2), p(3), p(k), p(k+1) \ytext p(k+2)$ son todas verdaderas, por principio de inducción $p(n)$ es verdadera $\paratodo n \en \naturales$.

\begin{aportes}
  \item \aporte{\dirRepo}{naD GarRaz \github}
\end{aportes}
