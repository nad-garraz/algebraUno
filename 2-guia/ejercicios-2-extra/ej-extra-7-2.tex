\begin{enunciado}{\ejExtra}
  Probar que, para todo $n \en \naturales$,
  $$
    \sumatoria{i=1}{n}(n-i)2^{i-1} = 2^n -n -1.
  $$
\end{enunciado}
Ejercicio de inducción \text{\small \rosa{\faIcon[regular]{meh-rolling-eyes}}}.

Queremos probar nuestras proposición $p(n)$:
$$
  p(n) : \sumatoria{i=1}{n}(n-i)2^{i-1} = 2^n - n -1 \quad \paratodo n \en \naturales
$$

\textit{Caso base  ¿$p(1)$ es verdadera?:}
$$
  p(\blue{1}) : \sumatoria{i=1}{\blue{1}}(\blue{1} - i)2^{i-1} = 0 =  2^{\blue{1}} - \blue{1} -1
$$
Por lo tanto $p(1)$ resulta verdadera.

\textit{Paso inductivo:}

Se asume que para algún $k\en \naturales$
$$
  p(\blue{k}) : \ub{
  \sumatoria{i=1}{\blue{k}}(\blue{k} - i)2^{i-1} =  2^{\blue{k}} - \blue{k} - 1
  }{\purple{\text{hipótesis inductiva}}}
$$

es verdadera. Y queremos probar que:

$$
  p(\blue{k+1}) : \sumatoria{i=1}{\blue{k+1}}(\blue{k+1}-i)2^{i-1} =  2^{\blue{k+1}} - (\blue{k+1}) - 1 = 2^{k+1} - k - 2
$$
también lo sea.

Empezando con $p(k+1)$. Acomodamos y vemos de hacer aparecer la \purple{hipótesis inductiva}:
$$
  \begin{array}{rcl}
    \sumatoria{i=1}{\blue{k+1}}(\blue{k+1}-i)2^{i-1} & \igual{\red{!!}}           & \sumatoria{i=1}{k}(k-i)2^{i-1} + \sumatoria{i=1}{k} 2^{i-1} + 0 \\
                                                     & \igual{\purple{\text{HI}}} & \purple{2^k - k - 1} +  \sumatoria{i=1}{k} 2^{i-1}
  \end{array}
$$
En el \red{!!} saqué el término $k+1$ y después agrupé de manera conveniente para separar las sumatorias. Nos queda ahora esa sumatoria que no es
otra cosa que una querida y odiada \textit{serie geométrica}, esa que no te acordás, que si empieza en 0 o en 1,
pero \hyperlink{2-teoria:geometrica}{acá tenés la formulita y coso en la teoría}.
$$
  \sumatoria{i=1}{k} 2^{i-1} \igual{\red{!}}
  \frac{1}{2} \cdot \sumatoria{i=1}{k} 2^i =
  \frac{1}{2} \cdot \frac{2^{k+1} - 2}{2 - 1}
  \igual{\red{!}}
  2^k - 1
$$
Por lo tanto con ese resultado:
$$
  \begin{array}{rcl}
    \sumatoria{i=1}{\blue{k+1}}(\blue{k+1}-i)2^{i-1} & \igual{\red{!!}} & \purple{2^k - k - 1} + 2^k - 1 =  2^{k+1} - k - 2
  \end{array}
$$
Por lo tanto $p(k+1)$ también es verdadera.

\bigskip

Se demostró que $p(1), p(k) \ytext p(k+1)$ son verdaderas y por el principio de inducción $p(n)$ también lo es $\paratodo n \en \naturales$.

\begin{aportes}
  \item \aporte{https://github.com/nad-garraz}{Nad Garraz \github}
  \item \aporte{https://github.com/JowinTeran}{Ale Teran \github}
\end{aportes}
