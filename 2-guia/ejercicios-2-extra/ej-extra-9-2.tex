\begin{enunciado}{\ejExtra}
  Probar que $\productoria{k=1}{n}\frac{10k - 5}{2k} > n 3^{n-1}$ para todo $n \en \naturales$.
\end{enunciado}

Inducción pura y dura.

\bigskip

Quiero probar la siguiente proposición:
$$
  p(n) :
  \productoria{k=1}{n}\frac{10k - 5}{2k} > n 3^{n-1} \quad \paratodo n \en \naturales
$$

\textit{Caso Base:}
$$
  p(\blue{1}) :
  \productoria{k=1}{\blue{1}}\frac{10k - 5}{2k} > \blue{1} 3^{\blue{1}-1}  = \frac{5}{2} > \blue{1} \cdot 3^{\blue{1} - 1} = 1
$$
Por lo tanto $p(\blue{1})$ resultó ser verdadera.

\medskip

Asumo ahora para algún $k \en \naturales$
$$
  p(\blue{h}) :
  \ub{\productoria{k=1}{\blue{h}}\frac{10k - 5}{2k} > \blue{h} 3^{\blue{h}-1}
    >
    \blue{h} \cdot 3^{\blue{h} - 1}}{\text{\purple{hipótesis inductiva}}}
$$
es verdadera. Y quiero probar que:
$$
  p(\blue{h+1}) : \productoria{k=1}{\blue{h+1}}\frac{10k - 5}{2k} > \blue{h+1} 3^{\blue{h+1}-1}
  >
  (\blue{h+1}) \cdot 3^{\blue{h+1} - 1} =
  (\blue{h+1}) \cdot 3^{\blue{h}}
$$
Partiendo del paso $(\blue{h + 1})-$ésimo:
$$
  \begin{array}{rcl}
    \productoria{k=1}{\blue{h+1}}\frac{10k - 5}{2k}
     & \igual{\red{!}}     &
    \frac{10(\blue{h+1}) - 5}{2(h+1)} \cdot
    \productoria{k=1}{\blue{h}}\frac{10k - 5}{2k}                    \\
     & \mayor{\purple{HI}} &
    \frac{10\blue{h} + 5}{2(h+1)} \cdot \blue{h}\cdot 3^{\blue{h}-1} \\
     & \mayor{\red{!}}     &
    (\blue{h+1}) \cdot 3^{\blue{h}}
  \end{array}
$$
Si podemos probar que el último $\mayor{\red{!}}$ es verdadero, listo, ganamos. Acomodo un poco, nada raro:
$$
  \frac{10\blue{h} + 5}{2(h+1)} \cdot \blue{h}\cdot 3^{\blue{h}-1}
  >
  (\blue{h+1}) \cdot 3^{\blue{h}}
  \sisolosi
  10h^2 + 5h \mayor{\red{!}} 6(h+1)^2
$$
Esa última desigualdad es verdadera $ \paratodo h\en \naturales_{>2}$.
  {\small
    \parrafoDestacado[\atencion]{
      Dependiendo del grado de paranoia que uno tenga se va a poner a probar eso
      de todas las formas existentes, con análisis de función límites, derivadas, transformada de Fourier y análisis complejo.
      Pero, yo no iría mucho más lejos que decir que el coeficiente principal de uno es más grande que el otro y son las dos crecientes y listo.
      Dicho eso es verdad que el criterio final será del corrector, asi, que andá a preguntale a alguien y listo.
    }
  }

Se puede probar a mano que $p(n = 2)$ es veradera.

Dado que $p(1), p(2), p(h) \ytext p(h+1)$ resultaron ser verdaderas, por el principio de inducción también lo es $p(n)$ para todo $n \en \naturales$ como se
quería mostrar.

\begin{aportes}
  \item \aporte{\dirRepo}{naD GarRaz \github}
\end{aportes}
