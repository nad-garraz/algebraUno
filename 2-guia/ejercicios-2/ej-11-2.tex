\begin{enunciado}{\ejercicio}
  \begin{enumerate}[label=\roman*)]
    \item Sea $(a_n)_{n \en \naturales }$ una sucesión de números reales todos del mismo signo y tales que
          $a_n > -1$ para todo $n \en \naturales $. Probar que
          $$
            \productoria{i=1}{n}  (1 + a_i) \geq 1 + \sumatoria{i=1}{n} a_i
          $$
          ¿En qué paso de la demostración se usa que $a_n > -1$ para todo $n \en \naturales $? ¿Y que todos los términos
          de la sucesión $(a_n)_{n \en \naturales }$ tienen el mismo signo?

    \item Deducir que si $a \en \reales $ tal que $a > -1$, entonces $(1+a)^n \geq 1 + na$.
  \end{enumerate}
\end{enunciado}

\setcounter{equation}{0}

\begin{enumerate}[label=\roman*)]
  \item Sea $(a_n)_{n \en \naturales }$ sucesión de números reales tales que
        \begin{align}
           & sg(a_n) = sg(a_1), \, \paratodo n \en \naturales                                      \\
           & a_n > -1, \, \paratodo n \en \naturales  \iff 1 + a_n > 0, \, \paratodo n \en \naturales
        \end{align}
        donde $sg(x)$ es la función signo de $x$
        \begin{align*}
          sg(x) = \begin{cases}
                    -1, \text{ si } x < 0 \\
                    \phantom{-}1, \text{ si } x > 0
                  \end{cases}
        \end{align*}

        Definamos nuestra proposición $P(n)$
        \begin{align*}
          P(n): \productoria{i=1}{n}  (1 + a_i) \geq 1 + \sumatoria{i=1}{n} a_i, \, n \en \naturales
        \end{align*}

        \underline{Caso Base}, $n = 1$:
        \begin{align*}
          P(1) & : \productoria{i=1}{n}  (1 + a_i) \geq 1 + \sumatoria{i=1}{1} a_i \\
          P(1) & : 1 + a_1 \geq 1 + a_1 \entonces P(1):V
        \end{align*}

        \underline{Paso inductivo}. Sea $n \en \naturales $:
        \begin{enumerate}
          \item[HI.] $P(n): V$
          \item[TI.] $P(n+1): \displaystyle \productoria{i=1}{n+1}  (1 + a_i) \geq 1 + \sumatoria{i=1}{n+1} a_i$
        \end{enumerate}
        Desarrollemos el lado izquierdo de la desigualdad
        \begin{align*}
          \productoria{i=1}{n+1}  (1 + a_i) & = (1 + a_{n+1}) \productoria{i=1}{n}  (1 + a_i)
            \taa{\text{HI}}[(2)]{\geq}
          (1 + a_{n+1}) \left(1 + \sumatoria{i=1}{n} a_i\right) = 1 + \sumatoria{i=1}{n} a_i + a_{n+1} + a_{n+1}\sumatoria{i=1}{n} a_i
          \igual{*}                                                                                                                          \\
                                            & \igual{*} 1 + \sumatoria{i=1}{n+1} a_i + a_{n+1}\sumatoria{i=1}{n} a_i \mayorIgual{\text{Aux}}
          1 + \sumatoria{i=1}{n+1} a_i \entonces P(n+1):V
        \end{align*}
        Hemos probado el caso base y el paso inductivo, entonces $P(n): V, \, \paratodo n \en \naturales $.
        \paragraph{Auxiliar}{Queremos ver que $a_{n+1}\sumatoria{i=1}{n} a_i > 0$}. La Ec.(1) nos dice que $a_n$ nunca cambia de
        signo mientras que la Ec.(2) nos dice que $a_n$ puede ser positiva o negativa. Entonces
        \begin{align*}
          sg(a_n) < 0 \, \lor sg(a_n) > 0
        \end{align*}
        Caso $sg(a_n) < 0 $
        \begin{align}
          sg(a_n) < 0 & \Entonces{(1)} sg(a_{n+1}) < 0            \\
          sg(a_n) < 0 & \Entonces{(1)} \sumatoria{i=1}{n} a_i < 0
        \end{align}
        Por la Ec.(3) y la Ec.(4), tenemos que
        \begin{align*}
          a_{n+1}\sumatoria{i=1}{n} a_i > 0
        \end{align*}

        Caso $sg(a_n) > 0 $
        \begin{align}
          sg(a_n) > 0 & \Entonces{(1)} sg(a_{n+1}) > 0            \\
          sg(a_n) > 0 & \Entonces{(1)} \sumatoria{i=1}{n} a_i > 0
        \end{align}
        Por la Ec.(5) y la Ec.(6), tenemos que
        \begin{align*}
          a_{n+1}\sumatoria{i=1}{n} a_i > 0
        \end{align*}

        \paragraph{¿En qué paso de la demostración se usa que $a_n > -1$ para todo $n \en \naturales $? }{En la misma parte
          donde utilizamos la HI, pues la desigualdad de la HI sigue siendo valida solo si multiplicamos a ambos lados por un
          número positivo. Ese número positivo esta en la Ec.(2).}

        \paragraph{¿Y que todos los términos de la sucesión $(a_n)_{n \en \naturales }$ tienen el mismo signo?}{En la parte
          donde utilizamos el calculo auxiliar.}

  \item Como $a > -1$ y $a \en \reales $, cumple con las condiciones impuestas sobre $(a_n)_{n \en \naturales }$ en el
        punto (i). Tomamos $a_n = a$, $\paratodo n \en \naturales $. Como $P(n)$ es verdadera
        \begin{align*}
          \productoria{i=1}{n}  (1 + a) & \geq 1 + \sumatoria{i=1}{n} a   \\
          (1 + a)^n                     & \geq 1 + a \sumatoria{i=1}{n} 1 \\
          (1 + a)^n                     & \geq 1 + n a
        \end{align*}
\end{enumerate}

% Contribuciones
\begin{aportes}
  %% iconos : \github, \enstagram, \tiktok, \linkedin
  %\aporte{url}{nombre icono}
  \item \aporte{https://github.com/koopardo/}{Marcos Zea \github}
\end{aportes}
