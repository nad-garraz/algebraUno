\begin{enunciado}{\ejercicio}
  Probar que
  \begin{enumerate}[label=\roman*)]
    \item $n! \geq 3^{n-1}, \ \paratodo n \geq 5$
    \item $3^n - 2^n > n^3, \ \paratodo n \geq 4$
    \item $ \sumatoria{i=1}{n} \frac{3^i}{i!} < 6n - 5, \ \paratodo n \geq 3$
  \end{enumerate}
\end{enunciado}

\begin{enumerate}[label=\roman*)]
  \item $P(n): n! \geq 3^{n-1}, \ \paratodo n \geq 5$ \par
        \underline{Caso Base}, $n = 5$:
        \begin{align*}
           & P(5): 5! \geq 3^{5-1}              \\
           & P(5): 120 \geq 81 \entonces P(5):V
        \end{align*}
        \underline{Paso inductivo.} $Sea n \geq 5$:
        \begin{enumerate}
          \item[HI.] $P(n): V$
          \item[TI.] $P(n+1): (n+1)! \geq 3^n$
        \end{enumerate}

        Desarrollemos el lado izquierdo de la desigualdad en la TI
        \begin{align*}
          (n+1)! = (n+1)n! \mayorIgual{HI} (n+1)3^{n-1} \mayorIgual{Aux} 3 \cdot 3^{n-1} = 3^n
          \entonces P(n+1):V
        \end{align*}
        Hemos probado el caso base y el paso inductivo. Entonces $P(n):V,$ $\paratodo n \geq 5$.

        \paragraph{Auxiliar}{$n \geq 5 \entonces n \geq 3 \entonces (n+1) \geq 3$}

  \item $P(n): 3^n - 2^n > n^3, \ \paratodo n \geq 4$ \\
        \underline{Caso Base}, $n = 4$:
        \begin{align*}
           & P(4): 3^4 - 2^4 > 4^3          \\
           & P(4): 65 > 64 \entonces P(5):V
        \end{align*}
        \underline{Paso inductivo.} $Sea n \geq 4$:
        \begin{enumerate}
          \item[HI.] $P(n): V$
          \item[TI.] $P(n+1): 3^{n+1} - 2^{n+1} > (n+1)^3$
        \end{enumerate}

        Desarrollemos el lado izquierdo de la desigualdad en la TI
        \begin{align*}
          3^{n+1} - 2^{n+1} = 3 \cdot 3^n - 2 \cdot 2^n \geq 3 \cdot 3^n - 3 \cdot 2^n = 3(3^n - 2^n)
          \mayor{\purple{HI}}
          3n^3
          \mayor{Aux}
          (n+1)^3 \entonces P(n+1):V
        \end{align*}
        Hemos probado el caso base y el paso inductivo. Entonces $P(n):V,$ $\paratodo n \geq 4$.

        \paragraph{Auxiliar}{$3n^3 > (n+1)^3 \iff \sqrt[3]{(3n^3)} > \sqrt[3]{(n+1)^3} \Sii{Aux}
            \sqrt[3]{3}n > n+1 \Sii{*} \\
            \Sii{*} \sqrt[3]{3}n - n > 1 \sii n(\sqrt[3]{3}-1) > 1 \sii n > \frac{1}{\sqrt[3]{3}-1} \simeq 2.3
            \entonces 3n^3 > (n+1)^3, \ \paratodo n \geq 4$}

  \item
        Hay que proba la siguiente proposición:
        $$
          p(n): \sumatoria{i=1}{n} \frac{3^i}{i!} < 6n - 5, ~ \paratodo n \geq 3
        $$

        \textit{Caso Base:}
        $$
          p(\blue{3}):
          \sumatoria{i=1}{\blue{3}} \frac{3^i}{i!} < 6\cdot\blue{3} - 5
        $$
        Haciendo la cuenta de la sumatoria queda:
        $$
          3 + \frac{9}{2} + \frac{27}{6} = 12 < 6 \cdot 3 - 5 \sii 12 < 13.
        $$
        Es así que la proposición $p(\blue{1})$ resultó verdadera.

        \medskip

        \textit{Paso inductivo:}

        Asumo que para algún $\blue{k} \en \naturales$ la siguiente proposición:
        $$
          p(\blue{k}):
          \ub{
            \sumatoria{i=1}{\blue{k}} \frac{3^i}{i!} < 6\blue{k} - 5, ~ \paratodo \blue{k} \geq 3
          }{
            \text{\purple{hipótesis inductiva}}
          }
        $$
        es verdadera, entonces quiero probar que
        $$
          p(\blue{k + 1}): \sumatoria{i=1}{\blue{k + 1}} \frac{3^i}{i!} < 6(\blue{k + 1}) - 5,
        $$
        también lo sea.

        Usamos el \textit{truquito de la sumatoria} para que aparezca la \purple{hipótesis inductiva}:
        $$
          \begin{array}{rcl}
            \sumatoria{i=1}{\blue{k + 1}} \frac{3^i}{i!}
             & =                   &
            \frac{3^{k+1}}{(k+1)!} + \sumatoria{i=1}{k} \frac{3^i}{i!} \\
             & \menor{\purple{HI}} &
            \frac{3^{k+1}}{(k+1)!} + 6k - 5 ~ \llamada1                \\
          \end{array}
        $$
        En la última expresión queda algo que quizás te moleste. Tenés que acotar. Es decir
        \textit{asegurar} que esa expresión va a ser \underline{menor a $6(\blue{k+1}) - 5 ~\paratodo k>3$ }. Si mirás fuerte
        al primer término y pensás en lo rápido que crece el factorial: ¿Qué tan grande puede ser?
        $\llamada1$:
        $$
          \frac{3^{k+1}}{(k+1)!} + 6k - 5 \menor{\red{!!}} \magenta{6} + 6k - 5 = 6(k+1) - 5
        $$
        Por lo tanto $p(\blue{k+1})$ es verdadera también.

        En el \red{!!} podría haber puesto un 3 o 4 también {\footnotesize(¿Por qué?)}, pero elegí el \magenta{6} para que me quede ya la expresión final que buscaba
        al sacar el último factor común.

        Dado que $p(3), p(k) \ytext p(k+1)$ son verdaderas, por principio de inducción también lo es $p(n) \paratodo n \en \naturales_{\geq 3}$

\end{enumerate}

% Contribuciones
\begin{aportes}
  \item \aporte{https://github.com/koopardo/}{Marcos Zea \github}
  \item \aporte{\dirRepo}{naD GarRaz \github}
\end{aportes}
