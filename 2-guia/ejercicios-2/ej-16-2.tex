\begin{enunciado}{\ejercicio}
  Hallar una fórmula para el término general de las sucesiones $(a_n)_{n \en \naturales }$ definidas recursivamente a
  continuación y probar su validez.
  \begin{multicols}{2}
    \begin{enumerate}[label=\roman*)]
      \item $a_1 = 1  \ytext  a_{n+1} = (1 +\sqrt{a_n})^2, \paratodo n \en \naturales $
      \item $a_1 = 3  \ytext  a_{n+1} = 2a_n + 3^n, \paratodo n \en \naturales $
      \item $a_1 = 1  \ytext  a_{n+1} = na_n, \paratodo n \en \naturales $
      \item $a_1 = 2  \ytext  a_{n+1} = 2 - \displaystyle \frac{1}{a_n}, \paratodo n \en \naturales $
    \end{enumerate}
  \end{multicols}
\end{enunciado}

\begin{enumerate}[label=\roman*)]
  \item Sea $(a_n)_{n \en \naturales }$ definida por
        \setcounter{equation}{0}
        \begin{align}
          a_1     & = 1                                              \\
          a_{n+1} & = (1 + \sqrt{a_n})^2, \paratodo n \en \naturales
        \end{align}
        Veamos si hay algún patrón entre los términos de la sucesión para poder conjeturar una fórmula para su término
        n-ésimo
        \begin{align}
          a_2 & = (1 +\sqrt{a_1})^2 = (1 + \sqrt{1})^2 = 4 = 2^2 \nonumber  \\
          a_3 & = (1 +\sqrt{a_2})^2 = (1 + \sqrt{4})^2 = 9 = 3^2 \nonumber  \\
          a_4 & = (1 +\sqrt{a_3})^2 = (1 + \sqrt{9})^2 = 16 = 4^2 \nonumber \\
              & \vdotswithin{=} \nonumber                                   \\
          a_n & = n^2
        \end{align}
        Tenemos que probar que la sucesión dada por recurrencia satisface la sucesión que conjeturamos en la Ec.(3).
        Hagámoslo por inducción
        \begin{align*}
          P(n): a_n = n^2,  n \en \naturales
        \end{align*}
        \underline{Caso Base}, $n = 1$:
        \begin{align*}
           & P(1): a_1 = 1^2 = 1 \igual{1} 1 \entonces P(1):V
        \end{align*}
        \underline{Paso inductivo.} Sea $n \geq 1$:
        \begin{enumerate}
          \item[HI.] $P(n): V$
          \item[TI.] $P(n+1): a_{n+1} = (n+1)^2$
        \end{enumerate}
        Desarrollemos el lado izquierdo de la igualdad en la TI
        \begin{align*}
          a_{n+1} \igual{(2)} (1 + \sqrt{a_n})^2 \igual{HI} (1 + \sqrt{n^2})^2 = (1 + n)^2
          \entonces P(n+1):V
        \end{align*}
        Hemos probado el caso base y el paso inductivo. Concluimos que $P(n):V,$ $\paratodo n \en \naturales $.

  \item Sea $(a_n)_{n \en \naturales }$ definida por
        \setcounter{equation}{0}
        \begin{align}
          a_1     & = 3                                       \\
          a_{n+1} & = 2 a_n + 3^n, \paratodo n \en \naturales
        \end{align}
        Veamos si hay algún patrón entre los términos de la sucesión para poder conjeturar una formula para su termino
        n-ésimo
        \begin{align*}
          a_2 & = 2 a_1 + 3^1 = 2 \cdot 3 + 3 = 9 = 3^2         \\
          a_3 & = 2 a_2 + 3^2 = 2 \cdot 9 + 3^2 = 9  = 27 = 3^3 \\
          a_4 & = 2 a_3 + 3^2 = 2 \cdot 27 + 3^3  = 81 = 3^4    \\
              & \vdotswithin{=}                                 \\
          a_n & = 3^n
        \end{align*}
        Tenemos que probar que la sucesión dada por recurrencia satisface la sucesión que conjeturamos en la Ec.(3).
        Hagámoslo por inducción
        \begin{align*}
          P(n): a_n = 3^n,  n \en \naturales
        \end{align*}

        \underline{Caso Base}, $n = 1$:
        \begin{align*}
           & P(1): a_1 = 3^1 = 3 \igual{(1)} 3 \entonces P(1):V
        \end{align*}
        \underline{Paso inductivo.} Sea $n \geq 1$:
        \begin{enumerate}
          \item[HI.] $P(n): V$
          \item[TI.] $P(n+1): a_{n+1} = 3^{n+1}$
        \end{enumerate}
        Desarrollemos el lado izquierdo de la igualdad en la TI
        \begin{align*}
          a_{n+1} \igual{(2)} 2 a_n + 3^n \igual{HI} 2 \cdot 3^n + 3^n = 3 \cdot 3^n = 3^{n+1}
          \entonces P(n+1):V
        \end{align*}
        Hemos probado el caso base y el paso inductivo. Concluimos que $P(n):V,$ $\paratodo n \en \naturales $.

  \item Sea $(a_n)_{n \en \naturales }$ definida por
        $$
          \begin{array}{rcl}
            a_1     & \igual{$\llamada1$} & 1                                 \\
            a_{n+1} & \igual{$\llamada2$} & n a_n, \paratodo n \en \naturales
          \end{array}
        $$

        Veamos si hay algún patrón entre los términos de la sucesión para poder conjeturar una formula de su termino
        n-ésimo

        $$
          \begin{array}{rcl}
            a_2 & =                   & 1 \cdot a_1 = 1 \cdot 1                              \\
            a_3 & =                   & 2 \cdot a_2 = 2 \cdot 1                              \\
            a_4 & =                   & 3 \cdot a_3 = 3 \cdot 2 = 3 \cdot 2 \cdot 1          \\
            a_5 & =                   & 4 \cdot a_4 = 4 \cdot 6 = 4 \cdot 3 \cdot 2  \cdot 1 \\
                & \vdots              &                                                      \\
            a_n & \igual{$\llamada3$} & (n-1)!
          \end{array}
        $$
        Tenemos que probar que la sucesión dada por recurrencia satisface la sucesión que conjeturamos en $\llamada3$.

        $$
          p(n): a_n = (n - 1)! \paratodo n \en \naturales
        $$

        \textit{Caso Base:}
        $$
          p(\blue{1}): a_1 = (\blue{1} - 1)! = 0! = 1 \igual{$\llamada1$} 1
        $$
        por lo que $p(1)$ es verdadera.

        \medskip
        \textit{Paso inductivo:}

        Asumo que para algún $\blue{k} \en \naturales$
        $$
          p(\blue{k})  : \ub{a_{\blue{k}} = (\blue{k} - 1)!}{\text{\purple{hipótesis inductiva}}}
        $$
        es verdadera. Por lo tanto quiero probar que:
        $$
          p(\blue{k+1})  : a_{\blue{k + 1}} = (\blue{k + 1} - 1)!
        $$
        también lo es.

        Está regalada: {\tiny \color{black!10!white}\textit{¡Como tu hermana!}}
        $$
          a_{\blue{k + 1}} = (\blue{k + 1} - 1)! = k!
          \igual{\red{!}}
          k \cdot (k - 1)!
          \igual{\purple{HI}}[$\llamada2$]
          k \cdot a_{\blue{k}}
          \entonces a_{k+1} = k \cdot a_{\blue{k}}
        $$
        Ese último paso es la definición de la sucesión, así que dicho formal y matemáticamente
        es más verdadero que \textit{que el agua moja}.

        Como $p(1),\, p(k) \ytext p(k+1)$ resultaron verdaderas, por el principio de inducción $p(n)$ también lo es $\paratodo n \en \naturales$

  \item Sea $(a_n)_{n \en \naturales }$ definida por
        \setcounter{equation}{0}
        \begin{align}
          a_1     & = 2                                             \\
          a_{n+1} & = 2 - \frac{1}{a_n}, \paratodo n \en \naturales
        \end{align}
        Veamos si hay algún patrón entre los términos de la sucesión para poder conjeturar una formula de su termino
        n-ésimo
        \begin{align}
          a_2 & = 2 - \frac{1}{a_1} = 2 - \frac{1}{2} = \frac{3}{2} \nonumber   \\
          a_3 & = 2 - \frac{1}{a_2} = 2 - \frac{1}{3/2} = \frac{4}{3} \nonumber \\
          a_4 & = 2 - \frac{1}{a_3} = 2 - \frac{1}{4/3} = \frac{5}{4} \nonumber \\
          a_5 & = 2 - \frac{1}{a_4} = 2 - \frac{1}{5/4} = \frac{6}{5} \nonumber \\
              & \vdotswithin{=} \nonumber                                       \\
          a_n & = \frac{n+1}{n}
        \end{align}
        Tenemos que probar que la sucesión dada por recurrencia satisface la sucesión que conjeturamos en la Ec.(3).
        Hagamoslo por inducción
        \begin{align*}
          P(n): a_n = \frac{n+1}{n}, \paratodo n \en \naturales
        \end{align*}
        \underline{Caso Base}, $n = 1$:
        \begin{align*}
           & P(1): a_1 = \frac{1+1}{1} = 2 \igual{(1)} 2 \entonces P(1):V
        \end{align*}
        \underline{Paso inductivo}. Sea $n \geq 1$:
        \begin{enumerate}
          \item[HI.] $P(n): V$
          \item[TI.] $P(n+1): a_{n+1} = \displaystyle \frac{n+2}{n+1}$
        \end{enumerate}
        Desarrollemos el lado izquierdo de la igualdad en la TI
        \begin{align*}
          a_{n+1} \igual{(2)} 2 - \frac{1}{a_n} \igual{HI} 2 - \frac{1}{(n+1)/n} = 2 - \frac{n}{n+1} =
          \frac{2(n+1) - n}{n+1} = \frac{n+2}{n+1}
          \entonces P(n+1):V
        \end{align*}
        Hemos probado el caso base y el paso inductivo. Concluimos que $P(n):V,$ $\paratodo n \en \naturales $.
\end{enumerate}

% Contribuciones
\begin{aportes}[3]
  \item \aporte{https://github.com/koopardo/}{Marcos Zea \github}
  \item \aporte{\dirRepo}{nad GarRaz \github}
  \item \aporte{https://github.com/fransureda}{Francisco Sureda \github}
\end{aportes}
