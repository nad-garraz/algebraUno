\begin{enunciado}{\ejercicio}
  \begin{enumerate}[label=\roman*)]
    \item Sea $(a_n)_{n \en \naturales }$ la sucesión definida por
          \begin{align*}
            a_1 = 1,\, a_2 = 3 \ytext a_{n+2} = a_{n+1} + 5a_n \, \, \, (n \en \naturales )
          \end{align*}
          Probar que $a_n < 1 + 3^{n-1}$ para todo $n \en \naturales $.

    \item Sea $(a_n)_{n \en \naturales }$ la sucesión definida por
          \begin{align*}
            a_1 = 1,\, a_2 = \frac{3}{2} \ytext a_{n+2} = a_{n+1} + \frac{2n+1}{n+2}a_n \, \, \, (n \en \naturales )
          \end{align*}
          Probar que $a_n > n + \displaystyle \frac{1}{3}$ para todo $n \en \naturales $, $n \geq 4$.
  \end{enumerate}
\end{enunciado}

\begin{enumerate}[label=\roman*)]
  \item Sea $(a_n)_{n \en \naturales }$ definida por
        \setcounter{equation}{0}
        \begin{align}
          a_1     & = 1                                            \\
          a_2     & = 3                                            \\
          a_{n+2} & = a_{n+1} + 5a_n,\, \paratodo n \en \naturales
        \end{align}
        Probemos la desigualdad que nos pide el ejercicio por inducción
        \begin{align*}
          P(n): a_n < 1 + 3^{n-1},\, n \en \naturales
        \end{align*}
        \underline{Caso Base}, $n = 1$ y $n = 2$:
        \begin{align*}
           & P(1): a_1 < 1 + 3^{1-1} \Entonces{(1)} P(1): 1 < 2 \entonces P(1):V \\
           & P(2): a_2 < 1 + 3^{2-1} \Entonces{(2)} P(2): 3 < 4 \entonces P(2):V
        \end{align*}
        \underline{Paso inductivo}. Sea $n \en \naturales $:
        \begin{enumerate}
          \item[HI.] $P(n):V \ytext P(n+1):V$, donde $P(n+1): a_{n+1} < 1 + 3^n$
          \item[TI.] $P(n+2): a_{n+2} < 1 + 3^{n+1}$
        \end{enumerate}
        Desarrollemos el lado izquierdo de la desigualdad en la TI
        \begin{align*}
           & a_{n+2} \igual{(3)} a_{n+1} + 5a_n \overset{\text{HI}}{<} 1 + 3^n + 5(1 + 3^{n-1}) =
          1 + 3^n + 5 + 5 \cdot 3^{n-1} \overset{\text{Aux}}{<} 1 + 3^n + 3^{n-1} + 5 \cdot 3^{n-1} \igual{*} \\
           & \igual{*} 1 + 3^n + 6 \cdot 3^{n-1} = 1 + 3^n + 2 \cdot 3 \cdot 3^{n-1}
          = 1 + 3^n + 2 \cdot 3^n = 1 + 3 \cdot 3^n = 1 + 3^{n+1},\, n \geq 3                                 \\
           & \entonces a_{n+2} < 1 + 3^{n+1},\, n \geq 3 \entonces P(n+2):V,\, n \geq 3
        \end{align*}
        Como solo probamos el paso inductivo para $n \geq 3$, deberiamos ver que $P(n)$ y $P(n+1)$ son verdaderas para $n = 3$. O sea,
        queremos ver que $P(3)$ y $P(4)$ son verdaderas.
        \begin{align*}
           & P(3): a_3 < 1 + 3^{3-1} \Entonces{(3)} P(3): 8 < 9 \entonces P(3):V   \\
           & P(4): a_4 < 1 + 3^{4-1} \Entonces{(3)} P(3): 23 < 27 \entonces P(4):V
        \end{align*}
        Probamos
        \begin{itemize}
          \item $P(1):V,\, P(2):V,\, P(3):V,\, P(4):V$
          \item $P(n):V \land P(n+1):V \entonces P(n+2):V, \paratodo n \geq 3$
        \end{itemize}
        Podemos concluir que $P(n):V,$ $\paratodo n \en \naturales $.

        \paragraph{Auxiliar}{Sea $n \en \naturales $
          \begin{align*}
            5 < 3^{n-1} & \Sii{\ } \log_3(5) < \log_3(3^{n-1}) \sii \log_3(5) < (n-1)\log_3(3) \sii \log_3(5) < n-1 \\
                        & \Sii{*}  n > 1 + \log_3(5) \simeq 2.5 \entonces 5 < 3^{n-1},\, \paratodo n \geq 3
          \end{align*}
        }

  \item Sea $(a_n)_{n \en \naturales }$ definida por
        \setcounter{equation}{0}
        \begin{align}
          a_1     & = 1                                                           \\
          a_2     & = \frac{3}{2}                                                 \\
          a_{n+2} & = a_{n+1} + \frac{2n+1}{n+2}a_n,\, \paratodo n \en \naturales
        \end{align}
        Probemos la desigualdad que nos pide el ejercicio por inducción
        \begin{align*}
          P(n): a_n > n + \frac{1}{3}, \, n \en \naturales _{\geq 4}
        \end{align*}
        \underline{Caso Base}, $n = 4$ y $n = 5$:
        \begin{align*}
           & P(4): a_4 < 4 + \frac{1}{3} \Entonces{(3)} P(4): \frac{35}{8} < \frac{13}{3} \entonces
          3 \cdot 35 > 8 \cdot 13 \entonces 105 > 104 \entonces P(4):V                              \\
           & P(5): a_5 < 5 + \frac{1}{3} \Entonces{(3)} P(5): \frac{63}{8} < \frac{16}{3} \entonces
          3 \cdot 63 > 8 \cdot 16 \entonces 189 > 128 \entonces P(5):V
        \end{align*}
        \underline{Paso inductivo}. Sea $n \en \naturales _{\geq 4}$:

        \begin{enumerate}
          \item[HI.] $P(n):V \ytext P(n+1):V$, donde $P(n+1): a_{n+1} > n + 1 + \displaystyle \frac{1}{3}$
          \item[TI.] $P(n+2): a_{n+2} > n + 2 + \displaystyle \frac{1}{3}$
        \end{enumerate}
        Desarrollemos el lado izquierdo de la desigualdad en la TI
        \begin{align*}
          a_{n+2} & \igual{(3)} a_{n+1} + \frac{2n+1}{n+2}a_n \overset{\text{HI}}{>}
          n + 1 + \frac{1}{3} + \frac{2n+1}{n+2}\left(n + \frac{1}{3}\right)
          = n + 1 + \frac{1}{3} + \frac{2n+1}{n+2}\left(n + \frac{1}{3}\right) + 1 - 1                                                          \\
                  & \igual{*} n + 2 + \frac{1}{3} + \frac{2n+1}{n+2}\left(n + \frac{1}{3}\right) - 1
          \overset{\text{Aux.1}}{\geq} n + 2 + \frac{1}{3} + 1 \cdot \left(n + \frac{1}{3}\right) - 1
          = n + 2 + \frac{1}{3} + n + \frac{1}{3} - 1                                                                                           \\
                  & \igual{**} n + 2 + \frac{1}{3} + \magenta{n - \frac{2}{3}} \ \mayorIgual{\magenta{Aux.2}} n + 2 + \frac{1}{3} + \magenta{0} \\
                  & \entonces a_{n+2} > n + 2 + \frac{1}{3} \entonces P(n+2):V
        \end{align*}
        Hemos probado el caso base y el paso inductivo. Podemos concluir que $P(n):V,$ $\paratodo n \en \naturales _{\geq 4}$.

        \paragraph{Auxiliar 1}{Sea $n \en \naturales $
          \begin{align*}
             & 1 \leq n \sii 1 + 1 \leq n + 1 \sii 2 \leq n + 1 \sii n + 2 \leq n + n + 1 \sii n + 2 \leq 2n + 1 \\
             & \sii 1 \leq \frac{2n+1}{n+2}
          \end{align*}
        }

        \paragraph{Auxiliar 2}{Sea $n \en \naturales $
          \begin{align*}
            n \geq  \frac{2}{3}
            \Sii{\red{!}}
            n - \frac{2}{3} \geq \magenta{0}
          \end{align*}
        }

\end{enumerate}

\begin{aportes}
  \item \aporte{https://github.com/koopardo}{Marcos Zea \github}
  \item \aporte{\dirRepo}{naD GarRaz \github}
\end{aportes}
