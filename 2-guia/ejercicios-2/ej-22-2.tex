\begin{enunciado}{\ejercicio}
  Probar que todo número natural $n$ se escribe como suma de distintas potencias de 2, incluyendo $2^0 = 1$.
\end{enunciado}
Este ejercicio es probar que que un número puede escribirse en \textit{binario}, onda:
$$
  27 = (\blue{11011})_2 = \blue{1} \cdot 2^4 + \blue{1} \cdot 2^3 + \blue{0} \cdot 2^2 + \blue{1} \cdot 2^1 + \blue{1} \cdot 2^0
$$
Probemos que todo número natural $n$ se escribe como suma de distintas potencias de 2 usando inducción:
$$
  p(n): n = \sumatoria{i = 1}{r}
  2^{\alpha_i},\,  \text{donde } \alpha_i \en \naturales_0 \text{ y si } \alpha_i \distinto 0 \entonces \alpha_i \distinto \alpha_j  \text{ si } i \distinto j
$$

\textit{Caso Base:}
$$
  p(\blue{1}): \blue{1} = 2^0
$$
por lo que $p(1)$ resulta verdadera

\textit{Paso inductivo}:

Asumo que para un $\blue{k} \en \naturales$ la proposición:
$$
  p(\blue{k}):
  \ub{
    \blue{k} = \sumatoria{i = 1}{r}
    2^{\alpha_i}
    ,\,  \text{donde } \blue{k} \ytext \alpha_i \en \naturales \ytext \alpha_i \distinto \alpha_j  \text{ si } i \distinto j
  }{
    \text{\purple{hipótesis inductiva}}
  }
$$
es verdadera. Entonces quiero probar que:
$$
  p(\blue{k+1}):
  \blue{k+1} =
  \sumatoria{i = 1}{\magenta{s}}
  2^{\alpha_i}
  ,\,  \text{donde } \alpha_i \en \naturales \ytext \alpha_i \distinto \alpha_j  \text{ si } i \distinto j
  \text{ con } \blue{k} \ytext \alpha_i \en \naturales
$$
también lo sea.

La hipótesis inductiva para un $\blue{k}$:
$$
  \begin{array}{rcl}
    \blue{k} =
    \sumatoria{i = 1}{\red{r}} 2^{\alpha_i}
     & = &
    2^0 + 2^1 + 2^2 +
    \cdots +
    2^{h-1} + 2^{h+1} +
    \cdots +
    2^{r-1} + 2^r \\
  \end{array}
$$
Sumándole 1:
$$
  \begin{array}{rcl}
    \blue{k + 1} =
    \blue{1} +
    \sumatoria{i = 1}{\red{r}} 2^{\alpha_i}
     & =               &
    \blue{1} +
    \ub{
      2^0 + 2^1 + 2^2 +
      \cdots +
      2^{\magenta{h-2}}+
      2^{\magenta{h-1}}
    }{
    \text{potencias con} \\ \text{exponentes consecutivos}
    }
    + 2^{\magenta{h+1}} +
    \cdots +
    2^{r-1} + 2^r        \\
     & \igual{\red{!}} &
    2^{\magenta{h}} + 2^{\magenta{h+1}} +
    \cdots +
    2^{r-1} + 2^r        \\
  \end{array}
$$
$p(\blue{k+1})$ resultó verdadera. Si la cuenta de recién quedó oscura, lo que pasó fue algo así como
$2^\alpha + 2^\alpha = 2^{\alpha+1}$, ejemplo:
$$
  \ub{1 + 2^0}{=2} + 2^1 + 2^2 + 2^3
  =
  \ub{2 + 2^1}{=2^2} + 2^2 + 2^3
  =
  \ub{2^2 + 2^2}{=2^3} + 2^3
  =
  \ub{2^3 + 2^3}{=2^4}
  = 2^4
$$
En cierta forma al sumar "se hace un efecto dominó", en el cual las potencias de 2, con exponente no nulos, consecutivas
\textit{desaparecen} formando la potencia con el siguiente exponente.
El efecto termina con la \textit{aparición} cuando la sucesión de exponentes consecutivos termina.

\parrafoDestacado{
  Este \textit{efecto} es el acarreo de toda la vida de la suma, el acarreo es ese \texit{me llevo uno}
  cuando sumás.
}

\bigskip

Finalmente $p(1), p(k) \ytext p(k+1)$ resultaron verdaderas y por el principio de inducción
$p(n)$ es verdadera $\paratodo n \en \naturales$

\begin{aportes}
  \item \aporte{https://github.com/koopardo/}{Marcos Zea \github}
  \item \aporte{\dirRepo}{naD GarRaz \github}
\end{aportes}
