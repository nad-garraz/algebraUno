\begin{enunciado}{\ejercicio}
  Calcular
  \begin{multicols}{2}
    \begin{enumerate}[label=\roman*)]
      \item $ \sumatoria{i=1}{n} (4i + 1) $
      \item $\sumatoria{i=6}{n} 2(i-5)$
    \end{enumerate}
  \end{multicols}
\end{enunciado}

Para resolver estos ejercicios haremos uso la notas teóricas, en particular
\ref{2-teoria:suma-prod} y \ref{2-teoria:suma-gauss}.

\begin{enumerate}[label=\roman*)]
  \item
        $ \sumatoria{i=1}{n} (4i + 1)
          = (\sumatoria{i=1}{n} 4i) + (\sumatoria{i=1}{n} 1)
          = (4 \cdot \sumatoria{i=1}{n} i) + n
          = 4 \cdot \frac{n(n+1)}{2} + n
          = \boxed{2n^2 + 3n}
        $\par

  \item
        $ \sumatoria{i=6}{n} 2(i-5)
          = 2 \cdot \sumatoria{i=6}{n} (i - 5)
          = 2 \cdot [(\sumatoria{i=6}{n} i) - (\sumatoria{i=6}{n} 5)]
          = 2 \cdot [(\sumatoria{i=0}{n} i) - (\sumatoria{i=0}{5} i) - 5(n-5)]\\
          = 2 \cdot (\frac{n(n+1)}{2} - \frac{5(5+1)}{2} - 5n + 25)
          = 2 \cdot (\frac{n(n+1)}{2} - 5n + 10)
          = n(n+1) - 10n + 20
          = \boxed{n^2 - 9n + 20}
        $
\end{enumerate}
