\begin{enunciado}{\ejercicio}
  Calcular
  \begin{multicols}{2}
    \begin{enumerate}[label=\roman*)]
      \item $\sumatoria{i=0}{n} 2^i$
      \item $\sumatoria{i=1}{n} q^i \quad q \en \reales$
      \item $\sumatoria{i=0}{n} q^{2i}, \quad q \en \reales - \set{0}$
      \item $\sumatoria{i=1}{2n} q^i \quad q \en \reales$
    \end{enumerate}
  \end{multicols}
\end{enunciado}

\begin{enumerate}[label=\roman*)]
  \item $\sumatoria{i=0}{n} 2^i
          \igual{}[$q \distinto 1$]
          \frac{\red{2}^{n+1} - 1}{\red{2} - 1} =
          2^{n+1} - 1 $

  \item $\sumatoria{i=1}{n} q^i =
          \red{-1 + 1} + \sumatoria{i=1}{n} q^i =
          \red{-1} + \sumatoria{i = \red{0}}{n} q^i =
          \llave{lll}{
            n + 1 \red{-1} = n                                                                                                       & \text{si} & q = 1         \\
            \frac{q^{n+1}-1}{q-1} \red{-1} = \ub{\frac{q^{n+1} - q}{q-1}}{\llamada1 \sumatoria{k = \red{1}}{n} q^i} & \text{si} & q \distinto 1
          }$

  \item $\sumatoria{i=0}{n} q^{2i} \igual{\red{!}}
          1 + \sumatoria{i=\magenta{1}}{n} q^{2i} =
          \llave{lll}{
            \scriptstyle
            1 + \underbrace{\scriptstyle q^2 + q^4 + \cdots + (q^{n-1})^2 + q^{2n}}_{\text{n elementos}} = n + 1 & \text{ si } & q = \pm 1         \\
            \scriptstyle
            1 + \underbrace{\scriptstyle(q^2)^1 + (q^2)^2 + \cdots + (q^2)^{n-1} + (q^2)^n}_{\text{n elementos}} \igual{$\scriptscriptstyle\llamada1$}
            \frac{(q^2)^{n+1} - q^2}{q^2 - 1} + 1 =
            \frac{q^{2(n+1)} - 1}{q^2 - 1}                                                                                                     & \text{ si } & q \distinto \pm 1
          }
        $
  \item $
          \sumatoria{i=1}{2n} q^i \igual{?}
          \llave{l}{
            2n \text{ si } q = 1 \\
            \frac{q^{2n-1} - q}{q-1} \text{ si } q \distinto 1
          }
        $
\end{enumerate}

% Contribuciones
\begin{aportes}
  %% iconos : \github, \instagram, \tiktok, \linkedin
  %\aporte{url}{nombre icono}
  \item \aporte{https://github.com/nad-garraz}{Nad Garraz \github}
\end{aportes}
