\begin{enunciado}{\ejercicio}
  Calcular
  \begin{multicols}{2}
    \begin{enumerate}[label=\roman*)]
      \item $\sumatoria{i=0}{n} 2^i$
      \item $\sumatoria{i=1}{n} q^i \quad q \en \reales$
      \item $\sumatoria{i=0}{n} q^{2i}, \quad q \en \reales - \set{0}$
      \item $\sumatoria{i=1}{2n} q^i \quad q \en \reales$
    \end{enumerate}
  \end{multicols}
\end{enunciado}

Acá \hyperlink{2-teoria:geometrica}{la fórmula de la serie geométrica click, click {\tiny \faIcon{mouse}}}

Pequeña demostración que nadie pidió y probablemente nadie necesite de una forma de llegar a la formulita de la serie geométrica, si:
$\sumatoria{i=0}{n} a \cdot q^i = S_{\ub{\scriptscriptstyle n+1}{\tiny\text{total de términos}}}$
$$
  \begin{array}{crcl}
            & S_{n+1}             & = & \blue{a} \cdot (1 + q + q^2 + \cdots + q^n)       \\
    \red{-} &                     &   &                                                   \\
            & q \cdot S_{n+1}     & = & \blue{a} \cdot (q + q^2 + \cdots + q^n + q^{n+1}) \\ \hline
            & (1\red{-} q)S_{n+1} & = & \blue{a} \cdot (1 - q^{n+1})                      \\
  \end{array}
  \Sii{$q\distinto 0$}
  \cajaResultado{
    S_{n+1} = \blue{a} \cdot \frac{1 - q^{n+1}}{1 - q} = \blue{a} \cdot \frac{q^{n+1} - 1}{q - 1}
  }
$$
\begin{enumerate}[label=\roman*)]
  \item  \textit{Formulini geométrini}:
        $$
          \sumatoria{i=0}{n} 2^i
          \igual{}[$q \distinto 1$]
          \frac{\red{2}^{n+1} - 1}{\red{2} - 1} =
          2^{n+1} - 1
        $$

  \item $$
          \sumatoria{i=1}{n} q^i =
          \red{-1 + 1} + \sumatoria{i=1}{n} q^i =
          \red{-1} + \sumatoria{i = \red{0}}{n} q^i =
          \llave{lll}{
            n + 1 \red{-1} = n                                                                                                       & \text{si} & q = 1         \\
            \frac{q^{n+1}-1}{q-1} \red{-1} = \frac{q^{n+1} - q}{q-1} & \text{si} & q \distinto 1
          }
        $$

  \item
        {\small
        $$
          \sumatoria{i=0}{n} q^{2i} \igual{\red{!}}
          1 + \sumatoria{i=\magenta{1}}{n} q^{2i} =
          \llave{lllcc}{
            1 + \ub{q^2 + q^4 + \cdots + (q^{n-1})^2 + q^{2n}}{n \text{ elementos}} &=& n + 1 & \text{ si } & q = \pm 1         \\
            1 + \ub{(q^2)^1 + (q^2)^2 + \cdots + (q^2)^{n-1} + (q^2)^n}{n \text{ elementos}}
            &\igual{\red{!}}&
            \frac{(q^2)^{n+1} - q^2}{q^2 - 1} + 1 =
            \frac{q^{2(n+1)} - 1}{q^2 - 1}                                                                                                     & \text{ si } & q \distinto \pm 1
          }
        $$
        }

  \item
        $$
          \sumatoria{i=n}{2n} q^i
          \igual{\red{!!}}
          \llave{rclcc}{
            1^n \cdot (1 + 1^1 + 1^2 \cdots + 1^{n-1} + 1^n)  &=& n + 1 &\text{ si }& q = 1 \\
            q^n \cdot (1+q+q^2+\cdots+q^n) &=& q^n \cdot \frac{q^{n+1} - 1}{q-1} &\text{ si } & q \distinto 1
          }
        $$
        Si no caes en el \red{!!}, escribí todo y después sacá factor común. La idea es que queda más fácil para contar la cantidad de términos que hay.
\end{enumerate}

% Contribuciones
\begin{aportes}
  \item \aporte{\dirRepo}{naD GarRaz \github}
  \item \aporte{https://github.com/JowinTeran}{Ale Teran \github}
\end{aportes}
