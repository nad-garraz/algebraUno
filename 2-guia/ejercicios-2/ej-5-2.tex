\begin{enunciado}{\ejercicio}
  Probar que, $\paratodo n \en \naturales,\, \sumatoria{i = 1}{n} (2i - 1) = n^2$:

  \begin{enumerate}[label=\roman*)]
    \item Contando de dos maneras la cantidad total de cuadraditos del diagrama.
          $$
            \begin{tikzpicture}
              [
                b/.style={draw=gray!50!white, fill=black},
                g/.style={draw=gray!50!white, fill=gray!60},
                scale = 0.4
              ]

              \matrix[
              matrix of nodes,
              rounded corners=1pt,
              row sep=1pt,
              column sep=1pt,
              nodes in empty cells] {
              |[g]| & |[g]| & |[g]| & |[g]| & |[g]| & |[g]| & |[g]| \\
              |[b]| & |[b]| & |[b]| & |[b]| & |[b]| & |[b]| & |[g]| \\
              |[g]| & |[g]| & |[g]| & |[g]| & |[g]| & |[b]| & |[g]| \\
              |[b]| & |[b]| & |[b]| & |[b]| & |[g]| & |[b]| & |[g]| \\
              |[g]| & |[g]| & |[g]| & |[b]| & |[g]| & |[b]| & |[g]| \\
              |[b]| & |[b]| & |[g]| & |[b]| & |[g]| & |[b]| & |[g]| \\
              |[g]| & |[b]| & |[g]| & |[b]| & |[g]| & |[b]| & |[g]| \\
              };
            \end{tikzpicture}
          $$

    \item Usando la suma aritmética (o suma de Gauss).

    \item Usando el principio de inducción.
  \end{enumerate}
\end{enunciado}

\begin{enumerate}[label=\roman*)]
  \item La forma fácil de contar sería multiplicando la cantidad de filas por la cantidad de columnas:
        $$
          7 \cdot 7 = 49
        $$
        También puedo sumar los cuadradadito construyendo la matriz, sumando 2 nuevas filas menos 1 \magenta{cuadradito} que me sobraría:
        $$
          \begin{tikzpicture}
            [
              b/.style={draw=gray!50!white, fill=black},
              g/.style={draw=gray!50!white, fill=gray!60},
              m/.style={draw=gray!50!white, fill=magenta!60},
              c/.style={draw=gray!50!white, fill=Cerulean!60},
              flecha/.style={-Latex, bend left = 50pt, draw = Cerulean},
              every node/.style={font={\tiny}},
              node distance = 2cm,
              matriz/.style={
                  matrix of nodes,
                  rounded corners=1pt,
                  row sep=1pt,
                  column sep=1pt,
                  nodes in empty cells
                },
            ]
            \matrix[
            matriz,
            ] (1) {
            |[g]|   \\
            };

            \matrix[
            matriz,
            right of = 1,
            ] (2) {
            |[c]| & |[c]|  \\
            |[g]| & |[c]|  \\
            };

            \matrix[
            matriz,
            right of = 2,
            ] (3) {
            |[c]| & |[c]| & |[c]|    \\
            |[b]| & |[b]| & |[c]|    \\
            |[g]| & |[b]| & |[c]|    \\
            };

            \matrix[
            matriz,
            right of = 3,
            ] (4) {
            |[c]| & |[c]| & |[c]| & |[c]|   \\
            |[g]| & |[g]| & |[g]| & |[c]|   \\
            |[b]| & |[b]| & |[g]| & |[c]|   \\
            |[g]| & |[b]| & |[g]| & |[c]|   \\
            };

            \matrix[
            matriz,
            right of = 4,
            ]
            (5)  {
            |[c]| & |[c]| & |[c]| & |[c]| & |[c]|  \\
            |[b]| & |[b]| & |[b]| & |[b]| & |[c]|  \\
            |[g]| & |[g]| & |[g]| & |[b]| & |[c]|  \\
            |[b]| & |[b]| & |[g]| & |[b]| & |[c]|  \\
            |[g]| & |[b]| & |[g]| & |[b]| & |[c]|  \\
            };

            \node[right of = 5] (dots) {$\dots$};

            \node[below of = 1, yshift = 28pt] (1l) {$n = 1$};
            \node[below of = 2, yshift = 28pt] (2l) {$n = 2$};
            \node[below of = 3, yshift = 28pt] (3l) {$n = 3$};
            \node[below of = 4, yshift = 28pt] (4l) {$n = 4$};
            \node[below of = 5, yshift = 28pt] (5l) {$n = 5$};

            \draw[flecha] (1.north) to node[midway, above]{$1+\blue{3}$}  (2.north);
            \draw[flecha] (2.north) to node[midway, above]{$4 + \blue{5}$}  (3.north);
            \draw[flecha] (3.north) to node[midway, above]{$9 + \blue{7}$}  (4.north);
            \draw[flecha] (4.north) to node[midway, above]{$16 + \blue{9}$}  (5.north);
            \draw[flecha] (5.north) to node[midway, above]{$25 + \blue{11}$}  (dots.north);

          \end{tikzpicture}
        $$

        La verdad que no sé si esto era la idea, pero bueh, así estaría dando la fórmula.

  \item Gauss
        $$
          \frac{n \cdot (n+1)}{2} = \sumatoria{1}{n} i
        $$
        Partiendo de la fórmula del enunciado:
        $$
          \sumatoria{i = 1}{n} 2i - 1 =
          2 \sumatoria{i = 1}{n} i - \sumatoria{1}{n} 1 \igual{\red{!}}
          2 \frac{n \cdot (n + 1)}{2} - n =
          n^2 + n - n = n^2
        $$

  \item Quiero probar que la siguiente proposición es verdadera:
        $$
          p(n):  \sumatoria{i = 1}{n} (2i - 1) = n^2 \paratodo n \en \naturales
        $$
        \textit{Caso base : }
        $$
          p(1) : \sumatoria{i = 1}{1} 2i -1 = 1 = 1^2  \entonces p(1)
        $$
        $p(1)$ resutló verdadera.

        \textit{Paso inductivo: }

        Para algún $h \en \naturales$ asumo que
        $$
          p(h) : \ub{
            \sumatoria{i = 1}{h} 2i -1 = h^2
          }{
            \text{
              \purple{hipótesis inductiva}
            }
          }
        $$
        es verdadera, entonce quiero probar que la proposición:
        $$
          p(h + 1) : \sumatoria{i = 1}{h + 1} 2i -1 = (h+1)^2
        $$
        también lo sea.

        Acomodo la sumatoria y uso la \purple{hipótesis inductiva}:
        $$
          \begin{array}{rcl}
            \sumatoria{i=1}{h+1} 2i -1 = (h+1)^2 & \sii & \sumatoria{i=1}{h+1} 2i -1 = (h + 1)^2                                                        \\
                                                 & \sii & \sumatoria{i=1}{h} (2i - 1) + 2(h+1) -1  \igual{\purple{HI}} \magenta{h^2} + 2h + 1 = (h+1)^2
          \end{array}
        $$

        Dado que $p(1),\, p(h),\, p(h+1)$ resultaron verdaderas, por criterio de inducción también lo es $p(n) \en \naturales$
\end{enumerate}

\begin{aportes}
  \item \aporte{\dirRepo}{naD GarRaz \github}
\end{aportes}
