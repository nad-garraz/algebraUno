\begin{enunciado}{\ejercicio}
  Probar que, $\paratodo n \en \naturales,\, \sumatoria{i = 1}{n} (2i - 1) = n^2$:

  \begin{enumerate}[label=\roman*)]
    \item Contando de dos maneras la cantidad total de cuadraditos del diagrama. (\red{hacer diagrama})

    \item Usando la suma aritmética (o suma de Gauss).

    \item Usando el principio de inducción.
  \end{enumerate}
\end{enunciado}

\begin{enumerate}[label=\roman*)]
  \item

  \item $\ub{s = \frac{n(n+1)}{2} = \sumatoria{1}{n} i }{Gauss}
          \to
          \sumatoria{i = 1}{n} 2i -1 =
          2 \sumatoria{i = 1}{n} i - \sumatoria{1}{n} 1 =
          2 \frac{n (n+1)}{2} - n = n^2 + n -n = n^2 \Tilde $

  \item \textit{Proposición: }\par
        $p(n):  \sumatoria{i = 1}{n} (2i - 1) = n^2 \paratodo n \en \naturales$\par
        \textit{Caso base : } $p(1) : \sumatoria{i = 1}{1} 2i -1 = 1 = 1^2  \entonces p(1) $  es verdadera. \Tilde\par
        \textit{Paso inductivo: } $p(h) : \sumatoria{i = 1}{h} 2i -1 = k^2
                \text{ verdadera con }h \en \enteros 
          \entonces
          \text{ quiero ver que }
          \sumatoria{i=1}{h+1} 2i -1 \igual{?} (h+1)^2.\\
          \sumatoria{i=1}{h+1} 2i -1 =
          \sumatoria{i=1}{h} (2i - 1) + 2(h+1) -1
          \igual{\magenta{HI}}
          \magenta{h^2} + 2h + 1 = (h+1)^2 \Tilde
        $\par
        Dado que $p(1),\, p(h),\, p(h+1)$ resultaron verdaderas, por criterio de inducción también lo es $p(n) \en \naturales$
\end{enumerate}
