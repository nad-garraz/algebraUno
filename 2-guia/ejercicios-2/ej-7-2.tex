\begin{enunciado}{\ejercicio}
  Probar que para todo $n \en \naturales$ se tiene
  \begin{multicols}{2}
    \begin{enumerate}[label=\roman*)]
      \item $\sumatoria{i=1}{n} (-1)^{i + 1} i^2 = \frac{(-1)^{n+1} n (n+1)}{2}$
      \item $\sumatoria{i=1}{n} (2i+1)3^{i - 1} = n3^n$
      \item $\sumatoria{i=1}{n} \frac{i2^i}{(i + 1)(i+2)} = \frac{2^{n+1}}{n+2} - 1$
      \item $\productoria{i=1}{n} (1 + a^{2^{i-1}}) = \frac{1 - a^{2^n}}{1 - a},\ a \en \reales - \set{1}$
      \item $\productoria{i=1}{n} \frac{n + i}{2i - 3} = 2^n(1-2n)$
    \end{enumerate}
  \end{multicols}
\end{enunciado}

\begin{enumerate}[label=\roman*)]

  \item
        \textit{Proposición: }
        $$
          p(n) : \sumatoria{i = 1}{n} (-1)^{i + 1} i^2 = \frac{(-1)^{n+1} n (n+1)}{2} \paratodo n \en \naturales
        $$

        \textit{Caso base: }
        $$
          p(\magenta{1}) : \sumatoria{i = 1}{\magenta{1}} (-1)^{i+1} i^2 =
          (-1)^2 \cdot 1 = 1
          \igual{\red{!}}
          \frac{(-1)^{\magenta{1}+1} \magenta{1} (\magenta{1} + 1)}{2} = 1  \Tilde
        $$
        $p(1)$ resulta ser verdadera.

        \textit{Paso inductivo: } Asumo que
        $$
          p(\blue{k}): \ub{\sumatoria{1}{\blue{k}} (-1)^{i+1} i^2 = (-1)^{\blue{k}+1} \frac{\blue{k}(\blue{k}+1)}{2}}{\text{\purple{hipótesis inductiva}}}
        $$
        es verdadera. Entonces quiero ver que:
        $$
          p(\blue{k+1}): \sumatoria{1}{\blue{k+1}}  (-1)^{i+1} i^2 = (-1)^{(\blue{k+1})+1} \frac{(\blue{k+1})(\blue{k+1} + 1)}{2}.
        $$
        también lo sea.\par

        Para probar arranco de $p(\blue{k+1})$ y en medio uso la \purple{hipótesis inductiva} para resolver:
        $$
          \scriptstyle
          \sumatoria{i=1}{k+1}  (-1)^{i+1} i^2 =
          \sumatoria{i=1}{k} (-1)^{i+1} i^2 + (-1)^{k+2} (k+1)^2
          \igual{\purple{HI}}[\purple{!}]
          (-1)^{k+1} \frac{k(k+1)}{2} + (-1)^k (-1)^2 (k+1)^2
          \igual{\red{!!!}}
          (-1)^k (k+1)\frac{(k+2)}{2} \Tilde
        $$

        Por lo tanto p(k+1) resulta ser verdadera también.\par
        Si te quedaste ahí pedaleando en el aire en el \red{!!!} es solo acomodar la expresión, nada raro, algún factor común y coso... \href{\justDoIt}{power!}.
        Si lo hago yo es puro spoiler y no ganás nada.

        Dado que $p(1),\, p(k),\, p(k+1)$ resultaron verdaderas, por criterio de inducción también lo es $p(n) \en \naturales$

  \item
        \textit{Proposición: }
        $$
          p(n) : \sumatoria{i=1}{n} (2i+1)3^{i - 1} = n3^n \paratodo n \en \naturales
        $$

        \textit{Caso base: }
        $$
          p(\magenta{1}) : \sumatoria{i=1}{\magenta{1}} (2i+1)3^{i - 1} = (2 \cdot \magenta{1} + 1)\cdot 3^{\magenta{1} - 1}=3
          \igual{\red!}
          \magenta{1}\cdot3^{\magenta{1}}=3
        $$
        $p(1)$  resulta ser verdadera \par

        \textit{Paso inductivo: } Asumo que
        $$
          \ub{p(\blue{k}):  \sumatoria{i=1}{n} (2i+1)3^{i - 1} = \blue{k}3^{\blue{k}}}{\text{\purple{hipótesis inductiva}}}
        $$
        es verdadera. Entonces quiero ver que
        $$
          p(\blue{k+1}):
          \sumatoria{1}{\blue{k+1}}(2i+1)3^{i - 1} = (\blue{k+1})3^{\blue{k+1}}
        $$
        también lo sea.\par

        Muy parecido al ejercicio anterior:
        $$
          \sumatoria{i=1}{k+1} (2i+1)3^{i - 1} = \sumatoria{i=1}{k} (2i+1)3^{i - 1} + (2(k+1) + 1)3^{(k+1) - 1}
          \igual{\purple{HI}}[\red{!}]
          k3^k + 3^k (2k + 3)
          \igual{\red{!!!}}
          (k+1) \cdot 3^{k+1}
        $$
        Y sí, en el \red{!!!} hay más cuentas. Pero ya a esta altura te vas dando cuenta de que la parte de \textit{inducción} no
        estaría siendo el desafío, sino que (en estos ejercicios) son las cuentas, por eso \underline{mirá fijo las cuentas} y dale tiempo  a tu \rosa{\faIcon{brain}} para que
        encuentre el factor común etc \href{\justDoIt}{¡Curtite, vieja!}.
        Las cuentas son en gran medida lo que complica los parciales, no tanto los temas.\par

        Dado que $p(1),\, p(k) \ytext p(k+1)$ resultaron verdaderas, por el criterio de inducción también lo es $p(n) \en \naturales$

  \item
        \textit{Proposición: } $ p(n) \sumatoria{i=1}{n} \frac{i2^i}{(i + 1)(i+2)} = \frac{2^{n+1}}{n+2} - 1 \paratodo n \en \naturales$\par
        \textit{Caso base: }   $p(1) : \sumatoria{i=1}{1} \frac{i2^i}{(i + 1)(i+2)} = \frac{1*2^1}{(1+1)(1+2)} = \frac{2}{6} = \frac{1}{3} = \frac{2^2 - 3}{1 + 2} \entonces p(1) $ es verdadera$ \Tilde$ \par
        \textit{Paso inductivo: } \text{Asumo }$p(n) : \sumatoria{i=1}{n} \frac{i2^i}{(i + 1)(i+2)} = \frac{2^{n+1}}{n+2} -1 \text{ como verdadera.}$\\

        $
          \entonces
          p(n+1) : \text{quiero ver que }
          \sumatoria{i=1}{n+1} \frac{i2^i}{(i + 1)(i+2)} = \frac{2^{(n+1)+1}}{(n+1)+2} - 1 \text{ también lo sea.}$\\

        $
          \sumatoria{i=1}{n+1} \frac{i2^i}{(i + 1)(i+2)}
          = \sumatoria{i=1}{n} \frac{i2^i}{(i + 1)(i+2)} + \frac{i2^{n+1}}{((n+1) + 1)((n+1)+2)}
          \igual{\magenta{HI}} \frac{2^{n+1}}{n+2} -1 + \frac{i2^{n+1}}{((n+1) + 1)((n+1)+2)}$\\

        $\text{Si } \frac{2^{n+1}}{n+2} -1 + \frac{i2^{n+1}}{((n+1) + 1)((n+1)+2)} = \frac{2^{(n+1)+1}}{(n+1)+2} - 1\entonces
          \text{ sera verdadero}$\\

        \begin{subequations}
          \begin{align*}
            \frac{2^{n+1}}{n+2} -1 + \frac{(n+1)2^{n+1}}{(n+ 2)(n+ 3)} & = \frac{2^{n+2}}{n+3} - 1 \\
            (n + 3) \frac{2^{n+1}}{n+2} + \frac{(n+1)2^{n+1}}{(n+ 2)}  & = 2^{n+2}                 \\
            (n + 3)2^{n+1} + (n+1)2^{n+1}                              & = 2^{n+2} (n+2)           \\
            (n + 3) + (n+1)                                            & = 2 (n+2)                 \\
            2n + 4                                                     & = 2 (n+2)                 \\
            2n + 4                                                     & = 2n + 4                  \\
            0                                                          & = 0
          \end{align*}
        \end{subequations}

  \item
        $\productoria{i=1}{n} \parentesis{ 1 + a^{2^{i-1}} } = \frac{1-a^{2^n}}{1-a}$\\
        $\llave{l}{
            \text{Primer caso } n = 1 \to
            \productoria{i=1}{1} ( 1 + a^{2^{i-1}} ) =
            1 + a^{2^0} = \magenta{1 + a} =
            \frac{1-a^{2^1}}{1 - a} = \frac{(1-a)(1+a)}{1-a} =
          \magenta{1 + a} \Tilde \\
          \text{Paso inductivo } n = k \to
          \productoria{i=1}{k} ( 1 + a^{2^{i-1}} ) =
          \frac{1-a^{2^k}}{1-a} \entonces
          n = k+1 \to  \productoria{i=1}{k+1} ( 1 + a^{2^{i-1}} ) \igual{?}
          \frac{1 - a^{2^{k+1}}}{1-a}\\
          \llave{l}{
            \productoria{i=1}{k+1} ( 1 + a^{2^{i-1}} ) =
            \HI{
              \productoria{i=1}{k} ( 1 + a^{2^k} )
            } \cdot
            \kmasuno{
              1 + a^{2^{i-1}}
            }  =
            \frac{1-a^{2^k}}{1-a} \cdot 1 + a^{2^k}
          \flecha{diferencia}[de cuadrados]
          \frac{1 - ( a^{2^k})^2}{1-a} =\\
          \frac{1 - a^{2 \cdot 2^k}}{1-a} = \frac{1 - a^{2^{k+1}}}{1-a}\Tilde
          }
          }
        $

  \item
        $\productoria{i=1}{n} \frac{n+i}{2i-3} = 2^n (1 -2n)$\par
        En este ejercicio conviene abrir la productoria y acomodar los factores. Por inducción:\par
        $$
          p(n):\ \productoria{i=1}{n} \frac{n+i}{2i-3} = 2^n (1 -2n)
        $$

        \textit{Caso Base: }
        $$
          p(1) : \productoria{i=1}{1} \frac{1+i}{2i-3} = \frac{1+1}{2 \cdot 1 - 3} = 2^1 (1 - 2\cdot 1) = -2
        $$

        \textit{Paso inductivo: }
        $$
          \begin{array}{c}
            p(\blue{k}): \ob{\productoria{i=1}{\blue{k}} \frac{\blue{k}+i}{2i-3} = 2^{\blue{k}} (1 -2\blue{k})}{\purple{\text{hipótesis inductiva}}}
            \text{ asumo verdadera para algún }\blue{k} \en \naturales \\
            \flecha{quiero}[ver que] p(\blue{k}+1): \productoria{i=1}{\blue{k}+1} \frac{\blue{k}+1+i}{2i-3} = 2^{\blue{k}+1} (1 -2(\blue{k}+1)
            \text{ también lo sea para algún } \blue{k} \en \naturales.
          \end{array}
        $$
        \textit{Nota que puede ser de utilidad:}
        Esta productoria tiene a la $n$ en el término general. Cuando pasa esto en el ejercicio, abrir la sumatoria
        para acomodar los factores y así formar la HI, suele ser el camino a seguir.\par
        \textit{Fin nota que puede ser de utilidad}

        $$
          \begin{array}{l}
            \productoria{i=1}{k} \frac{k+i}{2i-3} =
            \frac{k+1}{2 \cdot 1 - 3} \cdot \frac{k+2}{2 \cdot 2 - 3} \cdot \frac{k+3}{2 \cdot 3 - 3} \cdots \frac{2k}{2 \cdot k - 3} =
            \purple{2^k(1 - 2k)} \\

            \productoria{i=1}{k+1} \frac{k+1+i}{2i-3} =
            \frac{\red{k+2}}{\red{2 \cdot 1 - 3}} \cdot
            \frac{k+3}{2 \cdot 2 - 3} \cdots
            \frac{k+1 + (k-1)}{2(k-1) - 3} \cdot
            \frac{k+1 + k}{2k - 3} \cdot
            \frac{\ob{\scriptstyle k+1 + (k+1)}{\red{2k+2}}}{\red{2(k + 1) - 3}}
            \igual{$\llamada1$}
            \frac{1}{\red{2 \cdot 1 - 3}} \cdot
            \frac{\red{k+2}}{2 \cdot 2 - 3} \cdot
            \frac{k + 3}{2 \cdot 3 - 3} \cdots
            \frac{2 k}{2k - 3} \cdot
            \frac{2k + 1}{\red{2(k+1) - 3}}  \cdot
            \frac{\red{2k+2}}{1} \\
            \igual{$\llamada2$}
            \ub{
              \frac{\cyan{k+1}}{2 \cdot 1 - 3} \cdot
              \frac{k+2}{2 \cdot 2 - 3} \cdot
              \frac{k + 3}{2 \cdot 3 - 3} \cdots \frac{2 k}{2k - 3}}{\purple{\text{hipótesis inductiva}} } \cdot \frac{2k + 1}{2(k+1) - 3}
            \cdot \frac{2k+2}{\cyan{k+1}}
            \igual{\purple{HI}}
            \purple{2^k (1 - 2k)}\cdot \frac{2k + 1}{2(k+1) - 3} \cdot \frac{2k+2}{k+1}
            \igual{\red{!}}
            2^{k+1} (1 - 2(k+1)) \Tilde
          \end{array}
        $$

        En $\llamada1$ Corro los denominadores un lugar hacia la izquierda. Pinto con \red{rojo}
        las fracciones de los bordes solo para ayuda visual.\par
        En $\llamada2$ multiplico por $1 = \cyan{\frac{k+1}{k+1}}$ y lo ubico en los lugares
        \textit{apropiados} para que me aparezca la \purple{hipótesis inductiva}.\par
        Si te preguntás qué pasó en "\red{!}", eso son cuentas, \textit{simplificá}, \textit{factorizá} y yo
        que sé, que te dejo a vos, por mi parte yo \href{\dirRepo}{\Large\faIcon{hands-wash}}.\medskip

        Como $p(1),\, p(k) \ytext p(k+1)$ son verdaderas por el principio de inducción $p(n)$ es verdadera $\paratodo n \en \naturales$.
\end{enumerate}

% Contribuciones
\begin{aportes}
  %\aporte{url}{nombre icono}
  \item \aporte{https://github.com/misProyectosPropios}{Iñaki Frutos \github}
  \item \aporte{\dirRepo}{Nad Garraz \github}
\end{aportes}
