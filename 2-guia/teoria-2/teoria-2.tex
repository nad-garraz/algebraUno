\begin{itemize}[label=\tiny\faIcon{snowman}]
  \item\label{2-teoria:suma-prod} \textit{Propiedades de la sumatoria y productoria:}
        \begin{itemize}[label=\tiny\faIcon{pizza-slice}]
          \item
                $(\sumatoria{k=1}{n} a_k) + (\sumatoria{k=1}{n} b_k) = \sumatoria{k=1}{n} (a_k + b_k)$

          \item Sea $c$ un número dado:\par
                $\sumatoria{k=1}{n} (c \cdot a_k) = c \cdot \sumatoria{k=1}{n} a_k$

          \item
                $(\productoria{k=1}{n} a_k) \cdot (\productoria{k=1}{n} b_k) = \productoria{k=1}{n} (a_k b_k)$

          \item Sea $c$ un número dado:\par
                $\productoria{k=1}{n} (c \cdot a_k) = (\productoria{k=1}{n} c) \cdot (\productoria{k=1}{n} a_k) = c^n \cdot \productoria{k=1}{n} a_k$
        \end{itemize}

  \item\label{2-teoria:suma-gauss} \textit{Suma de Gauss:}
        $$
          \begin{array}{c}
            \paratodo n \en \naturales: \sumatoria{i = 1}{n} i =  1 + 2 + \cdots + (n-1) + n = \frac{n(n+1)}{2}
          \end{array}
        $$
        Y cuando empieza desde $i = 0$ u otro valor escribís lo que hay y hacés algo así:
        $$
          \sumatoria{i = \blue{0}}{n} i
          \igual{\red{!}}
          0 + \ub{1 + 2 + \cdots + (n-1) + n}{\sumatoria{i = \red{1}}{n} i} =  \sumatoria{i = \red{1}}{n} i
          \igual{\red{!}}
          \frac{n \cdot (n+1)}{2}
        $$
        o también puede pasar así:
        $$
          \sumatoria{i = \blue{3}}{n} i
          \igual{\red{!}}
          3 + 4 + \cdots + (n-1) + n =
          - \magenta{2} - \magenta{1} + \ub{\magenta{1} + \magenta{2} + 3 + 4 + \cdots + (n-1) + n}{\sumatoria{n=\red{1}}{n} i}
          \igual{\red{!}}
          -1 -2 +\frac{n \cdot (n+1)}{2}
        $$

  \item\hypertarget{2-teoria:geometrica}{\textit{Suma geométrica:} }
        $$
          \paratodo n \en \naturales:
          \sumatoria{i = 0}{n} q^i =
          1 + q + q^2 + \cdots  + q^{n-1} + q^n =
          \llave{lll}{
            n+1                                         & \text{si} & q = 1         \\
            \frac{q^{n+1}-1}{q-1} & \text{si} & q \distinto 1
          }
        $$
        Y cuando empieza desde $\blue{i = 1}$ u otro valor se hace:
        $$
          \sumatoria{\blue{i = 1}}{n} q^i
          \igual{\red{!}}
          \red{-1 + 1} + \sumatoria{\blue{i = 1}}{n} q^i
          \igual{\red{!}}
          \red{-1} + \sumatoria{i = \red{0}}{n} q^i =
          \llave{lll}{
            \red{-1} +  n+1 = n                                                                                  & \text{ si } & q = 1         \\
            \red{-1} + \frac{q^{n+1}-1}{q-1} = \frac{q^{n+1}-q}{q-1} & \text{ si } & q \distinto 1
          }
        $$

  \item \textit{Inducción:}\par
        Sea $H \subseteq \reales$ un conjunto. Se dice que $H$ es un conjunto \textit{inductivo} si se cumplen las dos condiciones siguiente:
        \begin{itemize}
          \item $1 \in H$
          \item $\paratodo x , x \in H \entonces x+1 \en H$
        \end{itemize}

  \item \textit{Principio de inducción:} \magenta{que se usará infinitas veces}\par
        Sea $p(n), n \in \naturales$ , una afirmación sobre los números naturales.\par
        Si $p(n)$ \underline{satisface}:
        \begin{itemize}[label=\small\faIcon{pray}]
          \item \textit{Caso Base: }
                $$
                  p(1) \text{ es Verdadera}.
                $$

          \item \textit{Paso inductivo:}
                $$
                  \paratodo \blue{h} \en \naturales,\, p(\blue{h}) \text{ es Verdadera}
                  \entonces p(\blue{h+1}) \text{ también es Verdadera},
                $$
                entonces $p(n)$ es Verdadera $\paratodo n \en \naturales$.
        \end{itemize}

  \item Principio de inducción \textit{corrido}: A los fines prácticos todos los principios corridos y la mar en coche, son iguales... bueh,
        son muy parecidos de resolver.
\end{itemize}
