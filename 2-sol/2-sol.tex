\documentclass[12pt, a4paper, spanish, twoside]{article}
% Sacar draft para que aparezcan las imagenes.
% Opciones: 12pt, 10pt, 11pt, landscape, twocolumn, fleqn, leqno...
% Opciones de clase: article, report, letter, beamer...

% Paquetes:
% =========
\usepackage[headheight=110pt, top = 2cm, bottom = 2cm, left=1cm, right=1cm]{geometry} %modifico márgenes
\usepackage[T1]{fontenc} % tildes
\usepackage[utf8]{inputenc} % Para poder escribir con tildes en el editor.
\usepackage[english]{babel} % Para cortar las palabras en silabas, creo.
\usepackage[ddmmyy]{datetime}
\usepackage{amsmath} % Soporte de mathmatics
\usepackage{mathtools} % Más herramientas para matemáctica
\usepackage{amssymb} % fuentes de mathmatics
\usepackage{array} % Para tablas y eso
\usepackage[dvipsnames]{xcolor} % Para colorear el texto: black, blue, brown, cyan, darkgray, gray, green, lightgray, lime, magenta, olive, orange, pink, purple, red, teal, violet, white, yellow.
\usepackage{enumitem} % Cambiar labels y más flexibilidad para el enumerate
\usepackage{multicol} 
\usepackage{tikz} % para graficar
\usepackage{cancel} % cancelar fórmulas
\usepackage{titlesec} % para editar titulos y hacer secciones con formato a medida
\usepackage{ulem}
\usepackage{centernot} % tacha cosas
\usepackage{bbding} % símbolos de donde uso FiveStar
\usepackage{skull} % símbolos de donde uso Skull
\usepackage{soul} % Para tachar texto en text y math mode
\usepackage{polynom} % para división de polinomios y mcd
\usepackage{fontawesome5} % fuentes "extras"
\usepackage{venndiagram} % Para los diagramas de Venn
\usepackage{qrcode} % genera código qr
%\usepackage{listings} % Escribir código

%\usepackage{algorithm}
%\usepackage{algpseudocode}
%\usepackage{algorithmicx}

\usepackage{fancyhdr} % Encabezados y pie de páginas
% \usepackage{lipsum} % dummy text
% \usepackage{caption} % Configuracion de figuras y tablas



% para hacer los graficos tipo grafos
\usetikzlibrary{shapes,arrows.meta, chains, matrix, calc, trees, positioning, fit}
\usetikzlibrary{external,decorations.pathreplacing,angles,quotes}

% En general quiero que este paquete sea el último en importarse
\usepackage{hyperref} % para que haya links navegables en el PDF
\hypersetup{
    colorlinks=true,
    linkcolor=blue,
    %filecolor=magenta,
    urlcolor=OliveGreen!90!black,
    pdftitle={Álgebra I - Resuelta, sueltísima},
    pdfauthor={Por los alumnos y exalumnos de Álgebra I}
    }
\urlstyle{same}

\setlength{\parindent}{0pt} % Para que no haya indentación en las nuevas líneas.

%% Info SOCIAL
\def\dirRepo{https://github.com/nad-garraz/algebraUno}
\def\dirTelegram{https://t.me/+1znt2GV1i8cwMTNh}
\newcommand{\dirGuia}[1]{\dirRepo/blob/main/#1-guia/#1-sol.pdf}


% Algunos paquetes quizás exclusivos de este archivo que no
% quiero poner en el preamble-general
% el día de mañana podría integrarse
\usetikzlibrary{external,angles,quotes}

% Definiciones y macros para que se me haga
% más ameno el codeo.
% Definiciones y nuevos comandos:def
% =============
% Conjuntos
\DeclareMathOperator{\partes}{\mathcal P}
\DeclareMathOperator{\relacion}{\,\mathcal{R}\,}
\DeclareMathOperator{\norelacion}{\,\cancel{\relacion}\,}
\DeclareMathOperator{\universo}{\mathcal U}
\DeclareMathOperator{\reales}{\mathbb R}
\DeclareMathOperator{\naturales}{\mathbb N}
\DeclareMathOperator{\enteros}{\mathbb Z}
\DeclareMathOperator{\racionales}{\mathbb Q}
\DeclareMathOperator{\irracionales}{\mathbb I}
\DeclareMathOperator{\complejos}{\mathbb C}


\DeclareMathOperator{\K}{\mathbb K} % cuerpo K
\DeclareMathOperator{\i}{\text{i}}
\DeclareMathOperator{\vacio}{\varnothing}
\DeclareMathOperator{\union}{\cup}
\DeclareMathOperator{\inter}{\cap}
\DeclareMathOperator{\diferencia}{\ \setminus \ }
\DeclareMathOperator{\y}{\land}
\def\o{\lor}
\def\neg{\sim}

\def\entonces{\Rightarrow}
\def\noEntonces{\centernot\Rightarrow}

\def\sisolosi{\iff} % largo
\def\sii{\Leftrightarrow} % corto

\def\clase{\overline}
\def\ord{\text{ord}}


\def\existe{\,\exists\,}
\def\noexiste{\,\nexists\,}
\def\paratodo{\ \, \forall}
\def\distinto{\neq}
\def\en{\in}
\def\talque{\;/\;}

% =====
\def\qvq{\text{ quiero ver que }}

%funciones
\DeclareMathOperator{\dom}{Dom}
\DeclareMathOperator{\cod}{Cod}
\def\F{\mathcal F}
\def\comp{\circ}
\def\inv{^{-1}}
\def\infinito{\infty}

% Llaves, paréntesis, contenedores
\newcommand{\llave}[2]{ \left\{ \begin{array}{#1} #2 \end{array}\right. }
\newcommand{\llaveInv}[2]{ \left\} \begin{array}{#1} #2 \end{array}\right. }
\newcommand{\llaves}[2]{ \left\{ \begin{array}{#1} #2 \end{array} \right\} }
\newcommand{\matriz}[2]{\left( \begin{array}{#1} #2 \end{array} \right)}
\newcommand{\deter}[2]{\left| \begin{array}{#1} #2 \end{array} \right|}
\newcommand{\lista}[2][(1)]{\begin{enumerate}[\bf #1]\setlength\itemsep{-0.6ex} #2 \end{enumerate}}
\newcommand{\listal}[2][-0.6ex]{\begin{enumerate}[\bf(a)]\setlength\itemsep{#1} #2 \end{enumerate}}

% naturales
\newcommand{\sumatoria}[2]{\sum\limits_{#1}^{#2}}
\newcommand{\productoria}[2]{\prod\limits_{#1}^{#2}}
\newcommand{\kmasuno}[1]{\underbrace{#1}_{k+1\text{-ésimo}}}
\newcommand{\HI}[1]{\underbrace{#1}_{\text{HI}}}

% % enteros
\def\divideA{\, | \,}
\def\noDivide{\centernot\divideA}
\def\congruente{\, \equiv \,}
\newcommand{\congruencia}[3]{#1 \equiv #2 \;(#3)}
\newcommand{\noCongruencia}[3]{#1 \not\equiv #2 \;(#3)}
\newcommand{\conga}[1]{\stackrel{(#1)}{\congruente}}
\newcommand{\divset}[2]{\mathcal{D}(#1) = \set{#2}}
\newcommand{\divsetP}[2]{\mathcal{D_+}(#1) = \set{#2}}
\newcommand{\ub}[2]{ \underbrace{\textstyle #1}_{\mathclap{#2}} }
\newcommand{\ob}[2]{ \overbrace{\textstyle #1}^{\mathclap{#2}} }
\def\cop{\, \perp \, }

% complejos
\DeclareMathOperator{\re}{Re}
\DeclareMathOperator{\im}{Im}
%\DeclareMathOperator{\arg}{arg}
\def\conj{\overline}

% Polinomios
\DeclareMathOperator{\cp}{cp}
\DeclareMathOperator{\gr}{gr}
\DeclareMathOperator{\mult}{mult}
\newcommand{\divPol}[2]{\polylongdiv[style=D]{#1}{#2}}
\newcommand{\mcd}[2]{\polylonggcd{#1}{#2}}


% =====
% Miscelanea
% =====
\def\ot{\leftarrow}
\newcommand{\estabien}{{\color{blue} Consultado, está bien. \checkmark}}
\newcommand{\hacer}{
  {\color{red!80!black}{\Large \faIcon{radiation} Falta hacerlo!}}\par
  {\color{black!70!white}
    \small Si querés mandarlo: Telegram $\to$ \href{https://t.me/+1znt2GV1i8cwMTNh}{\small\faIcon{telegram}},
    o  mejor aún si querés subirlo en \LaTeX $\to$ \href{https://github.com/nad-garraz/algebraUno}{\small \faIcon{github}}.
  }\par
}

\newcommand{\Hacer}{{\color{black!30!red}\Large Hacer!}}
\def\Tilde{\quad\checkmark}
\def\ytext{\text{ y }}
\def\otext{\text{ o }}

% Estrellita para hacer llamadas de atención, viene en divertidos colores
% para coleccionar.
\newcommand{\llamada}[1]{
  {\small{\textcolor{
          \ifcase \numexpr#1 mod 6\relax
            cyan\or magenta\or OliveGreen\or YellowOrange\or Cerulean\or Violet\or Purple\or
          \fi
        }
        {\text{{\small\FiveStar}}^{#1}}%
      }%
    }
}


% separadores
\def\separador{\rule{\linewidth}{0.4pt}\par}
\def\separadorCorto{\rule{0.5\linewidth}{0.4pt}\par\addvspace{10pt}}


% Colores
\newcommand{\red}[1]{\textcolor{red}{#1}}
\newcommand{\green}[1]{\textcolor{OliveGreen}{#1}}
\newcommand{\blue}[1]{\textcolor{Cerulean}{#1}}
\newcommand{\cyan}[1]{\textcolor{cyan}{#1}}
\newcommand{\yellow}[1]{\textcolor{YellowOrange}{#1}}
\newcommand{\magenta}[1]{\textcolor{magenta}{#1}}
\newcommand{\purple}[1]{\textcolor{purple}{#1}}

% Conjuntos entre llaves y paréntesis
% te ahorrás escribir los \left y \right, así dejando el código más legible.
\newcommand{\set}[1] { \left\{ #1 \right\} }
\newcommand{\parentesis}[1]{ \left( #1 \right) }

% Stackrel text. Es para ahorrarse ecribir el \text
\newcommand{\stacktext}[2]{ \stackrel{\text{#1}}{#2} }

% Dado que muchas veces ponemos cosas sobre un signo '='
%  acá está el comando para escribir \igual{arriba}[abajo] con texto!
\NewDocumentCommand{\igual}{m o}{%
  \IfNoValueTF{#2}{%
    \overset{\mathclap{\text{#1}}}=
  }{
    \overset{\mathclap{\text{#1}}}{\underset{\mathclap{\text{#2}}}=}
  }
}


%=======================================================
% Comandos con flechas extensibles.
%=======================================================
% *Flechita* extensible con texto {arriba} y [abajo] 
\NewDocumentCommand{\flecha}{m o}{%
  \IfNoValueTF{#2}{%
    \xrightarrow[]{\text{#1}}
  }{
    \xrightarrow[\text{#2}]{\text{#1}}
  }
}
% *Si solo si* extensible con texto {arriba} y [abajo] 
\NewDocumentCommand{\Sii}{m o}{%
  \IfNoValueTF{#2}{%
    \xLeftrightarrow[]{\text{#1}}
  }{
    \xLeftrightarrow[\text{#2}]{\text{#1}}
  }
}

%=======================================================
% fin comandos con flechas extensibles.
%=======================================================


% como el stackrel pero también se puede poner algo debajo
\newcommand{\taa}[3]{ % [t]exto [a]rriba y [a]bajo
  \overset{\mathclap{#1}}{\underset{\mathclap{#2}}{#3}}
}

%Update time
\def\update{
  actualizado: \today
}


%=======================================================
% sección ejercicio con su respectivo formato y contador
%=======================================================
\newcounter{ejercicio}[section] % contador que se resetea en cada sección
\renewcommand{\theejercicio}{\arabic{ejercicio}} % el contador es un número arabic
\newcommand{\ejercicio}{%
  \stepcounter{ejercicio}% incremento en uno
  \titleformat{\section}[runin]{\bfseries}{\theejercicio}{1em}{}%
  \section*{\theejercicio.}\labelEjercicio{ej:\theejercicio}
}

% Label y refencia para ejercicio hay alguna forma más elegante de hacer esto?
\newcommand{\labelEjercicio}[1]{
  \addtocounter{ejercicio}{-1} % counter - 1
  \refstepcounter{ejercicio} % referencia al anterior y luego + 1
  \label{#1}}
\newcommand{\refEjercicio}[1]{{ \bf\ref{#1}.}}

\def\fueguito{{\color{orange}{\faIcon{fire}}}}
\newcounter{ejExtra}[section] % contador que se resetea en cada sección
\renewcommand{\theejExtra}{\arabic{ejExtra}} % el contador es un número arabic
\newcommand{\ejExtra}{%
  \stepcounter{ejExtra}% incremento en uno
  \titleformat{\section}[runin]{\bfseries}{\theejExtra}{1em}{}%
  % Es como una sección. Le pongo un ícono, luego el número del ejercicio con la etiqueta para poder
  % linkearlo en el índice u otro lugar.
  % con \ref{ejExtra:{numero del ejercicio}} es que salto al ejercicio.
  \section*{\fueguito\theejExtra.}\labelEjExtra{ejExtra:\theejExtra}
}

% Label y refencia para ejercicio hay alguna forma más elegante de hacer esto?
\newcommand{\labelEjExtra}[1]{
  \addtocounter{ejExtra}{-1} % counter - 1
  \refstepcounter{ejExtra} % referencia al anterior y luego + 1
  \label{#1} % etiqueta para cada ejercicio extra
  \unskip
}
% Con esto llamos al ejercicio extra
\newcommand{\refEjExtra}[1]{
  {\fueguito\bf\ref{#1}.}
}

%=======================================================
% fin sección ejercicio con su respectivo formato y contador
%=======================================================

\newenvironment{enunciado}[1]{ % Toma un parametro obligatorio: \ejExtra o \ejercicio 
  \begin{minipage}{\textwidth}
          \par
    \vspace{5pt}
    \separador % linea sobre el enunciado
    #1
    }% contenido
    {\vspace{5pt}
    \par
    \separadorCorto % linea debajo del enunciado
  \end{minipage}
}



% Local acá que se usa mucha inducción
\def\V{\text{ verdadera }}
\def\eq?{\stackrel{\text{?}}}
\def\eqHI{\stackrel{\text{HI}}}
\def\eqDef{\stackrel{\text{def}}}

\begin{document}

\pagestyle{empty} % Para que no muestre el número en pie de página

% Info para armar título.
\title{Práctica 2 de álgebra 1} % título
\author{Nad Garraz} % autor
\date{\today} % Cambiar de ser necesario
% \maketitle  % Para que aprezca el título en el documento

\section*{Inducción, números naturales}

\begin{enumerate}
	\item $\paratodo n \en \naturales: \sumatoria{i = 1}{n} i =  1 + 2 + \cdots + (n-1) + n = \frac{n(n+1)}{2}$

	\item $\paratodo n \en \naturales: \sumatoria{i = 0}{n} q^i =
		      1 + q + q^2 + \cdots  + q^{n-1} + q^n =
		      \llave{lll}{
			      n+1 & \text{si} & q = 1\\
			      \frac{q^{n+1}-1}{q-1} & \text{si} & q \distinto 1\\
		      }$

	\item Inducción: Sea $H \subseteq \reales$ un conjunto. Se dice que $H$ es un conjunto \textit{inductivo} si se cumplen las dos condiciones siguiente:
	      \begin{itemize}
		      \item $1 \in H$
		      \item $\paratodo x , x \in H \entonces x+1 \en H$
	      \end{itemize}

	\item Principio de inducción: Sea $p(n), n \in \naturales$ , una afirmación sobre los números naturales.
	      Si $p$ satisface
	      \begin{itemize}
		      \item (Caso Base) $p(1)$ es Verdadera.
		      \item (Paso inductivo) $\paratodo h \en \naturales,\, p(h)$ \textit{Verdadera}
		            $\entonces p(h+1)$ \textit{Verdadera, entonces $p(n)$ es Verdadera} $\paratodo n \en \naturales$.
	      \end{itemize}

	\item Principio de inducción \textit{corrido}: Sea $n_0 \en \enteros$ y sea $p(n),\, n\geq n_0,\,$ una afirmación sobre $\enteros_{\geq n_0}$. Si $p$
	      satisface:
	      \begin{itemize}
		      \item (Caso Base) $p(n_0)$ es Verdadera.
		      \item (Paso inductivo) $\paratodo h \geq n_0,\, p(h)$ \textit{Verdadera}
		            $\entonces p(h+1)$ \textit{Verdadera, entonces $p(n)$ es Verdadera} $\paratodo n \en \naturales$.
	      \end{itemize}
\end{enumerate}

\begin{enumerate}
	\item explicación de las torres de Hanoi.
	      \begin{enumerate}[label=\arabic*)]
		      \item $a_1 = 1$
		      \item $a_3 = 7$
		      \item $a_4 = 15$
		      \item $a_9 = a_9 +1+a_9 = 2 a_9 +1$
	      \end{enumerate}
	      $\to$ \boxed{a{_n+1} = 2a_n + 1}

	\item Una sucesión $(a_n)_{n \en \naturales}$ como las torres de Hanoi $a_1 = 1 \y a_{n+1}= 2a_n + 1, \paratodo n \en \naturales$, es una
	      sucesión definida por recurrencia.\\

	\item El patrón de las torres de Hanoi parece ser $\underbrace{a_n = 2^n -1 }_{\text{término general}} \paratodo n \en \naturales$.
	      Esto puedo probarse por inducción.
	      $ \llave{l}{
			      \text{Proposición:} p(n): a_n = 2^n -1\\
			      \text{Caso Base: } p(1) \text{ es verdadero?} a_1 = 2^1 -1 =1 \Tilde\\
			      \text{Paso inductivo: } p(h) \text{ es verdadero}  \entonces p(h+1) V?\\

			      \llave{l}{
				      \text{HI}:  a_h = 2^h -1\\
				      \text{QPQ}: a_{h+1} = 2^{h+1}
			      } \to \text{cuentas y queda que }  \boxed{ p(n)\ es\ V, \paratodo n \en \naturales}
		      } $

	\item $\sum$ es una def por recurrencia $\to \sumatoria{k=1}{1} a_k = a_1 \y \sumatoria{k=1}{n+1} a_k =... facil $

\end{enumerate}


\textit{Principio de inducción III: } Sea $p(n)$ una proposición sobre $\naturales$. Si se cumple:
\begin{enumerate}
	\item  $p(1) \y p(2) \ V$
	\item $\paratodo h \en \naturales,\, p(h) \y p(h+1),\, V \entonces p(h+2)\ V\ (\text{paso inductivo})$,
	      entonces $p(n)$ es verdadera.
\end{enumerate}

$p(n): a_n = 3^n$\\

$\llave{l}{
		\text{caso base: } a_1 = 3, a_2 =9 \Tilde\\
		\text{Paso inductivo: } \paratodo h \en \naturales, p(h) \y p(h+1) \ V \entonces p(h+2)\ V\\

		\llave{l}{
			\text{HI: }  a_h = 3^h \y a_{h+1} = 3^{h+1}\\
			\text{Quiero probar que: } a_{h+2}= 3^{h+2}\\
			\text{Usando la fórmula de recurrencia sale enseguida}
		}
	}
$

\textit{Principio de inducción IV } Sea $p(n)$ una proposición sobre $\enteros_{\geq n_0}$. Si se cumple:
\begin{enumerate}
	\item  $p(n_0) \y p(n_0 + 1) \ V$
	\item $\paratodo h \en \enteros_{\geq n_0},\, p(h+1) \y p(h+2)\, V \entonces p(h+2)\ V\ (\text{paso inductivo})$,
	      entonces $p(n)$ es verdadera. $\paratodo n \geq n_0$\\
\end{enumerate}


\textit{Sucesión de Fibonacci}: $F_0 = 0, F_1 = 1, F_{n+2} = F_{n+1} + F_n, \paratodo n \geq 0$\\
Truco para sacar fórmulas a partir de Fibo.\\
$F_{n+2} - F_{n+1} - F_n = 0 \to x^2 - x -1 = 0 =
	\llaves{ l }{
		\Phi = \frac{1+\sqrt{5}}{2}\\
		\tilde\Phi = \frac{1-\sqrt{5}}{2}\\
	} \to \Phi^2 = \Phi + 1 \y \tilde\Phi^2  =\tilde\Phi + 1 $
\begin{itemize}
	\item  defino sucesiones $\Phi^n$ que satisfacen la recurrencia de la sucesión de Fibonacci pero no sus condiciones iniciales.
	\item puedo formar una combineta lineal talque: $(c_n)_{n\en \naturales_0} = (a\Phi^n + b\tilde\Phi^n)$ es la sucesión que satisface:
	      $\llave{ l }{
			      c_o = a+b\\
			      c_1 = a\Phi + b\tilde\Phi
		      }$ y la recurrencia de Fibonacci.\\
	      Resuelvo todo y llego a $\boxed{}$
\end{itemize}

\textit{Sucesione de Lucas}: Generalizaciones de Fibonacci.$(a_n)_{n\in\naturales_0}$\\

$a_0 = \alpha, a_1 = \beta \y a_{n+2} = \gamma a_{n+1} +\delta a_n,\, \paratodo n \geq 0,\, con \alpha, \beta, \gamma, \delta $ dados.\\
Esto lo meto en la ecuación característica: $x^2 - \gamma x -\delta = 0$, necesito raíces distintas.
Notar que $r^2 = \gamma r^1 + \delta$, y lo mismo es para $\tilde r$. Las sucesiones ($r^n$) y ($\tilde r^n$) satisfacen la recurrencia de Lucas,
pero no las condiciones iniciales $\alpha$ y $\beta$.
$c_n = (a r^n + b \tilde r^n)$, satisface Lucas, pero las condiciones iniciales son $c_0$ y $c_1$ o
$
	\llave{l}{
		a + b = \alpha\\
		r a +\tilde b = \beta\\
	}\to
	\llave{l}{
		ra +rb = r\alpha\\
		ra + \tilde r b = \beta\\
	}
$ luego hago lo mismo con $\tilde r$
Como resultado: $a = \frac{\beta - \tilde r \alpha}{r - \tilde r}$

\section*{Ejercicio fuera de la guía}
Se cumple que: $\frac{(2n)!}{n!^2} \leq (n+1)!, \, \paratodo n \en \naturales$?
\begin{enumerate}
	\item La proposición: $p(n): \frac{(2n)!}{n!^2} \leq (n+1)! $

	\item Caso base: $p(n = 1)$ es Verdadera? $\flecha{evalúo}[$n = 1$] \frac{(2\cdot 1)!}{1!^2} = 2 \leq (1+1)! $\Tilde

	\item Mi \textbf{HI} es \textit{que vale}  $ \frac{(2h)!}{h!^2} \leq (h+1)! $

	\item Quiero probar que $\frac{(2(h+1))!}{(h+1)!^2} \leq ((h+1)+1)!
		      \flecha{acomodo}
		      \frac{(2h + 2)!}{(h + 1)!^2} \leq (h+2)!$

	\item Hay que hacer cosas para poder meter la \textbf{HI} en las  cuentas del punto anterior.\\
	      Noto que: $\frac{(2h + 2)!}{(h + 1)!^2} =
		      \frac{(2h + 2) \cdot (2h + 1) \cdot \blue{(2h)!}}{(h + 1)^2 \cdot \blue{h!^2}} \stacktext{HI}\leq
		      \frac{(2h + 2) \cdot (2h + 1)}{(h + 1)^2 } \blue{(h+1)!} \leq (h+2)!$\\
	      Probando esa última desigualdad se prueba lo buscado.
\end{enumerate}

\subsubsection*{Ejercicio de la clase del 12/4}
Sea $(a_n)_{n \en \naturales_0}$ con
$\llave{l}{
		a_0 = 1\\
		a_1 = 3\\
		a_n = a_{n-1} - a_{n-2}\ \paratodo n \geq 2
	}$
\begin{enumerate}[label=(\alph*)]
	\item Probar que $a_{n+6} = a_n$\\
	      Por inducción: \boxed{p(n):   a_{n+6} = a_n \paratodo n \geq \naturales_0 \V?}\\
	      $\llave{l}{
		      \textit{ Caso Base:} \text{ Primero notar que,} \\
		      \to
		      \llaves{l}{
			      a_0 = 1\\
			      a_1 = 3\\
			      a_2 \eqDef= 2\\
			      a_3 \eqDef= -1\\
			      a_4 \eqDef= -3\\
			      a_5 \eqDef= -2
		      } \to
		      \llaves{l}{
			      a_6 \eqDef= 1\\
			      a_7 \eqDef= 3\\
			      a_8 \eqDef= 2\\
			      a_9 \eqDef= -1\\
			      a_{10} \eqDef= -3\\
			      a_{11} \eqDef= -2
		      } \to
		      \cdots \text{ Se ve que tiene un período de 6 elementos.}\\

		      p(n=2) \V? \to a_8 \eq?= a_2 \Tilde\\

		      \textit{Paso inductivo: } \text{Supongo } p(k) \V \entonces p(k+1) \V?\\
		      \textit{Hipótesis inductiva: }
		      \text{Supongo } a_{k+6} = a_k \paratodo k \in \naturales_0 \V ,\, \qvq a_{k+7} = a_{k+1}\\
		      \red{a_{k+7}} \eqDef=
		      a_{k+6} - a_{k+5} \stacktext{HI}{=}
		      a_k - a_{k+5} \eqDef=
		      a_k - (\ub{ a_k + a_{k+4}}{a_{k+5}}) = -a_{k+4}\\
		      \to \red{a_{k+7}} = -a_{k+4} \eqDef=
		      -(a_{k+3} - a_{k+2}) \eqDef =
		      - ( a_{k+2} - \red{a_{k+1}} - a_{k+2} ) = \red{a_{k+1}} \Tilde
		      }\\
	      $\\
	      Como $p(0) \y p(1) \y \cdots p(5)$ son verdaderas y $p(k)$ es verdadera así como $p(k+1)$ también lo es, por el principio de inducción $p(n)$ es verdadera $\paratodo n \in \naturales_0$

	\item Calcular $\sumatoria{k=0}{255} a_k$ \\
	      $\sumatoria{k=0}{255} a_k =
		      \underbrace{\textstyle a_0 + a_1 + a_2 + a_3 + a_4 + a_5}_{= 0} +
		      \underbrace{\textstyle a_6 + a_7 + a_8 + a_9 + a_{10} + a_{11} }_{=0} +
		      \cdots +
		      a_{252} + a_{253} + a_{254} + a_{255}
	      $\\
	      En la sumatoria hay \red{256 términos}. $256 = 42 \cdot 6 + 4$ por lo tanto van a haber 42 bloques que dan 0 y sobreviven los últimos 4 términos.
	      $\sumatoria{k=0}{255} a_k = \underbrace{\textstyle 0 + 0 + \dots + 0}_{42 \text{ ceros}} + a_{252} + a_{253} + a_{254} + a_{255} =
		      \cancel{a_{252}} + a_{253} + a_{254} + \cancel{a_{255}} = a_{253} + a_{254} = 5\\
	      $ Donde usé que: $a_n =
		      \llave{rcl}{
			      1 & \text{si} & n\mod6 = 0 \\
			      3 & \text{si} & n\mod6 = 1 \\
			      2 & \text{si} & n\mod6 = 2 \\
			      -1 & \text{si} & n\mod6 = 3 \\
			      -3 & \text{si} & n\mod6 = 4 \\
			      -2 & \text{si} & n\mod6 = 5 \\
		      }\longrightarrow
	      $
	      \boxed{\sumatoria{k=0}{255} a_k = 5} \Tilde
\end{enumerate}

\separador

Sea $(a_n)_{n\en \naturales}$ la sucesión definida por:
$a_1 = 1 \y a_{n+1} = \parentesis{\sqrt{a_n} - (n+1)}^2, \, \paratodo n \en \naturales$.
Voy a encontrar la fórmula general.

$
	\llave{ l }{
		a_1 = 1,\, a_2 = (1 - 2)^2,\, a_3 = 4,\, a_4 = 4,\, a_5 = 9,\, \dots\\
		a_n =
		\llave{l}{
			\parentesis{\frac{n+2}{3}^2} \text{si $n$ es impar}\\
			\parentesis{\frac{n}{3}^2} \text{si $n$ es par}\\
		}\\
		\to \text{ Se muestra por inducción } \hacer\\
	}$


\separador

Sea $(a_n)_{n\en \naturales}$ la sucesión definida por: $a_1 = 3, a_2 = 9 \y a_{n+2} = a+{n+1}+ 3a_n +3^{3+1},\, \paratodo n \en \naturales$\\
Tengo que encontrar el término general de esta  sucesión definida por recurrencia.
$a_1 =3, a_2 = 9. a_3 = a_2 + 3a_1 + 3^2=27 \to$ pinta ser $a_n = 3^n$. \\
Interesante que acá la HI dependería de muchos términos. Así que ahora viene una versión
cambiada del principio de inducción.\\

\section*{Ejercicios de la guía}
%1
\ejercicio
\hacer


%2
\ejercicio

\hacer


%3
\ejercicio
Calcular
\begin{multicols}{2}
	\begin{enumerate}[label=\roman*)]
		\item $ \sumatoria{i=1}{n} (4i + 1) $
		      \hacer


		\item $\sumatoria{i=6}{n} 2(i-5)$
		      \hacer

	\end{enumerate}
\end{multicols}

%4
\ejercicio Calcular \begin{enumerate}[label=\roman*)] \item $\sumatoria{i=0}{n}2^i$\\
	      $ \sumatoria{i=0}{n}2^i \stackrel{q\neq1} = \frac{\red{2}^{n+1} - 1}{\red{2} - 1} = 2^{n+1} - 1 $

	\item $\sumatoria{i=1}{n} q^i$\\
	      $\sumatoria{i=1}{n} q^i = \red{-1 + 1} + \sumatoria{i=1}{n} q^i = \red{-1} + \sumatoria{i = \red{0}}{n} q^i =
		      \llave{lll}{
			      n + 1 \red{-1} = n & \text{si} & q = 1\\
			      \frac{q^{n+1}-1}{q-1} \red{-1} = \frac{q^{n+1} - q}{q-1} & \text{si} & q \distinto 1\\
		      }$

	\item \hacer
	\item \hacer
\end{enumerate}

%5
\ejercicio
\begin{enumerate}[label=\roman*)]
	\item
	      \hacer

	\item $\underbrace{S = \frac{N(N+1)}{2} = \sumatoria{1}{N} i }_{Gauss} \to \sumatoria{1}{n} 2i -1 = 2 \sumatoria{1}{n}i - \sumatoria{1}{n} 1 = 2 \frac{n (n+1)}{2} - n = n^2 + n -n = n^2 \Tilde $

	\item $
		      \llaves{ l }{
			      \text{Primer caso } n = 1 \to \sumatoria{1}{1} 2i -1 = 1 = 1^2 \Tilde\\
			      \text{Paso inductivo } n = h \to \sumatoria{1}{k} 2i -1 = k^2 \Tilde \entonces \sumatoria{1}{k+1} 2i -1 \stackrel{\green{?}}= (k+1)^2\\
			      \sumatoria{1}{k+1} 2i -1 =
			      \HI{
				      \sumatoria{1}{k} 2i -1
			      } + \kmasuno{
				      2(k+1) -1
			      } = k^2 + 2k + 1 = (k+1)^2 \green{\Tilde}
		      } \to \boxed{\sumatoria{i = 1}{n} (2i - 1) = n^2}$
\end{enumerate}

%6
\ejercicio
\hacer

%7
\ejercicio
\begin{enumerate}[label=\roman*)]
	\item $\llave{ l }{
			      \text{Primer caso } n = 1 \to \sumatoria{1}{1} (-1)^{i+1} i^2 = (-1)^2 \cdot 1 = 1 \Tilde\\
			      \text{Paso inductivo }
			      \llave{ l }{
				      n = k \to \sumatoria{1}{k} (-1)^{i+1} i^2 = (-1)^{k+1} \frac{k(k+1)}{2} \\
				      \entonces\\
				      n = k+1 \to \sumatoria{1}{k+1}  (-1)^{i+1} i^2 \eq?= (-1)^{(k+1)+1} \frac{(k+1)(k+2)}{2}\\
				      \to \sumatoria{1}{k+1}  (-1)^{i+1} i^2 =
				      \HI{
					      \sumatoria{1}{k} (-1)^{i+1} i^2
				      } +
				      \kmasuno{
					      (-1)^{k+2} (k+1)^2
				      } =\\
				      (-1)^{k+1} \frac{k(k+1)}{2} + (-1)^k (-1)^2 (k+1)^2 \flecha{acomodar}[fáctor cómun] (-1)^k (k+1)\left[ -\frac{k}{2} + (k+1) \right] =\\
				      (-1)^k (k+1)\frac{(k+2)}{2} \Tilde
			      }
		      }\\ \to
		      \boxed{\sumatoria{i=1}{n} (-1)^{i+1} i^2 = (-1)^{n+1} \frac{n(n+1)}{2}}
	      $

	\item
	      \hacer

	\item
	      \hacer

	\item
	      $\productoria{i=1}{n} \parentesis{ 1 + a^{2^{i-1}} } = \frac{1-a^{2^n}}{1-a}$\\
	      $\llave{l}{
		      \text{Primer caso } n = 1 \to
		      \productoria{i=1}{1} ( 1 + a^{2^{i-1}} ) =
		      1 + a^{2^0} = \magenta{1 + a} =
		      \frac{1-a^{2^1}}{1 - a} = \frac{(1-a)(1+a)}{1-a} =
		      \magenta{1 + a} \Tilde \\
		      \text{Paso inductivo } n = k \to
		      \productoria{i=1}{k} ( 1 + a^{2^{i-1}} ) =
		      \frac{1-a^{2^k}}{1-a} \entonces
		      n = k+1 \to  \productoria{i=1}{k+1} ( 1 + a^{2^{i-1}} ) \eq?=
		      \frac{1 - a^{2^{k+1}}}{1-a}\\
		      \llave{l}{
		      \productoria{i=1}{k+1} ( 1 + a^{2^{i-1}} ) =
		      \HI{
			      \productoria{i=1}{k} ( 1 + a^{2^k} )
		      } \cdot
		      \kmasuno{
			      1 + a^{2^{i-1}}
		      }  =
		      \frac{1-a^{2^k}}{1-a} \cdot 1 + a^{2^k}
		      \flecha{diferencia}[de cuadrados]
		      \frac{1 - ( a^{2^k})^2}{1-a} =\\
		      \frac{1 - a^{2 \cdot 2^k}}{1-a} = \frac{1 - a^{2^{k+1}}}{1-a}\Tilde
		      }
		      }
	      $



	\item
	      $\productoria{i=1}{n} \frac{n+i}{2i-3} = 2^n (1 -2n)$\\
	      En este ejercicio conviene abrir la productoria y acomodar los factores. Por inducción:\\
	      $\llave{l}{
			      p(n):\ \productoria{i=1}{n} \frac{n+i}{2i-3} = 2^n (1 -2n)  \\
			      \textit{Caso Base: } p(1) \text{ V?} \to \productoria{i=1}{1} \frac{1+i}{2i-3} = \frac{1+1}{2 \cdot 1 - 3} = 2^1 (1 - 2\cdot 1) = -2\\
			      \textit{Paso inductivo: Supongo } p(k) \text{ Verdadero } \flecha{quiero ver}[que] p(k+1) \text{ Verdadero para algún } k \en \naturales.\\
			      \textit{Hipótesis inductiva: Supongo } \productoria{i=1}{k} \frac{k+i}{2i-3} = 2^k (1 -2k), \qvq   \productoria{i=1}{k+1} \frac{k+1+i}{2i-3} = 2^{k+1} (1 -2(k+1))\\

			      \llave{l}{
				      \productoria{i=1}{k} \frac{k+i}{2i-3} = \frac{k+1}{2 \cdot 1 - 3} \cdot \frac{k+2}{2 \cdot 2 - 3} \cdot \frac{k+3}{2 \cdot 3 - 3} \cdots \frac{2k}{2 \cdot k - 3} = 2^k(1 - 2k)\\
				      \productoria{i=1}{k+1} \frac{k+1+i}{2i-3} = \frac{k+2}{2 \cdot 1 - 3} \cdot \frac{k+3}{2 \cdot 2 - 3} \cdots \frac{k+1 + (k-1)}{2(k-1) - 3} \cdot \frac{k+1 + k}{2k - 3}\cdot \frac{k+1 + (k+1)}{2(k + 1) - 3} \\
				      \flecha{Masajear: multiplico por $1 = \frac{\cyan{k+1}}{\cyan{k+1}}$}[corro los denominadores una fracción hacia $\leftarrow$]
				      \frac{\cyan{k+1}}{\cyan{k+1}} \cdot \frac{1}{2 \cdot 1 - 3} \cdot \frac{k+2}{2 \cdot 2 - 3} \cdot \frac{k + 3}{2 \cdot 3 - 3} \cdots \frac{2 k}{2k - 3} \cdot \frac{2k + 1}{2(k+1) - 3}  \cdot \frac{2k+2}{1} = \\
				      \flecha{acomodo para que}[aparezca la HI]
				      \HI{
					      \textstyle \frac{\cyan{k+1}}{2 \cdot 1 - 3} \cdot \frac{k+2}{2 \cdot 2 - 3} \cdot \frac{k + 3}{2 \cdot 3 - 3} \cdots \frac{2 k}{2k - 3} \cdot \frac{2k + 1}{2(k+1) - 3}
				      }  \cdot \frac{2k+2}{\cyan{k+1}} =\\
				      = 2^k (1 -2k) \cdot \frac{2k + 1}{2(k+1) - 3}  \cdot \frac{2k+2}{\cyan{k+1}} = 2^k \cancel{(1 - 2k)} \cdot \frac{2k + 1}{(-1)\cancel{(1 - 2k)}}  \cdot \frac{2\cancel{(k + 1)}}{\cancel{\cyan{k + 1}}} = 2^{k+1}(-1)(2k + 1)=\\
				      = 2^{k+1} \parentesis{1 - 2(k+1)} \Tilde

			      }
		      }$
	      Como $p(1)$ es verdadero y $p(k)$ es verdadero y $p(k+1)$ también lo es, por el principio de inducción $p(n)$ es verdadera $\paratodo n\en \naturales $
\end{enumerate}

%8
\ejercicio
Sea $a,\, b \en \reales$. Probar que para todo $n \en \naturales,\, a^n - b^n = (a-b) \sumatoria{i = 1}{n}a^{i-1}b^{n-i}$. Deducir la fórmula de la serie
geométrica: para todo $a \neq 1,\, \sumatoria{i = 0}{n}a^i = \frac{a^{n+1} - 1}{a-1}$.\\
$\llave{ l }{
		\text{Primer paso: } n = 1 (a-b) \sumatoria{1}{1} a^{i-1} \cdot b^{1-1} = a - b = a^1 - b^1 \Tilde\\
		\text{Paso inductivo: } n = k  a^k - b^k = (a-b) \sumatoria{i = 1}{k} a^{i-1} \cdot b^{k-i} \entonces
		a^{k+1} - b^{k+1} \eq?= (a-b) \sumatoria{i = 1}{k+1} a^{i-1} \cdot b^{k+1-i}\\
		\llave{l}{
			(a-b) \sumatoria{i = 1}{k+1} a^{i-1} \cdot b^{k+1-i} =
			(a-b) \sumatoria{i = 1}{k} a^{i-1} \cdot \underbrace{b^{k+1-i}}_{b\cdot b^{k-i}} +
			\kmasuno{
				(a - b) a^k \cdot b^{k+1-(k+1)}
			} = \\
			b \cdot
			\HI{
				(a-b) \sumatoria{i = 1}{k} a^{i-1} \cdot b^{k-i}
			} + (a - b) a^k \eqHI= b \cdot a^k - b^k + (a - b) a^k = a^{k+1} - b^{k+1}\Tilde
		}\\
	}$\\
Para deducir la fórmula de la serie geométrica $b = 1 \to (a-1) \sumatoria{i=1}{n} a^{i-1} = a^n - 1 \to\\
	\llave{ l }{
		(a-1) \sumatoria{1}{n} a^{i-1} = 
        (a-1) \cdot (1 + a + a^2 + \cdots + a^{n-1})  = a^n - 1
        \flecha{multiplico por $a$ y}[divido por $(a - 1)$ M.A.M.]\\
		\sumatoria{1}{n} a^i = a + a^2 + \cdots + a^n  = \frac{a^{n+1} - a}{a-1}
        \flecha{sumo un $1$}[M.A.M.]
		\sumatoria{\magenta{i = 0}}{n} a^i = 1 + a + a^2 + \cdots + a^n   = \frac{a^n - a}{a-1} + 1 \to\\
		\boxed{ \frac{a^n + 1}{a-1} =  \sumatoria{i = 0}{n} a^i  }
	}$

%9
\ejercicio

%10
\ejercicio

%11
\ejercicio

%12
\ejercicio

%13
\ejercicio

%14
\ejercicio
Probar que para todo $n \geq 3$ vale que:
\begin{enumerate}[label=\roman*)]
	\item La cantidad de diagonales de un polígono de $n$ lados es $\frac{n(n-3)}{2}$.\\
	      Ejercicio donde hay que encontrar una fórmula a partir de algún método \textit{creativo} para luego probar por inducción.\\
	      \begin{tikzpicture}[scale=0.9]
		      \coordinate (n) at (0,0);
		      \coordinate (1) at (2,0);
		      \coordinate (2) at (3,1.5);
		      \coordinate (3) at (3.5,2.5);
		      \coordinate (k) at (2.5,3.8);
		      \coordinate (n-3) at (0.5,4);
		      \coordinate (n-2) at (-1,3.2);
		      \coordinate (n-1) at (-2,2);

		      \foreach \point/\position in {n/below left, 1/below, 2/right, 3/right, k/above right, n-3/above, n-2/above left, n-1/left}
			      {
				      \fill[OliveGreen] (\point) circle (3pt) node [\position] {$\point$};
				      \draw[dashed, magenta] (n) -- (\point);
			      }
		      \draw[dashed, green] (n-1) -- (1);
		      \draw[ultra thick, cyan] (n) -- (1) -- (2) -- (3) -- (k) -- (n-3) -- (n-2) -- (n-1) -- cycle;

		      \begin{scope}[xshift=8cm]
			      \coordinate (n) at (0,0);
			      \coordinate (1) at (2,0);
			      \coordinate (2) at (3,1.5);
			      \coordinate (3) at (3.5,2.5);
			      \coordinate (k) at (2.5,3.8);
			      \coordinate (n-3) at (0.5,4);
			      \coordinate (n-2) at (-1,3.2);
			      \coordinate (n-1) at (-2,2);
			      \foreach \point/\position in {n/below left, 1/below, 2/right, 3/right, k/above right, n-3/above, n-2/above left, n-1/left}
				      {
					      \fill[OliveGreen] (\point) circle (3pt) node [\position] {$\point$};
				      }
		      \end{scope}

		      \draw[ultra thick, red] (n-1) -- (1) -- (2) -- (3) -- (k) -- (n-3) -- (n-2) -- (n-1) -- cycle;

	      \end{tikzpicture}\\
	      Se desprende del gráfico el siguiente razonamiento: En el polígono \cyan{cyan} de $n$ lados voy a tener una cantidad de
	      diagonales dada por la sucesión $d_n$. El polígono \red{rojo} me genera polígono que tiene un lado menos y
	      un lado menos, cantidad que viene determinada por $d_{n-1}$. Las líneas punteadas son las diagonales de $d_n$ que no estarán en
	      $d_{n-1}$. Ahora voy a encontrar una relación entre ambas sucesiones. Al sacan un lado pierdo las \magenta{diagonales} desde $2$ hasta $n-2$ que
	      serían $n - 3$ en total y además pierdo la {\color{green}diagonal} que conectan el vértice $1$ con el $n-1$:
	      $\cyan{d_n} = \red{d_{n-1}} + \green{1} +\magenta{n-2} = d_{n-1} + n - 1$
	      $\to {d_n = d_{n-1} + n - 1}$\\
	      Ahora inducción:\\
	      $p(n): d_n = \frac{n(n-3)}{2} \paratodo n \geq 3\\
		      \llave{l}{
		      \textit{Caso Base: } p(3) \V? \to \frac{3(3-3)}{2} = 0,\text{ lo cual es verdad para el triángulo.\Tilde}\\
		      \textit{Paso inductivo: } p(k) \text{ es verdadero para algún } k \en \enteros_{\geq 3} \entonces p(k+1) \V?\\
		      \textit{Hipótesis Inductiva: } d_k = \frac{k(k-3)}{2} \entonces d_{k+1} = \frac{(k+1)(k-2)}{ 2 } \\
		      d={k+1} \eqDef = d_k + k-1 \eqHI = \frac{k(k-3)}{2} + k-1 = \frac{k^2 - k - 2}{2} = \frac{(k-2)(k+1)}{2} \Tilde\\
		      }\\
	      $
	      Como $p(3) \y p(k) \y p(k+1)$ resultaron verdaderas, por el principio de inducción $p(n)$ es verdadera $\paratodo n \en \naturales_{\geq3}$

	\item la suma de los ángulos interiores de un polígono de $n$ lados es $\pi(n-2)$.\\
	      \begin{tikzpicture}[scale=0.9]
		      \coordinate (n) at (0,0);
		      \coordinate (1) at (2,0);
		      \coordinate (2) at (3.2,1.5);
		      \coordinate (3) at (3.7,2.8);
		      \coordinate (k) at (2.5,3.8);
		      \coordinate (n-3) at (0.5,4);
		      \coordinate (n-2) at (-1,3.2);
		      \coordinate (n-1) at (-2,2);

		      \foreach \point/\position in {n/below left, 1/below, 2/right, 3/right, k/above right, n-3/above, n-2/above left, n-1/left}
			      {
				      \fill[OliveGreen] (\point) circle (3pt) node [\position] {$\point$};
			      }
		      \draw[dashed, green] (n-1) -- (1);
		      \draw[ultra thick, cyan] (n) -- (1) -- (2) -- (3) -- (k) -- (n-3) -- (n-2) -- (n-1) -- cycle;


		      \draw[dashed, green] (n-1) -- (1);
		      \draw[ultra thick, cyan] (n) -- (1) -- (2) -- (3) -- (k) -- (n-3) -- (n-2) -- (n-1) -- cycle;

		      \pic [draw, angle eccentricity=1.5] {angle = 1--n--n-1};
		      \pic [draw, angle eccentricity=1.5] {angle = 2--1--n};
		      \pic [draw, angle eccentricity=1.5] {angle = 3--2--1};
		      \pic [draw, angle eccentricity=1.5] {angle = k--3--2};
		      \pic [draw, angle eccentricity=1.5] {angle = n-3--k--3};
		      \pic [draw, angle eccentricity=1.5] {angle = n-2--n-3--k};
		      \pic [draw, angle eccentricity=1.5] {angle = n-1--n-2--n-3};
		      \pic [draw, angle eccentricity=1.5] {angle = n--n-1--n-2};

		      \pic [draw, red, angle eccentricity=1.5, angle radius=1cm] {angle = n-1--1--n};
		      \pic [draw, red, angle eccentricity=1.5, angle radius=0.7cm] {angle = 1--n--n-1};
		      \pic [draw, red, angle eccentricity=1.5, angle radius=1cm] {angle = n--n-1--1};
	      \end{tikzpicture}\\
	      En este caso estoy generando la suma de los ángulos internos de 2 polígonos, uno con $\alpha_n$ de $n$ lados y otro con $n-1, \alpha_{n-1}$
	      Es más claro en este caso que al sacarle un lado, estoy robádo un triángulo que tiene como suma de sus ángulos internos $\pi$, entonces afirmo
	      $\alpha_{n+1} = \alpha_n + \pi $. Ahora pruebo por inducción lo pedido.
	      $p(n): \alpha_n = \pi(n-2) \paratodo n \geq 3$\\
	      $ \llave{l}{
			      \textit{Caso Base: } p(3) \V? \to \pi(3-2) = \pi, \text{ lo cual es verdad para el triángulo.\Tilde}\\
			      \textit{Paso inductivo: } p(k) \text{ es verdadero para algún } k \en \enteros_{\geq 3} \entonces p(k+1) \V?\\
			      \textit{Hipótesis Inductiva: } \alpha_k = \pi(k-2) \entonces \alpha_{k+1} = \pi(k-1) \\
			      \llaves{l}{
				      \alpha_k \eqDef = \alpha_{k+1} - \pi\\
				      \alpha_k \eqHI= \pi(k-2)\\
			      } \to \alpha_{k+1} = \pi(k-2) + \pi = \pi(k-1)\Tilde
		      }$\\
	      Como $p(3) \y p(k) \y p(k+1)$ resultaron verdaderas, por el principio de inducción $p(n)$ es verdadera $\paratodo n \en \naturales_{\geq3}$\\
\end{enumerate}

\separador

\textit{Recurrencia}


%15
\ejercicio
\begin{enumerate}[label=\roman*)]
	\item
	      Sea $(a_n)_{n\ en \naturales}$ la sucesión de números reales definida recursivamente por:\\
	      $a_1 = 2$ y $a_{n+1} = 2 n a_n+ 2^{n+1}n!,\, \paratodo n \en \naturales$.\\
	      Probar que $a_n = 2^n n!$.\\
	      \magenta{Hecho en cuaderno, pasar}

	\item
	      Sea $(a_n)_{n\ en \naturales}$ la sucesión de números reales definida recursivamente por:\\
	      $a_1 = 0$ y $a_{n+1} = a_n + n(3n + 1),\, \paratodo n \en \naturales$\\
	      Probar que $a_n = n^2 (n-1)$.\\
	      \magenta{Hecho en cuaderno, pasar}

\end{enumerate}

%16
\ejercicio
Hallar la fórmula para el término general de las sucesiones $(a_n)_{n \en \naturales}$

%17
\ejercicio

\update
\end{document}
