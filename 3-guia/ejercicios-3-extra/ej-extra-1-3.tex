\begin{enunciado}{\ejExtra}
  Sea $\relacion \subseteq \partes(\naturales) \times \partes(\naturales)$ la relación de equivalencia
  $$
    X \relacion Y \sisolosi X \triangle Y \subseteq \set{4,5,6,7,8}.
  $$
  ¿Cuántos conjuntos hay en la clase de equivalencia de $X = \set{x \en \naturales : x \geq 6}$?

\end{enunciado}

Recordando que:

El conjunto $\partes{\naturales}$ es un conjunto que tiene todos los subconjuntos que me puedo formar con los
números naturales. El cardinal del conjunto $\partes(\naturales)$:
$$
  \#\partes(\naturales) = 2^{\#\naturales}.
$$

Una clase de equivalencia de $X$ se forma con los elementos de $\partes(\naturales)$ (acá son los $Y$) que estén relacionados con $X$, es
decir los pares $X \relacion Y$. Surge la pregunta:
\parrafoDestacado{
  \textit{¿Cómo debería ser un conjunto $Y$ para estar relacionado con $X$?}
}
\bigskip
Para esto me gusta pensar a la diferencia simétrica ($\triangle$) como:
\parrafoDestacado{
  \textit{Dados 2 conjuntos $A$ y $B$, su $A \triangle B$ es el conjunto formado por los elementos \ul{no comunes} entre $A$ y $B$}
}
Estos $Y$ que estoy buscando están sujetos a la condición:
$$
  X \triangle Y  \subseteq \set{4,5,6,7,8} \text{ con } X = \set{6,7,8,\cdots}
$$
Entonces un $Y$ \ul{no puede tener} los elementos 1, 2 y 3 porque \textit{rompería la relación}. Por otro lado $Y$ \ul{debe tener}
los elementos en $\set{n \en \naturales : n \geq 9}$, de manera de \textit{cancelar} todos esos valores de $X$, de
forma que $X \relacion Y \subseteq \set{4,5,6,7,8}$ y no tenga valores superiores a 8.

Con esta idea armo unos conjuntos $Z_i$
$$
  Z_i \en \partes(\set{4,5,6,7,8})\ \text{ con } i \en [1, 32],
$$

Se concluye que la clase de equivalencia será el conjunto, {\tiny(en \red{notación inventada})}:
$$
  \clase{Y}_1  =
  \bigg\{
  \ub{
    Z_1  \union \set{9,10,\dots}
  }{
    \clase{Y}_1
  },
  \ub{
    Z_2  \union \set{9,10,\dots}
  }{
    \clase{Y}_2
  },
  \dots,
  \ub{
    Z_{32}  \union \set{9,10,\dots}
  }{
    \clase{Y}_{32}
  }
  \bigg\}
  \quad
  \text{donde } \#\overline{X} = 2^5 = 32
$$
con los conjuntos:

En palabras:
\parrafoDestacado{
  Todos los $\clase{Y}_i$ tienen el $\set{9,10,\ldots}$ así que a la hora de contar, esa parte \ul{no aporta nada}, porque
  es igual en tooooodos los conjuntos $\clase{Y}_i$

  La papa está en armar todos los subconjuntos que pueda con los elementos $\set{4,5,6,7,8}$, los $Z_i$.
  Eso es equivalente a construir:
  $$
    \partes(\set{4,5,6,7,8}),
  $$
  que por definición es el conjunto con todos los subconjuntos de $\set{4,5,6,7,8}$.

  La cantidad de elementos de un conjunto $\partes(A)$ es $\#\partes(A) = 2^{\#A}$ es por eso
  que tengo
  $$
    \cajaResultado{
      32
    }
  $$
  conjuntos en la clase de equivalencias de $X$.
}

\begin{aportes}
  \item \aporte{\dirRepo}{naD GarRaz \github}
\end{aportes}
