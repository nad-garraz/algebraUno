\begin{enunciado}{\ejExtra}
  Sea $\relacion \subseteq \partes(\naturales) \times \partes(\naturales)$ la relación de equivalencia
  $$
    X \relacion Y \sisolosi X \triangle Y \subseteq \set{4,5,6,7,8}.
  $$
  ¿Cuántos conjuntos hay en la clase de equivalencia de $X = \set{x \en \naturales : x \geq 6}$?

\end{enunciado}

Bajo el enunciado a:
\begin{enumerate}
  \item La relación toma valores de $\partes(\naturales)$

  \item Los elementos del conjunto $\relacion \subseteq \partes(\naturales) \times \partes(\naturales)$

  \item El conjunto
        $$
          X = \set{\foreach \i in {6,...,10}{\i, }\dots}
        $$
        es simplemente un elemento de $\partes(\naturales)$. Los conjuntos $Y \en \partes(\naturales)$ tales que $X \relacion Y$
        van a ser los conjuntos que junto a $X$ formarán la clase de equivalencia.

        $ \clase{X} = \set{Y \en \partes(\naturales) : X \relacion Y}$
\end{enumerate}

Para tener una relación de equivalencia deben cumplirse:
\begin{itemize}
  \item \textit{Reflexividad:}
        $$
          X \triangle X = \vacio \stackrel{\checkmark}\subseteq \set{4,5,6,7,8}
        $$

  \item \textit{Simetría:}
        $$
          X \triangle Y \igual{\checkmark} Y \triangle X,\, \paratodo X,Y \in \partes(\naturales)
        $$
  \item \textit{Transitividad:}

        \hacer
\end{itemize}

Nos piden que encontremos la cantidad de conjuntos que hay en la clase de equivalencia de $X$, eso
es lo mismo que buscar el cardinal del conjunto.

Las clases son conjuntos formados por los pares $X \relacion Y$. Entonces tenemos que pensar ¿Cómo es que tiene que
ser un conjunto $Y$ para relacionarse con $X$?

\bigskip

Para esto me gusta pensar a la diferencia simétrica como: \textit{El conjunto formado por los elementos no comunes de los conjuntos:}.
Este conjunto:
$$
  X \triangle Y  \subseteq \set{4,5,6,7,8}
$$

El conjunto $Y$ \ul{no puede tener} los elementos, 1, 2 y 3, porque \textit{rompo la relación}. Por otro lado $Y$ \ul{debe tener}
los elementos en $\set{n \en \naturales : n \geq 9}$, de manera de \textit{cancelar} todos esos valores de $X$, de
forma que $X \relacion Y \subseteq \set{4,5,6,7,8}$ y no tenga valores superiores a 8.

Se concluye que la clase de equivalencia será el conjunto, {\tiny(en \red{notación inventada})}:
$$
  \clase{X}  =
  \bigg\{
  \ub{
    Y_1  \union \set{9,10,\dots}
  }{
    \clase{X}_1
  },
  \ub{
    Y_2  \union \set{9,10,\dots}
  }{
    \clase{X}_2
  },
  \dots,
  \ub{
    Y_{32}  \union \set{9,10,\dots}
  }{
    \clase{X}_{32}
  }
  \bigg\}
  \quad
  \text{donde } \#\overline{X} = 2^5 = 32
$$
con los conjuntos:
$$
  Y_i \en \partes(\set{4,5,6,7,8})\ \text{ con } i \en [1, 32]
$$

En palabras:
\parrafoDestacado{
  Todos los $\clase{X}_i$ tienen el $\set{9,10,\ldots}$ así que a la hora de contar, esa parte \ul{no aporta nada}, porque
  es igual en tooooodos los conjuntos $\clase{X}_i$

  La papa está en armar todos los subconjuntos que pueda con los elementos $\set{4,5,6,7,8}$, los $Y_i$.
  Eso es equivalente a construir:
  $$
    \partes(\set{4,5,6,7,8}),
  $$
  que por definición es el conjunto con todos los subconjuntos de $\set{4,5,6,7,8}$.

  La cantidad de elementos de un conjunto $\partes(A)$ es $\#\partes(A) = 2^{\#A}$
}

\begin{aportes}
  \item \aporte{\dirRepo}{naD GarRaz \github}
\end{aportes}
