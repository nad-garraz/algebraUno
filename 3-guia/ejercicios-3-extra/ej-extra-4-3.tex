\begin{enunciado}{\ejercicio}
        Hallar la cantidad de números naturales de exactamente 20 dígitos (o sea que no empiezan con 0) que se pueden
        formar con los dígitos 0, 2, 3 y 9 y que cumplen que la suma de los 7 últimos dígitos es igual a 6.
\end{enunciado}

El primer dígito puede valer solo 2, 3 o 9, es decir 3 opciones.
Del dígito 19 al dígito octavo puede valer solo 0, 2, 3 o 9, es decir 4 opciones. 
Para sumar 6 con los números que puedo usar, solo tengo 2+2+2 y 3+3:
En los últimos 7 dígitos tengo $\binom{7}{2} + \binom{7}{3}$ opciones

TODO:
HACER ESTO AGADABLE
