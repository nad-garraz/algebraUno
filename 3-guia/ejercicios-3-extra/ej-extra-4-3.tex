\begin{enunciado}{\ejExtra}
  Hallar la cantidad de números naturales de exactamente 20 dígitos (o sea que no empiezan con 0) que se pueden
  formar con los dígitos 0, 2, 3 y 9 y que cumplen que la suma de los 7 últimos dígitos es igual a 6.
\end{enunciado}

Un número de 20 dígitos, donde solo puedo poner, 0, 2, 3 o 9:

\begin{enumerate}[label=\purple{\faIcon{calculator}$_{\arabic*)}$}]
  \item El dígito más \textit{significativo}, digamos el vigésimo, tiene \blue{3} posibles valores 2, 3 o 9.

  \item Los dígitos $\ub{\text{desde el décimo noveno hasta el octavo}}{\orange{12} \text{ dígitos en total}}$ pueden tomar \orange{4} posibles valores, 0, 2, 3 o 9.

  \item Los últimos dígitos, del séptimo hasta el primero tienen que sumar 6. Solo es posible eso haciendo cosas de la pinta:
        $$
          \dots \ub{0. 0 0 0. 2 2 2}{\text{\magenta{3} bolitas en \magenta{7} cajitas}}
          \quad \otext \quad
          \dots \ub{0. 0 0 0 .0 3 3}{\text{\magenta{2} bolitas en \magenta{7} cajitas}}
        $$
\end{enumerate}

Con toda esa info, la cantidad de números de 20 cifras sería::
$$
  \cajaResultado{
    \blue{3} \cdot
    \ub{
      \orange{4}\cdot
      \orange{4}\cdot
      \orange{4}\cdot
      \orange{4}\cdot
      \orange{4}\cdot
      \orange{4}\cdot
      \orange{4}\cdot
      \orange{4}\cdot
      \orange{4}\cdot
      \orange{4}\cdot
      \orange{4}\cdot
      \orange{4}\cdot
    }{\orange{{4}^{12}}}
    \left(
    \binom{\magenta{7}}{\magenta{3}}
    +
    \binom{\magenta{7}}{\magenta{2}}
    \right)
  }
$$

\begin{aportes}
  \item \aporte{\dirRepo}{naD GarRaz \github}
  \item \aporte{\neverGonnaGiveYouUp}{Santino \youtube}
\end{aportes}
