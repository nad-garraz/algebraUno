\begin{enunciado}{\ejExtra}
  ¿Cuántas funciones $f: \set{1,2,\dots,10}\to \set{1,2,\dots,12}$ hay que \textbf{no} sean inyectivas y que
  al mismo tiempo cumplan que $f(1) < f(3) < f(5)$
\end{enunciado}

\textit{La receta:}
\begin{enumerate}[label=\arabic*)]
  \item Calcular tooodas las funciones que cumplan $f(1) < f(3) < f(5)$.
  \item Calcular todas las funciones \underline{inyectivas} que también cumplan $f(1) < f(3) < f(5)$.
  \item Restar los resultados obtenidos da lo pedido en el enunciado.
\end{enumerate}

\textit{A cocinar:}

\begin{enumerate}[label=\arabic*)]
  \item
        Entonces agarro 3 elementos del conjunto de llegada $\set{1,2,\dots,12}$ sin preocuparme por nada. En el conjunto
        hay un total de 12 elementos agarro 3 sin mirar \faIcon{blind}:
        $$
          \binom{12}{3} = \frac{12!}{9! \cdot 3!}
        $$
        Este número combinatorio me cuenta las distintas formas de sacar 3 elementos cualesquiera de un conjunto de 12 elementos.

        Si te hace ruido o pensás ¿Cómo sé que esto cumple las desigualdades? Podés pensar que todos los elementos son distintos y es imposible que elijas
        3 elementos $x_1, x_2, x_3$ y que esos elementos \textit{no cumplan que no sea mayor que otro o coso \href{\mindExplosion}{\rosa{\faIcon[regular]{brain}}}}.

        Una vez seleccionados estos \red{3} elementos para cumplir $f(1) < f(3) < f(5)$, no me importa que hago con los otros elementos restantes
        así que agarro:
        $$
          12^{10-\red{3}}
        $$
        Tengo entonces un total de:
        $$
          12^{10-\red{3}} \cdot \binom{12}{3} = 12^7 \cdot \binom{12}{3} \llamada1
        $$
        funciones que cumplirían que $f(1) < f(3) < f(5)$.

  \item
          Para calcular ahora  \textit{las funciones inyectivas} tenemos en cuenta que hay que agarrar 10 números de $\set{1,2,\dots,10}$ y mandarlos a 12 números
                de $\set{1,2,\dots,12}$. Esto con la restricción $f(1) < f(3) < f(5)$ (cálculo ya hecho), que me saca \red{3} elementos:
        $$
          \binom{12}{3} \cdot \frac{(12 - \red{3})!}{\big((12 - \red{3}) - (10 - \red{3})\big)!}
                =
                \binom{12}{3} \cdot \frac{9!}{2!}\llamada2
        $$

  \item
        Para calcular el número de funciones \textbf{no} inyectivas que cumplen la restricción restamos $\llamada1$ y $\llamada2$:
        $$
          \#funciones = \binom{12}{3} \cdot (12^7 - \frac{9!}{2!})
        $$
\end{enumerate}

\begin{aportes}
  \item \aporte{https://github.com/JowinTeran}{Ale Teran \github}
  \item \aporte{https://github.com/nad-garraz}{Nad Garraz \github}
\end{aportes}
