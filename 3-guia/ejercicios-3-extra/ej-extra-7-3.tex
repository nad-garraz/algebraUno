\begin{enunciado}{\ejExtra}

  Calcule la cantidad de anagramas de 6 letras que se pueden obtener usando las letras de la palabra

  PARTICULA, con la condición de que empiecen con C.
  Por ejemplo, posibles anagramas son CAPTAR y CRIPTA.
\end{enunciado}

Antes de hacer este ejercicio, si estás medio que no entendés el tema, mirate el ejercicio \ref{ej:23}{\tiny($\ot$ click acá)}.

\medskip

Hay que calcular anagrámas en una palabra \magenta{P\blue{A}RTICUL\blue{A}} con 9 letras, donde hay \magenta{8} posibles
valores, dado que \blue{2} son repetidas. A mí me gusta esta forma de resolver esto:

\begin{enumerate}[label=\faIcon{calculator}$_{(\arabic*)}$]
  \item Primero ubico la C:
        $$
          \llave{c c c c c c} {
            \text{C} & \_ & \_ & \_ & \_ & \_ \\
            1 & 2 & 3 & 4 & 5 & 6
          },
        $$
        Dado que la C solo va a estar ahí hay \ul{1} sola forma de hacer esto.

  \item Segundo cuento las versiones con 2 valores de $A$. Un ejemplo de eso sería:
        $$
          \llave{c c c c c c} {
            \text{C} & \blue{\text{A}} & \_ & \blue{\text{A}} & \_ & \_ \\
            1 & 2 & 3 & 4 & 5 & 6
          },
        $$
        En este caso tengo 5 lugares para poner estas \blue{2 A}. En total hay
        $$
          \binom{5}{\blue{2}}
        $$
        formas de hacer eso

  \item Por último tengo que ubicar las 3 de las \magenta{5} letras restantes $\set{\text{\magenta{P,R,T,I,U,L}}}$.
        Eso lo hacemos \textit{inyectando}, por ejemplo así:
        $$
          \llave{c c c c c c} {
            \text{C} & \blue{\text{A}} & \magenta{\text{P}} & \blue{\text{A}} & \magenta{\text{R}} & \magenta{\text{T}} \\
            1 & 2 & 3 & 4 & 5 & 6
          },
        $$
        Lo cual se podrá hacer de:
        $$
          6 \cdot 5 \cdot 4 =
          \frac{6!}{(6-3)!} =
          \frac{6!}{3!} = 120
        $$
        formas distintas.
\end{enumerate}
Por lo tanto para este caso en el que usamos 2 \blue{A} en cada palabra tenemos un total de:
$$
  \cajaResultado  {
  \text{\#ANAGRAMAS}_{\text{\blue{AA}} } =  1 \cdot \binom{5}{2} \cdot \frac{6!}{3!}
  }
$$

\bigskip

Esto mismo hay que hacerlo para el caso en que las palabras formadas tienen \underline{solo una \blue{A}} y después en el caso en que tengan
\underline{ninguna \blue{A}}. Eso no lo desarrollo, porque \textit{pajilla}, peeeero, como soy un tipazo te pongo lo que me dio así corroborás
y si lo hice mal yo y no me avisás te condeno \red{\faIcon[regular]{angry}} a recursar eternamente esta materia \red{\faIcon[regular]{angry}}.
\begin{center}
  {\tiny En el mismo párrafo pasé de ser un tipazo a un terrible hdp}.
\end{center}
$$
  \cajaResultado  {
  \text{\#ANAGRAMAS}_{\text{\blue{AA}} }  +
  \text{\#ANAGRAMAS}_{\text{\blue{A}} }  +
  \text{\#ANAGRAMAS}
  =
  1 \cdot \binom{5}{\blue{2}} \cdot \frac{6!}{3!} +
  1 \cdot \binom{5}{\blue{1}} \cdot \frac{6!}{2!} +
  1 \cdot \binom{5}{\blue{0}} \cdot \frac{6!}{1!}
  }
$$

\begin{aportes}
  \item \aporte{\dirRepo}{naD GarRaz \github}
\end{aportes}
