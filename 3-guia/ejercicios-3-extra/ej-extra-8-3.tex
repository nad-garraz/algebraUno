\begin{enunciado}{\ejExtra[recu 10/12/2024]}\fechaEjercicio{recuperatorio 10/12/2024}

  ¿Cuántos anagramas tiene la palabra $LOLLAPALOOZA$ tales que
  \begin{enumerate}[label=$\alph*)$]
    \item empiezan en consonante?
    \item no tienen dos vocales ni dos consonantes juntas?
  \end{enumerate}
\end{enunciado}

\begin{enumerate}[label=$\alph*)$]
  \item Tengo 6 consonantes $LLLPLZ \to \set{L,P,Z}$,  y 6 vocales $OAAOOA \to \set{O,A}$. Para ubicar una consonante en la primera
        ubicación, puedo poner una $L$, una $P$ o una $Z$. En el caso de usar la $\blue{L}$ voy a estar \textit{modificando el \textit{stock} de $\blue{L}$'s
          que quedan para el resto de la palabra}. Marco en \blue{azul} lo que es referente a las $\blue{L}$'s:
        $$
          \ub{
            \blue{1} \cdot \frac{11!}{\blue{3}! \cdot 3! \cdot 3!}
          }{
            \text{Uso una $\blue{L}$}\\
            \text{y completo}
          }
          +
          \ob{
            2 \cdot \frac{11!}{\blue{4}! \cdot 3! \cdot 3!}
          }{
            \text{Uso una $P$ o}\\
            \text{una $Z$ y completo}
          }
          = 277200
        $$

        El 2 en el segundo término es porque puedo elegir $P$ o $Z$ como primera consonante y después me quedan \blue{4} $\blue{L}$'s y vocales
        para completar el resto de las palabras. En el término donde uso la $\blue{L}$ tengo que recordar que solo quedan $\blue{3}$ luego para
        la parte de completar.

        \bigskip

        También se puede atacar el problema contando todas las palabras posibles y restarle las palabras que empiezan con una vocal:
        $$
          \ub{
            \frac{12!}{4! \cdot 3! \cdot 3!}
          }{
            \text{todas las}\\
            \text{posibles palabras}\\
            \text{distintas}
          }
          -
          \ob{
            2 \cdot (\frac{11!}{4! \cdot 3! \cdot 2!})
          }{
            \text{Uso una $A$ o}\\
            \text{una $O$ y completo}
          }
          = 277200
        $$

        Dio igual, así que me quedo tranquilo. Y si hay algo mal avisá!

  \item Quiero armar palabras así:
        $$
          \begin{array}{cc}
            V C V C V C V C V C V C & \to ~\text{Empiezan con consonante} \\
            C V C V C V C V C V C V & \to ~\text{Empiezan con vocal}
          \end{array}
        $$
        Para completar ese bosquejo del ejercicio, puedo pensar al problema de completar las vocales y las consonantes por separado.

        Poner las 6 consonantes en sus respectivos lugares contemplando las repeticiones me da:
        $$
          \frac{6!}{4!} = 30
        $$

        Poner las 6 vocales en sus respectivos lugares contemplando las repeticiones me da:
        $$
          \frac{6!}{3! \cdot 3!} = 20
        $$

        Es decir que por cada una de las 30 \textit{ubicaciones} de las consonantes tengo 20 \textit{ubicaciones} para las vocales:
        $$
          20 \cdot 30 = 600
        $$
        Multiplicando por 2, debido a que la palabra puede empezar con vocal o consonante se obtienen:
        $$
          \cajaResultado{
            1200
          }
        $$
        anagramas de $LOLLAPALOOZA$ que no tienen consonantes y vocales juntas.
\end{enumerate}

\begin{aportes}
  \item \aporte{\dirRepo}{naD GarRaz \github}
\end{aportes}
