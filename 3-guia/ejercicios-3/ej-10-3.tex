\begin{enunciado}{\ejercicio}
  Sean $A = \set{1,2,3,4,5}$ y $B = \set{1,2,3,4,5,6,7,8,9,10,11,12}$. Sea $\mathcal F$ el conjunto de todas las funciones
  $f: A \to B$.
  \begin{enumerate}[label=\roman*)]
    \item  ¿Cuántos elementos tiene el conjunto $\F$?
    \item  ¿Cuántos elementos tiene el conjunto $\set{f \en \F : 10 \not\en \im(f)}$?
    \item  ¿Cuántos elementos tiene el conjunto $\set{f \en \F : 10 \en \im(f)}$?
    \item  ¿Cuántos elementos tiene el conjunto $\set{f \en \F : f(1) \en \set{2,4,6} }$?
  \end{enumerate}
\end{enunciado}

Cuando se calcula la cantidad de funciones, haciendo el árbol se puede ver que va a haber
$\#\cod(f)$ de funciones que provienen de un elemento del dominio.

Por lo tanto si tengo dos conjuntos $A_n$ y $B_m$, con $\#(A) = n \ytext \#(B) = m$ la cantidad de funciones $f : A \to B$
será de $m^n$

\begin{enumerate}[label=\roman*)]
  \item Con lo expuesto ahí arriba el total, \textit{totalísimo} de funciones que me puedo armar sería:
        $$
          \# \F = 12^5 \llamada1,
        $$
        dado que tengo 12 opciones para cada elemento del $\dom(f)$.

  \item Parecido al anterior pero ahora el codominio no tiene el 10, entonces las posibilidades de formar funciones se achican:
        $$
          \#\set{f \en \F : 10 \not\en \im(f)} = 11^5 \llamada2
        $$
        dado que tengo 11 elementos para elegir para cada elemento del $\dom(f)$

  \item Para calcular esto lo pensamos con el complemento. Le sacamos al total de funciones $f$ las funciones que no tienen al 10 en su imagen:
        $$
          \#\set{f \en \F : 10 \en \im(f)}
          \igual{$\llamada1$}[$\llamada2$]
          12^5 - 11^5
        $$
        acá no se desperdicia nada, reciclando los resultados encontrados antes ahí están las funciones buscadas.

  \item Para atacar el problema está bueno pensarlo en 2 partes:
        \begin{enumerate}[label=\faIcon{calculator}$_{(\arabic*)}$]
          \item Cumplir $f(1) \en \set{2,4,6}$
          \item Contar lo que queda sin hacer \poo
        \end{enumerate}
        En el arte de escribir sin decir nada, empiezo por el principio:
        \begin{enumerate}[label=\faIcon{calculator}$_{(\arabic*)}$]
          \item Para que una función esté bien definida, siempre hay que agarrar todos los elementos de su dominio.
                En este caso me piden que $f(1) \en \set{2,4,6}$ por lo que tengo:
                $$
                  \binom{3}{1} = 3
                $$
                Lo cual no es una locura, dado que tengo solo 3 opciones:
                $$
                  \llave{rcl}{
                    f(1) &=& 2 \\
                    &\text{ o }&\\
                    f(1) &=& 4 \\
                    &\text{ o }&\\
                    f(1) &=& 6
                  }
                $$

          \item Ahora a contar sin... bueh \poo.
                Acá no hay que tener en cuenta que esta función no tiene restricciones, como ser inyectiva o yo que sé, es decir que
                para calcular todas las demás funciones que podemos formar, vamos a tener algo así:
                $$
                  A' = \set{2,3,4,5} \ytext B = \set{1,2,3,4,5,6,7,8,9,10,11,12}
                  \entonces
                  \#A = 4 \ytext \#B = 12
                $$
                Y contamos como al principio: $12^4$.

                \bigskip

                Por lo tanto el total de funciones que cumplen lo pedido será:
                $$
                  \cajaResultado{
                    \#\set{f \en \F : f(1) \en \set{2,4,6}} = 3 \cdot 12^4
                  }
                $$
        \end{enumerate}
\end{enumerate}

\begin{aportes}
  \item \aporte{https://github.com/Diego-Dev-Moros}{Diego Moros \github}
  \item \aporte{\dirRepo}{naD GarRaz \github}
\end{aportes}
