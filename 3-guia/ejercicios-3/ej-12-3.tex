\begin{enunciado}{\ejercicio}
  ¿Cuántos números de 5 cifras distintas se pueden armar usando los dígitos del 1 al 5?
  ¿ Y usando los dígitos del 1 al 7? ¿ Y usando los dígitos del 1 al 7 de manera que el dígito de las centenas no sea el 2?
\end{enunciado}

\begin{enumerate}[label=\arabic*)]
  \item Hay que usar $\set{1,2,3,4,5}$ y reordenarlos de todas las formas posibles. $5!$

  \item\label{ej-12:item_b} Hay que usar $\set{1,2,3,4,5,6,7}$ y ver de cuantas formas posibles pueden ponerse en 5 lugares:
        $$
          \llave{c c c c c}{
            \_ & \_ & \_ & \_ & \_ \\
            1  & 2  & 3  & 4  & 5
          },
        $$
        dado que no puedo repetir, a medida que voy llenando los valores, me voy quedando cada vez con menos elementos
        para elegir del conjunto, por lo tanto queda algo así:
        $$
          \llave{c c c c c}{
            \#7        & \#6        & \#5        & \#4        & \#3        \\
            \downarrow & \downarrow & \downarrow & \downarrow & \downarrow \\
            \_         & \_         & \_         & \_         & \_         \\
            1          & 2          & 3          & 4          & 5
          }
        $$
        Tengo:
        $$
          \cajaResultado{
            7 \cdot 6 \cdot 5 \cdot 4 \cdot 3 = \frac{7!}{2!}
          }
        $$

  \item Parecido al anterior pero fijo el 2 en el dígito de las centenas para encontrar los que
        \underline{sí} tienen el 2 en la cifra de las centenas

        $$\llave{c c c c c}{
            \#6        & \#5        & \#1        & \#4        & \#3        \\
            \downarrow & \downarrow & \downarrow & \downarrow & \downarrow \\
            \_         & \_         &  2        & \_          & \_         \\
            1          & 2          & 3          & 4          & 5
          }
        $$
        Tengo:
        $$
          6 \cdot 5 \cdot 1 \cdot 4 \cdot 3 = \frac{6!}{2!}
        $$
        Lo que conseguí es contar \underline{todos los números de 5 cífras que tienen al 2 en el dígito de las centenas}.
        Y ahora se lo resto al total que conseguí en el item \ref{ej-12:item_b}
        $$
          \cajaResultado{
            \quad  \ub{\displaystyle\binom{7!}{2!}}{\text{\tiny todos}} -
            \ob{\displaystyle\binom{6!}{2!}}{\substack{\text{\tiny 2 en dígito}\\\text{\tiny centena}}}  = 2160
          }
        $$
\end{enumerate}

\begin{aportes}
  \item \aporte{\dirRepo}{naD GarRaz \github}
  \item \aporte{https://github.com/olivportero}{Oliv Portero \github}
\end{aportes}
