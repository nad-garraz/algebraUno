\begin{enunciado}{\ejercicio}
  Sean $A = \set{1,2,3,4,5,6,7}$ y $B = \set{1,2,3,4,5,6,7,8,9,10}$.
  \begin{enumerate}[label=\roman*)]
    \item ¿Cuántas funciones inyectivas $f: A \to B$ hay?
    \item ¿Cuántas de ellas son tales que $f(1)$ es par?
    \item ¿Y cuántas tales que $f(1)$ y $f(2)$ son pares?
  \end{enumerate}
\end{enunciado}

\begin{enumerate}[label=\roman*)]
  \item Una pregunta equivalente a si tengo 10 pelotitas distintas y 7 cajitas cómo puedo ordenarlas.
        $$
          \llave{c c c c c c c}{
            \#10       & \#9        & \#8        & \#7        & \#6        & \#5        & \#4        \\
            \downarrow & \downarrow & \downarrow & \downarrow & \downarrow & \downarrow & \downarrow \\
            f(1)       & f(2)       & f(3)       & f(4)       & f(5)       & f(6)       & f(7)
          }
          \to
          \cajaResultado{
            \frac{10!}{3!}
          } = \magenta{\frac{\#B}{\#B - \#A} }
        $$

  \item Hay 5 números pares para elegir como imagen de $f(1)$:
        $$
          \llave{c c c c c c c}{
            \#5        & \#9        & \#8        & \#7        & \#6        & \#5        & \#4        \\
            \downarrow & \downarrow & \downarrow & \downarrow & \downarrow & \downarrow & \downarrow \\
            f(1)       & f(2)       & f(3)       & f(4)       & f(5)       & f(6)       & f(7)
          }
          \to
          \cajaResultado{
            5 \cdot \frac{9!}{3!}
          }
        $$

  \item Hay 5 números pares para elegir como imagen de $f(1)$, luego habrá 4 números pares para $f(2)$\\
        $$
          \llave{c c c c c c c}{
            \#5        & \#4        & \#8        & \#7        & \#6        & \#5        & \#4        \\
            \downarrow & \downarrow & \downarrow & \downarrow & \downarrow & \downarrow & \downarrow \\
            f(1)       & f(2)       & f(3)       & f(4)       & f(5)       & f(6)       & f(7)
          }
          \to
          \cajaResultado{
            5 \cdot 4 \cdot \frac{8!}{3!}
          }
        $$
\end{enumerate}

\begin{aportes}
  \item \aporte{\dirRepo}{naD GarRaz \github}
\end{aportes}
