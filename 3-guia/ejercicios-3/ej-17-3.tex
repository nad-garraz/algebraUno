\begin{enunciado}{\ejercicio}

  \begin{enumerate}[label=\roman*)]
    \item ¿Cuántos subconjuntos de 4 elementos tiene el conjunto
          $\set{\foreach \j in {1,...,6}{ \j, }7} $
    \item ¿Y si se pide que 1 pertenezca al subconjunto?
    \item ¿Y si se pide que 1 no pertenezca al subconjunto?
    \item ¿Y si se pide que 1 o 2 pertenezca al subconjunto, pero no simultaneamente los dos?
  \end{enumerate}

\end{enunciado}

El problema de tomar $k$ elementos de un conjunto de $n$ elementos se calcula con
$\binom{n}{k} = \frac{n!}{k!(n-k)!}$

\begin{enumerate}[label=\roman*)]
  \item\label{ej-17:item-i}  Tengo 7 elementos de donde elegir y tengo que tomar 4 al azar:
        $$\binom{7}{4} =
          \frac{7!}{4! \cdot (7-4)!} =
          \cajaResultado{
            35
          }
          .
        $$

  \item\label{ej-17:item-ii}
        Ahora tengo que formar cosas de la pinta: $\set{1,\text{\faIcon[regular]{smile}}, \text{\meh}, \text{\angry}}$.
        Tengo entonces un total de 6 elementos para elegir:
        $$
          \binom{6}{3} = \frac{6!}{3!\cdot 3!} =
          \cajaResultado{
            20
          }
          .
        $$
        Recordar que uno está armando conjuntos y pensar en elementos repetidos dentro del conjunto no tiene sentido, onda no puedo elegir otra vez al que
        ya viene por defecto.

  \item Es como si ahora el conjunto de donde elijo los valores sea: $\set{2,3,4,5,6,7}$. Tengo 6 elementos para elegir:
        $$
          \binom{6}{4} = \frac{6!}{4!\cdot 2!} =
          \cajaResultado{
            15
          }.
        $$
        También se puede encarar este tipo de problemas, pensando en sacarle a \underline{todos} los conjuntos que me puedo formar,
        la cantidad de elementos que tengan 1, usando así los resultados del item \ref{ej-17:item-ii} y del item \ref{ej-17:item-i}
        $$
          \binom{7}{4} - \binom{6}{4} = 35 - 20 =
          \cajaResultado{
            15
          }.
        $$

  \item
        Ahora tengo que formar cosas de la pinta: $\set{1,\text{\faIcon[regular]{smile}}, \text{\meh}, \text{\angry}}$ donde no puede
        estar el elemento 2, es decir que tengo para elegir entre $\set{3,4,5,6,7}$ y lo mismo para
        $\set{2,\text{\simpleicon{mcdonalds}}, \text{\simpleicon{burgerking}}, \text{\simpleicon{kfc}}}$ donde no puede
        estar el elemento 1, es decir que tengo para elegir entre $\set{3,4,5,6,7}$.
        $$
          \binom{5}{3} +
          \binom{5}{3} =
          2 \cdot \binom{5}{3} =
          \cajaResultado{
            20
          }.
        $$
\end{enumerate}

\begin{aportes}
  \item \aporte{\dirRepo}{naD GarRaz \github}
  \item \aporte{https://github.com/olivportero}{Oliv Portero \github}
\end{aportes}
