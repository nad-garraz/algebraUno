\begin{enunciado}{\ejercicio}
	¿Cuántos anagramas tienen las palabras \textit{estudio, elementos} y \textit{combinatorio}?
\end{enunciado}

El anagrama equivale a permutar los elementos del conjunto, en este caso las letras de las palabras.

Si no hay letras repetidas es una biyección $(\#(letras))!$, por ejemplo la palabra \textit{estudio} tiene
$$
	\cajaResultado{
            (\#\set{e,s,t,u,d,i,o}) ! = 7!
	}
$$
anagramas en total.

\bigskip

\textit{Elementos:}

Tiene 3 letras \textit{\blue{e}}, por lo tanto los elementos

no repetidos son 6:
$$
	\set{l,m,n,t,o,s}.
$$

Voy a realizar una \textit{inyección}:
\begin{itemize}
	\item Primero ubico lo que \underline{no} está repetido.
	\item Luego agrego, en una dada posición, a esos 3 o más elementos repetidos. Esto
	      no altera el conteo. Pensar que la palabra: \textit{lmntos\blue{eee}}
	      cuenta como \textit{lmntos\blue{\_}\blue{\_}\blue{\_}}.
\end{itemize}
Por lo tanto:
$$
	\cajaResultado{
		\frac{9!}{(9-6)!} = \frac{9!}{3!}
	}
$$
son todos los posibles anagramas de la palabra \textit{elementos}.

Otra forma de pensarlo, con combinatoria:
\begin{itemize}
	\item Primero ubico a las 3 letras \textit{\blue{e}}, por ejemplo:
	      $$
		      \llave{c c c c c c c c c} {
			      \blue{e} & \_ & \blue{e} & \_ & \blue{e} & \_ & \_ & \_ & \_\\
			      1 & 2 & 3 & 4 & 5 & 6 & 7 & 8 & 9
		      },
	      $$
	      donde esta es solo una de un total de
	      $$
		      \binom{9}{3}
	      $$
	      formas de hacer eso, y los elementos que quedan en el conjunto de letras se \textit{inyectan}.

	\item
	      Luego en los lugares vacíos que quedan, en este caso tengo 6 elementos \textit{distintos}
	      para ubicar en 6 lugares, lo que sería una biyección:
	      $$
		      \#\set{l,m,n,t,o,s}! =  6!
	      $$
\end{itemize}
Finalemente quedan:
$$
	\cajaResultado{
		\binom{9}{3} \cdot 6!  = \frac{9!}{3!}
	}
$$


\bigskip

\textit{Combinatorio}:

Tiene repetidas las letras \textit{\blue{i}} (x2) y la \textit{\blue{o}} (x3).
Tengo un conjunto de 7 elementos distintos:
$$
	\set{c,m.b,n,a,t,r}.
$$
Puedo ubicar las letras con en número combinatorio en 12 lugares \textit{\blue{o}} y luego las
\textit{\blue{i}} en los 9 lugares restantes.
Una vez hecho eso puedo \textit{inyectar (biyectar?)} las letras no repetidas restantes:
$$
	\cajaResultado{
		\binom{12}{3} \cdot \binom{9}{2} \cdot 7! =
		\normalsize \frac{12!}{3! \cdot 2!}
		= \frac{\foreach \i in {12,11,...,5}{\i \cdot}4}{2} = 39.916.800
	}
$$
\textit{Notar que:} Ese número que quedó es el total de biyecciones dividido entre las cantidades
de repeticiones de los elementos en cuestión.

\begin{aportes}
	\item \aporte{\dirRepo}{naD GarRaz \github}
\end{aportes}
