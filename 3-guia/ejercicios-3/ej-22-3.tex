\begin{enunciado}{\ejercicio}
  ¿Cuántas palabras se pueden formar permutando las letras de $cuadros$
  \begin{enumerate}[label=\roman*)]
    \item con la condición de que todas las vocales estén juntas?
    \item con la condición de que las consonantes mantengan el orden relativo original?
    \item con la condición de que nunca haya dos (o más) consonantes juntas?
  \end{enumerate}
\end{enunciado}

El conjunto de consonantes es $C = \set{c,d,r,s}$ y de vocales $V = \set{u,a,o}$
\begin{enumerate}[label=\roman*)]
  \item  Para que las vocales estén juntas pienso a las 3 como un solo elemento, fusionadas las 3 letras, con sus permutaciones,
        es decir que tengo 3! cosas de la siguiente pinta:\\
        $$
          \llave{c c c} % hay 3 en total
          {
            u & a & o \\
            u & o & a \\
            o & a & u \\
            o & u & a \\
            a & o & u \\
            a & u & o
          }
          \flecha{hay un total}[de] 3!
        $$
        Los anagramas para que las letras estén juntas los formo combinando $\binom{5}{1} = 5$ poniendo los 3!=6 valores así en cada uno de los
        5 lugares:\\
        $\llave{c c c c c} % hay 5
          {
            uao    & \_     & \_     & \_     & \_     \\
            \_     & uao    & \_     & \_     & \_     \\
            \_     & \_     & \_     & uao    & \_     \\
            \vdots & \vdots & \vdots & \vdots & \vdots \\ \hline
            1      & 2      & 3      & 4      & 5
          } $\\
        Ahora puedo \textit{inyectar} las 4 consonantese en los 4 lugares que quedan libres. Finalmente se pueden formar
        $\underbrace{4!}_{consonantes} \cdot \underbrace{ \binom{5}{1} \cdot 3!}_{vocales} =  720$ anagramas con la condición pedida.\\

  \item Supongo que el \red{orden relativo} es que aparezcan ordenadas así $"c \dots d \dots r \dots s"$, quiere decir que tengo que combinar
        un grupo de 4 letras en 7 que serían los lugares de la letras teniendo un total de $\binom{7!}{4!}$ y luego tengo 1! permutaciones o, \textit{no permuto
          dicho de otra forma},    dado que eso alteraría el orden y no quiero que pase eso. Obtengo cosas así:\\
        $$
          \llave{c c c c c c c c} % hay 7
          {
            c      & d      & r      & s      & \_     & \_     & \_     \\
            \_     & c      & \_     & d      & \_     & r      & s      \\
            c      & \_     & \_     & d      & r      & \_     & s      \\
            \vdots & \vdots & \vdots & \vdots & \vdots & \vdots & \vdots \\ \hline
            1      & 2      & 3      & 4      & 5      & 6      & 7
          } \to
        $$
        Lo cual deja 3 lugares libres para permutar con las 3 vocales, esa permutación es una \text{biyección} da $3!$.\\
        Por último se pueden formar:
        $$
          \ub{\binom{7}{4}\cdot 1!}{consonantes} \cdot \ub{3!}{vocales} =
          \frac{7!}{4!\cdot \cancel{3!}} \cdot \cancel{3!} = \frac{7 \cdot 6 \cdot 5 \cdot \cancel{4!}}{\cancel{4!}} = 210
        $$

  \item Ahora $C = \set{c,d,r,s}$ sin que estén juntas y sin ningún orden en particular.
        Esto quiere decir que puedo ordenar de pocas formas, muy pocas porque solo hay 7 lugares.
        $$
          \llave{c c c c c c c c} % hay 7 en total
          {
            c & \_ & d & \_ & r & \_ & s \\ \hline
            1 & 2  & 3 & 4  & 5 & 6  & 7
          } \to
        $$
        esta combinación es única $\binom{7}{7} = 1$, lo único que resta hacer es permutar las consonantes en esos espacios. 4
        espacios para 4 consonantes.
        Luego relleno \textit{inyectando} las vocales, como antes. El total de anagramas será
        $\underbrace{\binom{7}{7} \cdot 4!}_{consonantes} \cdot \underbrace{3!}_{vocales} = 144 $
\end{enumerate}

\begin{aportes}
  \item \aporte{\dirRepo}{naD GarRaz \github}
\end{aportes}
