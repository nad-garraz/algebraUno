\begin{enunciado}{\ejercicio}
  Un grupo de 15 amigos organiza un asado en un club al que llegarían en 3 autos distintos (4 por auto)
  y 3 irían caminando. Sabiendo que solo importa en qué auto están o si van caminando, determinar de
  cuántas formas pueden viajar si se debe cumplir que al menos uno entre Lucía, María y Diego debe
  ir en auto, y que Juan y Nicolás tienen que viajar en el mismo auto.
\end{enunciado}

Este ejercicio sale contando por el complemento: primero contamos las formas totales de viajar (con el hecho de que Juan y Nicolás tienen
que viajar en el mismo auto) y luego les restamos las formas en las que Lucía, María y Diego van caminando al mismo tiempo.

\begin{itemize}

  \item \underline{Formas totales}

        Teniendo en cuenta que Juan y Nicolás tienen que viajar en el mismo auto, primero elegimos en que auto van. Para esto,
        tenemos $\binom{3}{1}$ opciones.

        \bigskip

        Ahora completemos los dos lugares que faltan en el auto en el que van Juan y Nicolás. Como nos quedan 13 personas por asignar,
        tenemos $\binom{13}{2}$ opciones.

        \bigskip

        Completemos ahora otro auto. Como nos quedan 11 personas por asignar, tenemos $\binom{11}{4}$ opciones.
        Pero como cada auto \red{\ul{es distinto, son distinguibles, no es lo mismo que vayan en uno o en el otro}},
        debemos contemplar que vayan en el otro auto que queda también, de modo que hay que multiplicar ese número por un 2,
        pues son dos los autos que quedan.

        \bigskip

        Por último, debemos llenar el último auto. Como quedan 7 personas sin asignar, tenemos $\binom{7}{4}$ opciones.
        Respecto a los que van caminando, no hay que asignar nada, pues ya quedan asignados al haber llenados todos los autos.

        \bigskip

        Entonces,
        $$
          \text{Formas totales} =
          \ua{\binom{3}{1}}{\text{Auto para}\\ \text{J y N}} \cdot
          \oa{\binom{13}{2}}{\text{Completo auto }\\ \text{de J y N}} \cdot
          \ua{\binom{11}{4}}{\text{Completo }\\ \text{otro auto}} \cdot
          \oa{2}{\text{Dos}\\ \text{autos}} \cdot
          \ua{\binom{7}{4}}{\text{Último}\\ \text{auto}}
          =5405400
        $$
        Lo escribo de una segunda forma, un poco más explicita, para que se note la \textit{distinguibilidad} de los autos. El $+$ aparece
        cuando tengo \textit{una u otra opción}, primero tengo 3 autos, después 2 y por último 1. En cada corchete se eligió un color.
        $$
          \llave{c}{
            \text{\orange{\faIcon[regular]{car-side}}}\\
            \text{\blue{\faIcon[regular]{car-side}}}\\
            \text{\green{\faIcon[regular]{car-side}}}
          }
        $$
        $$
          \text{Formas totales} =
          \Big[
            \ua{1}{\text{\orange{\faIcon{car-side}} para}\\ \text{J y N}} \cdot
            \oa{\binom{13}{2}}{\text{Completo \orange{\faIcon{car-side}} }\\ \text{de J y N}}
            +
            \ua{1}{\text{\blue{\faIcon{car-side}} para}\\ \text{J y N}} \cdot
            \oa{\binom{13}{2}}{\text{Completo \blue{\faIcon{car-side}} }\\ \text{de J y N}}
            +
            \ua{1}{\text{\green{\faIcon{car-side}} para}\\ \text{J y N}} \cdot
            \oa{\binom{13}{2}}{\text{Completo \green{\faIcon{car-side}} }\\ \text{de J y N}}
            \Big]
          \cdot
          \Big[
            \ua{\binom{11}{4}}{\text{Completo }\\ \text{el \blue{\faIcon{car-side}}}}
            +
            \ua{\binom{11}{4}}{\text{Completo }\\ \text{el \green{\faIcon{car-side}}}}
            \cdot
            \Big]
          \cdot
          \Big[
            \ua{\binom{7}{4}}{\text{Completo}\\\text{último}\\ \green{\text{\faIcon{car-side}}}}
            \Big]
          =5405400
        $$
        O todavia hay una tercera forma de escribirlo \textit{GIGA explícito}:
        Voy llenando todo según este orden:
        $$
          \textit{Posibles viajes}:
          \llave{ccc}{
            1 & \to & \set{ \text{\orange{\faIcon{car-side}}, \blue{\faIcon{car-side}}, \green{\faIcon{car-side}}}}\\
            2 & \to & \set{ \text{\orange{\faIcon{car-side}}, \green{\faIcon{car-side}}, \blue{\faIcon{car-side}}}}\\
            3 & \to & \set{ \text{\green{\faIcon{car-side}}, \orange{\faIcon{car-side}}, \blue{\faIcon{car-side}}}}\\
            4 & \to & \set{ \text{\green{\faIcon{car-side}}, \blue{\faIcon{car-side}}, \orange{\faIcon{car-side}}}}\\
            5 & \to & \set{ \text{\blue{\faIcon{car-side}}, \green{\faIcon{car-side}}, \orange{\faIcon{car-side}}}}\\
            6 & \to & \set{ \text{\blue{\faIcon{car-side}}, \orange{\faIcon{car-side}}, \green{\faIcon{car-side}}}}
          }
        $$
        $$
          \text{Formas de llenar}(\set{ \text{\orange{\faIcon{car-side}}, \blue{\faIcon{car-side}}, \green{\faIcon{car-side}}}}) =
          \oa{\binom{13}{2}}{\text{Completo auto }\\ \text{de J y N}} \cdot
          \ua{\binom{11}{4}}{\text{Completo }\\ \text{otro auto}} \cdot
          \ua{\binom{7}{4}}{\text{Último}\\ \text{auto}}
          = 900900
        $$
        Es lo mismo para los demás casos, son 6 casos, \textit{because} $3!$ :
        $$
          \text{Formas totales} = 6 \cdot 900900 = 5405400
        $$

  \item \underline{Formas en las que Lucía, María y Diego van caminando}

        Como los tres que van caminando ya están asignados, solo tenemos que asignar en los autos, que es similar a lo que ya hicimos.

        \bigskip

        Teniendo en cuenta que Juan y Nicolás tienen que viajar en el mismo auto, primero elegimos en que auto van. Para esto,
        tenemos $\binom{3}{1}$ opciones.

        \bigskip

        Ahora completemos los dos lugares que faltan en el auto en el que van Juan y Nicolás. Como nos quedan 10 personas por asginar,
        tenemos $\binom{10}{2}$ opciones.

        \bigskip

        Completemos ahora otro auto. Como nos quedan 8 personas por asignar, tenemos $\binom{8}{4}$ opciones.
        Además, por lo mismo que antes, hay que multiplicar por 2.
        Con el último auto no queda nada por hacer, pues se asignan a las cuatro personas que quedan.

        \bigskip

        Entonces
        $$
          \text{Formas totales} =
          \ua{\binom{3}{1}}{\text{Auto para}\\ \text{J y N}} \cdot
          \oa{\binom{10}{2}}{\text{Completo auto}\\ \text{de J y N}} \cdot
          \ua{\binom{8}{4}}{\text{Completo}\\ \text{otro auto}} \cdot
          \oa{2}{\text{Dos}\\ \text{autos}}
          =18900
        $$

\end{itemize}

Entonces, la formas totales con Juan y Nicolás en el mismo auto y con al menos uno entre Lucía, María y Diego en auto son

$$
  \cajaResultado{
    \binom{3}{1} \cdot \binom{13}{2} \cdot \binom{11}{4} \cdot 2 \cdot \binom{7}{4}
    -
    \binom{3}{1} \cdot \binom{10}{2} \cdot \binom{8}{4} \cdot 2
    =5405400-18900 = 5386500
  }
$$

\bigskip

\parrafoDestacado{
  Volvieron del asado y \green{\faIcon{car-side}}\orange{\faIcon{car-crash}}.
  \it
  ¿De cuantas formas volvieron?

  \text{Formas totales} $= 1$. En pedo.

  ba-dum-tsss
}

\begin{aportes}
  \item \aporte{https://github.com/Nunezca}{Nunezca \github}
  \item \aporte{\dirRepo}{naD GarRaz \github}
\end{aportes}

