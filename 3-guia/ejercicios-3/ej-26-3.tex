\begin{enunciado}{\ejercicio}
  Probar que $ \binom{2n}{n} > n 2^n,$$ \paratodo n \geq 4$.
\end{enunciado}

Vamos a probarlo por inducción:

\textit{Proposición: }
$$
  p(n) : \binom{2n}{n} > n2^n,\, \paratodo n \geq 4
$$

\textit{Caso base: }
$$
  p(4) = \binom{8}{4} > 4\cdot2^4 = 70 > 64 \text{\Tilde}
$$

\textit{Paso inductivo: }

Ahora quiero probar que:
$$
  p(n) \entonces p(n + 1),
$$
o sea quiero ver que:
$$
  \binom{2(n+1)}{n+1} > (n+1) \cdot 2^{(n+1)},\, \paratodo n \geq 4
$$
Y va a ser clave tener esta expresión a mano:
$$
  \binom{2n}{n} = \frac{(2n)!}{n!(2n-n)!}
  \sisolosi
  \ub{\displaystyle\binom{2n}{n} = \frac{(2n)!}{(n!)^2} > n 2^n}{\text{\purple{hipótesis inductiva}}}
$$
Empezamos expandiendo el coeficiente binomial usando la fórmula con factoriales:
$$
  \everymath{\displaystyle}
  \begin{array}{rcl}
    \binom{2n+2}{n+1} = \frac{(2n+2)!}{(n+1)! \cdot (2n+2 - (n+1))!} & = & \frac{(2n+2)(2n+1)(2n)!}{(n+1)!\cdot(n+1)!}                                                                                             \\
                                                                     & = & \frac{(2n+2)(2n+1)\purple{(2n)!}}{(n+1)^2 \cdot \purple{(n!)^2}} \taa{\text{\purple{HI}}}{}> \frac{(2n+2)(2n+1)\purple{n 2^n}}{(n+1)^2}
  \end{array}
$$
Ahora quiero ver que:
$$
  \frac{(2n+2)(2n+1)n\cdot2^n}{(n+1)^2} > (n+1)\cdot2^{(n+1)}
  \Sii{\red{!!}}
  \frac{(2n + 1)n}{n+1} > n + 1
  \Sii{\red{!}}
  n^2 - n > 1
$$
En el \red{!!} y el \red{!}, son factores comunes, simplificaciones acomodar y nada raro. Pero te queda a vos, porque
nada te aportaría verlas.

Esto último es verdadero para $n \en \naturales_{>1}$, por ende para $n \geq 4$ la prueba inductiva será válida, y
queda probado por el principio de inducción que:
$$
  \binom{2n}{n} > n 2^n,\, \paratodo n \geq 4
$$

\begin{aportes}
  \item \aporte{https://github.com/sigfripro}{sigfripro \github}
  \item \aporte{\dirRepo}{naD GarRaz \github}
\end{aportes}
