\begin{enunciado}{\ejercicio}
  Sea $(a_n)_{n\en\naturales}$ la sucesión definida por
  $$
    a_1 = 2 \ytext a_{n+1} = 4a_n - 2\frac{(2n)!}{(n+1)!n!} \quad (n \en \naturales)
  $$

  Probar que $a_n = \binom{2n}n$.
\end{enunciado}

Ejercicio falopa si lo hay. Sale por inducción y rezándole a Dios para no caer en un infierno de cuentas
si uno va por el lugar equivocado.

\textit{Proposición:}
$$
  p(n) : a_n = \binom{2n}{n}.
$$

\textit{Casos base:}
$$
  \begin{array}{rcl}
    p(1) & : & a_1 \igual{def} 2 = \binom{2}{1} \Tilde                                                  \\
         &   & a_2 \igual{def} 4a_1 - 2 \frac{(2n)!}{(1+1)! 1!} \igual{\red{!}} 6 = \binom{4}{2} \Tilde
  \end{array}
$$

Resulta que $p(1)$ es verdadera.

\medskip

\textit{Paso inductivo:}
Voy a asumir como verdadera a
$$
  p(\blue{k}) :
  \ub{a_{\blue{k}} = \binom{2\blue{k}}{\blue{k}}}{\text{\purple{hipótesis inductiva}}}
$$
para algún $\blue{k} \en \naturales$.

Ahora quiero probar que:
$$
  p(k+1) : a_{k+1} = \binom{2(k+1)}{k+1}
$$
La idea es escribir la definición, meter la \purple{hipótesis inductiva}, y como siempre, rezar para que
se acomode todo y que aparezca lo que queremos que aparezca.
Voy a escribir la expresión:
$$
  a_{k+1}
  \igual{def}
  4a_k - 2 \frac{(2k)!}{(k+1)! k!}
$$
para masajearla y llegar a algo como esto:
$$
  a_{k+1} =
  \binom{2(k+1)}{k+1}
  \igual{def}
  \frac{(2k + 2)!}{(k+1)! (k+1)!}
$$

$$
  \begin{array}{rcl}
    a_{k+1}
     & \igual{def}         &
    4a_k - 2 \frac{(2k)!}{(k+1)! k!}                        \\
     & \igual{\purple{HI}} &
    4 \purple{\binom{2k}{k}} - 2 \frac{(2k)!}{(k+1)! k!}    \\
     & =                   &
    4 \frac{(2k)!}{k!\cdot k!}  - 2 \frac{(2k)!}{(k+1)! k!} \\
     & \igual{\red{!!}}    &
    2  \frac{(2k)!}{k!\cdot k!} \cdot (2 - \frac{1}{k+1})   \\
     & =                   &
    2  \frac{(2k)!}{k!\cdot k!} \cdot (\frac{2k +1}{k+1})   \\
     & \igual{\red{!!!}}   &
    \frac{(2k + 2)!}{(k+1)! (k+1)!}                         \\
     & \igual{def}         &
    \binom{2(k+1)}{k+1}
  \end{array}
$$
Oka. \textit{¿Qué carajos pasó en el \red{!!!} y en el \red{!!}?} Lo de siempre, factores comunes, sacar algún factor del factorial y coso.
En el \red{!!!} multipliqué y dividí por \textit{algo} y \textit{mirá fuerte a ese $2$ que está adelante de todo
    {\tiny \faIcon[regular]{smile-wink}}}, para que se alineen los planetas \href{\mindExplosion}{\faIcon{user-astronaut}}.

\bigskip

Por lo tanto $p(1), p(k) \ytext p(k+1)$ resultaron verdaderas. Por el principio de inducción $p(n)$ también es verdadera $\paratodo n \en \naturales$.

\begin{aportes}
  \item \aporte{\dirRepo}{naD GarRaz \github}
\end{aportes}
