\begin{enunciado}{\ejercicio}
  En este ejercicio no hace falta usar inducción.
  \begin{enumerate}[label=\alph*)]
    \item Probar que $\sumatoria{k = 0}{n} \binom{n}{k}^2 = \binom{2n}{n}. \qquad \parentesis{\text{sug: } \binom{n}{k} = \binom{n}{n-k}}$.
    \item Probar que $\sumatoria{k = 0}{n} (-1)^k \binom{n}{k} = 0$.
    \item Probar que $\sumatoria{k = 0}{2n} \binom{2n}{k} = 4^n$ y deducir que $\binom{2n}{n} < 4^n$.
    \item Calcular $\sumatoria{k = 0}{2n+1} \binom{2n+1}{k}$ y deducir $\sumatoria{k=0}{n} \binom{2n+1}{k}$.
  \end{enumerate}
\end{enunciado}

\begin{enumerate}[label=\alph*)]
  \item
        Voy a mostrarlo usando un argumento combinatorio. Básicamente voy a mostrar que con las dos expresiones estamos contando lo mismo.
        Imaginemos que tenemos un conjunto de $\rosa{n}$ cantidad de mujeres y otro de $\blue{n}$ cantidad de hombres.
        La suma de este conjunto tendría $\green{2n}$ personas. Ahora, yo quiero elegir $\green{n}$ personas de ese total
        de $\green{2n}$ personas, contar eso es:
        $$
          \binom{2n}{n}.
        $$

        \bigskip

        \parrafoDestacado{
          \textit{Un modelo de juguete:}

          En una caja con 6 \textit{bolitas}, podría sacar 3 de $\binom{6}{3} = \magenta{20}$ maneras diferentes.
          Ahora propongo agarrarlas de una forma diferente:

          Pinto \rosa{3 de rosa} y \blue{3 de azul}.
          Sigue habiendo 6 \textit{bolitas} solo que pintadas, ahora voy sacando de a 3, nuevamente, pero contando así:
          $$
            \binom{\rosa{3}}{\rosa{0}} \binom{\blue{3}}{\blue{3}} +
            \binom{\rosa{3}}{\rosa{1}} \binom{\blue{3}}{\blue{2}} +
            \binom{\rosa{3}}{\rosa{2}} \binom{\blue{3}}{\blue{1}} +
            \binom{\rosa{3}}{\rosa{3}} \binom{\blue{3}}{\blue{0}} =
            1 + 9 + 9 + 1 =
            \magenta{20}
          $$
          Cada término de esa suma es contar las formas de sacar $\rosa{k}$ \textit{bolitas} rosas
          de las \rosa{3 \textit{bolitas} rosas} que hay para luego multplicar eso por la cantidad de sacar
          \textit{\blue{$(3-k)$ bolitas azules}}. Como estoy sacando $\rosa{k} + \blue{(3-k)}$ es siempre
          estar sacando 3.
        }

        \bigskip

        \textit{El modelo más pulenta:}

        Hasta ahora todo bien. Notemos que esto lo puedo decir también, es decir, es lo mismo que elegir
        $\rosa{k}$ mujeres de las $\rosa{n}$ mujeres que hay y elegir $\blue{n} - \rosa{k}$ hombres entre los \blue{n} hombres que hay.

        Notar que $\rosa{k} + (\blue{n} - \rosa{k}) = \green{n}$, así que el grupo que elijamos como combinación
        de los dos siempre va a tener $\green{n}$ personas, y va a estar elegido una parte desde $\rosa{n}$ mujeres y la otra desde $\blue{n}$ hombres.

        \medskip

        Entonces voy a sumar todas las posibles formas de elegir $\green{n}$ personas de entre $\green{2n}$ personas,
        pero agarrando siempre de $\rosa{k}$ mujeres y $\blue{n} - \rosa{k}$ hombres :
        $$
          \binom{\green{2n}}{\green{n}} =
          \ob{
            \ub{\displaystyle \binom{\rosa{n}}{\rosa{0}}\binom{\blue{n}}{\blue{n} - \rosa{0}}}
            {\substack{\text{elijo  \rosa{0} mujeres}\\
                \ytext \\
                \text{ $\blue{n}$ hombres}
              }}
            +
            \ub{\displaystyle \binom{\rosa{n}}{\rosa{1}}\binom{\blue{n}}{\blue{n} - \rosa{0}}}
            {\substack{\text{elijo  $\rosa{1}$ mujeres}\\
                \ytext \\
                \text{ $\blue{n} - \rosa{1}$ hombres}
              }}
            +
            \cdots
            +
            \ub{\displaystyle \binom{\rosa{n}}{\rosa{n-1}}\binom{\blue{n}}{\blue{n} - (\rosa{n - 1})}}
            {\substack{\text{elijo  \rosa{$n-1$} mujeres}\\
                \ytext \\
                \text{ $\blue{1}$ hombres}
              }}
            +
            \ub{\displaystyle \binom{\rosa{n}}{\rosa{n}}\binom{\blue{n}}{\blue{n} - \rosa{n}}}
            {\substack{\text{elijo  \rosa{$n$} mujeres}\\
                \ytext \\
                \text{ $\blue{0}$ hombres}
              }}
          }{\text{formas de tomar \green{$n$} personas de un total entre $\green{2n}$}}
          = \llamada1
        $$
        La sumatoria en $\llamada1$ se puede reescribir por su simetría y además por la sugerencia del enunciado como:
        $$
          \llamada1 =
          \sumatoria{\rosa{k} = 0}{n} \binom{\rosa{n}}{\rosa{k}}\binom{\blue{n}}{\blue{n} - \rosa{k}}
          \igual{\red{!}}[sug.]
          \sumatoria{\rosa{k} = 0}{\green{n}} \binom{\rosa{n}}{\rosa{k}}^2
          = \binom{\green{2n}}{\green{n}}
        $$
        Como vimos, estamos contando lo mismo con las dos expresiones, por lo tanto queda probado que son iguales.
        $$
          \cajaResultado{
            \sumatoria{k = 0}{n} \binom{n}{k}^2
            = \binom{2n}{n}
          }
        $$

  \item
        ¿Vale usar \textit{crudamente} la fórmula del binomio de Newton acá?

        \bigskip

        \textit{Si \yellow{\underline{sí}}, se puede: }
        $$
          \textstyle
          \ub{(x + y)^n = \sumatoria{k=0}{n} \binom{n}{k} x^k y^{n-k}}{\text{Binomio de Newton}}
          \Entonces{$x = 1$}[$y = -1$]
          (1 - 1)^n = \sumatoria{k=0}{n} \binom{n}{k} 1^k \cdot (-1)^{n-k}
          \sii
          \cajaResultado{
            \sumatoria{k=0}{n} (-1)^{n-k} \binom{n}{k} = 0
          }
        $$

        \bigskip

        \bigskip

        \textit{Si \yellow{\underline{no}}, se puede: }
        Con una idea de por donde va esto de sumar los números combinatorios dada su simetría:

        \textit{Caso con $n$ impar:}
        $$
          \begin{array}{rcl}
            \sumatoria{k = 0}{n} (-1)^k \binom{n}{k}
             & =                      &
            (-1)^0  \binom{n}{0} +
            (-1)^1  \binom{n}{1} +
            \cdots +
            (-1)^{\frac{n}{2}} \binom{n}{\frac{n}{2}} +
            \cdots +
            (-1)^{(n-1)} \binom{n}{n-1} +
            (-1)^n  \binom{n}{n}        \\
             & \igual{\red{!}}        &
            \binom{n}{0} -
            \binom{n}{1} +
            \cdots \magenta{\pm}
            \binom{n}{\frac{n-1}{2}} \magenta{\mp}
            \binom{n}{\frac{n+1}{2}} \magenta{\pm}
            \cdots +
            \binom{n}{n-1} -
            \binom{n}{n}                \\
             & \igual{\red{!!}}[sug.] &
            \binom{n}{0} -
            \binom{n}{1} +
            \cdots \magenta{\pm}
            \binom{n}{\frac{n-1}{2}} \magenta{\mp}
            \binom{n}{\frac{n-1}{2}} \magenta{\pm}
            \cdots +
            \binom{n}{1} -
            \binom{n}{0}
            =0
          \end{array}
        $$

        \textit{Caso con $n$ par:}
        $$
          \begin{array}{rcl}
            \sumatoria{k = 0}{n} (-1)^k \binom{n}{k}
             & =                      &
            (-1)^0  \binom{n}{0} +
            (-1)^1  \binom{n}{1} +
            \cdots +
            (-1)^{\frac{n}{2}} \binom{n}{\frac{n}{2}} +
            \cdots +
            (-1)^{(n-1)} \binom{n}{n-1} +
            (-1)^n  \binom{n}{n}        \\
             & \igual{\red{!}}        &
            \binom{n}{0} -
            \binom{n}{1} +
            \cdots +
            \binom{n}{\frac{n}{2}} -
            \cdots -
            \binom{n}{n-1} +
            \binom{n}{n}                \\
             & \igual{\red{!!}}[sug.] &
            \binom{n}{0} -
            \binom{n}{1} +
            \cdots +
            \binom{n}{\frac{n}{2}} -
            \cdots -
            \binom{n}{1} +
            \binom{n}{0}                \\
             & =                      &
            2\binom{n}{0} -
            2\binom{n}{1} +
            \cdots -
            2\binom{n}{\frac{n}{2}-1} +
            \binom{n}{\frac{n}{2}}
          \end{array}
        $$

        \red{Tengo que probar que eso me da cero... a ver si alguien por favor me saca de este quilombo}

        \hacer

  \item
        Primero queremos probar que $\sumatoria{k = 0}{2n} \binom{2n}{k} = 4^n$. \\
        La forma mas sencilla de hacer esto es haciendo un cambio de variable y usando una identidad ya conocida:
        $\sumatoria{k=0}{n}\binom{n}{k} = 2^n$, haciendo cambio de variable $m = 2n$, tenemos
        $\sumatoria{k=0}{m}\binom{m}{k} = 2^m = 2^{2n} = 4^n$. \par\smallskip
        Otra manera de pensarlo es contando funciones de esta manera: \par\smallskip
        \textit{Recuerdo: }\par
        $\sumatoria{k=0}{n}\binom{n}{k} = 2^n$ esta identidad equivale a contar las partes $\partes$ de un conjunto de n elementos. 
        Que tambien se puede obtener contando las funciones $f:A \subseteq \naturales \to \set{0,1}, \, \#(A) = n$, esta notacion nos dice de manera intuitiva que 
        por cada elemento tenemos dos opciones, o lo incluimos ($1$), o no ($0$), todas las posibles maneras de relacionar el dominio con el codominio son 
        $\#(\set{0,1})^{\#(A)} = 2^5$ \par\smallskip   
        Ahora la idea es que tenemos el doble de elementos ($2n$) y hay que armar subconjuntos de hasta $2n$ elementos, la idea entonces 
        es extender esta idea de la funcion a dar la posibilidad de repetir las elecciones ya hechas, extendiendo el codominio a
        $\set{00, 01, 10, 11}$, de esta manera tenemos todas las posibilidades $2^n$ anteriores, y extendemos posibilidad de repetir o agregar elemento. Obteniendo
        todos los posibles subconjuntos de $2n$ elementos, ahora el codominio tiene 4 elementos, por lo que tenemos que la cantidad de funciones son $4^n$. \\
        Si no se entendió voy a hacer un ejemplo para que se vea mas visual: \smallskip
        Imaginemos un conjunto de $n = 2$ elementos $\set{a,b}$, ahora para calcular las partes voy a usar la idea de contar  
        las funciones, tenemos 4 posibilidades: \par\smallskip
        $
        \llave{l}{
          f(a) = 0 \\
          f(b) = 0
        }
        \flecha{queda}
        \vacio
        \quad$
        $
        \llave{l}{
          f(a) = 1 \\
          f(b) = 0
        }
        \flecha{queda}
        \set{a}
        \quad$
        $
        \llave{l}{
          f(a) = 0 \\
          f(b) = 1
        }
        \flecha{queda}
        \set{b}
        \quad$
        $
        \llave{l}{
          f(a) = 1 \\
          f(b) = 1
        }
        \flecha{queda}
        \set{a,b}
        \quad$ \par
        Si combinamos todos estos conjuntos obtenemos $\set{\set{a},\set{b},\set{a,b},\vacio}$ que precisamente son 
        las partes $\partes$ del conjunto $\set{a,b}$ \\
        Ahora veo la idea pero con el codominio extendido: \par\smallskip
        $
        \llave{l}{
          f(a) = 00 \\
          f(b) = 00
        }
        \flecha{queda}
        \vacio
        \hspace{0.75em}$
        $
        \llave{l}{
          f(a) = 10 \\
          f(b) = 00
        }
        \flecha{queda}
        \set{a}
        \hspace{0.75em}$
        $
        \llave{l}{
          f(a) = 01 \\
          f(b) = 00
        }
        \flecha{queda}
        \set{\tilde{a}}
        \hspace{0.75em}$
        $
        \llave{l}{
          f(a) = 11 \\
          f(b) = 00
        }
        \flecha{queda}
        \set{a,\tilde{a}}
        \hspace{0.75em}$ \par
        $
        \llave{l}{
          f(a) = 00 \\
          f(b) = 10
        }
        \flecha{}
        \set{b}
        \hspace{0.75em}$
        $
        \llave{l}{
        f(a) = 10 \\
        f(b) = 10
        }
        \flecha{}
        \set{a,b}
        \hspace{0.75em}$
        $
        \llave{l}{
        f(a) = 01 \\
        f(b) = 10
        }
        \flecha{}
        \set{\tilde{a},b}
        \hspace{0.75em}$
        $
        \llave{l}{
        f(a) = 11 \\
        f(b) = 10
        }
        \flecha{}
        \set{a,\tilde{a},b}
        \hspace{0.75em}$ \par
        $
        \llave{l}{
          f(a) = 00 \\
          f(b) = 01
        }
        \flecha{}
        \set{\tilde{b}}
        \hspace{0.75em}$
        $
        \llave{l}{
        f(a) = 10 \\
        f(b) = 01
        }
        \flecha{}
        \set{a,\tilde{b}}
        \hspace{0.75em}$
        $
        \llave{l}{
        f(a) = 01 \\
        f(b) = 01
        }
        \flecha{}
        \set{\tilde{a},\tilde{b}}
        \hspace{0.75em}$
        $
        \llave{l}{
        f(a) = 11 \\
        f(b) = 01
        }
        \flecha{}
        \set{a,\tilde{a},\tilde{b}}
        \hspace{0.75em}$ \par
        $
        \llave{l}{
          f(a) = 00 \\
          f(b) = 11
        }
        \flecha{}
        \set{b,\tilde{b}}
        \hspace{0.75em}$
        $
        \llave{l}{
        f(a) = 10 \\
        f(b) = 11
        }
        \flecha{}
        \set{a,b,\tilde{b}}
        \hspace{0.75em}$
        $
        \llave{l}{
        f(a) = 01 \\
        f(b) = 11
        }
        \flecha{}
        \set{\tilde{a},b,\tilde{b}}
        \hspace{0.75em}$
        $
        \llave{l}{
        f(a) = 11 \\
        f(b) = 11
        }
        \flecha{}
        \set{a,\tilde{a},b,\tilde{b}}
        \hspace{0.75em}$ \par

        \textit{Aclaración: } \par   
        El uso de las tildes $\tilde{~}$ en las letras es simplemente para marcar diferencia porq en un conjunto no puede haber elementos repetidos. \\
        Ahora llamo $c = \tilde{a}$ y $d = \tilde{b}$ y miren lo que tengo:
        \[
        \set{\vacio,\set{a},\set{b},\set{c},\set{d},\set{a,b},\set{a,c},\set{a,d},\set{b,c},\set{b,d},\set{c,d},
        \set{a,b,c},\set{a,b,d},\set{a,c,d},\set{b,c,d},\set{a,b,c,d}}
        \]
        Esto precisamente son las partes de un conjunto de 4 elementos!, que corresponde con contar $2^4 = 2^{2n} = 4^n$. Despues de todo este choclo, recordemos que originalmente
        teniamos $n = 2$ elementos $a,b$, y con esta idea de la funcion contamos las partes de un conjunto de $2n$ elementos. \par\bigskip
        Despues el enunciado nos pedia deducir que: $\binom{2n}{n} < 4^n$. \\
        Este no tiene mucho misterio, expandimos el lado derecha con lo que probamos recientemente y nos queda:
        \[
        \binom{2n}{n} < \sumatoria{k = 0}{2n} \binom{2n}{k}
        \]
        \[
        \red{\binom{2n}{n}} < \binom{2n}{0} + \cdots + \red{\binom{2n}{n}} + \cdots \binom{2n}{2n}
        \]
        Bueno todos los terminos de la derecha son positivos asi que se ve claramente que lo que nos planteaba el enunciado es verdadero.

  \item
        Para calcular lo que nos piden no damos mucha vuelta, hacemos cambio de variable $m = 2n + 1$ y vemos que:
        \[
        \sumatoria{k=0}{2n + 1} \binom{2n+1}{k} \xrightarrow{m = 2n + 1} \sumatoria{k=0}{m} \binom{m}{k} = 2^m \xrightarrow{m = 2n + 1} 2^{2n + 1} = 2^{2n} \cdot 2 = \cajaResultado{4^n \cdot 2}
        \] \\
        Ya tenemos la primera parte del ejercicio, ahora nos piden \textit{deducir} $\sumatoria{k = 0}{n} \binom{2n + 1}{k}$. Para hacer este vamos a usar un poco de intuicion del triangulo de pascal:\\
        \[
        \begin{tabular}{c|ccccccccccc}
        n=0&  &     &     &     &      & $\binom{0}{0}$   &      &      &     &     & \\
        &  &     &     &     &      &     &      &      &     &     & \\
        n=1&  &     &     &     & $\binom{1}{0}$    &     &  $\binom{1}{1}$     &      &     &     & \\
        &  &     &     &     &      &     &      &      &     &     & \\
        n=2&  &     &     & $\binom{2}{0}$     &      & $\binom{2}{1}$     &      & $\binom{2}{2}$      &     &     & \\
        &  &     &     &     &      &     &      &      &     &     & \\
        n=3&  &     & $\binom{3}{0}$     &     & $\binom{3}{1}$      &     &  $\binom{3}{2}$     &      & $\binom{3}{3}$     &     & \\
        &  &     &     &     &      &     &      &      &     &     & \\
        n=4&  & $\binom{4}{0}$     &     & $\binom{4}{1}$     &      & $\binom{4}{2}$     &      & $\binom{4}{3}$      &     & $\binom{4}{4}$     & \\
        &  &     &     &     &      &     &      &      &     &     & \\
        n=5&$\binom{5}{0}$   &     &  $\binom{5}{1}$    &     &  $\binom{5}{2}$    &     & $\binom{5}{3}$     &      &  $\binom{5}{4}$    &     & $\binom{5}{5}$  
        \end{tabular}
        \] \\
        Vemos que las filas con $n$ impares tienen una cantidad par de terminos, por ejemplo $n = 3$, tiene 4 terminos, esto se debe a que como el listado empieza
        en $0$. Como tienen una cantidad par de terminos, y sabemos que el triangulo de pascal es simétrico (por la identidad: $\binom{n}{k} = \binom{n}{n - k}$), con 
        saber la mitad ya sabemos el resto (en el caso de los $n$ impares). Entonces vamos a usar esta idea, nos piden deducir $\sumatoria{k = 0}{n} \binom{2n + 1}{k}$, vemos 
        que $2n + 1$ siempre va a ser un numero impar, asi que podemos aplicar esta idea de la simetria, la clave acá es que sumar hasta $n$ es sumar exactamente
        hasta la mitad de su respectiva fila en el triangulo de pascal. Por que? Tiene que ver con que el triangulo de pascal esta indexado en 0, pero mejor
        verlo graficamente, por ejemplo vean que en la fila 3 del triangulo de pascar, es como decir la fila $2m + 1,\, m=1$, y justamente en $m$ se parte al medio
        la fila. \\
        Bueno combinemos todas estas ideas y entonces para calcular $\sumatoria{k = 0}{n} \binom{2n + 1}{k}$ primero calculamos todos los elementos de la fila correspondiente
        al $2n + 1$ y luego dividimos por dos porque partimos al medio la fila. Entonces tenemos:
        \[
        \sumatoria{k = 0}{n} \binom{2n + 1}{k} = \frac{\sumatoria{k = 0}{2n + 1} \binom{2n + 1}{k}}{2} = \frac{4^n \cdot \cancel{2}}{\cancel{2}} = \cajaResultado{4^n}
        \]

\end{enumerate}

\begin{aportes}
  \item \aporte{https://github.com/sigfripro}{sigfripro \github}
  \item \aporte{\dirRepo}{naD GarRaz \github}
\end{aportes}
