\begin{enunciado}{\ejercicio}
  En este ejercicio no hace falta usar inducción.
  \begin{enumerate}[label=\roman*)]
    \item Probar que $\sumatoria{k = 0}{n} \binom{n}{k}^2 = \binom{2n}{n}$. \qquad sug: $\binom{n}{k} = \binom{n}{n-k}$.
    \item Probar que $\sumatoria{k = 0}{n} (-1)^k \binom{n}{k} = 0$.
    \item Probar que $\sumatoria{k = 0}{2n} \binom{2n}{k} = 4^n$ y deducir que $\binom{2n}{n} < 4^n$.
    \item Calcular $\sumatoria{k = 0}{2n+1} \binom{2n+1}{k}$ y deducir que $\sumatoria{k=0}{n} \binom{2n+1}{k}$.
  \end{enumerate}
\end{enunciado}

Binomio de Newton: $(x + y)^n = \sumatoria{k=0}{n} \binom{n}{k} x^n y^{n-k}$

\begin{enumerate}[label=\roman*)]
  \item
        Voy a mostrarlo usando un 
        argumento combinatorio, que es basicamente mostrar que con las dos expresiones estamos contando lo mismo. \\
        Veamos, imaginemos que tenemos un cojunto de $n$ cantidad de mujeres y otro de $n$ cantidad de hombres. La suma de este 
        conjunto tendria $2n$ personas, ahora yo quiero elegir $n$ personas de ese total de $2n$ personas, que contar eso es 
        $\binom{2n}{n}$. Hasta ahora todo bien, notemos que esto lo puedo decir tambien como elegir $k$ mujeres de las $n$ que hay 
        y elegir $n - k$ hombres de los $n$ que hay. Notar que $k + (n - k) = n$, asi que el grupo que elijamos como combinacion
        de los dos siempre va a tener $n$ personas, y va a estar elegido una parte desde $n$ mujeres y la otra desde $n$ hombres. 
        Entonces tenemos que:
        \[
        \binom{n}{0}\binom{n}{n} + \binom{n}{1}\binom{n}{n - 1} + \cdots + \binom{n}{n - 1}\binom{n}{1} + \binom{n}{n}\binom{n}{0} = \binom{2n}{n}
        \]
        Lo de arriba se puede reescribir como:
        \[
        \sum_{k = 0}^{n}\binom{n}{k}\binom{n}{n-k} = \binom{2n}{n}
        \]
        Y por simetria del numero combinatorio, queda como:
        \[
        \sum_{k = 0}^{n} \binom{n}{k}^2 = \binom{2n}{n}
        \]
        Como vimos, estamos contando lo mismo con las dos expresiones, por lo tanto queda probado que son iguales. 
        %\vspace{12pt}

  \item
        \hacer
  \item
        \hacer
  \item
        \hacer
\end{enumerate}

