\begin{enunciado}{\ejercicio}
  Sea $X = \set{1,2,3,4,5,6,7,8,9,10}$, y sea $R$ la relación de equivalencia en $\partes(X)$ definida por:
  $$
    A \relacion B \sisolosi A \inter \set{1,2,3} = B \inter \set{1,2,3}.
  $$
  ¿Cuántos conjuntos $B \en \partes(X)$ de exactamente 5 elementos tiene la clase de equivalencia $\overline A $ de $A = \set{1,3,5}$?
\end{enunciado}

Como $A$ tiene al 1 y al 3, los elementos $B$, \textit{conjuntos en este caso}, pertenecientes a la clase $\overline A$
deberían cumplir que si $B \subseteq \overline A \entonces
  \llaves{cc}{
    1 \en B&\\
    3 \en B&\\
    2 \notin B &\to \text{ si } 2 \en B \entonces A \cancel\relacion B
  } $.\\
Los conjuntos de 5 elementos serán de la forma:\\
$\llave{c c c c c} % hay 5 en total
  {
    1 & 3 & \_ & \_ & \_ \\
  } \flecha{5 elementos}[$ \inter \set{1,2,3} \igual{\checkmark} \set{1,3}$] \binom{7}{3} = 35$. Los 7 números usados son $\set{4,5,\red{6},7,8,9,10}$ \\
\red{¿Es solo eso o interpreto mal la $\relacion$ u otra cosa?}
