\begin{enunciado}{\ejercicio}
  \begin{enumerate}[label=\alph*)]
    \item Sea $A$ un conjunto con $2n$ elementos. ¿Cuántas relaciones de equivalencia pueden definirse en $A$ que cumplan la condición
          de que para todo $a \en A$ la clase de equivalencia de $a$ tenga $n$ elementos?

    \item Sea $A$ un conjunto con $3n$ elementos. ¿Cuántas relaciones de equivalencia pueden definirse en $A$ que cumplan la condición
          de que para todo $a \en A$ la clase de equivalencia de $a$ tenga $n$ elementos?

  \end{enumerate}
\end{enunciado}

\begin{enumerate}[label=\alph*)]
  \item Una relación de equivalencia en $A$ parte al conjunto en \ul{subconjuntos disjuntos}, que tiene las propiedades de
        ser \textit{reflexiva}, \textit{simétrica} y \textit{transitiva}.

        Por enunciado me piden que para cada $a \en A$ su clase de equivalencia tenga $n$ elementos, por lo tanto vamos a tener que partir
        a $A$ en 2 subconjuntos.
        La idea es la siguiente: Tomo un elemento $a_0 \en A$, lo relaciono consigo mismo y con otros $n - 1$ elementos más.
        De esa forma creo una clase con $n$ elementos:
        $$
          \clase{a_0} \ytext \#\clase{a_0} = n,
        $$
        Solo usé $n$ elementos de $A$ para armar $\clase{a_0}$, entonces automáticamente la otra clase de equivalencia queda determinada también
        con $n$ elementos.

        Dado que podría realizar esa construcción con cualquiera $n$ elementos de $A$, tengo en total:
        $$
          \binom{2n}{n}
        $$
        posibles clases.

        Una vez armada una clase en particular $\clase{a_0}$ me quedan $n$ elementos para armar la otra clase en particular.
        Haciendo la misma contrucción podría armar:
        $$
          \binom{n}{n}
        $$
        Por lo tanto es tentador pensar que hay en total:
        $$
          \binom{2n}{n} \cdot \binom{n}{n},
        $$
        \red{peeeero, hay que tener cuidado!} Hay que dividir por $2!$ ya que el orden
        en el que realizamos la construcción de cada clase no debería ser computado como distintos resultados totales.
        Finalmente:
        $$
          \cajaResultado{\binom{2n}{n} \cdot 1 \cdot \frac{1}{2!}}
        $$

  \item Este es similar al anterior, vamos a tener que construir en total $3$ clases de equivalencia.

        Primero de los $3n$ elementos elijo $n$, $\binom{3n}{n}$,
        me quedan $2n$ restantes, de los cuales tambien elijo $n$, $\binom{2n}{n}$,
        repito con los $n$ que quedan formando la clase final (ya quedo determinada por las elecciones anteriores).

        Finalmente dividimos por $3!$ ya que el orden en el que hagamos las operaciones no debería afectar el conteo final.
        $$
          \cajaResultado{\binom{3n}{n} \cdot \binom{2n}{n} \cdot 1 \cdot \frac{1}{3!}}
        $$
\end{enumerate}

\begin{aportes}
  \item \aporte{https://github.com/sigfripro}{sigfripro \github}
\end{aportes}
