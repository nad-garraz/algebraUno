\begin{enunciado}{\ejercicio}
  \begin{enumerate}[label=\alph*)]
    \item Sea $A$ un conjunto con $2n$ elementos. ¿Cuántas relaciones de equivalencia pueden definirse en $A$ que cumplan la condición
          de que para todo $a \en A$ la clase de equivalencia de $a$ tenga $n$ elementos?

    \item Sea $A$ un conjunto con $3n$ elementos. ¿Cuántas relaciones de equivalencia pueden definirse en $A$ que cumplan la condición
          de que para todo $a \en A$ la clase de equivalencia de $a$ tenga $n$ elementos?

  \end{enumerate}
\end{enunciado}

\begin{enumerate}[label=\alph*)]
 \item Recordemos que una relacion de equivalencia en $A$ parte al conjunto en suboconjuntos disjuntos, luego por 
 enunciado me piden que cada $a \in A$ su clase de equivalencia tenga $n$ elementos, por lo tanto vamos a tener que partir 
 a $A$ en 2 subconjuntos. La idea es la siguiente, fijo un elemento $a_0 \in A$, luego tengo que relacionar ese consigo mismo y con
 $n - 1$ elementos mas, de esa forma creo una clase con $n$ elementos, y automaticamente la otra clase de equivalencia queda determinada. 
 Esto seria $\binom{2n}{n}$. Y tenemos que dividir por $2!$ ya que el orden en el que realizamos las operaciones no afecta el conteo total. 
 Finalmente seria 
 $$
 \cajaResultado{\binom{2n}{n} \cdot 1 \cdot \frac{1}{2!}}
 $$

 \item Este es similar al anterior, vamos a tener que armar en total $3$ clases de equivalencia, primero de los $3n$ elementos elijo $n$, luego 
 me quedan $2n$ elementos restantes, de los cuales tambien elijo $n$, luego quedan $n$ que es la clase final (ya quedo determinada por las elecciones anteriores), luego 
 dividimos por $3!$ ya que el orden en el que hagamos las operaciones no cambia el conteo final. 
 $$
 \cajaResultado{\binom{3n}{n} \cdot \binom{2n}{n} \cdot 1 \cdot \frac{1}{3!}}
 $$
\end{enumerate}

\begin{aportes}
 \item \aporte{https://github.com/sigfripro}{sigfripro \github}
\end{aportes}