\begin{enunciado}{\ejercicio}
  Un estudiante puede elegir qué cursar entre 5 materias que se dictan este cuatrimestre. ¿De cuántas
  maneras distintas puede elegir qué materias cursar, incluyendo como posibilidad no cursar ninguna
  materia? ¿Y si tiene que cursar al menos dos materias?
\end{enunciado}

Hay 5 materias, el conjunto de materias lo bautizo $M$:
$$
  M = \set{m_1, m_2, m_3, m_4, m_5}
  \quad \text{con} \quad
  \#M = 5
$$

Si decide cursar 0 materias, eso se puede elegir de una sola manera:
$$
  \binom{5}{0} = 1
$$

\medskip

Si decide cursar 1 materia, eso se puede elegir así:
$$
  \binom{5}{1} = 5
$$

\medskip

Si decide cursar 2 materias, eso se puede elegir así:
$$
  \binom{5}{2} = 10
$$

\medskip

Si decide cursar 3 materias, eso se puede elegir así:
$$
  \binom{5}{3} = 10
$$

\medskip

Si decide cursar 4 materias, eso se puede elegir así:
$$
  \binom{5}{4} = 5
$$

\medskip

Si decide cursar 5 materias, eso se puede elegir así:
$$
  \binom{5}{5} = 1
$$

\bigskip

Entonces la forma de elegir que cosa cursar sería la suma de todo eso:
$$
  \binom{5}{0} +
  \binom{5}{1} +
  \binom{5}{2} +
  \binom{5}{3} +
  \binom{5}{4} +
  \binom{5}{5} =
  \cajaResultado{
    \sumatoria{i = 0}{5} \binom{5}{i} = 32
  }
$$

De \textit{yapa} se puede expresar así:
$$
  (x + y)^n = \sumatoria{i = 0}{n} \binom{n}{i} x^{n-i}y^{i}
  \Entonces{$x = 1,\, y = 1$}[$n=5$]
  \cajaResultado{
    2^5 = \sumatoria{i=0}{5} \binom{5}{i}
  }
$$

Si al menos tiene que cursar 2 materias, quiere decir que puede cursar 2, 3, 4 o 5. Sumando lo que corresponde:
$$
  \binom{5}{2} +
  \binom{5}{3} +
  \binom{5}{4} +
  \binom{5}{5} =
  \cajaResultado{
    \sumatoria{i = 2}{5} \binom{5}{i} = 26
  }
$$

\begin{aportes}
  \item \aporte{\dirRepo}{naD GarRaz \github}
\end{aportes}
