\begin{enunciado}{\ejercicio}
  Si $A$ es un conjunto con $n$ elementos ¿Cuántas relaciones en $A$ hay?
  ¿Cuántas de ellas son reflexivas?
  ¿Cuántas de  ellas son simétricas? ¿Cuántas de ellas son reflexivas y simétricas?
\end{enunciado}

Para dos conjuntos
$$
  A = \set{1,\cdots, n}
  \ytext
  B = \set{ a_i \en A \ytext a_j \en A \talque  a_i \relacion a_j \paratodo i,j \en [1, n] }
$$
Los cardinales de conjuntos de esta pinta:
$$
  \#A = n
  ,\quad
  \#B = n^2
  ,\quad
  \#\partes(B) = 2^{n^2}
$$

\bigskip

\textit{¿Cuántas relaciones reflexivas tengo en $A$?:}
Sé que las relaciones reflexivas son de la forma:
$$
  \paratodo a_i \en A \entonces a_i \relacion a_i,
$$
es decir, los $n$ elementos de la diagonal de una matriz:
$$
  \begin{array}{c|c|c|c|c|c|c|c|}
    \multicolumn{1}{c}{} & \multicolumn{1}{c}{a_1} & \multicolumn{1}{c}{a_2} & \multicolumn{1}{c}{a_3} & \multicolumn{1}{c}{\cdots} & \multicolumn{1}{c}{a_{n-2}} & \multicolumn{1}{c}{a_{n-1}} & \multicolumn{1}{c}{a_n} \\ \cline{2-8}
    a_1                  & R                       & \cdot                   & \cdot                   & \cdots                     & \cdot                       & \cdot                       & \cdot                   \\ \cline{2-8}
    a_2                  & \cdot                   & R                       & \cdot                   & \cdots                     & \cdot                       & \cdot                       & \cdot                   \\ \cline{2-8}
    a_3                  & \cdot                   & \cdot                   & R                       & \cdots                     & \cdot                       & \cdot                       & \cdot                   \\ \cline{2-8}
    \vdots               & \vdots                  & \vdots                  & \vdots                  & \ddots                     & \cdot                       & \cdot                       & \cdot                   \\ \cline{2-8}
    a_{n-2}              & \cdot                   & \cdot                   & \cdot                   & \ddots                     & R                           & \cdot                       & \cdot                   \\ \cline{2-8}
    a_{n-1}              & \cdot                   & \cdot                   & \cdot                   & \ddots                     & \cdot                       & R                           & \cdot                   \\ \cline{2-8}
    a_n                  & \cdot                   & \cdot                   & \cdot                   & \cdots                     & \cdot                       & \cdot                       & R                       \\ \cline{2-8}
  \end{array}
$$
deben estar en mi relación. Hay solo una forma de conseguir eso:
$$
  \binom{n}{n} = 1 \llamada1.
$$
solo hay una posibilidad para que pase eso.

Todo muy lindo, hago un ejemplo de juguete para entender esta verga. Ejemplo con $A = \set{a_1, a_2}$:
$$
  \begin{array}{c|c|c|}
    \multicolumn{1}{c}{} & \multicolumn{1}{c}{a_1} & \multicolumn{1}{c}{a_2} \\ \cline{2-3}
    a_1                  & R                       & \cdot                   \\ \cline{2-3}
    a_2                  & \cdot                   & R                       \\ \cline{2-3}
  \end{array}
$$
$$
  \llave{l}{
    \relacion_1 = \set{(a_1,a_1), (a_2,a_2)} \\
    \relacion_2 = \set{(a_1,a_1), (a_2,a_2), (a_{\cyan 2}, a_{\magenta 1})} \\
    \relacion_3 = \set{(a_1,a_1), (a_2,a_2), (a_{\magenta 1}, a_{\cyan 2})} \\
    \relacion_4 = \set{(a_1,a_1), (a_2,a_2), (a_1,a_2), (a_2, a_1)} \\
  }
$$
Tengo en total 4 posibles relaciones reflexivas. Se puede ver que los elementos \underline{no diagonales} son el
problema a tratar, o siendo menos dramáticos, los elementos no diagnonales me agregan posibles relaciones
a mi única ($\llamada1$) relación hasta el momento.

\medskip

\parrafoDestacado{
  ¿Cómo calculo todas las posibilidades de combinar los elementos en un conjunto $A = \set{a_1, \cdots, a_n }$?
}

\medskip

La cantidad total de pares $A \times A$ \underline{sin} los elementos de la diagonal, es decir $(a_i,a_j)$ con $i \distinto j$
que me puedo armar en esa matriz es:
$$
  n^2 - n
$$
Cada uno de estos $n^2 - n$ pares pueden \textit{aparecer o no} en la relación (como en el ejemplo de juguete).
\textit{Aparecer o no} son 2 posibilidades, \textit{ser o no ser \faIcon[regular]{skull}}. Habrá:
$$
  \ub{2 \cdot 2 \cdots 2}{\#(\textit{elementos no diagonales)}} =  2^{n^2 - n}
$$
Noto que ese resultado es $\#\partes(\set{\textit{elementos no diagonales}})$.

El total de relaciones \textit{reflexivas}:
$$
  \cajaResultado{
    2^{n^2 - n}
  }
$$

\vspace{1cm}

\textit{¿Cuántas relaciones simétricas habrá en $A$?:} Las relaciones simétricas serán aquellas que
$$
  a_i \relacion a_j \entonces a_j \relacion a_i,\quad
  \paratodo a_i \ytext a_j \en A.
$$
Arranco con un ejemplo de juguete, para un conjunto $A = \set{a_1, a_2}$, tengo las siguientes relaciones simétricas:
$$
  \begin{array}{c|c|c|}
    \multicolumn{1}{c}{} & \multicolumn{1}{c}{a_1} & \multicolumn{1}{c}{a_2} \\ \cline{2-3}
    a_1                  & S                       & \cdot                   \\ \cline{2-3}
    a_2                  & S                       & S                       \\ \cline{2-3}
  \end{array}
$$
$$
  \llave{l}{
    \relacion_1 = \set{(a_1,a_1)}                    \\
    \relacion_2 = \set{(a_2,a_2)}                    \\
    \relacion_3 = \set{(a_1,a_1), (a_2,a_2)} \\
    \relacion_4 = \set{(a_1,a_1), (a_1,a_2), (a_2, a_1)} \\
    \relacion_5 = \set{(a_2,a_2), (a_1,a_2), (a_2, a_1)} \\
    \relacion_6 = \set{(a_1,a_1), (a_2,a_2), (a_1,a_2), (a_2, a_1)} \\
    \relacion_7 = \set{(a_1,a_2), (a_2,a_1)}
  }
$$
Es así que se comprueba que las profundidades del infierno no están acotadas.
Por otro lado voy a ver cuántos elementos tengo para armar estas relaciones:
$$
  \begin{array}{c|c|c|c|c|c|c|c|}
    \multicolumn{1}{c}{} & \multicolumn{1}{c}{a_1} & \multicolumn{1}{c}{a_2} & \multicolumn{1}{c}{a_3} & \multicolumn{1}{c}{\cdots} & \multicolumn{1}{c}{a_{n-2}} & \multicolumn{1}{c}{a_{n-1}} & \multicolumn{1}{c}{a_n} \\ \cline{2-8}
    a_1                  & S                       & \cdot                   & \cdot                   & \cdots                     & \cdot                       & \cdot                       & \cdot                   \\ \cline{2-8}
    a_2                  & S                       & S                       & \cdot                   & \cdots                     & \cdot                       & \cdot                       & \cdot                   \\ \cline{2-8}
    a_3                  & S                       & S                       & S                       & \cdots                     & \cdot                       & \cdot                       & \cdot                   \\ \cline{2-8}
    \vdots               & \vdots                  & \vdots                  & \vdots                  & \ddots                     & \cdot                       & \cdot                       & \cdot                   \\ \cline{2-8}
    a_{n-2}              & S                       & S                       & S                       & \ddots                     & S                           & \cdot                       & \cdot                   \\ \cline{2-8}
    a_{n-1}              & S                       & S                       & S                       & \ddots                     & S                           & S                           & \cdot                   \\ \cline{2-8}
    a_n                  & S                       & S                       & S                       & \cdots                     & S                           & S                           & S                       \\ \cline{2-8}
  \end{array}
$$
La cantidad de elementos que hay marcados con una $S$ en la matriz y los $S$ debajo de la diagonal son respectivamente:
$$
  \frac{n \cdot (n+1)}{2}
  \ytext
  \frac{n \cdot (n - 1)}{2}
$$
Usando un razonamiento análogo a cuando calculamos las reflexivas estos elementos pueden aparecer o no aparecer
en la relación:
$$
  \ub{(2 \cdot 2 \cdots 2)}{\#(\textit{elementos $S$ en la matriz)}}
  =
  2^{\frac{(n^2 + n)}{2}}
$$

El total de relaciones \textit{simétricas}:
$$
  \cajaResultado{
    2^{\frac{(n^2 + n)}{2}}
  }
$$
\textit{Sanity check}: $n = 1$ tengo la relación \textit{simétrica} $(a_1, a_1)$ y $\vacio$.

\vspace{1cm}

\textit{¿Cuántas relaciones reflexivas y simétricas habrá?}

Como se laburó en la parte de reflexividad, tengo que agarrar \ul{todos los $n$ pares $(a_i,a_i)$} y para eso hay solo una forma de hacerlo.
Si quiero que las relaciones sean \textit{reflexivas} y \textit{simétricas} tengo que tomar de los elementos que conté para la
\textit{reflexividad} \ul{no diagonales} únicamente la mitad (por ejemplo todos los de abajo de la diagonal)
dado que cuando aparezca el elemento matricial $a_{ij}$ aparecerá el elemento $a_{ji}$
debido a la simetría, es decir no puedo tener la relación $\relacion$ que tenga al $a_{ij}$ sin su simétrico.

Serían un total de:
$$
  \oa{1}{\text{diagonales}} \cdot \ub{\frac{n\cdot (n-1)}{2}}{\text{no diagonales}}
$$

El total de relaciones \textit{simétricas} y \textit{reflexivas}:
$$
  \cajaResultado{
    2^{\frac{n^2 - n}{2}}
  }
$$

\begin{aportes}
  \item \aporte{\dirRepo}{naD GarRaz \github}
\end{aportes}
