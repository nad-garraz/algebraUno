\begin{enunciado}{\ejExtra}
  4400 ¿Cuántos divisores distintos tiene? ¿Cuánto vale la suma de sus divisores?
\end{enunciado}
Factorizo el número a estudiar:
$$
  4400 = 2^4 \cdot 5^2 \cdot 11
$$
Quiero encontrar los divisores $m$ de 4400, por lo tanto:
$$
  m \divideA 4400
  \sii
  m = \pm 2^\alpha \cdot 5^\beta \cdot 11^\gamma
  \quad\text{con}\quad
  \llaves{c}{
    0 \leq \alpha\leq \magenta{4}\\
    0 \leq \beta \leq \magenta{2}\\
    0 \leq \gamma \leq\magenta{1}
  }
$$

\hyperlink{teoria-4:cantidadDivisores}{Acá un poco de teoría sobre esto}.
Hay entonces un total de $(\magenta{4} + 1) \cdot (\magenta{2}+1) \cdot (\magenta{1} + 1) = 30$ divisores positivos y $60$ enteros.\\
Busco ahora la suma de esos divisores:
$$
  \sumatoria{i=0}{4} \sumatoria{j=0}{2}\sumatoria{k=0}{1} 2^i \cdot 5^j \cdot 11^k
  \igual{\red{!}}
  \parentesis{\sumatoria{i=0}{4} 2^i } \cdot \parentesis{ \sumatoria{j=0}{2} 5^j } \cdot \parentesis{ \sumatoria{k=0}{1} 11^k}\\
  \igual{\red{!!}}
  \frac{2^{4+1} - 1}{2 - 1} \cdot \frac{5^{2+1} - 1}{5 - 1} \cdot \frac{11^{1+1} - 1}{11 - 1} = 31 \cdot 31 \cdot 12 = 11532.
$$

Donde se separaran las sumatorias, porque los factores son independientes y luego se usó la fórmula geométrica.

\bigskip

Concluyendo hay un total de \yellow{60 divisores distintos}, cuya \yellow{suma es 11532}.

% Contribuciones
\begin{aportes}
  %% iconos : \github, \instagram, \tiktok, \linkedin
  %\aporte{url}{nombre icono}
  \item \aporte{\dirRepo}{naD GarRaz \github}
  \item \aporte{https://github.com/TobLoni}{Tobia Loni \github}
\end{aportes}
