\begin{enunciado}{\ejExtra}
  Caracterizar, para \textbf{cada} $a \en \enteros$, el valor de $(a^3 + 31: a^2 -a +1)$.
\end{enunciado}

Acá voy a \textit{simplificar la expresión}, y usar \textit{tabla de restos}. \par

\textit{Simplificar la expresión:} Siempre que se pueda está bueno usar el algoritmo de Euclides para bajarle el grado a la expresión:\par
\begin{center}
  \divPol{x^3 + 31}{x^2 - x +1}
\end{center}

$$
  d = (a^3 + 31: a^2 -a +1) \Entonces{Euclides} d = (a^2 - a + 1 : 30)
$$

Por lo tanto $d \en \set{1,2,3,5,6,10,15,30}$, muchísimos valores que puede tomar el mcd. Se descartan los pares dado que $a^2 - a +1$ es siempre impar:
$$
  \begin{array}{c}
    a = \ub{2k}{par} \entonces (2k)^2 - 2k + 1 = \ob{\ub{2 \cdot (2k^2 -k)}{par} + 1}{impar} \Tilde                 \\
    a = \ub{2k+1}{impar} \entonces (2k+1)^2 - 2(k+1) + 1 = \ob{\ub{2 \cdot (2k^2 + 3k + 2)}{par} + 1}{impar} \Tilde \\
  \end{array}
$$
Sin importar el valor de $a$, la expresión  $a^2 - a + 1$ no puede ser divisible por un número par es así que los posibles valores
de $d$ se reducen a $d \en \set{1,3,5,15}$, va tomando color.\par

\textit{Tabla de restos:} \par

Con la tabla de restos podemos ver si la expresión $a^2 - a + 1$ es divisible por algún valor de los que quedaron. \textit{Siempre empezar por los valores
  más bajos, porque ayudan a descartar los más altos}:

$$
  \begin{array}{|r|c|c|c|}
    \hline
    r_3(a)            & 0 & 1 & 2           \\ \hline
    r_3(a^2 - a + 1 ) & 1 & 1 & \magenta{0} \\ \hline
  \end{array}
$$

Cuando $\congruencia{a}{2}{3} $ la expresión es divisible por 3

$$
  \begin{array}{|r|c|c|c|c|c|}
    \hline
    r_5(a)            & 0 & 1 & 2 & 3 & 4 \\ \hline
    r_5(a^2 - a + 1 ) & 1 & 1 & 3 & 2 & 3 \\ \hline
  \end{array}
$$

La expresión no es divisible por 5 para ningún valor de $a \en \enteros$ y si no es divisible por 5 \textbf{tampoco lo será por 15} y así es como nos ahorramos
de calcular una tabla de restos de 15 horribles números \rosa{\faIcon{brain}}.

Se puede concluir que los valores para el mcd pedido:

$$
  d =
  \llave{rcl}{
    1 \sii \noCongruencia{a}{2}{3} \\
    3 \sii \congruencia{a}{2}{3}
  }
$$

\begin{aportes}
  %% iconos : \github, \instagram, \tiktok, \linkedin
  \item \aporte{https://github.com/nad-garraz}{Nad Garraz \github}
\end{aportes}
