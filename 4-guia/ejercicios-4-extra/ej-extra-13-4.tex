\begin{enunciado}{\ejExtra}
  Calcular $(a\cdot b^2 : 3a^2 + 3b^2) $ para cada par de enteros $a$ y $b$ tales que $(a:b) = 3$.
\end{enunciado}
Hay que \textit{comprimizar}, \textit{encontrar posibles divisores}, \textit{interpretar resultado}.

\textit{Coprimizar:} \par
$$
  (a:b) = 3 \sii (\frac{a}{3}: \frac{b}{3}) = 1 \Sii{$a=3A$}[$b=3B$] (A:B) = 1 \sii A \cop B.
$$

\textit{Reemplazo y acomodo:}
$$
  d = (a\cdot b^2 : 3a^2 + 3b^2)
  \Sii{\red{!}}
  d =  27(A \cdot B^2 : A^2 + B^2)
  \Sii{d = 27D }
  D =  (A \cdot B^2 : A^2 + B^2)
  \text{ con } A \cop B
$$

Dado que $D$ es el mcd, tiene que cumplir que:

$$
  \llave{l}{
    D \divideA A \cdot B^2 \\
    D \divideA A^2 + B^2
  }
  \Sii{\red{!!}}
  \llave{l}{
    D \divideA A^3 \\
    D \divideA B^4
  }
$$
Oka, ahí en el \red{!!} hice lo de siempre: Multipliqué una fila por $A$, la otra por $B^2$ y resté y coso.

Lo que nos queda es algo muy parecido a lo que pasó en el ejercicio \hyperlink{ejExtra:4-12-coprimos}{éste{\tiny(click)}}.

\medskip

\textit{Interpretación:}

Tenemos que $D$ por su condición de divisor común debe dividir a dos número \textit{coprimos},
dado que si $A \cop B$ también sucede que $A^3 \cop B^4$,
 \textit{because primos and shit}, y bueh, ¿Puede ser eso posible?.. Sí! Cuando $D = 1$.

Entonces:
$$
  D = 1 \entonces d = 27 \text{ para cada par } (a,b) \en \enteros \big/ (a:b) = 3
$$

% Contribuciones
\begin{aportes}
  \item \aporte{\dirRepo}{naD GarRaz \github}
\end{aportes}
