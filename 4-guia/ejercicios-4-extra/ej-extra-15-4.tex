\begin{enunciado}{\ejExtra}
    \red{Este ejercicio está pendiente de revisión, OJO} 
  Sean $a,\,b \en \enteros$ tales que $(a:b) = 1$. Calcular los posibles valores de $(a^2+3b^2 : 2a^2 + 11b^2)$
  y dar un ejemplo para cada uno de ellos.
\end{enunciado}

Si $d = (a^2+3b^2 : 2a^2 + 11b^2)$ entonces deber suceder:
$$
  \llave{l}{
    d \divideA a^2 + 3b^2 \\
    d \divideA 2a^2 + 11b^2
  }
  \Sii{$F_2 - 2 F_1 \to F_2 $}
  \llave{l}{
    d \divideA a^2 + 3b^2 \\
    \blue{d \divideA 5b^2}
  }
  \ytext
  \llave{l}{
    d \divideA a^2 + 3b^2 \\
    d \divideA 2a^2 + 11b^2
  }
  \Sii{$ 11 F_1 - 3 F_2 \to F_2 $}
  \llave{l}{
    d \divideA a^2 + 3b^2 \\
    \blue{d \divideA 5a^2}
  }
$$
De esta forma queda que el MCD:
$$
  d = (5a^2 : 5b^2)
  \sii
  d = 5(a^2 : b^2)
  \sii
  d = 5(a : b)^2
  \Sii{$a \cop b$}
  d = 5
$$

Si el \textit{máximo común divisor} de
$(a^2+3b^2 : 2a^2 + 11b^2)$ es 5, los valores que puede \textit{potencialmente} tomar la expresión son:
$$
  \set{1, 5}
$$

\textit{División por 1:}\par
El uno está por ejemplo para el par $(a,b) = (1,2)$ donde $a \cop b$.

\textit{División por 5:}
$$
  \begin{array}{|c|c|c|c|c|c|}
    \hline
    r_5(a)   & 0 & 1 & 2 & 3 & 4 \\ \hline
    r_5(a^2) & 0 & 1 & 4 & 4 & 1 \\ \hline
  \end{array}
  \ytext
  \begin{array}{|c|c|c|c|c|c|}
    \hline
    r_5(b)    & 0 & 1 & 2 & 3 & 4 \\ \hline
    r_5(3b^2) & 0 & 3 & 2 & 2 & 3 \\ \hline
  \end{array}
  \ytext
  \cajaResultado{
    \begin{array}{|c|c|c|c|c|c|}
      \hline
      r_5(a^2 + 3b^2) & 0 & 4 & 1 & 1 & 4 \\ \hline
    \end{array}
  }
$$
Ese resultado dice que para que suceda que $ 5 \divideA a^2 + 3b^2$ se requiere que:
$$
  \congruencia{a}{0}{5}
  \ytext
  \congruencia{b}{0}{5}
$$
Peeeeeero, por enunciado $(a:b) = 1$ así que se concluye que no hay par de $(a,b)$ con $a\cop b$ tal que $ 5 \divideA a^2 + 3b^2$.

\bigskip

Así que el único valor que puede tomar la expresión $(a^2+3b^2 : 2a^2 + 11b^2)$ es 1.

\begin{aportes}
  \item \aporte{https://github.com/JowinTeran}{Ale Teran \github}
  \item \aporte{\dirRepo}{naD GarRaz \github}
\end{aportes}
