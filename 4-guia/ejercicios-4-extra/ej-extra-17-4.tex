\begin{enunciado}{\ejExtra}
  Hallar todos los valores de $a \en \enteros$ tales que $(3a+6 : 7a^2 - a -3) \distinto 1$.
\end{enunciado}
Si el mcd es $d$:
$$
  d = (3a+6 : 7a^2 - a -3)
$$
% Puedo usar Euclides para simplificar la expresión del mcd:
% $$
%   \polyset{vars=a}
%   \divPol{7a^2-a-3}{3a+6}
% $$
% Por lo tanto $d$ queda:
% $$
%   d = (3a+6 : 7a^2 - a -3) =
%   \cajaResultado{(3a + 6 : 27)}
% $$
Tengo que $d$ es un \textit{divisor común a ambas expresiones:}
$$
  \llave{l}{
    d \divideA 7a^2 - a - 3\\
    d \divideA 3a + 6
  }
  \Sii{$3F_1 - 7aF_2 \to F_1$}
  \llave{l}{
    d \divideA - 45a - 9\\
    d \divideA 3a + 6
  }
  \Sii{$F_1 + 15F_2 \to F_1$}
  \llave{l}{
    d \divideA  81 \\
    d \divideA 3a + 6
  }
$$
Como $81 = 3^4$, los posibles divisores $d$:
$$
  d \en \set{1, 3, 9, 27, 81}
$$
Empiezo a ver si es divisible por $d = 3$:
Tabla de restos para $d = 3$:
$$
  \begin{array}{|r||c|c|c|}
    \hline
    r_3(a)            & 0 & 1 & 2       \\ \hline\hline
    r_3(3a+6)         & 0 & 0 & 0       \\ \hline
    r_3(7a^2 - a - 3) & 0 & 0 & \red{2} \\ \hline
  \end{array}
$$
Cuando tenga valores de:
$$
  \llave{l}{
    \congruencia{a}{0}{3} \\
    \otext                \\
    \congruencia{a}{1}{3}
  }
  \sisolosi
  d \distinto 1
$$
Cuando
$
  \congruencia{a}{2}{3}
$
no son \underline{ambas expresiones} divisibles por 3, por eso se descarta.

Dado que los otros posibles divisores (9, 27, 81) son potencias de 3, se concluye que solo valdrá que:
$$
  \cajaResultado{
    d \neq 1
    \sisolosi
    \llave{l}{
      \congruencia{a}{0}{3} \\
      \otext                \\
      \congruencia{a}{1}{3}
    }
  }
$$

Importante que en este ejercicio no pidieron encontrar los posibles valores de $d$, solo que fueran \underline{distintos de uno}. De no ser así
habría que haber hecho, por ejemplo la tabla de restos 9, para ver si el nuevo era un posible $d$ y así con los posibles divisores.

\begin{aportes}
  \item \aporte{\dirRepo}{naD GarRaz \github}
  \item \aporte{https://github.com/daniTadd}{Dani Tadd \github}
  \item \aporte{https://github.com/olivportero}{Olivia Portero \github}
\end{aportes}
