\begin{enunciado}{\ejExtra}
  Sea $a \en \enteros$. Calcule todos los posibles valores de $d = (a^2 + a : a^3 + 3a^2 + 2a + 14)$. Para
  cada posibilidad de $d$ hallada, caracterice todos los $a \en \enteros$ para los cuales se obtiene dicho valor de $d$.
\end{enunciado}

Simplifico al MCD, $d$ usando Euclides:
$$
  \polyset{vars=a}
  \divPol{a^3 + 3a^2 + 2a + 14}{a^2 + a}
$$
Por lo tanto:
$$
  d = (a^2 + a :  14)
  \entonces
  \llave{l}{
    d \divideA a^2 + a\\
    \ytext\\
    d \divideA 14
  }
  d \en \set{1,2,7,14}
$$

\bigskip

\textit{Tabla de restos:}

Empiezo por los números menores.

\medskip

\textit{¿$d = 2$ divide las expresiones?:}
$$
  \begin{array}{|c|c|c|}
    \hline
    r_2(a)       & 0           & 1           \\\hline
    r_2(a^2 + a) & \magenta{0} & \magenta{0} \\\hline
  \end{array}
$$
Se concluye que el 2 es un \textit{divisor común} para \underline{cualquier valor de $a$}. Peeeeero como yo quiero al \textit{MAYOR divisor común}, sigo
probando con los otros posibles valores de $d$.

\medskip

\textit{¿$d = 7$ divide las expresiones?:}
$$
  \begin{array}{|c|c|c|c|c|c|c|c|}
    \hline
    r_2(a)       & 0           & 1 & 2 & 3 & 4 & 5 & 6           \\\hline
    r_2(a^2 + a) & \magenta{0} & 2 & 6 & 5 & 6 & 2 & \magenta{0} \\\hline
  \end{array}
$$
Se concluye que el 7 es un \textit{divisor común} para:
$$
  \congruencia{a}{0}{7}
  \otext
  \congruencia{a}{6}{7}
$$

\medskip

\textit{¿$d = 14$ divide las expresiones?:}

No hace falta hacer la tabla. ¿Por qué?

Bueno, resulta que 14 va a ser un \textit{divisor común} ¡Cuando tanto 2 y 7 lo sean! Por
lo tanto 14 es un \textit{divisor común} cuando:
$$
  \congruencia{a}{0}{7}
  \otext
  \congruencia{a}{6}{7}
$$

Vamos redondeando. Los valores de el MCD, $d$ van a ser:
$$
  \cajaResultado{
    d =
    \llave{ccl}{
      14 &\sii&
      \llave{l}{
        \congruencia{a}{0}{7}\\
        \otext\\
        \congruencia{a}{6}{7}
      }\\
      \\
      2 &\sii& \text{otro caso}
    }.
  }
$$

\begin{aportes}
  \item \aporte{\dirRepo}{naD GarRaz \github}
  \item \aporte{https://github.com/daniTadd}{Dani Tadd \github}
\end{aportes}

