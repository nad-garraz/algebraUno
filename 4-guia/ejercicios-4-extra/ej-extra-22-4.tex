\begin{enunciado}{\ejExtra}
	Hallar el menor $n \en \naturales$ tal que $(11n^2 : 2^2 \cdot 3^4 \cdot 5 \cdot 11) = 99$ y
	que tenga exactamente 12 divisores positivos.
\end{enunciado}

Lindo enunciado. Bautizo:
$$
	d = (11n^2 : 2^2 \cdot 3^4 \cdot 5 \cdot 11)
$$
Factorizando ese $d$:
$$
	99 = 3^2 \cdot 11
$$
Por la forma en que está expresado el número este ejercicio sale entendiendo que:
\parrafoDestacado{
	El \textit{máximo común divisor}, $d = (a:b)$, es el producto de los
	primos comunes, de las factorizaciones de $a$ y $b$, elevados al menor exponente.
}
Para que $d$ sea 99 entonces:
$$
	n = 3 \cdot k \quad \text{con } k \en \naturales.
$$
Ya está con eso tenemos que $d = 99$ siempre que lo que agreguemos en $k$ \underline{no tenga ningún primo en común} con:
$$
	2^2 \cdot 3^4 \cdot 5 \cdot 11.
$$
Tenemos que agregar el primo más chico que no rompa esa condición para conseguir el menor $n$ posible:
$$
	7,\text{ en la quiniela \textit{revólver}.}
$$
Entonces:
$$
	\cajaResultado{
		n = 3^{\blue{1}} \cdot 7^{\blue{5}}
	}
$$
Ese número tiene $(\blue{1} + 1) \cdot (\blue{5} + 1) = 12$ divisores.
\hyperlink{teoria-4:cantidadDivisores}{Acá un poco de teoría sobre esto \click}.

\begin{aportes}
	\item \aporte{\dirRepo}{naD GarRaz \github}
\end{aportes}
