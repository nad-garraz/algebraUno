\begin{enunciado}{\ejExtra[1er cuatrimestre 2025]} \fechaEjercicio{1er cuatrimestre 2025}

  Hallar el menor $n \en \naturales$ tal que $(11n^2 : 2^2 \cdot 3^4 \cdot 5 \cdot 11) = 99$ y
  que tenga exactamente 12 divisores positivos.
\end{enunciado}

Lindo enunciado. Bautizo:
$$
  d = (11n^2 : 2^2 \cdot 3^4 \cdot 5 \cdot 11)
$$
Factorizando ese $d$:
$$
  99 = 3^2 \cdot 11
$$
Por la forma en que está expresado el número este ejercicio sale entendiendo que:

\parrafoDestacado[\red{\small\atencion}]{\Large
  El \textit{máximo común divisor}, $d = (a:b)$, es el producto de los
  primos, de las factorizaciones de $a$ y $b$, \ul{comunes elevados al menor exponente}.
}

Para que $d$ sea 99 entonces:
$$
  n = 3^{\yellow{1}} \cdot k \quad \text{con } k \en \naturales.
$$
Donde le puse a 3 el exponente $\yellow{1}$ para que no rompa la condición del $d = 99$.

Ya está con eso tenemos que $d = 99$ siempre que lo que agreguemos en $k$ \ul{no tenga ningún primo en común} con:
$$
  2^2 \cdot 3^4 \cdot 5.
$$
Tenemos que agregar entonces la combinación de primos que no rompan nada y que además generen un $n$ con 12 divisores,
ya sea $7^{\blue{\alpha}}, 11^{\green{\beta}}, 13^{\magenta{\gamma}}, 17^{\violet{\delta}}, \ldots$, peeeeero
para que la cantidad de divisores de $n$ sea 12 no puede haber más de 3 primos en la factorización,
sino \ul{habría más de 3 divisores} (\hyperlink{teoria-4:cantidadDivisores}{Acá un poco de formulitas sobre esto \click}).

Armo sistema con los 3 primos más chicos que puedo elegir:
$$
  \llave{rcl}{
    n & = & 3^{\yellow{1}} \cdot 7^{\blue{\alpha}} \cdot 11^{\green{\beta}}\\
    (\yellow{1} + 1) \cdot (\blue{\alpha} + 1) \cdot (\green{\beta} + 1) & = &  12
    \quad \sisolosi \quad
    (\blue{\alpha} + 1) \cdot (\green{\beta} + 1) = 6
  }
$$
Quedan solo 4 opciones:
$$
  \llave{rcl}{
    (\blue{\alpha},\green{\beta)} = (1,2) & \entonces & n = 3^{\yellow{1}} \cdot 7^{\blue{1}} \cdot 11^{\green{2}}\quad\text{\tiny\skull}\\
    (\blue{\alpha},\green{\beta)} = (2,1) & \entonces & n = 3^{\yellow{1}} \cdot 7^{\blue{2}} \cdot 11^{\green{1}}  \quad\text{\tiny\faIcon[regular]{smile}}\\
    (\blue{\alpha},\green{\beta)} = (5,0) & \entonces & n = 3^{\yellow{1}} \cdot 7^{\blue{0}} \cdot 11^{\green{5}}  \quad\text{\tiny\skull}\\
    (\blue{\alpha},\green{\beta)} = (0,5) & \entonces & n = 3^{\yellow{1}} \cdot 7^{\blue{5}} \cdot 11^{\green{0}}  \quad\text{\tiny\skull}
  }
$$
La combinación que da \ul{el menor $n$} es:
$$
  \cajaResultado{
    n = 3^1 \cdot 7^2 \cdot 11^1
  }
$$

\begin{aportes}
  \item \aporte{\dirRepo}{naD GarRaz \github}
\end{aportes}
