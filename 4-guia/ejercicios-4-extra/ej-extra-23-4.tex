\begin{enunciado}{\ejExtra}\fechaEjercicio{(7/10/25) primer parcial}
  \begin{center}
    Probar que para todo $n \en \naturales$ vale que $36 \divideA 19^n -18n^2 - 1$.
  \end{center}
\end{enunciado}

Acomodo el enunciado como:
$$
  36 \divideA 19^n -18n^2 - 1
  \Sii{def}
  \congruencia{19^n -18n^2 - 1}{0}{36}
$$

\textit{Inducción:}
Quiero probar que la proposición $p(n)$:
$$
  p(n) : \congruencia{19^n -18n^2 - 1}{0}{36} \paratodo n \en \naturales
$$
es verdadera.

\bigskip

\textit{Caso base:}
Quiero ver que para $n=1$ la proposición es verdadera.
$$
  p(\blue{1}) : \congruencia{19^{\blue{1}} -18 \cdot \blue{1}^2 - 1 = 0}{0}{36}.
$$
Por lo tanto $p(1)$ es verdadera.

\textit{Paso inductivo:}
Para algún $k \en \naturales$ asumo que la proposición:
$$
  p(\blue{k}) :
  \ub{
    \congruencia{19^{\blue{k}} - 18 \blue{k}^2 - 1}{0}{36}
  }{
    \text{\purple{hipótesis inductiva}}
  }
$$
es verdadera. Entonces quiero probar que:
$$
  p(\blue{k + 1}) : \congruencia{19^{\blue{k + 1}} -18 (\blue{k + 1})^2 - 1}{0}{36}
$$
también lo sea.

Partiendo del paso $k+1$:
$$
  \begin{array}{rcl}
    19^{\blue{k + 1}} -18 (\blue{k + 1})^2 - 1
     & \igual{\red{!}} &
    19 \cdot 19^{\blue{k}} - 18 \blue{k}^2 - 36\blue{k} - 19  \\
     & =               &
    19 \cdot (19^{\blue{k}} - 1) - 18 \blue{k}^2 - 36\blue{k} \\
     & \conga{36}      &
    19 \cdot (19^{\blue{k}} - 1) - 18 \blue{k}^2 \llamada1
  \end{array}
$$
Usando la \purple{hipótesis inductiva}:
$$
  \congruencia{19^{\blue{k}} - 18 \blue{k}^2 - 1}{0}{36}
  \Sii{$\llamada2$}
  \congruencia{19^{\blue{k}} - 1}{18 \blue{k}^2}{36}
$$
puedo escribir a $\llamada1$ como:
$$
  \begin{array}{rcl}
    \llamada1 \to 19 \cdot (19^{\blue{k}} - 1) - 18 \blue{k}^2
     & \conga{36}[\llamada2] &
    19 \cdot (18\blue{k}^2) - 18 \blue{k}^2 \\
     & =                     &
    18 \cdot (18\blue{k}^2)                 \\
     & =                     &
    18^2 \blue{k}^2                         \\
     & \conga{36}[\red{!}]   &
    0
  \end{array}
$$

Probando así que $p(\blue{k + 1})$ es verdadera.

Dado que $p(1), p(\blue{k}) \ytext p(\blue{k+1})$ resultaron verdaderas, por principio de inducción también lo es $p(n) \paratodo n \en \naturales$.

\begin{aportes}
  \item \aporte{\dirRepo}{naD GarRaz \github}
\end{aportes}

