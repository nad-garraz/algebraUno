\begin{enunciado}{\ejExtra[11/02/26 🌶Ejercicio de la Práctica]}
  \fechaEjercicio{11/02/26 Ejercicio de la Práctica}

  Sea $b \en \enteros$ impar y sea $a_n$ la sucesión de enteros dada por
  $$
    \llave{ccl}{
      a_1 & = & -6 \\
      a_{n+1} & = & 41 a_n + b^{2n} + 7
    }.
  $$
  Probar que $\congruencia{a_n}{2}{8},\, \paratodo n \en \naturales$.
\end{enunciado}

Sale por inducción:

Dada una suceción definida por recurrencia
$
  \llave{ccl}{
    a_1 & = & -6 \\
    a_{n+1} & = & 41 a_n + b^{2n} + 7
  }
$

Quiero probar que la siguiente proposición es verdadera:
$$
  p(n) :  \congruencia{a_n}{2}{8}
$$

\textit{Caso base:}

Quiero probar que
$$
  p(\blue{1}) :  \congruencia{a_{\blue{1}}}{2}{8}
$$
es verdadera.

Es trivialmente verdadera dado que $a_1 \igual{def} -6 \conga8 2$.

\bigskip

\textit{Paso inductivo:}

Para algún $\blue{k} \en \naturales$ asumo que la proposición:
$$
  p(\blue{k}) :  \ub{
    \congruencia{a_{\blue{k}}}{2}{8}
  }{
    \text{\purple{hipótesis inductiva}}
  }
$$
es verdadera. Entonces quiero ver que la proposición:
$$
  p(\blue{k+1}) :  \congruencia{a_{\blue{k+1}}}{2}{8}
$$
también lo sea.

Usando la \purple{hipótesis inductiva} en la definición de la sucesión:
$$
  \begin{array}{rcl}
    a_{\blue{k+1}} = 41 a_{\blue{k}} + b^{2\blue{k}} + 7
     & \conga{8}[\purple{\text{HI}}\red{!!}] &
    2 + b^{2\blue{k}} + 7                      \\\\
     & \conga{8}                             &
    b^{2\blue{k}} + 1 ~ \llamada1
  \end{array}
$$
Si puedo probar que $\congruencia{b^{2\blue{k}}}{1}{8}$ listo gané,
dado que me quedaría que $\congruencia{a_{\blue{k+1}}}{2}{8}$.
Lo voy a probar de 2 formas distintas, cuál se te ocurre a vos es cosa tuya. La primera es más \textit{"creativa y elegante"},
la segunda más \textit{"intuitiva y mecánica"}.

\parrafoDestacado{
  Mirar los ejercicios \refEjercicio{ej:4} y \refEjercicio{ej:13} puede servir de inspiración.
  Anyways, si no salió ahí va.
}

\parrafoDestacado[\magic]{
  \it
  ¿Cómo verga se hace esto?
  Es un salto creativo para pensar un ratito y probar cositas hasta que salga.
}

Resulta que $b$ es impar:
$$
  b \igual{$\llamada2$} 2\orange{i} + 1
  \entonces
  \llave{rcl}{
    b^{2\blue{k}}
    & =                   &
    (b^2)^{\blue{k}}         \\
    & \igual{$\llamada2$} &
    ((2\orange{i} + 1)^2)^{\blue{k}}\\
    & \igual{\red{!}}&
    (4
    \ub{
      \orange{i}(\orange{i+1})
    }{
      \text{esto es} \\ \text{siempre}\\ \text{par}
    } + 1 )^{\blue{k}}\\
    & \igual{\red{!!!}}&
    (8\magenta{j} + 1 )^{\blue{k}}\\
    & \conga{8} &
    (1 )^{\blue{k}} = 1\\
  }
$$

Si te perdiste en los \red{!}, en uno son cuentas y en otro escribo un número par en su forma genérica.

\bigskip
\parrafoDestacado{
  \it
  Estás pensando:

  ¡Eso no se me ocurre ni en pedo en un parcial! Fair enough, tabla de restos
}

$$
  \begin{array}{|c|c|c|c|c|c|c|c|c|c|}
    \hline
    r_8(\orange{i})          & 0 & 1 & 2 & 3 & 4 & 5 & 6 & 7 \\ \hline
    r_8((2\orange{i} + 1)^2) & 1 & 1 & 1 & 1 & 1 & 1 & 1 & 1 \\ \hline
  \end{array}
$$

Mucho más fácil honestamente, menos elegante y \textit{mathy snoby}, pero mucho más fácil. Elige tu propia aventura.

Así queda demostrado $\llamada1$ que $\congruencia{a_{\blue{k+1}}}{2}{8}$.

\bigskip

Dado que $p(1),\, p(\blue{k}), p(\blue{k+1})$ resultaron verdaderas, por el principio de inducción también lo es
$p(n) \paratodo n \en \naturales$.

\begin{aportes}
  \item \aporte{https://github.com/ivdou}{Ivan Doumerc \github}
  \item \aporte{\dirRepo}{naD GarRaz \github}
  \item \aporte{https://github.com/Diego-Dev-Moros}{Diego Moros \github}
\end{aportes}
