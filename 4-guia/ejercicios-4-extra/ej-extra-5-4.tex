\begin{enunciado}{\ejExtra}
  Determinar los posibles valores de $d = (a^2 - 2a -5: a-1)$ para $a \en \enteros$. Exhibir un valor de $a$
  correspondiente a cada uno de los valores de $d$ hallados.
\end{enunciado}

Parecido a cosas que ya se hicieron en otros ejercicios. Simplificamos si se puede con Euclides y después con tabla de restos
filtramos los máximos común divisores que quedaron.

\bigskip

\textit{Euclides con División de polinomios}
$$
\polyset{vars=a}
  \divPol{a^2 - 2a -5}{a-1}
$$

Que en el resto quede un número es una excelente noticia, podemos reescribir al mcd:
$$
  d = (a^2 - 2a -5: a-1) = (a-1 : -6)
$$

Con ese resultado y dado que $d  \divideA a - 1$ y también $d\divideA 6$:
$$
  d \en \set{1, 2, 3, 6}
$$

Tabla de restos para ver para que valores de $a$ se divide la expresión $a-1$
$$
  \begin{array}{|r|cc|}
    \hline
    r_2(a)     & 0 & 1           \\ \hline
    r_2(a - 1) & 1 & \magenta{0} \\ \hline
  \end{array}
  \quad
  \begin{array}{|r|ccc|}
    \hline
    r_3(a)     & 0 & 1           & 2 \\ \hline
    r_3(a - 1) & 2 & \magenta{0} & 1 \\ \hline
  \end{array}
  \quad
  \begin{array}{|r|cccccc|}
    \hline
    r_6(a)     & 0 & 1           & 2 & 3 & 4 & 5 \\ \hline
    r_6(a - 1) & 5 & \magenta{0} & 1 & 2 & 3 & 4 \\ \hline
  \end{array}
$$
Ahora hay que elegir un valor $a$ de forma tal que $d$ sea un valor que cumpla con los resultados.

Hay que tener cuidado, porque los conjuntos de $a$ que salen de la tabla de restos no son disjuntos.

Los siguientes valores salen a ojímetro:
$$
  \begin{array}{c}
    \text{si } a = 0 \entonces d = 1 \\
    \text{si } a = 5 \entonces d = 2 \\
    \text{si } a = 4 \entonces d = 3 \\
    \text{si } a = 7 \entonces d = 6
  \end{array}
$$

\begin{aportes}
  \item \aporte{\dirRepo}{naD GarRaz \github}
\end{aportes}
