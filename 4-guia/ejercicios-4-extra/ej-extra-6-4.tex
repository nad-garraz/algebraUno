\begin{enunciado}{\ejExtra}
  Sean $a$, $b \en \enteros$ tal que $(a:b) = 6$.
  Hallar todos los $d = (2a + b : 3a - 2b)$ y dar un ejemplo en cada caso.
\end{enunciado}

Conviene \textit{coprimizar} para hacer menos cuentas:
$$
  (a:b) = 6
  \sisolosi
  \llaves{l}{
    a = 6A\\
    b = 6B
  }
  ~\text{con}~
  (A:B) \igual{$\llamada{1}$} 1
$$

Uso ahora las nuevas variables coprimas entre sí, $A$ y $B$. Con esto la expresión de $d$ queda:
$$
  d =
  (2\cdot 6A + 6B : 3\cdot 6A - 2\cdot 6B) =
  (6\cdot( 2 \cdot A + B) : 6\cdot (3\cdot A - 2\cdot B)) =
  6 \cdot (2A + B : 3A - 2B) = D
$$

Entonces ahora puedo estudiar $D = (2A + B : 3A - 2B)$. Busco \textit{divisores comunes}:
$$
  \llave{l}{
    D \divideA 2A + B  \\
    D \divideA 3A - 2B
  } \Sii{\red{!}}
  \llave{l}{
    D \divideA 7B \\
    D \divideA 7A
  }
  \sii
  D = (7A:7B) = 7 \cdot (A:B) \igual{$\llamada{1}$} 7
$$
Por lo tanto los posibles divisores comúnes:
$$
  D \en \divsetP{7}{1,7},
$$
pero yo quiero encontrar ejemplos de $A$ y $B$:

\textit{Para $D = 7$}
$$
  A = 2
  \ytext
  B = 3
$$
Traduciendo esto para los valores de $a, b$ y $d$:
$$
  \cajaResultado{
    d = 42
    \ytext
    \llave{l}{
      a = 12\\
      b = 18
    }
  }
$$

\bigskip

\textit{Para $D = 1$}
$$
  A = 0
  \ytext
  B = 1
$$
Traduciendo esto para los valores de $a, b$ y $d$:
$$
  \cajaResultado{
    d = 6
    \ytext
    \llave{l}{
      a = 0\\
      b = 6
    }
  }
$$

\begin{aportes}
  \item \aporte{\dirRepo}{naD GarRaz \github}
\end{aportes}
