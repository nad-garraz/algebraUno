\begin{enunciado}{\ejExtra}
  Sea $a\en \enteros$ tal que $\congruencia{32a}{17}{9}$. Calcular $(a^3 + 4a + 1 : a^2 + 2)$

\end{enunciado}
Simplifico un poco:
$$
  \congruencia{32a}{17}{9}
  \sii
  \congruencia{5a}{8}{9}
  \Sii{$\times 2$}[$(\Leftarrow) 2 \cop 9$]
  \congruencia{a}{7}{9} \llamada1 \Tilde
$$

Simplifico la exprecion del MCD con euclides:\par
\begin{center}
  \polyset{vars=a}
  \divPol{a^3+4a+1}{a^2 + 2}
\end{center}

Entonces puedo escribir:
$$
  d = (a^3 + 4a + 1 : a^2 + 2) =
  (a^2 + 2 : 2a+1)
$$

Busco potenciales $d$:
$$
  \llave{l}{
    d \divideA a^2 + 2 \\
    d \divideA 2a + 1
  }
  \Sii{$2F_1 - aF_2$}
  \llave{l}{
    d \divideA -a + 4 \\
    d \divideA 2a + 1
  }
  \Sii{$2F_1 + F_2$}
  \llave{l}{
    d \divideA -a + 4 \\
    d \divideA 9
  }
$$

Por lo tanto la versión más simple quedó en: $ d = (-a+4 : 9) $. Posibles $d: \set{1,3,9}\Tilde $\par\medskip

Hago tabla de restos 9 y 3, para ver si las expresiones $(a^2 + 2 : 2a+1)$ son divisibles por mis potenciales $d$.\par

\textit{Tabla de restos para $d=9$}:
$$
  \begin{array}{|c|c|c|c|c|c|c|c|c|c|c|}
    \hline
    r_9(a)      & 0 & 1 & 2 & 3 & \magenta{4} & 5  & 6  & 7  & 8  \\ \hline\hline
    r_9(-a + 4) & 4 & 3 & 2 & 1 & \magenta{0} & -1 & -2 & -3 & -4 \\ \hline
  \end{array}
$$
Entonces los $a$ que cumplen \magenta{$\congruencia{a}{4}{9}$}, son candidatos para obtener $d$.\par

\textit{Tabla de restos para $d=3$}:
$$
  \begin{array}{|c|c|c|c|}
    \hline
    r_3(a)      & 0 & \magenta{1} & 2 \\ \hline\hline
    r_3(-a + 4) & 2 & \magenta{0} & 2 \\ \hline
  \end{array}
$$
Entonces los $a$ que cumplen \magenta{$\congruencia{a}{1}{3}$}, también con candidatos para obtener $d$ .\par

Estos resultados deben cumplir la condición $\llamada1 \congruencia{a}{7}{9}$ como se pide en el enunciado,
lo cual no es compatible con el resultado de la tabla de $r_9$, pero sí con la tabla $r_3$.
Notar que: $a = 9k + 7 \conga3 1 $.\par\bigskip

Finalmente el MCD con $a \en \enteros$ que cumplan que $\congruencia{32a}{17}{9}$
$$
  \boxed{
    (a^3 + 4a + 1 : a^2 + 2) = 3
  } \Tilde
$$

\begin{aportes}
  \item \aporte{\dirRepo}{naD GarRaz \github}
\end{aportes}
