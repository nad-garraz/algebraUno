\begin{enunciado}{\ejExtra}

  Determinar todos los $a \en \enteros$ que cumplen que

  $$
    \frac{2a - 1}{5} - \frac{a - 1}{2a - 3} \en \enteros.
  $$

\end{enunciado}

Busco una fracción. Para que esa fracción $\ en \enteros$ es necesario que el
denominador divida al numerador. Fin.
$$
  \frac{2a - 1}{5} - \frac{a - 1}{2a - 3} = \frac{4a^2 -13a + 8}{10a - 15}
$$

$$
  \llamada1\llave{l}{
    10a - 15 \divideA  4a^2 -13a + 8\\
    10a - 15 \divideA  10a - 15
  }
  \flecha{operaciones}[varias]
  \llave{l}{
    10a - 15 \divideA  -25 \llamada2\\
    10a - 15 \divideA  10a - 15.
  }
$$
Para que ocurra $\llamada1$, debe ocurrir $\llamada2$.
$$
  10a - 15 \divideA  -25
  \sisolosi
  10a-25 \en \set{\pm 1, \pm 5, \pm 25}\llamada3 \text{ para algún } a \en \enteros. \Tilde
$$
De paso observo que $|10a - 25| \leq 25$. Busco $a$:
$$
  \llave{rlcl}{
    \text{Caso: } & d = 10a - 15 = 1 & \sisolosi & a = \frac{8}{5} \quad \text{\faIcon{skull}}\\
    \text{Caso: } & d = 10a - 15 = -1 & \sisolosi & a = \frac{8}{5} \quad \text{\faIcon{skull}}\\
    \text{Caso: } & d = 10a - 15 = 5 & \sisolosi & a = \magenta{2} \Tilde\\
    \text{Caso: } & d = 10a - 15 = -5 & \sisolosi & a = \green{1}  \Tilde\\
    \text{Caso: } & d = 10a - 15 = 25 & \sisolosi & a = \blue{4}  \Tilde\\
    \text{Caso: } & d = 10a - 15 = -25 & \sisolosi & a = \yellow{-1}  \Tilde
  }
$$
Los valores de $a \en \enteros$ que cumplen $\llamada2$ son $\set{-1,1,2,4}$. Voy a evaluar y así encontrar
para cual de ellos se cumple $\llamada1$, es decir que el númerador sea un múltiplo del
denominador para el valor de $a$ usado.\par
$$
  \begin{array}{llcllr}
    d = 5   & a = \magenta{2} & \entonces & 4 \cdot \magenta{2}^2 - 13 \cdot \magenta{2} + 8 = -2     & \to & 5 \noDivide -2 \quad \text{\faIcon{skull}} \\
    d = -5  & a = \green{1}   & \entonces & 4 \cdot \green{1}^2 - 13 \cdot \green{1} + 8 = 1          & \to & -5 \noDivide 1 \quad \text{\faIcon{skull}} \\
    d = 25  & a = \blue{4}    & \entonces & 4 \cdot \blue{4}^2 - 13 \cdot \blue{4} + 8 = 4            & \to & 25 \noDivide 4 \quad \text{\faIcon{skull}} \\
    d = -25 & a = \yellow{-1} & \entonces & 4 \cdot (\yellow{-1})^2 - 13 \cdot (\yellow{-1}) + 8 = 25 & \to & -25 \divideA 25 \Tilde
  \end{array}
$$

El único valor de $a \en \enteros$ que cumple lo pedido es:
$$
  \cajaResultado{a = -1}
$$

\separadorCorto

\textit{Notas extras sobre el ejercicio:}\par
Para $a = \red{-1}$ se obtiene $\frac{2a - 1}{5} - \frac{a - 1}{2a - 3} = -1$. Más aún, si hubiese encarado el
ejercicio con tablas de restos para ver si lo de arriba es divisible por los divisores en $\llamada3$, calcularía:\par
$$
  r_5(4a^2 -13a + 8) \ytext  r_{25}(4a^2 -13a + 8)
$$

$$
  r_5 (4a^2 -13a + 8) = 0
  \sii
  \llave{l}{
    \congruencia{a}{3}{5}\\
    \congruencia{a}{4 \congruente \red{-1}}{5}
  }
  \ytext
  r_{25} (4a^2 -13a + 8) = 0
  \sii
  \llave{l}{
    \congruencia{a}{23}{25}\\
    \congruencia{a}{24 \congruente \red{-1}}{25}
  }
$$
Se puede ver también así que el único valor de $a \en \enteros$,
que cumple $\llamada1$ es $a = -1$

\begin{aportes}
  \item \aporte{\dirRepo}{naD GarRaz \github}
\end{aportes}
