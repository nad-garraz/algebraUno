\noindent\textit{\underline{Divisibilidad}}
\ejercicio
Decidir si las siguientes afirmaciones son verdaderas $\paratodo a,\, b,\, c \en \enteros$:\\
Calcular
\begin{enumerate}[label=\roman*)]
	\item $a \cdot b \divideA c \entonces a \divideA c$ \ y \ $b \divideA c$ \\
	      \separadorCorto

	      $\llave{l}{
			      c = k\cdot a \cdot b = \underbrace{h}_{k \cdot b} \cdot a \entonces a \divideA c \Tilde\\
			      c = k\cdot a \cdot b = \underbrace{i}_{k \cdot a} \cdot b \entonces b \divideA c\Tilde
		      }$

	\item $4 \divideA a^2 \entonces 2 \divideA a $\\
	      \separadorCorto

	      $ a^2 = k \cdot 4 = \underbrace{h}_{k \cdot 2} \cdot 2 \entonces a^2 \divideA 2
		      \flecha{si $a\cdot b \divideA c$}[$\entonces a \divideA c \y b \divideA c$]
		      a \divideA 2 \Tilde$

	\item $2 \divideA a \cdot b \entonces 2 \divideA a $ \ o \ $2 \divideA b$\\
	      \separadorCorto

	      Si $2 \divideA a \cdot b \entonces
		      \llaves{c}{
			      a \text{ tiene que ser } par \\
			      \o \\
			      b \text{ tiene que ser } par \\
		      } \flecha{para que} a \cdot b$ sea par. Por lo tanto si  $2 \divideA a \cdot b \entonces 2 \divideA a $ \ o \ $2 \divideA b$.

	\item $9 \divideA a\cdot b \entonces 9 \divideA a  $ \ o \ $9 \divideA b$\\
	      \separadorCorto

	      Si $a = 3 \y b = 3$, se tiene que $9 \divideA 9$, sin embargo $9 \noDivide 3$

	\item $a \divideA b + c \entonces a \divideA b $ \ o \  $a \divideA c$\\
	      \separadorCorto

	      $12 \divideA 20 + 4 \entonces 12 \noDivide 20$  \ y\   $ 12\noDivide 4 $

	\item
	      \separadorCorto
	      \hacer
	\item
	      \separadorCorto
	      \hacer
	\item
	      \separadorCorto
	      \hacer
	\item $a \divideA b + a^2 \entonces a \divideA b$\\
	      \separadorCorto

	      $  a \divideA b + a^2 \entonces b + a^2 = k \cdot a \flecha{acomodo} b = (k - a) \cdot a = h \cdot a \entonces a | b \Tilde$\\
	      $ \flecha{también puedo}[decir si:]
		      \llaves{l}{
			      a \divideA a^2 \\
			      a \divideA b - a^2
		      } \flecha{por}[propiedad] a \divideA (b - a^2) + (a^2) = b \entonces a \divideA b \Tilde $


	\item $a \divideA b \entonces a^n \divideA b^n, \paratodo n \en \naturales$\\
	      \separadorCorto

	      Pruebo por inducción. $p(n)\  :\  a \divideA b \entonces a^n \divideA b^n $\\
	      $
		      \llave{l}{
			      \text{Caso base: } n = 1 \entonces a|b \entonces a^1 \divideA b^1 \Tilde\\
			      \text{Paso inductivo: } \paratodo h \en \naturales, p(h) \ V \entonces p(h+1)\ V?\\
			      \text{Si } a \divideA b \entonces a^k \divideA b^k  \entonces a^k \cdot c = b^k
			      \flecha{multiplico por}[$b$ M.A.M]
			      b \cdot a^k \cdot c = b^{k+1}
			      \flecha{$a\divideA b$}[$a \cdot d = b$]
			      a\cdot d \cdot a^k \cdot c =  a^{k+1}\cdot(cd) = b^{k+1}\\
			      \flecha{concluyendo}[que] a^{k+1} \divideA b^{k+1} \text{como quería mostrarse.}
		      }
	      $ Como $p(1) \y p(k) \y p(k+1)$ resultaron verdaderas, por el principio de inducción $p(n)$
	      es verdadera $\paratodo n \en \naturales$\\
	      \textit{Este resultado es importante y se va a ver en muchos ejercicios.\\
		      $a \divideA b \entonces a^n \divideA b^n \sisolosi
			      \congruencia{b}{0}{a} \entonces \congruencia{b^n}{\ub{0}{\stackrel{(a^n)}\congruente a^n}}{a^n} \sisolosi \congruencia{b^n}{a^n}{a^n}$}


\end{enumerate}
