\textit{\underline{Divisibilidad}}

\begin{enunciado}{\ejercicio}
  Decidir si las siguientes afirmaciones son verdaderas
  $\paratodo a,\, b,\, c \en \enteros$
  \begin{multicols}{2}
    \begin{enumerate}[label=\alph*)]
      \item $a \cdot b \divideA c \entonces a \divideA c  \ytext  b \divideA c$
      \item $4 \divideA a^2 \entonces 2 \divideA a $
      \item $2 \divideA a \cdot b \entonces 2 \divideA a  \otext 2 \divideA b$
      \item $9 \divideA a\cdot b \entonces 9 \divideA a   \otext  9 \divideA b$
      \item $a \divideA b + c \entonces a \divideA b  \otext  a \divideA c$
      \item $a \divideA c \ytext b \divideA c \entonces a \cdot b \divideA c$
      \item $a \divideA b \entonces a \leq b$
      \item $a \divideA b \entonces |a| \leq |b|$
      \item $a \divideA b + a^2 \entonces a \divideA b$
      \item $a \divideA b \entonces a^n \divideA b^n, \paratodo n \en \naturales$
    \end{enumerate}
  \end{multicols}
\end{enunciado}

\begin{enumerate}[label=\alph*)]
  \item $a \cdot b \divideA c \entonces a \divideA c$ \ y \ $b \divideA c$\par
        $\llave{l}{
            c = k\cdot a \cdot b = \underbrace{h}_{k \cdot b} \cdot a \entonces a \divideA c \Tilde\\
            c = k\cdot a \cdot b = \underbrace{i}_{k \cdot a} \cdot b \entonces b \divideA c\Tilde
          }$

  \item $4 \divideA a^2 \entonces 2 \divideA a $\par
        $ a^2 = k \cdot 4 = \underbrace{h}_{k \cdot 2} \cdot 2 \entonces a^2 \divideA 2
          \flecha{si $a\cdot b \divideA c$}[$\entonces a \divideA c \y b \divideA c$]
          a \divideA 2 \Tilde$

  \item $2 \divideA a \cdot b \entonces 2 \divideA a $ \ o \ $2 \divideA b$\par
        Si $2 \divideA a \cdot b \entonces
          \llaves{c}{
            a \text{ tiene que ser } par \\
            \o \\
            b \text{ tiene que ser } par \\
          } \flecha{para que} a \cdot b$ sea par. Por lo tanto si  $2 \divideA a \cdot b \entonces 2 \divideA a $ \ o \ $2 \divideA b$.

  \item $9 \divideA a\cdot b \entonces 9 \divideA a  $ \ o \ $9 \divideA b$\par
        Si $a = 3 \y b = 3$, se tiene que $9 \divideA 9$, sin embargo $9 \noDivide 3$

  \item $a \divideA b + c \entonces a \divideA b $ \ o \  $a \divideA c$\par
        $12 \divideA 20 + 4 \entonces 12 \noDivide 20$  \ y\   $ 12\noDivide 4 $

  \item
        $a\divideA c \ytext b \divideA c \entonces a \cdot b \divideA c$\par%
        $4\divideA 12 \ytext 6 \divideA 12 \text{ pero } 24 \noDivide 12$

  \item
        $a\divideA b \entonces a \leq b$\par
        $2\divideA -4 \text{ pero } 2 > -4$
  \item
        $a\divideA b \entonces |a| \leq |b|$\par
        $4 \divideA 0 \text{ pero }|4| > 0$\par
        Igual, si $b \neq 0$ es verdadero, el enunciado no lo especifica pero ese caso es el que esta bueno ver.\par
        $\text{si } a\divideA b,\, b = k \cdot a$, tomo el modulo de todo porque que todo sea positivo no va a cambiar la veracidad de la prueba.\par
        $|b| = |k|\cdot|a|$, $|k| \geq 1$ asi que acoto por esa expresion, queda $|b| = |k|\cdot|a| \geq 1\cdot|a|$, finalmente $|b| \geq |a|,\, b \neq 0$

  \item $a \divideA b + a^2 \entonces a \divideA b$\par
        $  a \divideA b + a^2
          \entonces
          b + a^2 = k \cdot a
          \flecha{acomodo}
          b = (k - a) \cdot a = h \cdot a
          \entonces
          a | b \Tilde$\par
        $ \flecha{también puedo}[decir si:]
          \llaves{l}{
            a \divideA a^2 \\
            a \divideA b - a^2
          } \flecha{por}[propiedad]
          a \divideA (b - a^2) + (a^2) = b \entonces a \divideA b \Tilde $

  \item $a \divideA b \entonces a^n \divideA b^n, \paratodo n \en \naturales$\par
        Pruebo por inducción.
        $$
          p(n)  :  a \divideA b \entonces a^n \divideA b^n
        $$

        \textit{Caso base: }
        $$
          n = 1 \entonces a \divideA b \entonces a^1 \divideA b^1 \Tilde
        $$

        $p(1)$ resulta verdadera.\par

        \textit{Paso inductivo: }\par
        Asumo $\ub{p(h): a \divideA b \entonces a^h \divideA b^h}{\text{\purple{hipótesis inductiva}}}$
        verdadera $\entonces$ quiero ver que $p(h+1): a \divideA b \entonces a^{h+1} \divideA b^{h+1}$\par
        Parto de la \purple{hipótesis inductiva} y voy llegar a $p(k+1)$. Si:
        $$
          a \divideA b
          \Entonces{\purple{HI}}
          a^k \divideA b^k
          \sii
          a^k \cdot c = b^k
          \Sii{$\times b$}
          b \cdot a^k \cdot c = b^{k+1}
          \Sii{$a \divideA b$}[$a \cdot d = b$]
          a \cdot d \cdot a^k \cdot c =
          a^{k+1} \cdot (cd) = b^{k+1}\\
          \sii
          a^{k+1} \divideA b^{k+1}.
        $$
        Como $p(1),\, p(k) \ytext p(k+1)$ resultaron verdaderas, por el principio de inducción $p(n)$
        es verdadera $\paratodo n \en \naturales.$\medskip

        \textit{Este resultado es importante y se va a ver en muchos ejercicios:}
        $$
          a \divideA b \entonces a^n \divideA b^n \sisolosi
          \congruencia{b}{0}{a}
          \entonces
          \congruencia{b^n}{0}{a^n}
          \Sii{$0 \conga{a^n} a^n$}
          \congruencia{b^n}{a^n}{a^n}
        $$
        $$
          \boxed{
            a \divideA b
            \entonces
            \congruencia{b^n}{a^n}{a^n}
          }
        $$
\end{enumerate}

% Contribuciones
\begin{aportes}
  %% iconos : \github, \instagram, \tiktok, \linkedin
  %\aporte{url}{nombre icono}
  \item \aporte{\dirRepo}{naD GarRaz \github}
  \item \aporte{https://github.com/sigfripro}{sigfripro \github}
\end{aportes}
