\textit{\underline{Divisibilidad}}

\begin{enunciado}{\ejercicio}
  Decidir si las siguientes afirmaciones son verdaderas
  $\paratodo a,\, b,\, c \en \enteros$
  \begin{multicols}{2}
    \begin{enumerate}[label=\alph*)]
      \item $a \cdot b \divideA c \entonces a \divideA c  \ytext  b \divideA c$
      \item $4 \divideA a^2 \entonces 2 \divideA a $
      \item $2 \divideA a \cdot b \entonces 2 \divideA a  \otext 2 \divideA b$
      \item $9 \divideA a\cdot b \entonces 9 \divideA a   \otext  9 \divideA b$
      \item $a \divideA b + c \entonces a \divideA b  \otext  a \divideA c$
      \item $a \divideA c \ytext b \divideA c \entonces a \cdot b \divideA c$
      \item $a \divideA b \entonces a \leq b$
      \item $a \divideA b \entonces |a| \leq |b|$
      \item $a \divideA b + a^2 \entonces a \divideA b$
      \item $a \divideA b \entonces a^n \divideA b^n, \paratodo n \en \naturales$
    \end{enumerate}
  \end{multicols}
\end{enunciado}

Este ejercicio clave para el resto de la materia.

\begin{enumerate}[label=\alph*)]
  \item\label{ej1:itemA} Esta proposición es \green{verdadera}:
        $$
          a \cdot b \divideA c \Sii{def}  c = a \cdot b \cdot q \quad \text{con } q \en \enteros
        $$
        Mirando cada número por separado:
        $$
          \llave{l}{
            c =
            a \cdot b \cdot q
            \igual{\red{!}} a \cdot q'  \Sii{def} a \divideA c \\
            c =
            a \cdot b \cdot q
            \igual{\red{!}} b \cdot q^{''}  \Sii{def} b \divideA c
          }
        $$

  \item La proposición es \green{verdadera}:
        $$
          4 \divideA a^2
          \Sii{def}
          a^2 = 4 \cdot q = 2 \cdot 2 \cdot q = 2 \cdot q'
        $$
        Es decir que $a^2$ es un número par, más aún:
        $$
          a^2 \text{ es par } \sisolosi a \text{ es par} \quad \text{con}~  a \en \enteros
        $$
        Y bueh, si $a$ es par entonces $2 \divideA a$.

  \item La afirmación es \green{verdadera}.
        $$
          2 \divideA a \cdot b
          \Sii{def}
          a \cdot b = 2 \cdot q \quad \llamada1
        $$
        El producto de 2 números es par, si y solo si alguno es par:

        \textit{2 \blue{pares}}:
        $$
          \blue{2n} \cdot \blue{2m} = 2 \cdot \ub{(2\cdot n \cdot m)}{q'} = 2 \cdot q'
        $$
        \textit{1 \blue{par} y el otro \orange{impar}}
        $$
          \blue{2n} \cdot (\orange{2m - 1}) = 2 \cdot (\ub{2\cdot n \cdot m - n}{q'}) = 2 \cdot q'
        $$
        Con este resultado en $\llamada1$ queda:
        $$
          a \cdot b = 2 \cdot q' \sisolosi 2 \divideA a ~\lor~ 2 \divideA b
        $$

  \item La proposición es \red{falsa}.

        \textit{Contraejemplo:} Si $a = 3 \y b = 3$, se tiene que $9 \divideA 9$, sin embargo $9 \noDivide 3$.

  \item La proposición es \red{falsa}.

        \textit{Contraejemplo}: Se tiene que $12 \divideA 20 + 4$, sin embargo $12 \noDivide 20  ~~\text{ni}~~  12\noDivide 4$

  \item La proposición es \red{falsa}.

        \textit{Contraejemplo}: Se tiene que $4 \divideA 12 ~\land~ 6 \divideA 12$, sin embargo $24 \noDivide 12$

  \item La proposición es \red{falsa}.

        \textit{Contraejemplo}: Se tiene que $2 \divideA -4$, sin embargo $2 \not\leq -4$

  \item La proposición es \red{falsa}.

        \textit{Contraejemplo}: Se tiene que $4 \divideA 0$, sin embargo $|4| > 0$

        En el caso en que $b \distinto 0$ la proposición es \green{verdadera}.
        $$
          a \divideA b
          \Sii{def}[$q \en \enteros_{\distinto 0}$]
          b = a \cdot q
          \Sii{\red{!}}
          |b| = |a \cdot q|
          \sii
          |b| = |a| \cdot |q|
          \Sii{\red{!}}
          |b| \geq |a|
        $$

  \item La proposición es \green{verdadera}.
        $$
          a \divideA b + a^2
          \Sii{def}
          b + a^2 = a \cdot q
          \Sii{\red{!}}
          b = a \cdot (q - a)
          \sii
          b = a \cdot q'
          \Sii{def}
          a \divideA b
        $$

  \item La proposición es \green{verdadera}.

        Pruebo por inducción. Quiero probar que la siguiente proposición es verdadera:
        $$
          p(n)  :  a \divideA b \entonces a^n \divideA b^n
        $$

        \textit{Caso base: }
        $$
          p(\blue{1}) :  a \divideA b \entonces a^{\blue{1}} \divideA b^{\blue{1}}
        $$
        $p(\blue{1})$ resultó verdadera.

        \textit{Paso inductivo:}

        Asumo  que para algún $\blue{h} \en \naturales$ la proposición:
        $$
          p(\blue{h}): \ub{a \divideA b \entonces a^{\blue{h}} \divideA b^{\blue{h}}}{\text{\purple{hipótesis inductiva}}}
        $$
        es verdadera, entonces quiero probar que la proposición:
        $$
          p(\blue{h+1}): a \divideA b \entonces a^{\blue{h + 1}} \divideA b^{\blue{h + 1}}
        $$
        también lo sea.

        Si:
        $$
          \begin{array}{rcl}
            a \divideA b
             & \Entonces{\purple{HI}}                 &
            a^{\blue{h}} \divideA b^{\blue{h}}                                                        \\
             & \Sii{def}                              &
            b^{\blue{h}} = a^{\blue{h}} \cdot q                                                       \\
             & \Sii{$\times b$}                       &
            b^{\blue{h + 1}} =  b \cdot a^{\blue{h}} \cdot q                                          \\
             & \Sii{$a \divideA b$}[$b = a \cdot q'$] &
            b^{\blue{h + 1}} =  a \cdot q' \cdot a^{\blue{h}} \cdot q = a^{\blue{h + 1}} \cdot q^{''} \\
             & \Sii{def}                              &
            a^{\blue{h + 1}} \divideA b^{\blue{h + 1}}.
          \end{array}
        $$
        Como $p(\blue{1}),\, p(\blue{h}) \ytext p(\blue{h+1})$ resultaron verdaderas, por el principio de inducción la proposición
        $p(n)$ es verdadera $\paratodo n \en \naturales$.

        \textit{Este resultado es importante y se va a ver en muchos ejercicios:}
        $$
          a \divideA b \entonces a^n \divideA b^n \sisolosi
          \congruencia{b}{0}{a}
          \entonces
          \congruencia{b^n}{0}{a^n}
          \Sii{$0 \conga{a^n} a^n$}
          \congruencia{b^n}{a^n}{a^n}
        $$
        $$
          \boxed{
            a \divideA b
            \entonces
            \congruencia{b^n}{a^n}{a^n}
          }
        $$
\end{enumerate}

% Contribuciones
\begin{aportes}
  %% iconos : \github, \instagram, \tiktok, \linkedin
  %\aporte{url}{nombre icono}
  \item \aporte{\dirRepo}{naD GarRaz \github}
  \item \aporte{https://github.com/sigfripro}{sigfripro \github}
\end{aportes}
