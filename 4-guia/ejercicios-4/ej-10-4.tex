\begin{enunciado}{\ejercicio}
  \begin{enumerate}[label=\alph*)]
    \item
          Si $\congruencia{a}{22}{14}$, hallar el resto de dividir a $a$ por 14, por 2 y por 7.
    \item
          Si $\congruencia{a}{13}{5}$, hallar el resto de dividir a $33a^3 + 3a^2 -197a +2$ por 5.
    \item Hallar, para cada $n \en \naturales$, el resto de la división de $\sumatoria{i=1}{n} (-1)^i \cdot i!$ por 12
  \end{enumerate}

\end{enunciado}

\begin{enumerate}[label=\alph*)]
  \item $\llave{l}{
            \congruencia{a}{22}{14} \to a = 14 \cdot q + \ub{22}{14 + 8} = 14 \cdot (q + 1) + 8 \flecha{el resto}[es] r_{14}(a) = 8 \Tilde\\
            \congruencia{a}{22}{14} \to a = \ub{14 \cdot q}{2 \cdot (7 \cdot q)} + \ub{22}{2 \cdot 11} = 2 \cdot (7q + 11) + 0 \flecha{el resto}[es] r_{2}(a) = 0 \Tilde\\
            \congruencia{a}{22}{14} \to a = \ub{14 \cdot q}{7 \cdot (2 \cdot q)} + \ub{22}{1 + 7 \cdot 3} = 7 \cdot (2q + 3) + 1 \flecha{el resto}[es] r_{7}(a) = 1 \Tilde\\
          }$

  \item  Dos números congruentes tienen el mismo resto. $\congruencia{a}{13}{5}  \sisolosi \congruencia{a}{3}{5}$
        $r_5(33a^3 + 3a^2 -197a +2) = r_5( 3 \cdot r_5(a)^3 + 3 \cdot r_5(a)^2 - 2\cdot r_5(a) + 2 )\\
          \flecha{como $\congruencia{a}{13}{5}$}[$r_5(a) = 3$]
                r_5(33a^3 + 3a^2 -197a +2) = 4$

  \item Queremos hallar el resto, planteemos una ecuacion de congruencia:
        \[\congruencia{\sumatoria{i=1}{n} (-1)^i \cdot i!}{r}{12}\]
         \\
        Ahora expandimos la suma para ver el patron que tiene y ver si podemos simplificar algo: 
        \[\congruencia{-1+2-6+24-\cdots+(-1)^k\cdot n!}{r}{12}\] \\
        Notemos que el $24$ es divisor de $12$, además todos los terminos subsiguientes, van a tener de factores $4\cdot3$, por lo cual 
        los terminos que le sigue al 24 todos van a ser divisibles por 12, o sea podemos reescribir la ecuacion como:
        \[\congruencia{-1+2-6+12h}{r}{12}\] \\
        Luego:
        \[\congruencia{-1+2-6}{r}{12}\]
        \[\congruencia{-5}{r}{12}\]
        Sumo $12$ a ambos miembros 
        \[\congruencia{-5 + 12}{r + 12}{12}\]
        Finalmente
        \[\congruencia{7}{r}{12}\]
        Entonces concluimos que el resto para los $n \geq 4$ al dividir por 12 es de $7$. Para los primeros terminos
        va a haber que calcularlo manualmente. \\
        Para $n = 1$, $\congruencia{-1}{11}{12}$ \\
        Para $n = 2$ $\congruencia{-1 + 2}{1}{12}$ \\
        Para $n = 3$, $\congruencia{-1+2-6}{7}{12}$ \\
        La respuesta final es entonces: \cajaResultado{$\text{Para $n = 1$, $r = 11$, $n = 2$, $r = 1$, $n \geq 3$, $r=7$}$}

\end{enumerate}

\begin{aportes}
  \item \aporte{https://github.com/sigfripro}{sigfripro \github}
\end{aportes}
