\begin{enunciado}{\ejercicio}
  \begin{enumerate}[label=\alph*)]
    \item Probar que $\congruencia{a^2}{-1}{5} \sisolosi \congruencia{a}{2}{5}  \o \congruencia{a}{3}{5}$
    \item Probar que no existe ningún entero $a$ tal que $\congruencia{a^3}{-3}{7}$
    \item Probar que $\congruencia{a^7}{a}{7} \paratodo a \en \enteros$
    \item Probar que $7 \divideA a^2 + b^2 \sisolosi 7 \divideA a \y 7 \divideA b.$
    \item Probar que $5 \divideA a^2 + b^2 + 1 \entonces 5 \divideA a \otext 5 \divideA b$.
          ¿Vale la implicación recíproca?
  \end{enumerate}
\end{enunciado}

\begin{enumerate}[label=\alph*)]
  \item Me piden que pruebe una congruencia es válida solo para ciertos $a \en \enteros$.
        Pensado en términos de \textit{restos} quiero que el resto al
        poner los $a$ en cuestión cumplan la congruencia.\par

        $
          \llave{l}{
            \congruencia{a^2}{-1}{5} \sii
            \congruencia{a^2}{4}{5}  \sii
            \congruencia{a^2 - 4}{0}{5} \sii
            \congruencia{(a-2)\cdot (a+2)}{0}{5}\\
            \flecha{quiero}[resto 0]
            r_5(a^2 + 1) =
            r_5(a^2 - 4) =
            r_5(r_5(a-2) \cdot r_5(a+2)) =
            \ub{r_5(( r_5(a) - 2) \cdot (r_5(a) + 2) )}{\llamada1} = 0\\
            r_5(a^2 + 1) = 0
            \Sii{$\llamada{1}$}
            r_5(( r_5(a) - 2) \cdot (r_5(a) + 2) ) = 0
            \llave{rcl}{
              r_5(a) = 2 &\Leftrightarrow&  \congruencia{a}{2}{5} \Tilde\\
              r_5(a) = -2 &\Leftrightarrow&  \congruencia{a}{3}{5} \Tilde
            }
          }$\par
        \textit{Más aún}:\par
        Para una congruencia módulo 5 habrá solo 5 posibles restos,
        por lo tanto se pueden ver todos los casos haciendo una \textit{table de restos}.\par
        $
          \begin{array}{|c|c|c|c|c|c|}
            \hline
            a        & 0 & 1 & 2 & 3 & 4 \\ \hline\hline
            r_5(a)   & 0 & 1 & 2 & 3 & 4 \\ \hline
            r_5(a^2) & 0 & 1 & 4 & 4 & 1 \\ \hline
          \end{array}
          \to   $ La tabla muestra que para un dado $a\\ \to
          r_5(a) =
          \llaves{l}{
            2
            \sisolosi
            \congruencia{a}{2}{5}
            \sisolosi
            \congruencia{a^2}{4}{5}
            \sisolosi
            \congruencia{a^2}{-1}{5} \\
            3
            \sisolosi
            \congruencia{a}{3}{5}
            \sisolosi
            \congruencia{a^2}{4}{5}
            \sisolosi
            \congruencia{a^2}{-1}{5} \\
          }$

  \item Empezamos reescribiendo la expresión $\congruencia{a^3 + \red{7}}{-3 +\red{7}}{7} \longrightarrow \congruencia{a^3}{4}{7}$. \\
        Ahora solo nos basta con probar todas las posibilidades para $ 0 \leq a \leq |6|$ ya que si tuvieramos $7$
        directamente el resto seria 0, y si tenemos algun numero mayor que $7$, la ecuacion se puede reescribir como:
        \[\congruencia{(7+m)^3}{4}{7}\]
        \[\congruencia{(\underbrace{7^3+7^2\cdot m + 7\cdot m^2}_{\text{divisibles por 7}} + m^3)}{4}{7}\]
        \[\congruencia{m^3}{4}{7}\]
        Eso se puede seguir haciendo iterativamente hasta eventualmente tener un $m \leq 7$, en donde entrarian nuestros casos base,
        esto se podria y se deberia enunciar con induccion global para ser mas formal, pero bueno la idea se entiende. \\
        Entonces veamos los $|a|\leq 6$:\\
        \begin{center}
          Para $|a| = 0$ es trivial, el resto es 0\par
          Para $|a| = 1$, $a^3 = \pm 1$, $\congruencia{1}{1}{7}$, $\congruencia{-1}{6}{7}$\par
          Para $|a| = 2$, $a^3 = \pm 8$, $\congruencia{8}{1}{7}$, $\congruencia{-8}{6}{7}$\par
          Para $|a| = 3$, $a^3 = \pm 27$, $\congruencia{27}{6}{7}$, $\congruencia{-27}{1}{7}$\par
          Para $|a| = 4$, $a^3 = \pm 64$, $\congruencia{64}{1}{7}$, $\congruencia{-64}{6}{7}$\par
          Para $|a| = 5$, $a^3 = \pm 125$, $\congruencia{125}{6}{7}$, $\congruencia{-125}{1}{7}$\par
          Para $|a| = 6$, $a^3 = \pm 216$, $\congruencia{216}{6}{7}$, $\congruencia{-216}{1}{7}$\par
        \end{center}
        Como vemos, ninguno es congruente con 4, por ende queda probado que no existe nigun entero $a$
        tal que $\congruencia{a^3}{-3}{7}$

  \item Me piden que exista una dada congruencia para todo $a \en \enteros$.
        Eso equivale a probar a que al dividir el \textit{lado izquierdo}
        entre el \textit{divisor}, el \textit{resto} sea lo que está en el
        \textit{lado derecho} de la congruencia.\par
        $\congruencia{a^7 - a}{0}{7}
          \sisolosi
          \congruencia{a \cdot \ub{(a^6 - 1)}{(a^3 - 1) \cdot (a^3 + 1)}}{0}{7}
          \sisolosi
          \congruencia{a \cdot (a^3 - 1) \cdot (a^3 + 1)}{0}{7}
          \flecha{tabla de restos con}[sus propiedades lineales]\\
          \begin{array}{|c|c|c|c|c|c|c|c|}
            \hline
            a            & 0 & 1 & 2 & 3 & 4 & 5 & 6 \\ \hline\hline
            r_7(a)       & 0 & 1 & 2 & 3 & 4 & 5 & 6 \\ \hline
            r_7(a^3 - 1) & 6 & 0 & 0 & 5 & 0 & 5 & 5 \\ \hline
            r_7(a^3 + 1) & 1 & 2 & 2 & 0 & 2 & 0 & 0 \\ \hline
          \end{array}
          \to $ Cómo para todos los $a$, alguno de los factores del resto siempre se anula, es decir:\par
        $r_7(a^7 - a) =
          r_7(r_7(a) \cdot  r_7(a^3 - 1) \cdot r_7(a^3 + 1)) =
          0 \paratodo a \en \enteros$
  \item
        Tenemos una doble implicación, asique hay que probar los dos lados:\\
        $\Rightarrow)$\\
        Que $7\divideA a^2 + b^2$ puede reescribirse como $\congruencia{a^2+b^2}{0}{7}$, lo que queremos
        entonces es que la suma de los restos de dividir por $7$ a $a^2$ y a $b^2$ sumen 0, para eso podemos armar
        una tablita, primero veamos cuales son los restos de dividir por 7 a un numero de la forma $m^2$: \\
        $r_7(0^2) = 0\quad r_7(1^2) = 1\quad r_7(2^2)=4\quad r_7(3^2) = 2\quad r_7(4^2) = 2\quad
          r_7(5^2) = 4\quad r_7(6^2) = 1$\par

        Entonces ahora armamos una tablita: \\
        $$
          \begin{array}{|c|c|c|c|c|}
            \hline
            _{r_7(b^2)}\big\backslash^{r_7(a^2)} & ^0      & ^1 & ^2 & ^4 \\ \hline
            _0                                   & \red{0} & 1  & 2  & 4  \\
            _1                                   & 1       & 2  & 3  & 5  \\
            _2                                   & 2       & 3  & 4  & 6  \\
            _4                                   & 4       & 5  & 6  & 1  \\\hline
          \end{array}
        $$
        De la tabla vemos que los unicos posibles restos de $7$ que sumados dan $0$ como resto de $7$, son $0$ y $0$, y
        el unico numero $m$ del $0$ al $6$ tal que $r_7(m^2) = 0$ es $m = 0$, o sea que $a$ y $b$ ambos tienen que ser $0$.
        Dicho de otro modo, la solucion de la ecuacion de congruencia $\congruencia{a^2 + b^2}{0}{7}$ es $(a,b) = (0,0)$.
        Está claro que el $7$ divide al $0$ asique queda probada la primera implicacion. \\
        $\Leftarrow)$ \\
        Ahora el otro lado, este es mas sencillo. Tenemos de hipotesis que $7\divideA a$ y $7\divideA b$, por ende $7\divideA a+b$ y $7\divideA \red{a\cdot b}$. \\
        Entonces tambien se cumple que $7\divideA (a+b)\cdot(a+b)$, por lo que $7\divideA a^2 + 2\red{a\cdot b} + b^2$, y finalmente como $2ab$ es divisible por 7
        se puede reescribir finalmente todo como $7\divideA a^2 + b^2$.\\
        Finalmente se han probado las dos implicaciones, por lo tanto la proposicion inicial es verdadera

  \item
\end{enumerate}

\begin{aportes}
  \item \aporte{https://github.com/sigfripro}{sigfripro \github}
\end{aportes}
