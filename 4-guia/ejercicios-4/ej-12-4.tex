\begin{enunciado}{\ejercicio}

  \begin{enumerate}[label=(\alph*)]

    \item Probar que  $\congruencia{2^{5k}}{1}{31}$ para todo $k \en \naturales$.

    \item Hallar el resto de la división de $2^{51833}$ por 31.

    \item Sea $k \in \naturales$. Sabiendo que $\congruencia{2^k}{39}{31}$, hallar el resto de la división de $k$ por 5.

    \item Hallar el resto de la división de $43 \cdot 2^{163} + 11 \cdot 5^{221} + 61^{999}$ por 31.

  \end{enumerate}

\end{enunciado}

\begin{enumerate}[label=(\alph*)]

  \item Probémoslo por inducción.

  Sea la proposición $P(k): \congruencia{2^{5k}}{1}{31}$, $\paratodo \ k \en \naturales$

  \begin{itemize}
  

    \item \textbf{Caso base}: $P(1)$

    $P(1): \congruencia{2^{5 \cdot 1}}{1}{31} \sisolosi \congruencia{32}{1}{31} \Tilde$

    Luego, $P(1)$ es verdadera.

    \item \textbf{Paso inductivo} $P(k) \entonces P(k+1)$

    Asumiendo verdadero $\congruencia{2^{5k}}{1}{31}$, queremos probar que $\congruencia{2^{5(k+1)}}{1}{31}$ es verdadero. \par

    De la hipótesis inductiva tenemos que 

    $$
    \congruencia{2^{5k}}{1}{31}
    \Entonces{$\congruencia{32}{1}{31}$}
    \congruencia{2^{5k} \cdot 32}{1 \cdot 1}{31}
    \sisolosi 
    \congruencia{2^{5k} \cdot 2^5}{1}{31}
    \sisolosi
    \congruencia{2^{5(k+1)}}{1}{31} \Tilde
    $$

    Luego, $P(k+1)$ es verdadera.
  
  \end{itemize}

  Como $P(1)$ es verdadera y $P(k) \entonces P(k+1), \paratodo\ k \en \naturales$, por el principio de inducción, $P(k)$ es verdadera para todo $k \en \naturales$


  \item Queremos hallar el resto de la división de $2^{51833}$ por 31, lo que es lo mismo que buscar a qué es congruente $2^{51833}$ módulo 31.

Observemos que $\congruencia{2^5}{1}{31}$, con lo que dividiendo 51833 por 5, tenemos que $51833=5 \cdot 10366 + 3$. Luego

$$
2^{51833} \equiv 2^{5 \cdot 10366 + 3} \equiv 2^{5\cdot 10366} \cdot 2^3 \equiv (2^5)^{10366}\cdot 8 \equiv \congruencia{1^{10366} \cdot 8}{8}{31}
$$

Entonces, $\boxed{r_{31}\parentesis{2^{51833}}=8}$

\item Como $\congruencia{39}{8}{31}$, tenemos que $\congruencia{2^k}{8}{31}$. Busquemos ahora que valores puede tomar $k$.

Si van probando valores, van a darse cuenta que el 3, 8, 13, 18, ... funcionan, lo que nos permite conjetuar que $k=3+5q, \ q \en \naturales$. 
Entonces podemos conjeturar que 

$$
\congruencia{2^k}{8}{31}
\sisolosi
k=3+5q, \ q \en \naturales
$$

Probemos la doble implicación.

\begin{itemize}

  \item $\Leftarrow$

  Reemplazando $k=3+5q$, tenemos que 

  $$
  2^k \equiv 2^{3+5q} \equiv 2^3 \cdot 2^{5q} \equiv 8 \cdot 32^q \equiv \congruencia{8 \cdot 1^q}{8}{31} \Tilde 
  $$

  \item $\entonces$

  Tenemos que probar que $k$ solo puede ser de la forma $k=3+5q$. Para esto debemos verificar que si $k$ es igual a $c + 5q$, con $c \en \set{0,1,2,4}$
  entonces $\noCongruencia{2^k}{8}{31}$. Pues así estariamos viendo todas las posibilidades. Reemplacemos entonces $k=c+5q$:

  $$
  2^k \equiv 2^{c+5q} \equiv 2^c \cdot 2^{5q} \equiv 2^c \cdot 32^q \equiv \congruencia{2^c \cdot 1^q}{2^c}{31}
  $$

  Veamos ahora los valores de $c$

  
  \begin{align*}
    c=0 \rightarrow 2^k \equiv 2^0 \equiv \noCongruencia{1}{8}{31} \\
    c=1 \rightarrow 2^k \equiv 2^1 \equiv \noCongruencia{2}{8}{31} \\
    c=2 \rightarrow 2^k \equiv 2^2 \equiv \noCongruencia{4}{8}{31} \\
    c=4 \rightarrow 2^k \equiv 2^4 \equiv \noCongruencia{16}{8}{31} \\
  \end{align*}

  Dado que ninguno es congruente a 8 módulo 31, llegamos a la conclusión de que los únicos valores que puede tomar $k$ son los de la forma $k=3+5q, \ q \en \naturales$.
\end{itemize}

Por último, dado que $k=3+5q$, es evidente que $\congruencia{3+5q}{3}{5}$. Entonces $\boxed{r_5(k)=3}$

\item Reduzcamos cada término módulo 31.

Es fundamental notar que $\congruencia{2^5}{1}{31}$, que $\congruencia{5^3}{1}{31}$ y que $\congruencia{61}{-1}{31}$ \par

Entonces

\begin{align*}
  \congruencia{43}{12}{31} \\
  2^{163} \equiv 2^{5 \cdot 32 + 3} \equiv \congruencia{(2^5)^{32} \cdot 8}{8}{31} \\
  5^{221} \equiv 5^{3 \cdot 73+2} \equiv \congruencia{(5^3)^{73} \cdot 25}{25}{31} \\
  61^{999} \equiv \congruencia{(-1)^{999}}{-1}{31}
\end{align*}

Juntando todo

$$
43 \cdot 2^{163} + 11 \cdot 5^{221} + 61^{999} 
\equiv
12 \cdot 8 + 11 \cdot 25 -1
\equiv
\congruencia{370}{29}{31}
$$

Luego, $\boxed{r_{31}(43 \cdot 2^{163} + 11 \cdot 5^{221} + 61^{999})=29}$
\end{enumerate}



\begin{aportes}
    \item \aporte{https://github.com/Nunezca}{Nunezca \github}
\end{aportes}
