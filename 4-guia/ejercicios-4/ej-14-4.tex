\begin{enunciado}{\ejercicio}
  \begin{enumerate}[label=(\alph*)]
    \item Hallar el desarrollo en base 2 de
          \begin{multicols}{4}
            \begin{enumerate}[label=\roman*.]
              \item 1365
              \item 2800
              \item $3\cdot 2^{12}$
              \item $13 \cdot 2^n + 5 \cdot 2^{n-1}$
            \end{enumerate}
          \end{multicols}

    \item Hallar el desarrollo en base 16 de 2800.
  \end{enumerate}
\end{enunciado}
{
\footnotesize
\parrafoDestacado{
  A veces escribo $(12)_2 = 1100$ para decir
  \textit{El número $12$ en base dos es $1100$} y a veces puede ser que se escriba como
  $12 = (1100)_2$. A mí me gusta la primera, pero bueh, mientras se entienda, \faIcon{hands-wash}.
}
}

Para hallar el desarrollo en base $2$ de algún número dado su representación decimal, simplemente dividimos por $2$, truncamos, anotamos el resto
y seguimos iterando hasta que lleguemos a $0$, luego escribimos los restos que conseguimos en orden inverso y esta es la representación.

\medskip

Pequeño ejemplo para que se vea: Consideremos el número $6$, que su representación en base $2$ es:
$$
  6 = 2^2 \cdot \blue{1} + 2^1 \cdot \blue{1} + 2^0 \cdot \blue{0}
  \entonces
  (6)_2 =  \blue{110}
$$
Lo mismo pero usando una notación de $n \flecha{$r_2(n)$} \lfloor \frac{n}{2} \rfloor$:
$$
  6 \flecha{$\magenta{0}$} 3 \flecha{$\magenta{1}$} 1 \flecha{$\magenta{1}$} 0 \quad \llamada1
$$
Ahora ponemos en reversa los restos que obtuvimos en $\llamada1$, y nos queda:
$$
  2^2 \cdot \magenta{1} + 2^1 \cdot \magenta{1} + 2^0 \cdot \magenta{0}
  \entonces
  (6)_2 =  \magenta{110}
$$

\begin{enumerate}[label=(\alph*)]
  \item
        \begin{enumerate}[label=\roman*.]
          \item
                $$
                  1365 \flecha{\magenta{1}}
                  682 \flecha{$\magenta{0}$}
                  341 \flecha{\magenta{1}}
                  170 \flecha{\magenta{0}}
                  85 \flecha{\magenta{1}}
                  42 \flecha{\magenta{0}}
                  21 \flecha{\magenta{1}}
                  10 \flecha{\magenta{0}}
                  5 \flecha{\magenta{1}}
                  2 \flecha{\magenta{0}}
                  1 \flecha{\magenta{1}} 0.
                $$
                Hacemos el display en reversa y nos da:
                $$
                  (1365)_2 = \magenta{10101010101},
                $$
                cada $1$ representa que multiplica al respectivo $2^d$ y $0$ que lo hace cero. En este caso
                lo escribimos con $0$ y $1$ porque es más conveniente que andar poniendo tal multiplicado por $2^d$

          \item
                $$
                  2800 \flecha{\magenta{0}}
                  1400 \flecha{\magenta{0}}
                  700 \flecha{\magenta{0}}
                  350 \flecha{\magenta{0}}
                  175 \flecha{\magenta{1}}
                  87 \flecha{\magenta{1}}
                  43 \flecha{\magenta{1}}
                  21 \flecha{\magenta{1}}
                  10 \flecha{\magenta{0}}
                  5 \flecha{\magenta{1}}
                  2 \flecha{\magenta{0}}
                  1 \flecha{\magenta{1}}
                  0.
                $$
                Hacemos el display en reversa y nos da:
                $$
                  (2800)_2 = \magenta{10101111000}.
                $$

          \item $$
                  3 \cdot 2^{12} =
                  (2 + 1)\cdot 2^{12} = 2^{13} + 2^{12}
                  \igual{\red{!}}
                  2^{13} \cdot \magenta{1} + 2^{12} \cdot \magenta{1} + \sumatoria{i=0}{11} 2^{11 - i} \cdot \magenta{0}
                $$
                Queda como:
                $$
                  (3\cdot 2^{12})_2 = \magenta{11000000000000}.
                $$

          \item $$
                  13 \cdot 2^n + 5 \cdot 2^{n - 1} =
                  (2^3 + 2^2 + 2^0) \cdot 2^n + (2^2 + 2^0) \cdot 2^{n - 1} =
                  2^{n + 3} + 2^{n + 2} + 2^n + 2^{n + 1} + 2^{n - 1}
                $$
                Para algún $n\en \naturales$ queda como:
                $$
                  (13 \cdot 2^n + 5 \cdot 2^{n - 1})_2 =
                  \magenta{11111}\ub{\magenta{0 \cdots 0}}{n - 1 \text{ dígitos}}
                $$
        \end{enumerate}
  \item Así como en base 2, un dígito puede tomar 2 valores, el 0 o el 1, en base 16 cada dígito de nuestro número puede tomar 16 posibles valores:
        $$
          0,\,1,\,2,\,3,\,4,\,5,\,6,\,7,\,8,\,9,\,A,\,B,\,C,\,D,\,E \otext F
        $$
        Es igual a lo anterior, pero ahora los posibles restos serán valores entre $0$ y $\ua{F}{15}$:
        $$
          2800 \flecha{\magenta{0}} 175 \flecha{\magenta{F}} 10 \flecha{\magenta{A}} 0.
        $$
        Entonces su representación en base $16$:
        $$
          (2800)_{16} = \magenta{AF0}
        $$
\end{enumerate}

\begin{aportes}
  \item \aporte{https://github.com/sigfripro}{sigfripro \github}
  \item \aporte{\dirRepo}{naD GarRaz \github}
\end{aportes}
