\begin{enunciado}{\ejercicio}
  Sea $a = (a_d a_{d-1} \ldots a_1 a_0)_2$ un número escrito en base $2$ (o sea escrito en bits). Determinar
  simplemente como son las escrituras en base $2$ del número $2a$ y del número $a/2$ cuando $a$ es par, o sea
  las operaciónes "multiplicar por 2" y "dividir por 2" cuando se puede. Estas operaciónes se llaman $shift$ en inglés,
  o sea corrimiento, y son operaciónes que una computadora hacer en forma sencilla.

\end{enunciado}

$$
  a = (\blue{a_d a_{d-1} \ldots a_1 a_0})_2
$$
podemos escribirlo en base 10 como:
$$
  \llamada1
  a = 2^d \cdot \blue{a_d} + 2^{d-1} \cdot \blue{a_{d-1}} + \cdots + 2^1 \cdot \blue{a_1} + 2^0 \cdot \blue{a_0}.
$$
Multiplicamos por $2$ y obtenemos:
$$
  2a = 2^{d + 1} \cdot \blue{a_d} + 2^d \cdot \blue{a_{d-1}} + \cdots + 2^1 \cdot \blue{a_0} + \red{2^0 \cdot 0}
$$
En base $2$ este número sería:
$$
  2a = (\blue{a_d a_{d-1} \ldots a_1 a_0 \red{0}})_2.
$$
Vemos que los números se \textit{corrieron a la izquierda}.
Esta operación es el \textit{left shift}.

\bigskip

Hacemos lo mismo pero dividiendo $\llamada1$ por 2:
$$
  \frac{a}{2} = 2^{d - 1} \cdot \blue{a_d} + 2^{d - 2} \cdot \blue{a_{d-1}} + \cdots + 2^1 \blue{a_2} + 2^0 \blue{a_1} + \red{r},
$$
escrito en base $2$ sería:
$$
  \frac{a}{2} = (\blue{a_{d} a_{d-1} \ldots a_2 a_1})_2
$$
ya que el resto se elimina. Vemos que perdimos información del último dígito, porque dividimos por 2 y truncamos.

\begin{aportes}
  \item \aporte{https://github.com/sigfripro}{sigfripro \github}
\end{aportes}
