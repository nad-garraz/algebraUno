\begin{enunciado}{\ejercicio}
  En cada uno de los siguientes casos calcular el máximo común divisor entre $a$ y $b$ y escribirlo como combinación lineal entera de $a$ y $b$:
  \begin{enumerate}[label=\roman*)]
    \item $a = 2532,\ b = 63$.
    \item $a = 131,\ b = 23$.
    \item $a = n^4 - 3,\ b = n^2 + 2\ (n \en \naturales)$.
  \end{enumerate}
\end{enunciado}

\begin{enumerate}[label=\roman*)]
  \item
        Para calcular el máximo común divisor hay que usar el \textit{algoritmo de Euclides}, en forma matricial
        lo que hago es ir restando las filas hasta encontrar el $\magenta{0}$, el valor de la fila anterior es le \blue{mcd}.

        Lo que me gusta de este método es que en cada línea vas a escribiendo una combinación entera
        entre los números sin mucho esfuerzo.
        $$
          \begin{array}{|c|c|c|l|} \hline\rowcolor{red!5!white}
            1           & 0            & 2532     & F_1               \\ \hline\rowcolor{red!5!white}
            0           & 1            & 63       & F_2               \\ \hline
            1           & -40          & 12       & F_3 = F_1 - 40F_2 \\ \hline \rowcolor{Cerulean!5!white}
            \orange{-5} & \orange{201} & \blue{3} & F_4 = F_2 - 5F_3  \\ \hline
            -           & -            & \red{0}  & F_5 = F_3 - 4F_4  \\ \hline
          \end{array}
        $$
        El \textit{máximo común divisor} entre 2532 y 63 escrito como combinación entera de los mismos:
        $$
          \cajaResultado{
            \blue{3} = \orange{-5} \cdot 2532 + \orange{201} \cdot 63
          }
        $$

        También se puede encontar el $(2532 : 63)$ haciendo la factorización en primos
        y agarrando las \ul{potencias comunes elevadas al menor exponente}:
        $$
          2532 = 2^2 \cdot 3^1 \cdot 211
          \ytext
          63 = 3^2 \cdot 7
          \entonces
          (2532 : 63) = \blue{3^1}
        $$

  \item Dado que 131 y 23 son dos números primos, sabemos que $(131:23)$. El tema es la combinación entera
        para lo cual vuelvo a usar el método matricial de Euclides.
        $$
          \begin{array}{|c|c|c|l|} \hline\rowcolor{red!5!white}
            1            & 0           & 131      & F_1              \\ \hline\rowcolor{red!5!white}
            0            & 1           & 23       & F_2              \\ \hline
            1            & -5          & 16       & F_3 = F_1 - 5F_2 \\ \hline
            -1           & 6           & 7        & F_4 = F_2 - F_3  \\ \hline
            3            & -17         & 2        & F_5 = F_3 - 2F_4 \\ \hline \rowcolor{Cerulean!5!white}
            \orange{-10} & \orange{57} & \blue{1} & F_6 = F_4 - 3F_5 \\ \hline
            -            & -           & 0        & F_7 = F_5 - 2F_6 \\ \hline
          \end{array}
        $$
        El \textit{máximo común divisor} entre 2532 y 63 escrito como combinación entera de los mismos:
        $$
          \cajaResultado{
            \blue{1} = \orange{-10} \cdot 131 + \orange{57} \cdot 23
          }
        $$

  \item
        En este caso con estos polinomios hago división polinomial:
        $$
          \polyset{vars=n}
          \divPol{n^4-3}{n^2+2}
        $$
        Tengo según Euclides:
        $$
          (n^4-3 : n^2+2) = (n^2+2 : \blue{1}) = 1
        $$
        Por el algoritmo de división sé que:
        $$
          n^4 - 3 =  (n^2+2) \cdot (n^2 - 2) + \blue{1}
          \sii
          \cajaResultado{
            \blue{1} = \orange{1} \cdot (n^4 - 3) + \orange{-(n^2 - 2)} \cdot (n^2 + 2)
          }
        $$
\end{enumerate}

\begin{aportes}
  \item \aporte{\dirRepo}{naD GarRaz \github}
\end{aportes}
