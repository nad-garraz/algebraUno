\begin{enunciado}{\ejercicio}
  Sean $a,b \in \enteros$. Sabiendo que el resto de dividir a $a$ por $b$ es $27$ y que el resto de dividir $b$ por $27$
  es $21$, calcular $(a:b)$
\end{enunciado}

Traducimos lo que nos da el enunciado a congruencias y tenemos:
$$
  \llave{rcl}{
    \congruencia{a}{27}{b} & \Sii{def} & a = b \cdot \magenta{j} + 27 \text{ con } \magenta{j} \en \enteros \llamada1\\
    \congruencia{b}{21}{27} & \Sii{def} & b = 27 \cdot \green{k} + 21 \text{ con } \green{k} \en \enteros \llamada2.
  }
$$
Reescribimos el \textit{máximo común divisor} $(a:b)$:
$$
  (a:b)
  \igual{$\llamada1$}
  (b \cdot \magenta{j} + 27 : b)
  \igual{\red{!}}
  (27 : b)
  \igual{$\llamada2$}
  \big(27 : (27 \cdot \green{k} + 21)\big)
  \igual{\red{!}}
  (27 : 21)
  \igual{\red{!}}
  (21 : 6)
  \igual{\red{!}}
  (6 : 3)
  =
  (3^2 : 3^1)
  =3
$$
Quizás los últimos pasos son medio un \textit{overkill}, pero quedó claro.

Si en los \red{!} te quedás pensando: No es otra cosa que el algoritmo de Euclides.

\begin{aportes}
  \item \aporte{https://github.com/sigfripro}{sigfripro \github}
  \item \aporte{\dirRepo}{naD GarRaz \github}
\end{aportes}
