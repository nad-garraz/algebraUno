\begin{enunciado}{\ejercicio}
  Sea $a \en \enteros$.
  \begin{enumerate}[label=\alph*)]
    \item Probar que
          $(5a + 8 : 7a + 3) = 1 \otext 41.$
          Exhibir un valor de $a$ para el cual da 1, y verificar
          que efectivamente para $a = 23$ da $41$.

    \item Probar que $(2 a^2 + 3a -1 : 5a +6) = 1 \otext 43.$ Exhibir un valor de $a$ para el cual da 1, y verificar
          que efectivamente para $a = 16$ da $43$

    \item Probar que $(a^2-3a+2 : 3a^3 -5a^2) = 2 \otext 4$, y exhibir un valor de $a$ para cada caso.

          (Para este item es \textbf{indispensable} mostrar que el máximo común divisor nunca puede ser 1).
  \end{enumerate}
\end{enunciado}

\begin{enumerate}[label=\roman*)]
  \item
        Si $d = (5a + 8 : 7a + 3)$, entonces $d$ divide ambas expresiones:
        $$
          \llave{l}{
            d \divideA 5a + 8 \\
            d \divideA 7a + 3
          }
          \Sii{$7F_1 - 5F_2 \to F_2$}
          \llave{l}{
            d \divideA 5a + 8 \\
            d \divideA 41
          }
        $$
        Para que se cumpla ese último resultado:
        $$
          d \en \set{\pm1, \pm 41}
        $$
        Pero el \textit{máximo común divisor} es positivo así que:
        $$
          d \en \set{1, 41}
        $$
        Si $a = 1$ queda $d = (13 : 10) = 1$. En el caso en que $a = 23$:
        $$
          d = (123:164)
          \igual{\red{!}}
          (123:41) = 41
        $$
        Donde en el \red{!} usé Euclides. Haciendo el algoritmo de Euclides, recordar que el valor del
        \textit{máximos común divisor} se encuentra en el último resto distinto de 0.

  \item
        Si $d = (2a^2 + 3a - 1 : 5a + 6)$, entonces $d$ divide ambas expresiones:
        $$
          \llave{l}{
            d \divideA 2a^2 + 3a - 1 \\
            d \divideA 5a + 6
          }
          \Sii{$5F_1 - 2a F_2 \to F_1$}
          \llave{l}{
            d \divideA 3a - 5\\
            d \divideA 5a + 6
          }
          \Sii{$5F_1 - 3 F_2 \to F_2$}
          \llave{l}{
            d \divideA 3a -5\\
            d \divideA -43
          }
        $$
        Para que se cumpla ese último resultado:
        $$
          d \en \set{\pm1, \pm 43}
        $$
        Pero el \textit{máximo común divisor} es positivo así que:
        $$
          d \en \set{1, 43}
        $$
        Si $a = 0$ queda $d = (-1 : 6) = 1$. En el caso en que $a = 16$:
        $$
          d = (559:86)
          \igual{\red{!}}
          (86:42)
          \igual{\red{!}}
          (86:43)
          = 43
        $$
        Donde en el \red{!} usé Euclides. Haciendo el algoritmo de Euclides, recordar que el valor del
        \textit{máximos común divisor} se encuentra en el último resto distinto de 0.

  \item  Dado que voy a hacer cuentas en la cuales $a$ no puede ser 0. Me saco de encima eso primero:
        $$
          a = 0 \entonces (a^2 -3a + 2 : 3a^3 - 5a^2) = (0:-2) = 2
        $$

        Acomodo un poco y hago Euclides haciendo división de polinomios para achicar la expresión:
        $$
          \polyset{vars=a}
          \divPol{3a^3 - 5a^2}{a^2 - 3a + 2}
        $$
        Con ese resultado:
        $$
          d =
          (a^2 -3a + 2 : 3a^3 - 5a^2)
          =
          (a^2 -3a + 2 : 6a - 8)
        $$
        Procedo ahora a encontrar los posibles divisores comunes:
        $$
          \llave{l}{
            d \divideA a^2 -3a + 2\\
            d \divideA 6a - 8
          }
          \Sii{$6F_1 - 2a F_2 \to F_1$}[$a \distinto 0$]
          \llave{l}{
            d \divideA  -2a + 12\\
            d \divideA 6a - 8
          }
          \Sii{$3F_1 + F_2 \to F_1$}
          \llamada1
          \llave{l}{
            d \divideA  28\\
            d \divideA 6a - 8
          }
        $$
        Para que se cumpla ese último resultado teniendo en cuenta que $28 = 2^2 \cdot 7$:
        $$
          d \en \set{\pm1, \pm2, \pm 4, \pm 7, \pm 14, \pm 28}
        $$
        Pero el \textit{máximo común divisor} es positivo así que:
        $$
          d \en \set{1, 2, 4, 7, 14, 28}
        $$
        Hay que sacar posibles valores del MCD. De $\llamada1$ necesito que $d \divideA 6a-8$.

        Puedo hacer una tabla de restos para estudiar eso:
        $$
          \begin{array}{|c|c|c|c|c|c|c|c|} \hline
            r_7(a)      & 0 & 1 & 2 & 3 & 4 & 5 & 6 \\ \hline
            r_7(6a - 8) & 6 & 5 & 4 & 3 & 2 & 1 & 5 \\ \hline
          \end{array}
        $$
        La tabla de restos dice que el $7$ no divide núnca a la expresión $6a -8$, por lo que podemos
        descartar todos los divisores que sean multiplos enteros de 7. Actualizo los posibles valores para $d$:
        $$
          d \en \set{1, 2, 4}
        $$
        Necesito como dice la \textbf{sugerencia}, \ul{descartar la posibilidad de que $d = 1$ para algún valor de $a$}:
        $$
          \begin{array}{|c|c|c|} \hline
            r_2(a)      & 0 & 1 \\ \hline
            r_2(6a - 8) & 0 & 0 \\ \hline
          \end{array}
        $$
        La tabla dice que \ul{para todo valor de $a$} la expresión va a ser divisible por 2. Eso descarta que
        1 pueda ser \textit{un máximos común divisor}.
        $$
          \begin{array}{|c|c|c|c|c|} \hline
            r_4(a)      & 0 & 1 & 2 & 3 \\ \hline
            r_4(6a - 8) & 0 & 2 & 0 & 2 \\ \hline
          \end{array}
        $$
        Juntando todos estos resultados se puede concluir:
        $$
          d =
          \llave{rcl}{
            2 & \text{ si } & \congruencia{a}{1}{2}  \lor a = 0\\
            4 & \text{ si } & \congruencia{a}{0}{4} \land a \distinto 0
          }
        $$
        O puesto de otra manera
        $$
          d =
          \llave{rcl}{
            2 & \text{ si } & a \text{ es impar} \lor a = 0\\
            4 & \text{ si } & a \text{ par} \land a\distinto 0
          }
        $$
        Solo falta el ejemplo de $a$ para que $d = 4$:
        $$
          a = 2 \entonces (a^2 -3a + 2 : 3a^3 - 5a^2) = (0:4) = 4
        $$
\end{enumerate}

\begin{aportes}
  \item \aporte{\dirRepo}{naD GarRaz \github}
\end{aportes}
