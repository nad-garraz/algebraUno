\def\enumeracion{\alph}

\begin{enunciado}{\ejercicio}
  \begin{enumerate}[label=\enumeracion*)]
    \item Determinar todos los $a,b \en \enteros$ coprimos tales que $\frac{b+4}{a} + \frac{5}{b} \en \enteros$.
    \item Determinar todos los $a,b \en \enteros$ coprimos tales que $\frac{9a}{b} + \frac{7a^2}{b^2} \en \enteros$.
    \item Determinar todos los $a \en \enteros$ tales que $\frac{2a + 3}{a + 1} + \frac{a + 2}{4} \en \enteros$.
  \end{enumerate}
\end{enunciado}

\begin{enumerate}[label=\enumeracion*)]
  \item Acomodo el enunciado como:
        $$
          \frac{b+4}{a} + \frac{5}{b} =
          \frac{b^2 + 4b + 5a}{ab}
        $$

        Quiero que esa fracción sea entera, lo cual es lo mismo que decir:

        $$
          ab \divideA b^2 + 4b + 5a
        $$
        Dado que $a \cop b$:
        $$
          \llave{l}{
            a | b^2 + 4b +5a \\
            b | b^2 + 4b +5a
          }
          \sii
          \llave{l}{
            a | b^2 + 4b \\
            b | 5a
          }
          \Entonces{sé que $b\noDivide a$}[$b$ debe dividir a 5]
          \llave{l}{
            a | b\cdot(b+4) \llamada1 \\
            b | 5
          }
        $$
        Sale que:
        $$
          b \en \set{\pm1, \pm5}
        $$
        Puedo entonces probar valores de $b$ y así ver que valor de $a$ queda:

        \textit{Si $b = 1$}
        $$
          \Entonces{$\llamada1$}
          a \divideA 5
          \entonces (a,b)
          =
          \llave{l}{
            (\pm1, 1) \\
            (\pm5, 1)
          }
        $$

        \textit{Si $b = -1$}
        $$
          \Entonces{$\llamada1$}
          a \divideA -3
          \entonces (a,b)
          =
          \llave{l}{
            (\pm1, -1) \\
            (\pm3, -1)
          }
        $$

        \textit{Si $b = 5$}
        $$
          \Entonces{$\llamada1$}
          a \divideA 45
          \entonces (a,b)
          \igual{\red{!!}}
          \llave{l}{
            (\pm1, 5) \\
            (\pm3, 5) \\
            (\pm9, 5)
          }
        $$

        \textit{Si $b = -5$}
        $$
          \Entonces{$\llamada1$}
          a \divideA 5
          \entonces (a,b)
          \igual{\red{!!}}
          \llave{l}{
            (\pm1, 5)
          }
        $$

        En el \red{!!} se usa que $a \cop b$

  \item
        Acomodo el enunciado sacando denominador común:
        $$
          \frac{9a}{b} + \frac{7a^2}{b^2} = \frac{9ab + 7 a^2}{b^2}
        $$

        Quiero que esa fracción sea entera, lo cual es lo mismo que decir:

        $$
          b^2 \divideA 9ab + 7 a^2
          \Entonces{\red{!!!}}
          b \divideA 9ab + 7 a^2
          \sii
          b \divideA 7 a^2
          \Sii{$a \cop b$}
          b \divideA 7
        $$
        Sale que:
        $$
          b \en \set{\pm1, \pm7}
        $$

        Puedo entonces probar valores de $b$ y así ver que valor de $a$ queda:

        \textit{Si $b = \pm 1$:}
        $$
          \frac{\pm 9a + 7 a^2}{1} \en \enteros \paratodo a \en \enteros
          \entonces (a,b)
          =
          (a, \pm1)
        $$

        \textit{Si $b = 7$:}
        $$
          \frac{ 9 a \cdot 7 + 7 a^2}{49} =
          \frac{9 a + a^2}{7}
        $$
        Esto último va a ocurrir cuando:
        $$
          \congruencia{a^2 + 9a}{0}{7}
          \sii
          \congruencia{a\cdot (a + 2)}{0}{7}
          \Sii{$a \cop 7$}[$b = 7$]
          \congruencia{a + 2}{0}{7}
          \sii
          \congruencia{a}{5}{7}
        $$
        Los pares $(a,b)$ para satisfacer lo pedido son de la forma:
        $$
          (a, b) =
          (a, 7) \quad \text{ con } \quad \congruencia{a}{5}{7}
        $$

        \textit{Si $b = -7$:}
        El desarrollo es igual que para el caso anterior y los pares quedan:
        $$
          (a, b) =
          (a, -7) \quad \text{ con } \quad \congruencia{a}{2}{7}
        $$

  \item
        $$
          \frac{2a + 3}{a+1} + \frac{a+2}{4} = \frac{a^2 + 11a + 14}{4a+4} \llamada{1}
        $$

        Para que $\frac{a^2 + 11a + 14}{4a+4} \en \enteros$ debe ocurrir que

        $$
          4a + 4 \divideA a^2 + 11a + 14
        $$
        Busco eliminar la a del lado derecho:
        $$
          \llave{l}{
            4a + 4 \divideA a^2 + 11a + 14 \\
            4a + 4 \divideA 4a + 4
          }
          \flecha{\red{!}}
          \llave{l}{
            4a + 4 \divideA 16 \\
            4a + 4 \divideA 4a + 4
          }
        $$
        Las cuentas del \red{!} te las dejo a vos.

        $4a+4$ tiene que dividir a 16, onda $\frac{16}{4a + a} \en \enteros$ por lo tanto necesitamos:
        $$
          4a + a \en \set{\pm1,\pm2,\pm4,\pm8,\pm16}
        $$.
        Teniendo en cuenta que $4a + 4 \en \enteros$ y también que $a \en \enteros$, quedan como únicos posibles valores:
        $$
          \begin{array}{rcc}
            4 \cdot \magenta{(-5)} + 4 & = & \blue{-16} \\
            4 \cdot \magenta{(-3)} + 4 & = & \blue{-8}  \\
            4 \cdot \magenta{(-2)} + 4 & = & \blue{-4}  \\
            4 \cdot \magenta{0} + 4    & = & \blue{4}   \\
            4 \cdot \magenta{1} + 4    & = & \blue{8}   \\
            4 \cdot \magenta{3} + 4    & = & \blue{16}
          \end{array}
        $$
        reemplazando esos valores de $a$ en $\llamada1$ se obtiene que  $ \llamada1 \en \enteros$.
\end{enumerate}

\begin{aportes}
  \item \aporte{\dirRepo}{naD GarRaz \github}
  \item \aporte{https://github.com/a18delsol}{Juan Iglesias \github}
\end{aportes}
