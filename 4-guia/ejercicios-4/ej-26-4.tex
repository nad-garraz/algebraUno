\begin{enunciado}{\ejercicio}
    Decidir si existen enteros $a$ y $b$ no nulos que satisfagan
    \begin{multicols}{2}
      \begin{enumerate}[label=\alph*)]
        \item $a^2=3b^3$
        \item $7a^2=8b^2$
      \end{enumerate}
    \end{multicols}
\end{enunciado}

\begin{enumerate}[label=\alph*)]
   \item 
   Observando que hay un 3 del lado derecho, a ojo se puede ver que, por ejemplo, $(a,b)= (3^2,3)$ cumple.

   \item
   A simple vista, no parece haber una solución obvia. Veamos la factorización en primos para ver si encontramos una contradicción. \par
   Por TFA, se tiene que 

   $$
   \llave{l}{
     a = (P_1)^{m_1}...(P_r)^{m_r}, ~ m_1,...,m_r ~ \en \naturales_0 \\
     b = (P_1)^{n_1}...(P_r)^{n_r}, ~ n_1,...,n_r ~ \en \naturales_0
    }
   \entonces
   \llave{l}{
    a^{2} = (P_1)^{2m_1}...(P_r)^{2m_r} \\
    b^{2} = (P_1)^{2n_1}...(P_r)^{2n_r}
    }  
   $$

   Entonces

   $$
   7a^2=8b^2
   \sisolosi
   7^1 \cdot (P_1)^{2m_1}...(P_r)^{2m_r} = 2^3 \cdot (P_1)^{2n_1}...(P_r)^{2n_r}
   $$

   Del lado izquierdo de la igualdad, el 7 aparece con el exponente $2m_7 +1$. \par
   Del lado derecho de la igualdad, el 7 aparece con el exponente $2n_7$. \par
   Entonces, por unicidad de la factorización, se deberia tener que 

   $$
   2m_7 +1=2n_7
   $$

   Lo cual es absurdo, pues un número es impar y el otro par. \par
   Luego, $\noexiste a, b \en \enteros$  no nulos tal que $7a^2=8b^2$
   
\end{enumerate}
   
   
\begin{aportes}
    \item \aporte{https://github.com/Nunezca}{Nunezca \github}
\end{aportes}





