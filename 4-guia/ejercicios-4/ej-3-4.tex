\ejercicio Sean $a.\,b \en \enteros$.
\begin{enumerate}[label=\roman*)]
	\item Probar que $a-b \divideA a^n - b^n$ para todo $n \en \naturales \y a \distinto b \en \enteros$
	\item Probar que si $n$ es un número natural par y $a \distinto -b$, entonces $a+b \divideA a^n - b^n$.
	\item Probar que si $n$ es un número natural impar y $a \distinto -b$, entonces  $a+b \divideA a^n + b^n$.
\end{enumerate}
\separadorCorto
%=============
%Macro local
\def\aMenosB{\stackrel{(a-b)}\congruente}
\def\aMasB{\stackrel{(a-b)}\congruente}
%FinMacro local
%=============
\begin{enumerate}[label=\roman*)]
	\item  Si $a-b \divideA a^n - b^n \stackrel{def}\sisolosi \congruencia{a^n}{b^n}{a-b} \sisolosi
		      \llaves{l}{
			      \congruencia{a}{b}{a-b} \\
			      \congruencia{a^2}{\ub{a}{\aMenosB b}\cdot b}{a-b} \to \congruencia{a^2}{b^2}{a-b} \\
			      \quad\vdots\\
			      \congruencia{a^n}{b^n}{a-b}
		      }\\
		      \to \congruencia{a^n}{b^n}{a-b} \sisolosi a-b \divideA a^n - b^n$

	\item  Sé que $\congruencia{a}{-b}{a+b} \sisolosi
		      \llaves{l}{
			      \congruencia{a^2}{\ub{a}{\aMasB -b} \cdot b}{a+b} \flecha{propiedad}[congruencia] \congruencia{a^2}{(-1)^2 \cdot b^2}{a+b} \\
			      \quad \vdots \quad \llamada{1}{\ot} \\
			      \congruencia{a^n}{(-1)^n \cdot b^n}{a+b} \to
			      \llaves{ll}{
				      \congruencia{a^n}{ b^n}{a+b} & \text{con n par}  \\
				      \congruencia{a^n}{(-1)^n \cdot b^n}{a+b}   & \text{con n impar} \\
			      }\llamada{2}{}\\
		      }\\
		      \llamada{2}{}
		      \boxed{
			      \llave{l}{
				      \text{Con $n$ par: }  \congruencia{a^n}{ b^n}{a+b} \entonces a+b \divideA a^n - b^n  \\
				      \text{Con $n$ impar: }  \congruencia{a^n}{ - b^n}{a+b} \entonces a+b \divideA a^n + b^n
			      }
		      }
	      $

	      $\llamada{1}{}
		      \textit{Inducción:}\\
		      \llave{l}{
			      p(n)\ :\ \magenta{\congruencia{a}{-b}{a+b}} \entonces
			      \congruencia{a^n}{(-1)^n \cdot b^n}{a+b} \paratodo n \en \naturales.\\
			      \textit{Caso base: }\\
			      p(1)\ :\
			      \magenta{\congruencia{a}{-b}{a+b}} \entonces \congruencia{a^1}{(-1)^1\cdot b^1}{a+b}
			      \entonces
			      \magenta{\congruencia{a}{-b}{a+b}} \text{ Verdadero}.\\
			      \textit{Hipótesis inductiva: }\\
			      p(k)\ V \entonces p(k+1)\ V? \\
			      \magenta{\congruencia{a}{-b}{a+b}} \entonces \congruencia{a^k}{(-1)^k\cdot b^k}{a+b}
			      \entonces
			      \magenta{\congruencia{a}{-b}{a+b}} \entonces \congruencia{a^{k+1}}{(-1)^{k+1}\cdot b^{k+1}}{a+b} \\
			      \text{Parto de } p(k):\\
			      \llave{l}{
				      \congruencia{a^k}{(-1)^k\cdot b^k}{a+b}\\
				      \flecha{multiplico}[por $a$]
				      \congruencia{a\cdot a^k}{(-1)^k\cdot \ub{a}{\aMasB -b}\cdot b^k}{a+b}\\
				      \flecha{y acomodo}
				      \congruencia{a^{k+1}}{(-1)^{k+1} \cdot b^{k+1}}{a+b} \Tilde
			      }
		      }
	      $
	      \\
	      Como $p(1),\ p(k),\ p(k+1)$ son verdaderas por principio de inducción lo es también $p(n) \paratodo n \en \naturales$


	\item hecho en el anterior.
\end{enumerate}
