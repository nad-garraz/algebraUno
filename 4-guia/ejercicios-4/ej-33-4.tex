\begin{enunciado}{\ejercicio}
  Hallar todos los $n \en \naturales$ tales que
  \begin{enumerate}[label=(\alph*)]

    \item $(n:945)=63, ~ (n:1176)=84 ~ \text{y} ~ n \leq 2800$

    \item $(n:1260)=70$ y $n$ tiene 30 divisores positivos.

  \end{enumerate}
\end{enunciado}

\begin{enumerate}[label=(\alph*)]

  \item Trabajemos con la primera condición
        $$
          (n:945) = 63
          \sisolosi
          (n:3^3 \cdot 5 \cdot 7)= 3^2 \cdot 7
        $$
        De aca tenemos que, en su factorización, $n$ tiene un $3^2$, no tiene un 5 y tiene un $7^m, ~ m \geq 1$.

        \bigskip

        Veamos ahora la segunda condición
        $$
          (n:1176)=84
          \sisolosi
          (n:2^3 \cdot 3 \cdot 7^2)= 2^2 \cdot 3 \cdot 7
        $$
        De aca tenemos que, en su factorización, $n$ tiene un $2^2$, tiene un $3^k, ~ k \geq 1$ y tiene un 7.

        Juntando todo, tenemos que
        $$
          n = 2^2 \cdot 3^2 \cdot 7 \cdot (P_1)^{m_1} \cdots (P_r)^{m_r}, ~ m_1, \dots, m_r ~ \en \naturales_0
        $$

        Veamos ahora la tercer condición.

        Si $n$ no tiene ningún primo más, tenemos que $n= 2^2 \cdot 3^2 \cdot 7 = 252 \leq 2800 \Tilde$

        Si $n$ tiene un primo más, sabiendo que el 5 no puede ser, probamos con el 11, que es el que le sigue al 7. Entonces, tenemos
        que $n= 2^2 \cdot 3^2 \cdot 7 \cdot 11 = 2772 \leq 2800 \Tilde$ \par
        Si agregamos otro 11, ya nos pasamos, pues $n=2^2 \cdot 3^2 \cdot 7 \cdot 11^2=30492$. Por otro lado, si no agregamos el 11 sino el que le sigue, el 13,
        tenemos que $n=2^2 \cdot 3^2 \cdot 7 \cdot 13 =3276$, con lo que también nos pasamos. De este modo, es evidente que con cualquier otro primo mayor a 13 también
        nos pasariamos. Así, solo puede haber como máximo un 11 más.

        \bigskip

        Luego, los unicos $n$ que cumplen son
        $$
          n=2^2 \cdot 3^2 \cdot 7 = \boxed{252}
        $$
        $$
          n=2^2 \cdot 3^2 \cdot 7 \cdot 11 = \boxed{2772}
        $$

  \item Veamos la primera condición
        $$
          (n:1260)=70
          \sisolosi
          (n:2^2 \cdot 3^2 \cdot 5 \cdot 7 )= 2 \cdot 5 \cdot 7
        $$

        De aca deducimos que, en su factorización, $n$ tiene un 2, no tiene un 3, tiene un $5^m, ~ m \geq 1$ y tiene un $7^k, ~ k \geq 1$. \par
        Así, tenemos que
        $$
          n = 2^1 \cdot 5^m \cdot 7^k \cdot (P_1)^{m_1} \cdots (P_r)^{m_r}, ~ m_1, \dots, m_r ~ \en \naturales_0
        $$
        De la segunda condición tenemos que
        $$
          \#Div_+(n)=30
          \sisolosi
          30 = (1+1)(m+1)(k+1)(m_1+1) \cdots (m_r+1)
          \sisolosi
          15 =(m+1)(k+1)(m_1+1) \cdots (m_r+1)
        $$
        Notemos ahora que las únicas maneras de escribir a $15=3 \cdot 5$ como un producto de dos o más números es haciendo $15 \cdot 1$ o $3 \cdot 5$ \par
        Para empezar, estos nos dice que no hay más primos en la factorización de $n$, además del 2, 5 y 7. Luego, tenemos que
        $$
          15 =(m+1)(k+1)
        $$
        Como ambos factores son mayores iguales a 2 (pues k y m son mayores a 1), tenemos que la única manera que el producto de 15 es que uno sea
        igual a 3 y el otro a 5. Con lo que tenemos dos opciones:
        $$
          (m+1)= 3 \ytext (k+1)=5
        $$
        $$
          (m+1)= 5 \ytext (k+1)=3
        $$
        De la primera obtenemos que $m= 2$ y que $k=4$. Con lo que $n=2 \cdot 5^2 \cdot 7^4$.
        De la segunda obtenemos que $m= 4$ y que $k=2$. Con lo que $n=2 \cdot 5^4 \cdot 7^2$.

        \bigskip

        Luego, los unicos $n$ que cumplen son
        $$
          \cajaResultado{
            n = 2 \cdot 5^2 \cdot 7^4 = 120050
          }
          \ytext
          \cajaResultado{
            n = 2 \cdot 5^4 \cdot 7^2 = 61250
          }
        $$

\end{enumerate}

\begin{aportes}
  \item \aporte{https://github.com/Nunezca}{Nunezca \github}
\end{aportes}
