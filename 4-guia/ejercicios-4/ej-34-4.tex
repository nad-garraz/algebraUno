\begin{enunciado}{\ejercicio}
        Hallar el menor número natural $n$ tal que $(n : 3150) = 45$ y $n$ tenga exactamente 12 divisores positivos.
\end{enunciado}

Trabajemos con la primera condición: 
\\
$$
(n:3150)=45
\sisolosi
(n:2\cdot3^{2}\cdot5^{2}\cdot7)=3^{2}\cdot5
$$


Utilizando que el MCD se calcula como primos en común a la menor potencia, concluimos que $n$ no tiene en su factorización al 2 ni al 7
y que si tiene en su factorización un 5 y un $3^{i}$, con $i \geq 2$. Es decir:
\\

$$
n = 3^{i}\cdot5\cdot(P_1)^{m_1}...(P_k)^{m_k} ~,~ i \geq 2 ~ y ~  m_j \geq 0 
$$

De la segunda condición tenemos que
\\
$$
12 =2(i+1)(m_1+1)...(m_k+1)
\sisolosi
6=(i+1)(m_1+1)...(m_k+1)
$$

Como  $i \geq 2 \entonces i+1 \geq 3$ y como queremos que el producto nos de 6, esto nos deja dos opciones:

\begin{itemize}
        \item $(i+1)= 6 \ytext (m_1+1)...(m_k+1)= 1$
        \item $(i+1)=3 \ytext (m_1+1)...(m_k+1)=2$
\end{itemize}

De la primera tenemos que $i=5$ y que no hay otro primo en la factorización. De modo que $n=3^{5}\cdot5=1215$
\\
\\
De la segunda tenemos que $i=2$ y que solo puede haber otro primo en la factorización con $m_1=1$.
Como nos piden el menor $n$, elegimos el menor primo que le sigue a 5 que no sea el 7, es decir, el 11. Entonces, $n=3^{2}\cdot5\cdot11=495$
\\
\\
Luego, elegiendo el menor entre los dos, la respuesta es $\boxed{n=495}$

\begin{aportes}
        \item \aporte{https://github.com/Nunezca}{Nunezca \github}
\end{aportes}