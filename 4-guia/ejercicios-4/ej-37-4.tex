\begin{enunciado}{\ejercicio}
  Sean $a,b \en \enteros$ tales que $(a:b)=5$.

  \begin{enumerate}[label=(\alph*)]

    \item Calcular los posibles valores de $(ab:5a-10b)$ y dar un ejemplo para cada uno de ellos.

    \item Para cada $k \en \naturales$, calcular $(a^{k-1}b:a^k+b^k)$.
  \end{enumerate}

\end{enunciado}

\begin{enumerate}[label=(\alph*)]

\item\label{ej-31:a}

Coprimizo: defino $5a'=a$ y $5b'=b$, con lo que $(a:b)=(5a':5b')=5(a':b')=5$, de modo que $(a':b')=1$. \bigskip

Reemplazo en $d=(ab:5a-10b):$

$$
  d=(ab:5a-10b) =
  (25a'b':25a'-50b')=
  25(a'b':a'-2b')
$$

Sea ahora $d'=(a'b':a'-2b')$. Entonces, $d=25d'$.
Trabajemos ahora con $d'$
$$
  \llave{l}{
    d' \divideA a'b' \\
    d' \divideA a'-2b'
  }
  \llave{l}{
    \Entonces{$b' \cdot F_2$}
    \llave{l}{
      d' \divideA a'b' \\
      \\
      d' \divideA a'b'-2(b')^2
    }
    \Entonces{$F_1-F_2$}
    d' \divideA 2(b')^2 \\
    \\
    \Entonces{$4 \cdot F_1$}[$ 2a' \cdot F_2$]
    \llave{l}{
      d' \divideA 4a'b' \\
      d' \divideA 2(a')^2 -4b'a'
    }
    \Entonces{$F_1+F_2$}
    d' \divideA 2(a')^2
  }
  \entonces
  d' \divideA (2(b')^2:2(a')^2 )
$$
De eso último:
$$
  d' \divideA (2(b')^2:2(a')^2 )=2((b')^2:(a')^2) \igual{\red{!!}} 2 \cdot 1 = 2
  \entonces
  d' \divideA 2
  \sii
  \cajaResultado{
    d' \en \set{1,2}
  }
$$
En $\red{!!}$ uso que  $(b':a')=1 \sisolosi ((b')^2:(a')^2)=1$ \bigskip

Como $d'= 1 \otext 2$, entonces $d= 25 \otext 50$. Veamos ahora que ambos valores son posibles:
$$
  \llave{l}{
    (a,b)=(0,5)
    \flecha{$(0,5)=5$}
    d=(0:-50)=50 \Tilde \\
    (a,b)=(5,15)
    \flecha{$(5,15)=5$}
    d=(75:25-150)=(50:-125)=25 \Tilde
  }
$$

Finalmente:
$$
  \cajaResultado{d = 25 \otext d = 50}
$$

\item Procedimiento similar al ítem \ref{ej-31:a}

Coprimizo: defino $5a'=a$ y $5b'=b$, con lo que $(a:b)=(5a':5b')=5(a':b')=5$, de modo que $(a':b')=1$.

\medskip

Reemplazo en $d=(a^{k-1}b:a^k+b^k):$

$$
  \begin{array}{rcl}
    d=(a^{k-1}b:a^k+b^k) & = & ((5a')^{k-1}5b':(5a')^k+(5b')^k)                   \\
                         & = & (5^k(a')^{k-1}b':5^k(a')^k+5^k(b')^k)              \\
                         & = & 5^k\big(\ub{ (a')^{k-1}b':(a')^k+(b')^k }{d'}\big) \\
                         & = & 5^k d'
  \end{array}
$$
Si $d'=((a')^{k-1}b':(a')^k+(b')^k)$, entonces $d=5^kd'$

Trabajemos ahora con $d'$
$$
  \llave{l}{
    d' \divideA (a')^{k-1}b' \\
    \\
    d' \divideA (a')^k+(b')^k
  }
  \llave{l}{
    \Entonces{$a'(b')^{k-1} \cdot F_1$}[$(a')^k \cdot F_2$]
    \llave{l}{
      d' \divideA (a')^k(b')^k \\
      d' \divideA (a')^{2k}+(b')^k(a')^{k}
    }
    \Entonces{$F_2-F_1$}
    d' \divideA (a')^{2k} \\
    \\
    \Entonces{$a'(b')^{k-1} \cdot F_1$}[$(b')^k \cdot F_2$]
    \llave{l}{
      d' \divideA (a')^k(b')^k \\
      d' \divideA (b')^{2k}+(b')^k(a')^{k}
    }
    \Entonces{$F_2-F_1$}
    d' \divideA (b')^{2k}
  }
  \entonces
  d' \divideA ((a')^{2k}:(b')^{2k}) \igual{\red{!!}} 1
$$

$$
  \entonces
  d' \divideA 1
  \entonces
  d' = 1
$$

En $\red{!!}$ uso que  $(a':b')=1 \sisolosi ((a')^{2k}:(b')^{2k})=1$ \bigskip

Luego, como $d'=1$, tenemos que $\cajaResultado{d=5^k, \text{para cada $k \en \naturales$}}$

\begin{aportes}
  \item \aporte{https://github.com/Nunezca}{Nunezca \github}
\end{aportes}

