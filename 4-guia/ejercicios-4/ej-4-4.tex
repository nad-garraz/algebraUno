\begin{enunciado}{\ejercicio}
  Sea $a \in \enteros$ impar.
  Probar que $2^{n+2} \divideA a^{2^n} - 1$ para todo $n \en \naturales$
\end{enunciado}

Pruebo por inducción:

$$
  p(n): 2^{n+2} \divideA a^{2^n} - 1,\, \text{ con } a \en \enteros \text{ e impar.} \paratodo n \en \naturales.
$$

\textit{Caso base: }
$$
  p(1) : 2^3 = 8 \divideA a^2 - 1 = (a - 1) \cdot (a + 1)
$$
Como $a \en \enteros$ es impar, puedo escrirla como:
$$
  a \igual{$\llamada1$} 2m -1
$$
Entonces
$$
  \begin{array}{rcl}
    (a - 1) \cdot (a + 1) & \igual{$\llamada1$} & (2m - 2)\cdot(2m)                         \\
                          & \igual{\red{!}}     & 4 \cdot \ub{m \cdot (m-1)}{\text{seguro}  \\\text{es par} }                     \\
                          & \igual{\red{!!}}    & 4 \cdot 2 \violet{h} = 8 \cdot \violet{h}
  \end{array}
$$
Es decir que $a^2 - 1 = 8 \cdot \violet{h}$ con $\violet{h} \en \enteros$. Por lo tanto:
$$
  8 \divideA 8h = (a - 1) \cdot (a + 1) \text{ para algún } h \en \enteros
$$

La proposición $p(1)$ es verdadera.

\textit{Paso inductivo: }
Asumo que para algún valor de $\blue{k} \en \naturales$ que:
$$
  p(\blue{k}): \ob{
    2^{\blue{k}+2} \divideA a^{2^{\blue{k}}} - 1
  }{
    \purple{\text{hipótesis inductiva}}
  },
$$
es verdadera, entonces quiero probar que la proposición:
$$
  p(\blue{k}+1) : 2^{\blue{k} + 3} \divideA a^{2^{\blue{k} + 1}} - 1,
$$
también lo sea.
$$
  \begin{array}{c}
    2^{k+3} \divideA a^{2^{k+1}} - 1
    \Sii{\red{!!}}
    2^{k+2} \cdot 2 \divideA (a^{2^k} - 1)
    \cdot
    \ob{(a^{2^k} + 1)}{\text{\green{par !}}}                                                           \\
    \Sii{Si $a \divideA b \ytext c \divideA d \entonces ac \divideA bd$}[\purple{hipótesis inductiva}] \\
    \purple{2^{k+2}} \cdot 2 \divideA \purple{(a^{2^k} - 1)} \cdot \ub{(a^{2^k} + 1)}{\text{\green{par}}}.
  \end{array}
$$

El \red{!!} es todo tuyo (\textit{hints:} diferencia de cuadrados, propiedades de exponentes... \faIcon{hands-wash})

En el último paso se comprueba que $p(k+1)$ es vedadera.

Como $p(1), p(k) \ytext p(k+1)$ resultaron verdaderas, por el principio de inducción también lo será ${p(n) \paratodo n \en \naturales}$.

\begin{aportes}
  \item \aporte{\dirRepo}{naD GarRaz \github}
\end{aportes}
