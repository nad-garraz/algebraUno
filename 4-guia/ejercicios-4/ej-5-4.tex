\begin{enunciado}{\ejercicio}
  \begin{enumerate}[label=\roman*)]
    \item Probar que si $n$ es compuesto, entonces $2^n - 1$ es compuesto. 
    \item Probar que si $2^n + 1$ es primo, entonces $n$ es una potencia de 2. 
  \end{enumerate}
\end{enunciado}

Para el ejercicio vamos a considerar 2 identidades que son las siguientes: 
\begin{enumerate}[label=\alph*)]
\item $a^n - 1 = (a - 1)(a^{n-1} + a^{n-2} + \cdots + a + 1)$

Prueba: Consideramos la serie geometrica: 
\begin{align*}
\sum_{k=0}^{N - 1} a^k &= \frac{a^N - 1}{a - 1} \\
1 + a + \cdots + a^{N - 1} &= \frac{a^N - 1}{a - 1} \\
(1 + a + \cdots + a^{N - 1})(a - 1) &= a^N - 1
\end{align*}

\item $a^m + 1 = (a + 1)(a^{m-1} - a^{m-2} + \cdots - x + 1)$ con $m$ impar 

Pruba por inducción: $p(1)$ verdadero. Asumo $p(2k + 1)$ verdadero, luego qvq $p(2k + 3)$ es verdadero. 
Considero la divivsion $\frac{a^{2k + 3} + 1}{a + 1}$, que es igual a $(a + 1)(a^{2k + 2} - a^{2k + 1})
+ \red{\underbrace{\frac{a^{2k + 1}}{a + 1}}_{\text{hipotesis inductiva}}}$, como se queria probar.

\end{enumerate}

Ahora encaramos el ejercicio
\begin{enumerate}[label = \roman*)]
 \item si $n$ es compuesto entonces se puede escribir como $ab$. 
 
 Sigue que $2^{ab} - 1 = (2^a)^b - 1 = (2^a - 1)((2^a)^{b-1} + (2^a)^{b - 2} + \cdots + 2^a + 1)$, 
 claramente este numero es compuesto, como se queria probar.

 \item Para probar esta implicacion vamos a probar el contrarreciproco ($p \entonces q = \lnot q \entonces \lnot p$). 
 Entonces, queremos probar que dado $n$ divisible por un primo impar entonces $2^n + 1$ es compuesto. 

 $p \divideA n \sii n = kp$, luego $2^n + 1$ = $2^{kp} + 1 = (2^k)^p + 1 = (2^k + 1)((2^k)^{p - 1} - (2^k)^{p - 2} + \cdots - 2^k + 1)$. 
 Que es un numero compuesto, como se queria probar. 
\end{enumerate}

\begin{aportes}
 \item \aporte{https://github.com/sigfripro}{sigfripro \github}
\end{aportes}