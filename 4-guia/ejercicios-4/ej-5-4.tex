\begin{enunciado}{\ejercicio}
  \begin{enumerate}[label=\roman*)]
    \item Probar que si $n$ es compuesto, entonces $2^n - 1$ es compuesto.
    \item Probar que si $2^n + 1$ es primo, entonces $n$ es una potencia de 2.
  \end{enumerate}
\end{enunciado}

Un número compuesto es un número natural que tiene más de 2 divisores. Un número compuesto $n$ se puede
escribir como el producto de 2 números menores, en particular son dos divisores de n, $n = d_1 \cdot d_2$.

\begin{enumerate}[label=\roman*)]
  \item  Mirá el ejercicio \refEjercicio{ej:3} a ver si se te ocurre como usar ese resultado en esta parte.

        A partir del resultado dle ejercicio \refEjercicio{ej:3}:
        $\congruencia{a^n - b^n}{0}{a - b} \paratodo n \en \naturales \ytext a\distinto b \en \enteros$.
        $$
          \begin{array}{rcl}
            \congruencia{a^n - b^n}{0}{a - b}
             & \Sii{def}                      &
            a^n - b^n = (a - b) \cdot k                 \\
             & \Sii{$a = 2^{d_1}$}[$n = d_2$] &
            (2^{d_1})^{d_2} - b = (2^{d_1} - b) \cdot k \\
             & \Sii{$b = 1$}                  &
            2^{d_1 \cdot d_2} - 1 = (2^{d_1} - 1) \cdot k
          \end{array}
        $$
        Por lo tanto una potencia de 2 con un exponente \textit{compuesto} menos 1, es otro número compuesto.

  \item Mirá el ejercicio \refEjercicio{ej:4} a ver si se te ocurre como usar ese resultado en esta parte.

        Por contrarrecíproco:
        $$
          (p \entonces q) \sisolosi (\lnot q \entonces \lnot p)
        $$
        Entonces, queremos probar que dado $n$ divisible por un \ul{primo impar}, $\blue{p}$, es decir que $n \distinto 2^m$,
        entonces $2^n + 1$ es compuesto, es decir que \ul{no es primo} ni a palos.

        Dado un $\blue{p}$ impar:
        $$
          p \divideA n \Sii{def} n = \blue{p} \cdot \green{q},
        $$
        Asegurando así que $n \distinto 2^m$.
        Del ejercicio \refEjercicio{ej:4} sé que para un natural impar, $\blue{p}$ en este caso:
        $$
          a + b \divideA a^{\blue{p}} + b^{\blue{p}}
          \Sii{def}
          a^{\blue{p}} + b^{\blue{p}} = (a + b) \cdot k
          \Sii{$a = 2^{\green{q}}$}[$b = 1$]
          (2^{\green{q}})^{\blue{p}} + 1 = (2^{\green{q}} + 1) \cdot k
          \sii
          \cajaResultado{
            2^{\blue{p} \cdot \green{q}} + 1 = (2^{\green{q}} + 1) \cdot k
          }
        $$
        El último resultado muestra que:
        \begin{center}
          \big(
          Si $n \distinto 2^k \entonces 2^n + 1$ no es primo
          \big)
          \quad$\sisolosi$\quad
          \big(
          Si $2^n + 1$ es primo $\entonces n$ es una potencia de 2.
          \big)
        \end{center}
\end{enumerate}

\begin{aportes}
  \item \aporte{\dirRepo}{naD GarRaz \github}
  \item \aporte{https://github.com/sigfripro}{sigfripro \github}
\end{aportes}
