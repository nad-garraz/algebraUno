\begin{enunciado}{\ejercicio}
  \begin{enumerate}[label=\alph*)]
    \item Probar que el producto de $n$ enteros consecutivos es divisible por $n!$
    \item Probar que $\binom{2n}{n}$ es divisible por 2.
  \end{enumerate}
\end{enunciado}

\begin{enumerate}[label=\alph*)]
  \item
        Este es bastante sencillo de ver, tenemos el producto de $n$ enteros consecutivos como:
        $$
          \productoria{i=\green{m}}{n + \green{m} - 1} i = \green{m} \times (\green{m} + 1) \times \cdots \times (n + \green{m} - 1)
        $$
        Por ejemplo:
        $$
          \productoria{i=4}{7} i  = 4\times5\times6\times7
        $$
        Tenemos que ver que es divisible por $n!$, si uno se lo pone a pensar intuitivamente, es bastante logico
        ya que $n!$ va a siempre compartir factores, de hecho el factorial se puede definir como $\productoria{i=1}{n}  i = n!$, asi que
        si empezamos desde otro indice no cambia nada, igual sigue siendo divisible por $n!$.

        Vamos a probarlo por inducción porque es lo más fácil, sino se podría generalizar con productorias pero es un \textit{quilombo de notación}.

        Tenemos que nuestra \purple{hipotesis inductiva} es:
        $$
          p(n):\productoria{i=\green{m}}{n} i \text{ es divisible por } n!
        $$
        Probemos el caso base $p(1)$:
        $$
          p(1): \productoria{i=\green{m}=1}{1}  i = 1 \text{ es divisible por 1!}
        $$
        Ahora queremos ver que $p(n) \entonces p(n+1)$
        $$
          \productoria{i=\green{m}}{n+1}  i \text { es divisible por $(n+1)!$ ?}
        $$
        $$
          (\red{n+1})  \times \ub{\displaystyle \productoria{i=\green{m}}{\red{n}}  i}{\text{\purple{hipótesis inductiva}}}
          \text{ es divisible por }
          \ub{n! \times (\red{n+1})}{=(n+1)!} \llamada1
        $$
        Finalmente entonces probamos que $p(n+1)$ es verdadera, completando la prueba.

        $\llamada1 :$ Aca usamos que
        $
          a\divideA b
          \entonces
          t\cdot a \divideA t \cdot b,\, t \en \enteros
        $

  \item
        Expandimos el coeficiente binomial con su fórmula de factoriales:
        $$
          \binom{2n}{n} = \frac{(2n)!}{(n!)^2}
        $$
        Si fuera par, eso significa que al dividirlo por dos obtengo un número entero, entonces la idea va a ser esa:
        $$
          \frac{(2n)!}{(n!)^2} \cdot \frac{1}{2} = \frac{(\cancel{2}n)(2n-1)!}{(n!)^2} \cdot \frac{1}{\cancel{2}}
        $$
        Entonces me basta ver que
        $$
          \frac{n(2n-1)!}{(n!)^2} \quad \en \enteros
        $$
        Manipulamos la expresión un poco:
        $$
          \frac{\cancel{n}(2n-1)!}{(n!)\cancel{n}(n-1)!}
        $$
        Ahora podemos ver que esta expresión es lo mismo que $\binom{2n-1}{n-1}$, el cual es un número entero. Concluimos
        entonces que $\binom{2n}{n}$ es par.

        \parrafoDestacado[$\triangle$]{
          Para más intuición, fijarse que en el triángulo de pascal, el coeficiente $\binom{2n}{n}$ es justo la mitad de la fila
          con $n$ par (en este caso el $n$ seria el $2n$ de nuestro coeficiente binomial), pueden ver que todos los valores que están en el medio
          en el triángulo de pascal surgen de la suma de los dos de arriba, que son exactamente iguales.
        }

\end{enumerate}

\begin{aportes}
  \item \aporte{https://github.com/sigfripro}{sigfripro \github}
\end{aportes}
