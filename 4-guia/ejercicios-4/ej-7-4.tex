\begin{enunciado}{\ejercicio}
  Proba que las siguientes afirmaciones son vedaderas para todo $n \en \naturales.$
  \begin{multicols}{2}
    \begin{enumerate}[label=\alph*)]
      \item $99 \divideA 10^{2n} + 197$
      \item $9 \divideA 7 \cdot 5^{2n} + 2^{4n+1}$
      \item $56 \divideA 13^{2n} + 28n^2 - 84n -1$
      \item $256 \divideA 7^{2n} + 208n - 1$
    \end{enumerate}
  \end{multicols}
\end{enunciado}

\begin{enumerate}[label=\alph*)]
  \item $99 \divideA 10^{2n} + 197 \Sii{def}
          \congruencia{10^{2n} + 197}{0}{99} \to
          \congruencia{10^{2n} + 198}{1}{99} \to
          \congruencia{10^{2n} + \ub{198}{\conga{99} 0}}{1}{99} \to \congruencia{100^n}{1}{99}\to \\
          \llave{l}{
            \flecha{sé}[que] \congruencia{100}{1}{99} \sisolosi \congruencia{100^2}{\ub{100}{ \conga{99} 1 }}{99} \to
            \congruencia{100^2}{1}{99} \sisolosi \dots \sisolosi \congruencia{100^n}{1}{99} \\
          }
        $\par
        Se concluye que  $99 \divideA 10^{2n} + 197 \sisolosi 99 \divideA \ub{100 - 1}{99}$

  \item
        $9 \divideA 7 \cdot 5^{2n} + 2^{4n+1}
          \Sii{def}
          \congruencia{7\cdot5^{2n} + 2^{4n+1}}{0}{9}
          \flecha{sumo $2\cdot 5^{2n}$}[M.A.M]
          \congruencia{\ub{9 \cdot 5^{2n}}{\conga9 0} + 2\cdot 2^{4n}}{2 \cdot 5^{2n}}{9}\\
          \flecha{simplifico}[y acomodo]
          \congruencia{2^{4n}}{5^{2n}}{9} \to
          \congruencia{16^n}{25^n}{9}
          \flecha{simetría}[congruencia]
          \congruencia{25^n}{16^n}{9}
          \flecha{$25\conga9 16$}
          \congruencia{25}{16}{9} =
          \congruencia{9}{0}{9}\\
        $
        Se concluye que $9 \divideA 7 \cdot 5^{2n} + 2^{4n+1} \sisolosi 9 \divideA 9$ \red{$\leftarrow$ ¿Se concluye esto...?}

  \item \hacer

  \item  Hermoso ejercicio en el que sin fe en el todo poderoso Gauss sencillamente uno tira la toalla.

        Sale por inducción:

        \textit{Quiero ver que:}
        $$
          p(n) : 256 \divideA 49^n + 208n - 1
        $$

        O en notación de congruencia:

        $$
          p(n) : \congruencia{49^n + 208n - 1}{0}{256}
        $$

        \textit{Caso base:}
        $$
          p(\blue{1}) : 256 \divideA 49^{\blue{1}} + 208 \cdot \blue{1} - 1 \Tilde
        $$
        Por lo tanto $p(1)$ resulta verdadera.

        \textit{Paso inductivo:}

        Uso la notación de congruencia de acá en adelante, porque es mucho más cómodo.
        Supongo que:
        $$
          p(\blue{k}) : \ub{\congruencia{49^{\blue{k}} + 208 \cdot \blue{k} - 1}{0}{256}}{\purple{\text{hipótesis inductiva}}}
          \quad \text{ para algún } \blue{k} \en \enteros
        $$
        es  una proposición verdadera.
        Entonces quiere probar que:
        $$
          p(\blue{k+1}) : \congruencia{49^{\blue{k+1}} + 208 \cdot (\blue{k+1}) - 1}{0}{256},
        $$
        también sea verdadera. Arranco del paso $(k+1)$ y haciendo un poco de \textit{matemagia}:
        $$
          \begin{array}{rcl}
            49^{k+1} + 208 \cdot (k+1) - 1 & =                       & 49 \cdot 49^k + 208k + 208 - 1
            \conga{(256)}[\purple{\text{HI}}]
            49 \cdot (-208k + 1) + 208k + 208 - 1                                                                                \\
                                           & \conga{256}             & 49 \cdot (48k + 1) - 48k -48 - 1 = 2352 k + 49 - 48k - 49 \\
                                           & \conga{(256)}[\red{!!}] & 48 k + 49 - 48k - 49 = 0 \Tilde
          \end{array}
        $$
        En \red{!!} y gracias a Gauss $\congruencia{2352}{48}{256}$ ¿Casualidad? No sé y no me importa.\par

        Dado que $\congruencia{49^{k+1} + 208 \cdot (k+1) - 1}{0}{256}$, la proposición $p(k+1)$ resultó verdadera.\par\bigskip

        Dado que $p(1), p(k) \ytext p(k+1)$ resultaron verdaderas, por principio de inducción $p(n)$ también lo es para todo $n \en \naturales$.

\end{enumerate}

% Contribuciones
\begin{aportes}
  \item \aporte{\dirRepo}{naD GarRaz \github}
\end{aportes}

