\textit{\underline{Algoritmo de División}}:\par

\begin{enunciado}{\ejercicio}
  Calcular el cociente y el resto de la división de $a$ por $b$ en los casos:
  \begin{multicols}{2}
    \begin{enumerate}[label=\alph*)]
      \item $a =133,\quad b = -14.$
      \item $a =13,\quad b = 111.$
      \item $a =3b+7,    \quad b \distinto 0.$
      \item $a = b^2 - 6,\quad b \distinto 0.$
      \item $a = n^2 + 5,\quad b = n+2 \ (n \en \naturales).$
      \item $a = n + 3,   \quad b = n^2 + 1 \ (n \en \naturales).$
    \end{enumerate}
  \end{multicols}
\end{enunciado}

Voy a usar $\blue{q}$ para el \blue{cociente} y \magenta{$r$} para el \magenta{resto}:
\parrafoDestacado[\purple{\atencion}]{
  $b$ el divisor tiene que cumplir: $|b| \leq |a|$

  $\magenta{r}$ tiene que cumplir condición de resto: $0 \leq \magenta{r} \leq |a|$.
}

\begin{enumerate}[label=\alph*)]
  \item $133 \div (-14) \entonces 133 = \ua{\blue{(-9)}}{\blue{q}} \cdot (-14) + \ua{\magenta{7}}{\magenta{r}}$

  \item $13 \div 111  \entonces 13 = 111 \cdot \ua{\blue{0}}{\blue{q}} + \ua{\magenta{13}}{\magenta{r}}$

        ¿Qué onda esto? Si bien el divisor $b = 111 > 13 = a$ ¿El algoritmo de división vale igual entonces y
        valen el cociente y el resto?

  \item $a = 3b + 7 \to
          \text{me interesa: }\to
          \llaves{l}{
            |b| \leq |a| \Tilde\\
            0 \leq \magenta{r} < |b| \Tilde \\
          }$

        $$
          \llave{l}{
            \text{Si: } |b| > 7 \to (\blue{q},\magenta{r}) = (3,7)\\
            \text{Si: } |b| \leq 7 \to (\blue{q},\magenta{r}) = (3,7)\\
            \begin{array}{|l|l|l|l|l|l|l|l|}
              \hline
              (a,b)                  & (-14,-7) & (-11,-6) & (-8,-5) & (-5,-4) & (4, -1) & \dots \\ \hline
              (\blue{q},\magenta{r}) & (2,0)    & (2,1)    & (2,2)   & (2,3)   & (4, 0)  & \dots \\ \hline
            \end{array}
          }
        $$

  \item $a = b^2 - 6,\quad b \distinto 0$:

        Quiero hacer $\frac{a}{b} = \frac{b^2 - 6}{b}$:
        $$
          \polyset{vars=b}
          \divPol{b^2 - 6}{b}
          \qquad \to \qquad
          b^2 - 6 = \ua{b}{d} \cdot \oa{b}{\blue{q}} - 6 \ \llamada1
        $$
        Entonces los escenarios posibles:
        \begin{center}
          $b$ el divisor tiene que cumplir: $|b| \leq |a|$

          $\magenta{r}$ tiene que cumplir condición de resto: $0 \leq \magenta{r} \leq |a|$.
        \end{center}
        $$
          \begin{array}{|c|c|c|c|c|}
            \hline
            (a,b)                  & (-5,1) & (-5,-1) & (-2,2) & (-2,-2) \\ \hline
            (\blue{q},\magenta{r}) & (-5,0) & (5,0)   & (-1,0) & (1,0)   \\ \hline
          \end{array}
        $$
        Caso cuando $a = -5$ y $b = 1$ meto en $\llamada1$:
        $$
          -5 =
          1 \cdot 1 - 6
          \igual{\red{!}}
          1 \cdot \blue{(-5)} + \magenta{0}
        $$
        Caso cuando $a = -5$ y $b = -1$ meto en $\llamada1$:
        $$
          -5 =
          (-1) \cdot (-1) - 6
          \igual{\red{!}}
          (-1) \cdot \blue{(-5)} + \magenta{0}
        $$
        Caso cuando $a = -2$ y $b = 2$ meto en $\llamada1$:
        $$
          -2 =
          2 \cdot \blue{(-1)} + \magenta{0}
        $$
        Caso cuando $a = -2$ y $b = -2$ meto en $\llamada1$:
        $$
          -2 =
          (-2) \cdot \blue{1} + \magenta{0}
        $$

  \item \hacer

  \item \hacer
\end{enumerate}

\begin{aportes}
  \item \aporte{\dirRepo}{naD GarRaz \github}
\end{aportes}
