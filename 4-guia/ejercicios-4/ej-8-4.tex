\textit{\underline{Algoritmo de División}}:\par

\begin{enunciado}{\ejercicio}
  Calcular el cociente y el resto de la división de $a$ por $b$ en los casos:
  \begin{multicols}{2}
    \begin{enumerate}[label=\alph*)]
      \item $a =133,\quad b = -14.$
      \item $a =13,\quad b = 111.$
      \item $a =3b+7,    \quad b \distinto 0.$
      \item $a = b^2 - 6,\quad b \distinto 0.$
      \item $a = n^2 + 5,\quad b = n+2 \ (n \en \naturales).$
      \item $a = n + 3,   \quad b = n^2 + 1 \ (n \en \naturales).$
    \end{enumerate}
  \end{multicols}
\end{enunciado}

Voy a usar $\blue{q}$ para el \blue{cociente} y \magenta{$r$} para el \magenta{resto}:
\parrafoDestacado[\purple{\atencion}]{

  $\magenta{r}$ tiene que cumplir condición de resto: $0 \leq \magenta{r} \leq |d|$.
}

Congruencia:
        $$
          d \divideA a
          \Sii{def} \congruencia{a}{0}{d}
        $$

\begin{enumerate}[label=\alph*)]
  \item $a = 133 = -14 \cdot \ua{\blue{-9}}{\blue{q}} + \ua{\magenta{7}}{\magenta{r}}$
  
  \item $13 \div 111  \entonces 13 = 111 \cdot \ua{\blue{0}}{\blue{q}} + \ua{\magenta{13}}{\magenta{r}}$

  \item $a = 3b + 7,\quad b = b$
        $$
          \begin{array}{r|c|c|c|c|c|c|c|c|c|c|c|c|c|c|c|} \cline{2-15}
            \text{dividendo}\to & a = 3b + 7                  & -5 & -2 & 1  & 4  & 10 & 13 & 16 & 19 & 22 & 25 & 28 & 31 & 34 \\ \cline{2-15}
            \text{divisor}  \to & b                           & -4 & -3 & -2 & -1 & 1  & 2  & 3  & 4  & 5  & 6  & 7  & 8  & 9  \\ \cline{2-15}
            \text{cociente} \to & \blue{q}                    & 2  & 1  & 0  & -4 & 10 & 6  & 5  & 4  & 4  & 4  & 4  & 3  & 3  \\ \cline{2-15}
            \text{resto}    \to & 0 \leq \magenta{r} \leq |d| & 3  & 1  & 1  & 0  & 0  & 1  & 1  & 3  & 2  & 1  & 0  & 7  & 7  \\ \cline{2-15}
          \end{array}
        $$
        Pasando en limpio, para $b \en \enteros_{\distinto 0}$:
        $$
          \llave{rcl}{
            \blue{q_b(a)} = 3  & \text{ para } & |b| \geq 8\\
            \magenta{r_b(a)} = 7  & \text{ para } & |b| \geq 8\\
            \text{ver tabla}& \text{ para } & |b| < 8
          }
        $$

  \item $a = b^2 - 6,\quad b \distinto 0$:

        Quiero hacer $\frac{a}{b} = \frac{b^2 - 6}{b}$:
        $$
          b^2 - 6 = \ua{b}{d} \cdot \oa{b}{\blue{q}} - 6
        $$
        Posibles valores para $a,\ b,\ \blue{q} \ytext \magenta{r}$:
        $$
          \begin{array}{r|c|c|c|c|c|c|c|c|c|c|} \cline{2-10}
            \text{dividendo}\to & a = b^2 - 6                 & -5 & -5 & -2 & -2 & 3 & 3  & 10 & 10 \\ \cline{2-10}
            \text{divisor}  \to & b                           & 1  & -1 & 2  & -2 & 3 & -3 & 4  & -4 \\ \cline{2-10}
            \text{cociente} \to & \blue{q}                    & -5 & 5  & -1 & 1  & 1 & -1 & 2  & -2 \\ \cline{2-10}
            \text{resto}    \to & 0 \leq \magenta{r} \leq |d| & 0  & 0  & 0  & 0  & 0 & 0  & 2  & 2  \\ \cline{2-10}
          \end{array}
        $$

  \item $a = n^2 + 5,\quad b = n + 2 \ (n \en \naturales).$

        Quiero hacer $\frac{a}{b} = \frac{n^2 + 5}{n + 2}$:
        $$
          \polyset{vars=n}
          \divPol{n^2 + 5}{n + 2}
          \qquad \to \qquad
          n^2 + 5 = \ub{(n + 2)}{d} \cdot \ob{(n - 2)}{\blue{q_n}} + 9
        $$
        Posibles valores para $a,\ b,\ \blue{q} \ytext \magenta{r}$:
        $$
          \begin{array}{r|c|c|c|c|c|c|c|c|c|c|c|c|c|c} \cline{2-13}
                                & n                           & 1 & 2 & 3  & 4  & 5  & 6  & 7  & 8  & 9  & 10  & 11  \\ \cline{2-13}
            \text{dividendo}\to & a = n^2 + 5                 & 6 & 9 & 14 & 21 & 30 & 41 & 54 & 69 & 86 & 105 & 126 \\ \cline{2-13}
            \text{divisor}  \to & b = n + 2                   & 3 & 4 & 5  & 6  & 7  & 8  & 9  & 10 & 11 & 12  & 13  \\ \cline{2-13}
            \text{cociente} \to & \blue{q}                    & 2 & 2 & 2  & 3  & 4  & 5  & 6  & 6  & 7  & 8   & 9   \\ \cline{2-13}
            \text{resto}    \to & 0 \leq \magenta{r} \leq |d| & 0 & 1 & 4  & 3  & 2  & 1  & 0  & 9  & 9  & 9   & 9   \\ \cline{2-13}
          \end{array}
        $$
        Entonces en la tabla están los valores de los cocientes. Los primeros están sacados a manopla:
        $$
          \llave{ccl}{
            \blue{q_b(a)} = n - 2 & \text{ para } & n \geq 8 \\
            \magenta{r_b(a)} = 9 & \text{ para } & n \geq 8
          }
        $$

  \item Cuando $b$ sea mayor que $a$ el resto va a ser simplemente $a$, asi que busco cuando esto se cumpla
  Digo que $n^2 + 1 > n + 3 \sii n^2 - n - 2 > 0 \sii n > 2 \sii n \geq 3$. Hago los casos separados para $n = 1$ y $n = 2$. 

  $n = 1$. $a = 4$, $b=2$, $\blue{q} = 2$, $\magenta{r} = 0$

  $n = 2$. $a = 5$, $b=5$, $\blue{q} = 1$, $\magenta{r} = 0$

  Finalmente 
  $$
  \blue{q_b(a)} = 
  \begin{cases}
  2 \text{ si } n = 1 \\
  1 \text{ si } n = 2 \\
  0 \text{ si } n \geq 3 
  \end{cases}
  \magenta{r_b(a)} =
  \begin{cases}
  0 \text{ si } n < 3 \\
  n + 3 \text{ si } n \geq 3
  \end{cases}
  $$

\end{enumerate}

\begin{aportes}
  \item \aporte{\dirRepo}{naD GarRaz \github}
  \item \aporte{https://github.com/sigfripro}{sigfripro \github}
\end{aportes}
