\begin{enunciado}{\ejercicio}
  Sabiendo que el resto de la división de un entero
  $a$ por 18 es 5, calcular el resto de:
  \begin{enumerate}[label=\alph*)]
    \begin{multicols}{2}
      \item la división de $a^2 -3a +11$ por 18.
      \item la división de $a$ por 3.
      \item la división de $4a+1$ por 9.
      \item la división de $7a^2 + 12$ por 28.
    \end{multicols}
  \end{enumerate}
\end{enunciado}

\begin{enumerate}[label=\alph*)]
  \item $r_{18}(a) =
          r_{18}( \ub{r_{18}(a)^2}{5^2} - \ub{r_{18}(3)}{3} \cdot \ub{r_{18}(a)}{5} + \ub{r_{18}(11)}{11} ) =
          r_{18}(21) = 3 $

        \separadorCorto

  \item $
          \llaves{l}{
            a = 3 \cdot q + r_3(a)\\
            6 \cdot a = 18 \cdot q + \ub{\green{6 \cdot r_3(a)}}{r_{18}(6a)}\\
          } \to
          r_{18}(6a) = r_{18}( r_{18}(6) \cdot r_{18}(a) ) = r_{18}(30) = 12\\
          \entonces \green{6 \cdot r_3(a)} = r_{18}(6a) \to  r_3(a) = 2
        $
        \separadorCorto

  \item $r_9(4a+1) = \ub{r_9(4 \cdot r_9(a) + 1)}{\blue{*1}} \to\\
          a = 18 \cdot q + 5 = 9 \cdot \ub{( 9 \cdot q)}{q'} + \ub{5}{r_9(a)}
          \flecha{\blue{*1}}
          r_9(a) = r_9(21) = 3
        $

  \item A ver el resto de esa expresión:
        $$
          r_{28}(7a^2 + 12) = r_{28}(7 \cdot r_{28}(a)^2 + 12)
        $$
        La pregunta ahora es ¿qué \poo es $r_{28}(a)\llamada1$? Sé por enunciado que:
        $$
          r_{18}(a) = 5
          \Sii{def}
          a = 18 \cdot q + 5
        $$
        La idea ahora es modificar esa expresión para que me quede un 28 de divisor:
        $$
          a = 18 \cdot q + 5 = 2 \cdot 9 q + 5
          \Sii{$\times$ 14}[M.A.M]
          14 \cdot a = \ub{14 \cdot 2}{28} \cdot 9 \cdot q + 70
        $$
        Ahora tengo que \textit{toquetear} esa expresión para que quede algo de la pinta
        $\green{\text{divisor}} \times \blue{\text{cociente}} + \purple{\text{resto}}$:
        $$
          14 \cdot a = \green{28} \cdot \blue{9q} + \purple{70}
          \sii
          14 \cdot a = \green{28} \cdot \blue{9q} + \purple{2\cdot 28 + 14}
          \igual{\red{!}}
          \green{28}\cdot \ub{\blue{(9q + 2)}}{\blue{q'}}  + \purple{14}
        $$
        Entonces queda que:
        $$
          14 \cdot a = \green{28}\cdot \blue{q'}  + \purple{14}
          \Sii{def}
          \congruencia{14a}{\purple{14}}{\green{28}}
          \sii
          \congruencia{a}{1}{28}
          \sii r_{28}(a) = 1
        $$
        Lo que logré fue encontrar una expresión donde el divisor sea 28 ¡Y así contesté $\llamada1$!

        Ahora que sé que $r_{28}(a) = 1$ sale que
        $$
          r_{28}(7a^2 + 12) = r_{28}( 7 \cdot r_{28}(a)^2 + 12) = r_{28}(7 \cdot 1 + 12 ) = r_{28}(a) = 19
        $$
\end{enumerate}

\begin{aportes}
  \item \aporte{\dirRepo}{naD GarRaz \github}
\end{aportes}
