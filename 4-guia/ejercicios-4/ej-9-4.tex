\begin{enunciado}{\ejercicio}
  Sabiendo que el resto de la división de un entero
  $a$ por 18 es 5, calcular el resto de:
  \begin{enumerate}[label=\alph*)]
    \begin{multicols}{2}
      \item la división de $a^2 -3a +11$ por 18.
      \item la división de $a$ por 3.
      \item la división de $4a+1$ por 9.
      \item la división de $7a^2 + 12$ por 28.
    \end{multicols}
  \end{enumerate}
\end{enunciado}

\begin{enumerate}[label=\purple{\alph*)}]
  \item $r_{18}(a) = 5
          \entonces
          r_{18}
          \big(
          a^2 -3a +11
          \big) =
          r_{18}\big(
          \ub{r_{18}(a)^2}{5^2} - \ub{r_{18}(3)}{3} \cdot \ub{r_{18}(a)}{5} + \ub{r_{18}(11)}{11}
          \big) =
          r_{18}(21) = 3 $


  \item Dividir a cualquier valor $a$ por 3 es hacer:
        $$
          a = 3 \cdot q + r_3(a)\llamada1
        $$
        Busco que en el algoritmo de división quede como divisor $18$ para poder usar el dato de que $r_{18}(a) = 5$.
        Multiplico en $\llamada1$ por 6 ambos miembros:
        $$
          \blue{6} \cdot a = \blue{6} \cdot 3 \cdot q + \blue{6} \cdot r_3(a)
          \sisolosi
          \blue{6a} = 18 \cdot q + \ub{\blue{6} \cdot r_3(a)}{\text{esto debe ser} \\ r_{18}(\blue{6a})}
        $$
        Por lo tanto usando el \textit{dato} de $r_{18}(a) = 5$:
        $$
          r_{18}(\blue{6a}) = 6 \cdot r_3(a)
          \Sii{\red{!!}}[\textit{dato}]
          30 = 6 \cdot r_3(a)
          \sii
          5 = r_3(a)
          \Sii{\textit{condición}}[\textit{de resto}]
          \cajaResultado{
            r_3(a) = 2
          }
        $$


  \item
        Dividir a cualquier valor $4a + 1$ por 9 es hacer:
        $$
          4a + 1 = 9 \cdot q + r_9(4a + 1)\llamada1
        $$
        Busco que en el algoritmo de división quede como divisor $18$ para poder usar el dato de que $r_{18}(a) = 5$.
        Multiplico en $\llamada1$ por 2 ambos miembros:
        $$
          \blue{2} \cdot (4a + 1) = \blue{2} \cdot 9 \cdot q + \blue{2} \cdot r_9(4a + 1)
          \sisolosi
          \blue{8a + 2} = 18 \cdot q + \ub{\blue{2} \cdot r_9(4a + 1)}{\text{esto debe ser} \\ r_{18}(\blue{8a + 2})}
        $$
        Por lo tanto usando el \textit{dato} de $r_{18}(a) = 5$:
        $$
          r_{18}(\blue{8a + 2}) = 2 \cdot r_9(\blue{4a + 1})
          \Sii{\red{!!}}[\textit{dato}]
          8 \cdot 5 + 2 = 2 \cdot r_9(\blue{4a + 1})
          \sii
          21
          = r_3(a)
          \Sii{\textit{condición}}[\textit{de resto}]
          \cajaResultado{
            r_9(4a + 1) = 2
          }
        $$

  \item A ver el resto de esa expresión:
        $$
          r_{28}(7a^2 + 12) = r_{28}(7 \cdot r_{28}(a)^2 + 12)
        $$
        La pregunta ahora es ¿qué \poo es $r_{28}(a)\llamada1$? Sé por enunciado que:
        $$
          r_{18}(a) = 5
          \Sii{def}
          a = 18 \cdot q + 5
        $$
        La idea ahora es modificar esa expresión para que me quede un 28 de divisor:
        $$
          a = 18 \cdot q + 5 = 2 \cdot 9 q + 5
          \Sii{$\times$ 14}[M.A.M]
          14 \cdot a = \ub{14 \cdot 2}{28} \cdot 9 \cdot q + 70
        $$
        Ahora tengo que \textit{toquetear} esa expresión para que quede algo de la pinta
        $\green{\text{divisor}} \times \blue{\text{cociente}} + \purple{\text{resto}}$:
        $$
          14 \cdot a \ua{=}{\text{\tiny también se podría}\\\text{\tiny tomar congruencia 28}\\\text{\tiny en ambos miembros }} \green{28} \cdot \blue{9q} + \purple{70}
          \sii
          14 \cdot a = \green{28} \cdot \blue{9q} + \purple{2\cdot 28 + 14}
          \igual{\red{!}}
          \green{28}\cdot \ub{\blue{(9q + 2)}}{\blue{q'}}  + \purple{14}
        $$
        Entonces queda que:
        $$
          14 \cdot a = \green{28}\cdot \blue{q'}  + \purple{14}
          \Sii{def}
          \congruencia{14a}{\purple{14}}{\green{28}}
          \sii
          \congruencia{a}{1}{28}
          \sii r_{28}(a) = 1
        $$
        Lo que logré fue encontrar una expresión donde el divisor sea 28 ¡Y así contesté $\llamada1$!

        Ahora que sé que $r_{28}(a) = 1$ sale que
        $$
          r_{28}(7a^2 + 12) = r_{28}( 7 \cdot r_{28}(a)^2 + 12) = r_{28}(7 \cdot 1 + 12 ) = r_{28}(a) = 19
        $$
\end{enumerate}

\begin{aportes}
  \item \aporte{\dirRepo}{naD GarRaz \github}
\end{aportes}
