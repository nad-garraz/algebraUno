\begin{itemize}
  \item $d$ divide a $a \to
          d \divideA a \sisolosi \existe k \en \enteros : a = k \cdot d$
  \item $ \divset{-a}{-|a|,\dots,-1,1,\dots,|a|}$.
  \item $d \divideA 0 $, dado que $0 = 0\cdot d$. Se desprende que $\divset{0}{\enteros - \set{0}}$
  \item $\llave{l}{
            d \divideA a \sisolosi -d \divideA a \text{ (pues }a = k \cdot d \sisolosi a = (-k) \cdot (-d))\\
            d \divideA a \sisolosi d \divideA -a \text{ (pues} a = k \cdot d \sisolosi (-a) = (-k) \cdot d)\\
            \entonces d \divideA a \sisolosi |d| \,\divideA\, |a|
          }$

  \item $\llave{l}{
            d \divideA a \ytext d \divideA b \entonces d \divideA a + b\\
            d \divideA a \ytext d \divideA b \entonces d \divideA a - b\\
            d \divideA a \entonces d \divideA c \cdot a, \paratodo c \en \enteros\\
            d \divideA a \entonces d \divideA c \cdot a\\
            d \divideA a \entonces d^2 \divideA a^2 \ytext d^n \divideA a^n  \paratodo n \en \naturales\\
            d \divideA a \cdot b \text{ no implica } d \divideA a \o d \divideA b. \text{ Por ejemplo } 6 \divideA 3 \cdot 4
          }$

  \item
        $\llave{l}{
            \textit{$a$ es congruente a $b$ módulo $d$} \text{ si }   d \divideA a-b \text{. Se nota } \congruencia{a}{b}{d}\\
            \congruencia{a}{b}{d} \sisolosi d \divideA a-b
          }$

  \item $
          \llave{c}{
            \congruencia{a_1}{b_1}{d}\\
            \vdots\\
            \congruencia{a_n}{b_n}{d}
          }
          \entonces \congruencia{a_1 + \cdots + a_n}{a_b + \cdots + b_n}{d}
        $.
  \item $
          \llave{c}{
            \congruencia{a_1}{b_1}{d}\\
            \vdots\\
            \congruencia{a_n}{b_n}{d}
          }
          \entonces \congruencia{a_1 \cdots a_n}{a_b \cdots b_n}{d} \flecha{$a_i = a \y b_i = b$}[$\paratodo i \en \set{1,\dots, n}$] \congruencia{a^n}{b^n}{d}$
\end{itemize}

\textit{\underline{Algoritmo de división:}}\\
% macro local
\newcommand{\condicionResto}[1]{\ub{0 \leq #1 < |d|}{\text{\tiny cumple condición de resto}}}
\begin{itemize}
  \item
        Dados $a, \, d \en \enteros$ con $d \distinto 0$,
        \textit{\underline{existen}} $k$ (cociente),
        $r \text{(resto)} \en \enteros$ tales que:\\
        \[
          \llaves{c}{
            a =  k \cdot d + r,\\
            \text{con } 0\leq r < |d|.
          }
        \]
        Y además estos $k$ y $r$ son \textit{\underline{únicos}}.\\

  \item \textit{Notación: } $r_d(a)$ es el resto de dividir a $a$ entre $d$

  \item $\condicionResto{r} \entonces r = r_d(r)$. Un número que cumple condición de resto, es su resto.

  \item $r_d(a) = 0 \sisolosi d \divideA a \sisolosi \congruencia{a}{0}{d}$

  \item $\congruencia{a}{r_d(a)}{d}$. Tiene mucho sentido.

  \item $\congruencia{a}{r}{d}$ con $\condicionResto{r} \entonces r = r_d(a)$

  \item $\congruencia{r_1}{r_2}{d}$ con $\condicionResto{r_1,r_2} \entonces r_1 = r_2$

  \item $\congruencia{a}{b}{d} \sisolosi r_d(a) = r_d(b)$. Dos números que son congruentes, tienen igual resto.

  \item $r_d(a+b) = r_d(r_d(a) + r_d(b))$ ya que si
        $ \llaves{c}{
            \congruencia{a}{r_d(a)}{d}\\
            \congruencia{b}{r_d(b)}{d}
          } \to \congruencia{a + b}{r_d(a) + r_d(b)}{d}
        $

  \item $r_d(a \cdot b) = r_d(r_d(a) \cdot r_d(b))$ ya que si
        $ \llaves{c}{
            \congruencia{a}{r_d(a)}{d}\\
            \congruencia{b}{r_d(b)}{d}
          } \to \congruencia{a \cdot b}{r_d(a) \cdot r_d(b)}{d}$
\end{itemize}

\textit{\underline{Sistema de numeración: }}
\begin{itemize}
  \item Sea $d \en \naturales, d \geq 2$. Entonces $\paratodo a \en \naturales_0$ se puede
        escribir en la forma
        $$
          a = r_n d^n + r_{n-1} d^{n-1} + \cdots + r_1 d^1 + r_0
        $$
        con $0 \leq r_i < d$ para $0 \leq i \leq n$ con $r_n,\dots, r_0$ son únicos
        en esas condiciones.

  \item \textit{Notación:} $a = (r_n r_{n-1}\cdots r_1 r_0)_d =
          \llave{l}{
            2020 = (2020)_{10} \\
            2020 = (7E4)_{16} \\
            2020 = (31040)_{5}
          }$
  \item $d^n = (1 \ub{0\cdots 0}{n})$

  \item ¿Cuál es el número más grande que puedo escribir usando n cifras en base $d$?\\
        $(\ub{d-1\quad d-1\quad \cdots\quad d-1}{n \text{ cifras}})_d =  \sumatoria{i=0}{n-1}(d-1)d^i = d^n -1$

  \item ¿Cuántos números hay con $\leq n$ cifras?\\
        Hay del $0$ hasta el $d^n -1$, es decir $d^n$.

  \item ¿Cuál es la forma más rápida de calcular $2^{16}$

\end{itemize}
\textit{\underline{Máximo común divisor: }}
%%%%%% Macro local
\def\mcd{(a:b)}
\def\D{\mathcal D}
\def\cz{s\cdot a + t \cdot b}
%%%%%% fin Macro local
\begin{itemize}
  \item Sean $a,b \en \enteros$, \underline{no ambos nulos}. El MCD entre $a$ y $b$ es el mayor de los divisores
        común entre $a$ y $b$ y se nota $\mcd$

  \item $\mcd \en \naturales$ (pues $\mcd \geq 1$) siempre existe.
        $\D com_+(a,b) = \D_+(a) \inter \D_+(b) \distinto \vacio
          \text{ pues } 1 \en \D com_+(a,b)$.
        Se ve también que está acotado por el menor entre $a$ y $b$, pues si
        $d \divideA a \y d \divideA b \entonces d \leq |a| \y d \leq|b|$ y es \underline{único}.

  \item Sean $a$ y $b \en \enteros$, no ambos nulos.\\
        \begin{itemize}
          \item $\mcd = (\pm a : \pm b)$
          \item $\mcd = (b:a)$
          \item $(a:1) = 1$
          \item $(a:0) = |a|,\ \paratodo a \en \enteros -\set{0}$
          \item si $b \divideA a \entonces \mcd = |b|$ si $b \en \enteros - \set{0}$
          \item $\mcd = (a: b+na)$ con $n \en \enteros$
          \item $\mcd = (a: r_a(b))$ con $n \en \enteros$
        \end{itemize}

  \item \textit{Algoritmo de Euclides}:
        Sean $a, b \en \enteros$ con $b \distinto 0$, entonces, $\paratodo k \en \enteros$, se tiene:
        $\mcd = (b:a-kb)$. En particular, como $r_b(a) = a-kb$, con $k$ el cociente (para $b\distinto 0$), se tiene
        $\mcd = (b: r_b(a))$

  \item \textit{Combinacion Entera}:
        Sean $a,b \en \enteros$ no ambos nulos, entonces $\existe s,\ t \en \enteros$ tal que $\mcd = \cz$.
        \begin{itemize}
          \item Todos los divisores comunes entre $a$ y $b$ dividen al $\mcd$. Sean $a,b \en \enteros$ no ambos nulos, $d \en \enteros - \set{0}$. Entonces:
                \[
                  d \divideA a \ytext d \divideA b \sisolosi d \divideA \ub{\mcd}{\cz}
                \].
          \item Sea $c \en \enteros$ entonces $\existe s', t' \en \enteros$ con $c = s'a + t'b \sisolosi \mcd \divideA c$.

          \item Todos los números múltiplos del MCD se escriben como combinación entera de $a$ y $b$.
          \item Si un número es una combinación entera de $a$ y $b$ entonces es un múltiplo del MCD.
          \item Sean $a,\ b \en \enteros$ no ambos nulos, y sea $k \en \naturales$
                \[
                  (ka:kb) = k(a:b)
                \]
        \end{itemize}
  \item \textit{Coprimos: }
        \begin{itemize}
          \item
                Dados $a,b \en \enteros$, no ambos nulos, se dice que son \textit{coprimos} si $\mcd = 1$
                \[
                  \begin{array}{c}
                    a \cop b \sisolosi \mcd = 1 \\
                    \qquad a\cop b \sisolosi \existe s,\ t \en \enteros \text{ tal que } 1 = \cz
                  \end{array}
                \]
          \item
                Sean $a,b \en \enteros$, no ambos nulos. Entonces $\frac{a}{\mcd}\cop \frac{b}{\mcd}$.
          \item Coprimizar es :
                $\llaves{l}{
                    a = \mcd \cdot a'\\
                    b = \mcd \cdot b'
                  }\to a' \ytext b'$ son coprimos.

          \item Sean $a, c, d \en \enteros$ con $c,d$ no nulos. Entonces:
                \[
                  c \divideA a \ytext d \divideA a \ytext c \cop d \red{\sisolosi} c\cdot d \divideA a
                \]
          \item Sean $a, b, d \en \enteros$ con $d \distinto 0$. Entonces:
                \[
                  d \divideA a \cdot b \ytext d \cop a   \entonces d \divideA b
                \]

        \end{itemize}

  \item \textit{\underline{Primos y Factorización:}}
        \begin{itemize}
          \item Sea $p$ primo y sean $a,b \en \enteros$. Entonces:
                \[
                  p \divideA a\cdot b \entonces p \divideA a \o p \divideA b
                \]
          \item \textit{Si $p$ divide a algún producto de números, tiene que dividir a alguno de los factores $\to$}\\
                Sean $a_1,\dots, a_n \en \enteros$:\\
                \begin{center}
                  $
                    \llave{l}{
                      p \divideA a_1 \cdot a_2 \cdots a_n \entonces p \divideA a_i \text{ para algún } i \text{ con } 1 \leq i \leq n.\\
                      p \divideA a^n \entonces p \divideA a.
                    }$
                \end{center}
          \item Si $a \en \enteros$, $p$ primo:\\
                \begin{center}
                  $\llave{l}{
                      (a:p) = 1 \sisolosi p \noDivide a\\
                      (a:p) = p \sisolosi p \divideA a
                    }$
                \end{center}
          \item Sea $n \en \enteros - \set{0},\,
                  n = \ub{s }{\set{-1,1}} \cdot \productoria{i=1}{k} p_i^{\alpha_i} =
                  p_1^{\alpha_1} \cdots p_k^{\alpha_k}$
                su factorización en primos. Entonces todo divisor $m$ positivo de $n$ se escribe como:\\
                \[
                  \llave{c}{
                  \text{Si } m \divideA n \to  m = p_1^{\beta_1} \cdots p_k^{\beta_k}
                  \text{ con } 0 \leq \beta_i \leq \alpha_i,\, \paratodo i\, 1\leq i \leq k\\
                  \text{ y hay } \\
                  (\alpha_1 + 1) \cdot (\alpha_2 + 1)\cdots (\alpha_k + 1) = \productoria{i=1}{k} \alpha_i +1 \\
                  \text{divisores positivos de } n.
                  }\]

          \item Sean $a$ y $b \en \enteros$ no nulos, con
                $
                  \llave{l}{
                  a = \pm p_1^{m_1}\cdots p_r^{m_r} \text{ con } m_1,\cdots, m_r \en \enteros_0\\
                  b = \pm p_1^{n_1}\cdots p_r^{n_r} \text{ con } n_1,\cdots, n_r \en \enteros_0\\
                  \llave{l}{
                  \entonces \mcd = p_1^{min\set{m_1,n_1}}\cdots p_r^{min\set{m_r,n_r}}\\
                  \entonces [a:b] = p_1^{max\set{m_1,n_1}}\cdots p_r^{max\set{m_r,n_r}}
                  }
                  }
                $

          \item Sean $a, d \en\enteros$ con $d \distinto 0$ y sea $n \en \naturales$. Entonces
                $$
                  d \divideA a \sisolosi d^n \divideA a^n.
                $$

          \item Sean $a,b,c \en \enteros$ no nulos:
                \begin{itemize}
                  \item $a \cop b \sisolosi \text{no tienen primos en común}.$
                  \item $\mcd = 1 \ytext (a : c) = 1 \sisolosi (a : bc) = 1$
                  \item $\mcd = 1 \sisolosi (a^m: b^n) = 1 ,\, \paratodo m, n \en \naturales$
                  \item $(a^n:b^n) = (a:b)^n \paratodo n \en \naturales$
                \end{itemize}

          \item Si $a \divideA m \ \y \  b \divideA m$, entonces $[a:b] \divideA m$

          \item $\mcd \cdot [a:b] = |a \cdot b|$
        \end{itemize}

\end{itemize}
