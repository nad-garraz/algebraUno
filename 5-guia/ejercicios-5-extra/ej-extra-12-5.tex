\begin{enunciado}{\ejExtra}
  Determinar todos los primos $p \en \naturales$ que satisfacen que:
  $$
    5p \divideA 6^{p-1} + 10^{p^2} + 119
  $$
\end{enunciado}
Si $ 5p \divideA 6^{p-1} + 10^{p^2} + 119$ entonces:
$$
  p \divideA 6^{p-1} + 10^{p^2} + 119
$$
y ahí puedo sacar info sobre $p$:
$$
  p \divideA 6^{p-1} + 10^{p^2} + 119
  \Sii{def}[\red{!}]
  \congruencia{6^{p-1} + 10^{p^2} + 119}{0}{p}
$$
Las factorizaciones en primos de los números del ejercicio:
$$
  \congruencia{(2\cdot 3)^{p-1} + (2 \cdot 5)^{p^2} + 17 \cdot 7}{0}{p} \llamada1
$$

Si $p = 2$ en $\llamada1$:
$$
  \congruencia{0 + 0 + 1}{0}{2}
$$
$p = 2$, no sirve $\llamada2$.

\bigskip

Si $p = 3$ en $\llamada1$:
$$
  \congruencia{0 + 1 + 2}{0}{3}
$$
\cajaResultado{p = 3}, sí sirve. $\llamada3$

\bigskip

Si $p = 5$ en $\llamada1$:
$$
  \congruencia{1 + 0 -1}{0}{5}
$$
\cajaResultado{p = 5}, sí sirve. $\llamada4$

\bigskip

Si $p = 17$ en $\llamada1$:
$$
  \congruencia{6^0 + 10^1}{0}{17}
$$
$p = 17$, no sirve.$\llamada5$

\bigskip

\bigskip

Hago el caso con $p$ genérico $\llamada1$:
$$
  \congruencia{6^{p-1} + 10^{p^2} + 119}{0}{p}
  \Sii{PTF}
  \congruencia{6^0 + 10^1 + 119}{0}{p}
  \sii
  \congruencia{130}{0}{p}
  \sii
  \congruencia{2 \cdot 5 \cdot 13}{0}{p}
$$
Encuentro de esta forma que el
\cajaResultado{p = 13}, sí sirve. $\llamada6$

\bigskip

\bigskip


De $\llamada2, \llamada3, \llamada4, \llamada5$ y $\llamada6$ sé que el 2 y el 17 no sirven pero 3, 5 y 13 sí.

\bigskip

Todo muy lindo, ahora quiero ver si para $p = 3$:
$$
  5 \cdot 3 \divideA 6^{p-1} + 10^{p^2} + 119
  \Sii{def}
  \congruencia{6^2 + 10^9 + 14}{0}{15}
  \Sii{\red{!!!}}
  \congruencia{6 + 10 + 14}{0}{15} \Tilde
$$
En \red{!!!} cálculo fue el de $\congruencia{10^9}{10}{15}:\quad
  \congruencia{X}{10^9}{15}
  \leftrightsquigarrow
  \llave{l}{
    \congruencia{X}{0}{5} \\
    \congruencia{X}{1}{3}
  }
  \Sii{\href{\chinito}{TCH}}
  \congruencia{X}{10}{15}
$

Ahora para $p = 5$:
$$
  5 \cdot 5 \divideA 6^{p-1} + 10^{p^2} + 119
  \Sii{def}
  \congruencia{6^4 + 10^{25} + 19}{0}{25}
  \Sii{\red{!}}
  \congruencia{21 + 0 + 19}{15 \not\equiv 0 }{25} \quad \text{\red{\faIcon{skull}}}
$$

\bigskip

\text{Para $p = 13$}:
$$
  5 \cdot 13 \divideA 6^{12} + 10^{169} + 119
  \sii
  \congruencia{6^{12} + 10^{169} + 54}{0}{65}
  \Sii{\red{!!!}}
  \congruencia{1 + 10 + 54}{0}{65}
$$
En \red{!!!} como se hizo antes el cálculo fue el de $\congruencia{10^{169}}{10}{65}$:
$$
  \congruencia{X}{10^{169}}{65}
  \leftrightsquigarrow
  \llave{l}{
    \congruencia{X}{0}{5} \\
    \congruencia{X}{10^{169}}{13}
    \Sii{PTF}
    \congruencia{X}{10}{13}
  }
  \Sii{\href{\chinito}{TCH}}
  \congruencia{X}{10}{65}
$$

\bigskip

Se concluye de esta forma que los únicos $p$ primos que cumplen que $5p \divideA 6^{p-1} + 10^{p^2} + 119$ son:
$$
  \cajaResultado{p \en \set{3, 13}}
$$

% Contribuciones
\begin{aportes}
  %% iconos : \github, \instagram, \tiktok, \linkedin
  \item \aporte{\dirRepo}{naD GarRaz \github}
  \item \aporte{\neverGonnaGiveYouUp}{Nacho \youtube}
  \item \aporte{https://github.com/daniTadd}{Dani Tadd \github}
\end{aportes}
