\begin{enunciado}{\ejExtra}
  Hallar todos los $a\en \enteros$ tales que el resto de la división de $15a^{831}$ por
  $77$ es $6$.
\end{enunciado}

Teniendo en cuenta que:
$$
  831 = 3 \cdot 277
$$
Queremos ver para cuál valor de $a$:
$$
  \congruencia{15a^{831}}{6}{77}
$$
Para poder usar PTF, necesitamos que el divisor sea un número primo:
$$
  \congruencia{15a^{831}}{6}{77}
  \leftrightsquigarrow
  \llave{c}{
    \congruencia{a^{831}}{6}{7} \llamada1 \\
    \congruencia{4a^{831}}{6}{11} \llamada2
  }
$$

\textit{Estudio $\llamada1$:}
$$
  \congruencia{a^{831}}{6}{7}
  \sii
  \llave{l}{
    \Sii{$\congruencia{a}{0}{7}$} \congruencia{0}{6}{7} \text{ \red{\faIcon{skull}}} \\
    \Sii{$\noCongruencia{a}{0}{7}$}[PTF]
    \congruencia{a^3}{6}{7}
    \Sii{\red{!!}}[$\llamada3$]
    \llave{l}{
      \congruencia{a}{3}{7} \\
      \congruencia{a}{5}{7} \\
      \congruencia{a}{6}{7}
    }
  }
$$
donde en \red{!!} hacés una hermosa tabla de restos.
\medskip

\textit{Estudio $\llamada2$:}
$$
  \congruencia{4a^{831}}{6}{11}
  \sii
  \llave{l}{
    \Sii{$\congruencia{a}{0}{11}$} \congruencia{0}{6}{11} \text{ \red{\faIcon{skull}}} \\
    \Sii{$\noCongruencia{a}{0}{11}$}[PTF]
    \congruencia{4a^1}{6}{11}
    \Sii{$3 \cop 11$}
    \congruencia{a}{7}{11}
  }
$$

Con los resultados encontrados quedan 3 sistemas, uno por cada ecuación en $\llamada3$:
$$
  \begin{array}{c}
    \llave{l}{
    \congruencia{a}{7}{11} \\
      \congruencia{a}{3}{7}
    }
  \end{array}
  \quad, \quad
  \begin{array}{c}
    \llave{l}{
    \congruencia{a}{7}{11} \\
      \congruencia{a}{5}{7}
    }
  \end{array}
  \ytext
  \begin{array}{c}
    \llave{l}{
    \congruencia{a}{7}{11} \\
      \congruencia{a}{6}{7}
    }
  \end{array}
$$

Tengo solución por \href{\chinito}{TCH}, dado que los divisores son coprimos en todos los sistemas:\par
Desarrollo el sistema con $a\conga7 6$, y los otros, \textit{pajilla}:
$$
  \begin{array}{c}
    \congruencia{a}{7}{11}
    \Sii{def} a = 11 \cdot \blue{k} + 7 \\
    \flecha{reemplazo}                  \\
    \congruencia{11\cdot \blue{k} + 7}{6}{7}
    \sii
    \congruencia{4\cdot \blue{k}}{6}{7}
    \Sii{$2\cop 7$}[\red{!}]
    \congruencia{\blue{k}}{5}{7}
    \Sii{def}
    \blue{k} = 7\cdot \yellow{q} + 5    \\
    \flecha{reemplazo}[en $a$]          \\
    a = 11 \cdot ( 7\cdot \yellow{q} + 5) + 7 = 77 \cdot \yellow{q} + 62
    \Sii{def}
    \cajaResultado{\congruencia{a}{62}{77}}
  \end{array}
$$

Las soluciones de los otros sistemas:
$$
  \begin{array}{rcl}
    a \conga7 3 & \to & \cajaResultado{\congruencia{a}{73}{77}} \\
    a \conga7 5 & \to & \cajaResultado{\congruencia{a}{40}{77}}
  \end{array}
$$

\begin{aportes}
  \item \aporte{https://github.com/nad-garraz/algebraUno}{Nad Garraz \github}
  \item \aporte{https://github.com/daniTadd}{Dani Tadd \github}
\end{aportes}
