\begin{enunciado}{\ejExtra} \fechaEjercicio{final 30/07/2024}
  Determinar todos los primos positivos $p$ que satisfacen:
  $$
    2p \divideA 13^{p^2 + p} + 3 \cdot 2^p + 295^{p-1}.
  $$
\end{enunciado}
Arranco acomodando el enunciado:
$$
  2p \divideA 13^{p^2 + p} + 3 \cdot 2^p + 295^{p-1}
  \Sii{def}
  \congruencia{13^{p^2 + p} + 3 \cdot 2^p + 295^{p-1}}{0}{2p}
$$
Quiero ver para cuales valores de $p$ esa ecuación de congruencia es verdadera. Laburo el sistema equivalente, quebrando:
\parrafoDestacado[\red{\atencion}]{
  Hay que observar que el sistema quebrado funciona bien por el
  \href{\superIdol}{\textit{teorema chino del resto}}. En particular cuando $p = 2$ también
  funciona {\small(¡Mostralo, sino estás haciendo cagadas!)}:
  $$
    2p \divideA a
    \entonces
    p \divideA a \y 2 \divideA a
  $$
  peeeeero, la vuelta no vale en general para $p = 2$.
}
$$
  \congruencia{13^{p^2 + p} + 3 \cdot 2^p + 295^{p-1}}{0}{2p}
  \equivalente
  \llave{l}{
    \congruencia{{1} + 3 \cdot 26p + 295^{p-1} \conga{2}[\red{!}] 0}{0}{2}\\
    \congruencia{13^{p^2 + p} + 3 \cdot 2^p + 295^{p-1}}{0}{p} \llamada1
  }
$$
Hay que estudiar $\llamada1$ para distintos valores de $p$. Se va a usar fuerte el \hyperlink{teoria-5:PTF}{\textit{pequeño teorema de Fermat} \click}.
Factorizando $295 = 5 \cdot 59$ para un $\blue{p} \not\en \set{2, 5, 13, 59}$ así uso PTF:
$$
  \llamada1
  \Sii{PTF}
  \llamada2
  \llave{l}{
    \congruencia{2^{\blue{p}^2 + \blue{p}}}{\green{2}}{\blue{p}}\\
    \congruencia{13^{\blue{p}^2 + \blue{p}}}{\green{13^2}}{\blue{p}}\\
    \congruencia{295^{\blue{p} - 1}}{\green{1}}{\blue{p}}
  }
$$
Donde usé:
{\tiny
$$
  \polyset{vars=p}
  \divPol{p^2 + p}{p-1}
  \quad
  \ytext
  \quad
  \divPol{p - 1}{p-1}
$$
}
Metiendo toda la info calculada en $\llamada2$ en $\llamada1$:
$$
  \congruencia{
    \green{13^2} + 3\cdot \green{2} + \green{1}
    \igual{\red{!}} 2^4 \cdot 11
  }{0}{\blue{p}}
  \sisolosi \blue{p} \en \set{2, 11}
$$
\parrafoDestacado{
  Muchas cuentas. Pará respirá e interpretá. ¿Qué son estos valores encontrados? ¿Listo? ¿Terminó el ejercicio?
}

Las cuentas hechas habían arrancado con cierta restricción en $\blue{p}$, tengo que laburar los valores que propuse
que $\blue{p}$ \ul{no} podía tomar para usar el PTF. Por lo tanto ahora voy a ver que pasa $\llamada1$ con
los valores de $\magenta{p} \en \set{2,5,13,59}$:

\bigskip

\textit{$\magenta{p} = \magenta{2}$:}
$$
  \congruencia{ 13^{\magenta{2}^2 + \magenta{2}} + 3 \cdot 2^{\magenta{2}} + 295^{\magenta{2}-1} }{0}{\magenta{2}}
$$

\textit{$\magenta{p} = \magenta{5}$:}
$$
  \congruencia{ 13^{\magenta{5}^2 + \magenta{5}} + 3 \cdot 2^{\magenta{5}} + 295^{\magenta{5}-1} }{0}{\magenta{5}}
$$

\textit{$\magenta{p} = \magenta{13}$:}
$$
  \noCongruencia{ 13^{\magenta{13}^2 + \magenta{13}} + 3 \cdot 2^{\magenta{13}} + 295^{\magenta{13}-1} \congruente 7}{0}{\magenta{13}} \quad \red{\skull}
$$

\textit{$\magenta{p} = \magenta{59}$:}
$$
  \noCongruencia{
    13^{\magenta{59}^2 + \magenta{59}} + 3 \cdot 2^{\magenta{59}} + 295^{\magenta{59}-1}
    \taa{\red{!!}}\congruente 54
  }{0}{\magenta{59}} \quad \red{\skull}
$$

\bigskip

Los $p$ que cumplen que:
$
  2p \divideA 13^{p^2 + p} + 3 \cdot 2^p + 295^{p-1}
$
son:
$$
  \cajaResultado{
    p \en \set{2, 5, 11}
  }
$$

\begin{aportes}
  \item \aporte{https://www.gitlab.com/tizisf}{Tizi S. F. \gitlab}
  \item \aporte{\dirRepo}{naD GarRaz \github}
\end{aportes}
