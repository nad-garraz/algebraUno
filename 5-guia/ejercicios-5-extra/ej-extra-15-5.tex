\begin{enunciado}{\ejExtra[integrador 16/12/2025]}\fechaEjercicio{integrador 16/12/2025}
  Calcular el resto de dividir por 11 al producto de todos los divisores positivos de $23 \cdot 5^{32}$.
\end{enunciado}

Por suerte el número $23 \cdot 5^{32}$ ya es un producto de primos por lo que es fácil escribir
todos los divisores positivos:
$$
  \llave{rcl}{
    23^{\blue{\alpha}} & \text{con} & 0 \leq \blue{\alpha} \leq 1 \\
    5^{\green{\beta}} & \text{con} & 0 \leq \green{\beta} \leq 32
  }
$$
Si bien no se pidió hay un total de $(1+1) \cdot (32 + 1) = 66$ divisores.

El producto de todos los divisores va a ser algo así:
$$
  23^{\blue{0}} 5^{\green{0}} \cdot
  23^{\blue{0}} 5^{\green{1}} \cdots
  23^{\blue{0}} 5^{\green{31}} \cdot
  23^{\blue{0}} 5^{\green{32}} ~\red{\cdot}~
  23^{\blue{1}} 5^{\green{0}} \cdot
  23^{\blue{1}} 5^{\green{1}} \cdots
  23^{\blue{1}} 5^{\green{31}} \cdot
  23^{\blue{1}} 5^{\green{32}}
  \igual{\red{!!}}
  23^{32} \cdot 5^{1056}
$$
En el \red{!!} es solo multiplicar potencias, nada súper raro, bueh, usé la suma de \textit{Gauss}.

Calculo el resto haciendo un par de cuentas y usando \textit{PTF}. 11 es primo y $11\noDivide 5$:
$$
  \congruencia{23^{32} \cdot 5^{1056}}{1^{32} \cdot 5^{r_{10}(1056)}}{11}
  \sisolosi
  \cajaResultado{
    \congruencia{23^{32} \cdot 5^{1056}}{5}{11}
  }
$$
Por lo tanto:
$$
  \cajaResultado{
    r_{11}(\textit{\tiny el producto de los divisores de} ~ 23\cdot 5^{32}) = 5
  }
$$

\begin{aportes}
  \item \aporte{\dirRepo}{naD GarRaz \github}
\end{aportes}
