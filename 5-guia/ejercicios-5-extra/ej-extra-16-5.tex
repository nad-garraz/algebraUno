\begin{enunciado}{\ejExtra[recuperatorio 21/03/2025]}\fechaEjercicio{recuperatorio 21/03/2025}

  Encuentre todos los divisores de $3^{200}$ que tengan resto 82 en la división por 119.
\end{enunciado}

Tiene pinta de que voy a usar PTF. Con $119 = 7 \cdot 17$.

Los divisores de $3^{200}$ son números de la pinta:
$$
  \pm3^\alpha ~ \text{con} ~ \alpha \en [0,200]
$$
Trato a los valores de positivos y negativos por separado:
$$
  \begin{array}{rcl}
    \congruencia{3^{\alpha}}{82}{119}
     & \taa{$\llamada1$}{\equivalente} &
    \llave{rcl}{
      \congruencia{3^{\alpha}}{5}{7}
     & \Sii{PTF}[7 primo]              &
      \congruencia{3^{r_6(\alpha)}}{5}{7}
    \\
      \congruencia{3^{\alpha}}{14}{17}
     & \Sii{PTF}[17 primo]             &
      \congruencia{3^{r_{16}(\alpha)}}{14}{17}
    }
    \\ \\
    \congruencia{-3^{\alpha}}{82}{119}
    \sii
    \congruencia{3^{\alpha}}{37}{119}
     & \taa{$\llamada2$}{\equivalente} &
    \llave{rcl}{
      \congruencia{3^{\alpha}}{2}{7}
     & \Sii{PTF}[7 primo]              &
      \congruencia{3^{r_6(\alpha)}}{2}{7}
    \\
      \congruencia{3^{\alpha}}{3}{17}
     & \Sii{PTF}[17 primo]             &
      \congruencia{3^{r_{16}(\alpha)}}{3}{17}
    }
  \end{array}
$$
Uso tablas de restos para resolver:
$$
  \begin{array}{|c|c|c|c|c|c|c|}
    \hline
    r_6(\alpha)          & 0 & 1 & 2 & 3 & 4 & 5 \\  \hline
    r_7(3^{r_6(\alpha)}) & 1 & 3 & 2 & 6 & 4 & 5 \\ \hline
  \end{array}
$$
$$
  \begin{array}{|c|c|c|c|c|c|c|c|c|c|c|c|c|c|c|c|c|}
    \hline
    r_{16}(\alpha)             & 0 & 1 & 2 & 3  & 4  & 5 & 6  & 7  & 8  & 9  & 10 & 11 & 12 & 13 & 14 & 15 \\  \hline
    r_{17}(3^{r_{16}(\alpha)}) & 1 & 3 & 9 & 10 & 13 & 5 & 15 & 11 & 16 & 14 & 8  & 7  & 4  & 12 & 2  & 6  \\ \hline
  \end{array}
$$
\parrafoDestacado{
  \textit{ \color{gray!40}
    (¿Cómo harías esto más elegante sin toda esa tabla fea? Mandame un mesaje con esa idea así lo cambiamos, que no me gusta como quedó.)
  }
}

Para poder satisfacer la ecuaciones de congruencias necesito que:
$$
  \flecha{para}[$\llamada1$]
  \llave{l}{
    \congruencia{\alpha}{5}{6}
    \equivalente
    \llave{l}{
      \congruencia{\alpha}{2}{3}\\
      \congruencia{\alpha}{1}{2}\\
    }\\
    \congruencia{\alpha}{9}{16}
    \equivalente
    \llave{l}{
      \congruencia{\alpha}{1}{2}\\
    }
  }
$$

$$
  \flecha{para}[$\llamada2$]
  \llave{l}{
    \congruencia{\alpha}{2}{6}
    \equivalente
    \llave{l}{
      \congruencia{\alpha}{2}{3}\\
      \congruencia{\alpha}{0}{2}
    }\\
    \congruencia{\alpha}{1}{16}
    \equivalente
    \llave{l}{
      \congruencia{\alpha}{1}{2} \quad \red{\skull}
    }
  }
$$

Tengo un sistema compatible solo para $\llamada1$, lo que quiere decir que ningún \textit{divisor negativo} de $3^{200}$ satisface lo que se pide.
Resuelvo para encontrar valores de $\alpha$ que cumplan $\llamada1$:
$$
  \llave{l}{
    \congruencia{\alpha}{2}{3}\\
    \congruencia{\alpha}{9}{16}
  }
$$
Por los divisores con coprimos, por \href{\chinito}{TCH} habemos solución, la cual no quiero desarrollar todo el procedimiento porque \textit{pajilla}, pero
es $\congruencia{\alpha}{41}{48}$.

\bigskip

Los divisores de $3^{200}$ que cumplen que $r_{119}(3^{200}) = 82$:
$$
  \cajaResultado{
    \set{
      3^{41};\;
      3^{89};\;
      3^{137};\;
      3^{185}
    }
  }
$$

\begin{aportes}
  \item \aporte{\dirRepo}{naD GarRaz \github}
\end{aportes}
