\begin{enunciado}{\ejExtra}
  Sea $a \en \enteros$ tal que $(a^{197} - 26 : 15) = 1$. Hallar los posibles valores de
  $(a^{97} - 36 : 135)$
\end{enunciado}

Como $135 = 3^3 \cdot 5$, los posibles valores sin tener en cuenta la restricción que puede tomar el MCD serían:
$$
  \llamada1 \text{MCD} \en \set{1, 3, 5, 9, 15, 27, 45, 135}
$$.

Por otro lado tengo que:
$$
  15 = 3 \cdot 5 \ytext (a^{197} - 26 : 15) = 1,
$$
lo cual se cumple para:
$$
  \llamada2
  \llave{l}{
    \noCongruencia{a^{197} - 26}{0}{5}
    \sii
    \noCongruencia{a^{197}}{1}{5} \\
    \noCongruencia{a^{197} - 26}{0}{3}
    \sii
    \noCongruencia{a^{197}}{2}{3}
  }
$$

$$
  \noCongruencia{a^{197}}{1}{5}
  \Sii{5 es}[primo]
  \llave{l}{
    \Sii{$\noCongruencia{a}{0}{5}$}[PTF]
    \noCongruencia{a}{1}{5} \\
    \Sii{$\congruencia{a}{0}{5}$}
    \noCongruencia{0}{1}{5}
  }
$$
$$
  \noCongruencia{a^{197}}{2}{3}
  \Sii{3 es}[primo]
  \llave{l}{
    \Sii{$\noCongruencia{a}{0}{3}$}[PTF]
    \noCongruencia{a}{2}{3} \\
    %---
    \Sii{$\congruencia{a}{0}{3}$}
    \noCongruencia{0}{2}{3}
  }
$$
Por lo tanto tengo que los \blue{valores permitido de $a$} para calcular los MCD son:
$$
  \begin{array}{c}
    \noCongruencia{a^{197}}{1}{5} \sii \noCongruencia{a}{1}{5}
    \llave{l}{
    \blue{\congruencia{a}{0}{5}} \\
    \blue{\congruencia{a}{2}{5}} \\
    \blue{\congruencia{a}{3}{5}} \\
      \blue{\congruencia{a}{4}{5}}
    }
    \vspace{5pt}
    \\
    \noCongruencia{a^{197}}{2}{3} \sii \noCongruencia{a}{2}{3}
    \llave{l}{
    \blue{\congruencia{a}{0}{3}} \\
    \blue{\congruencia{a}{1}{3}} \\
    }
  \end{array}
$$
\bigskip
\textit{Divisibilidad por 3. ¿Es la expresión $a^{97} - 36$ divisible por 3 para alguno de los valores de $a$ permitidos?}:
$$
  \congruencia{a^{97} - 36 }{0}{3}
  \sii
  \congruencia{a^{97}}{0}{3}
  \Sii{3 es}[primo]
  \llave{l}{
    \Sii{\purple{$\noCongruencia{a}{0}{3}$}}[PTF]
    \congruencia{a}{0}{3} \quad \text{incompatible con} \quad \purple{\noCongruencia{a}{0}{3}} \\
    \Sii{\purple{$\congruencia{a}{0}{3}$}}
    \congruencia{0}{0}{3} \quad \text{compatible con} \quad \purple{\congruencia{a}{0}{3}}
    \ytext
    \llaves{l}{
      \blue{\congruencia{a}{0}{3}}\\
      \blue{\congruencia{a}{0}{5}}
    }
  }
$$

Se concluye que la expresión $a^{97} - 36$ es divisible por 3 para alguno de los valores de $a$ permitidos.\par \bigskip

\textit{Divisibilidad por 5. ¿Es la expresión $a^{97} - 36$ divisible por 5 para alguno de los valores de $a$ permitidos?}:
$$
  \congruencia{a^{97} - 36 }{0}{5}
  \sii
  \congruencia{a^{97}}{1}{5}
  \Sii{5 es}[primo]
  \llave{l}{
    \Sii{\purple{$\noCongruencia{a}{0}{5}$}}[PTF]
    \congruencia{a}{1}{5}
    \quad \text{incompatible con } \blue{\noCongruencia{a}{1}{5}} \\
    \Sii{\purple{$\congruencia{a}{0}{5}$}}
    \congruencia{0}{1}{5} \quad \text{incompatible}
  }
$$
Se concluye que la expresión $a^{97} - 36$ \ul{no es divisible por 5 para los valores de $a$ permitidos}.
Esto reduce la cantidad de \textit{posibles} MCD a:
$$
  \text{MCD} \en \set{1,3,9,27}
$$

\textit{\underline{Divisibilidad por 9}}.

Teniendo en cuenta que los posibles MCD, son todos potencias de 3 y que $\congruencia{a}{0}{3}$
\textit{¿Es la expresión $a^{97} - 36$ divisible por 9 para alguno de los valores de $a$ permitidos?}:
$$
  \congruencia{a^{97} - 36 }{0}{9}
  \sii
  \congruencia{a^{97}}{0}{9}
  \Sii{$\blue{\congruencia{a}{0}{3}} \entonces \congruencia{a}{0}{9}$}
  \congruencia{0}{0}{9}
$$

Se concluye que la expresión $a^{97} - 36$ es divisible por 9 para $\congruencia{a}{0}{9}$ los cuales son valores que cumplen
$\congruencia{a}{0}{3}$. De esa manera \ul{descarto que el 3 pueda ser MCD}.

\bigskip

\textit{\underline{Divisibilidad por 27}.}

Teniendo en cuenta que hasta el momento que el 9 es el MCD.
\textit{¿Es la expresión $a^{97} - 36$ divisible por 27 para alguno de los valores de $a$ permitidos?}:
$$
  \congruencia{a^{97} - 36}{0}{27}
  \sii
  \congruencia{a^{97}}{9}{27}
  \sii
  \llave{l}{
    \Sii{$\blue{\congruencia{a}{0}{3}}$}
    \congruencia{0}{9}{27} \quad \text{noup!} \\
    \Sii{$\blue{\congruencia{a}{1}{3}}$}
    \congruencia{1}{9}{27} \quad \text{noup!}
  }
$$

Se concluye que la expresión $a^{97} - 36$ no es divisible por 27 para ninguno de los valores de $a$ permitidos.\par \medskip

\bigskip

Hasta el momento obtuvimos que $\set{9}$ es el único MCD posible. Falta analizar el caso del 1:

Encuentro que si:
$$
  a = 4
  \entonces
  \congruencia{4^{97} - 36}{0}{3}
  \sii
  \congruencia{1}{0}{3}
$$
Entonces $a = 4$ es un valor de $a$ para el cual el MCD no es divisible por 3, por lo que:
$$
  (4^{97} -36 : 135) = 1.
$$
De esta formar $a =4$ es un valor permitido por $\blue{\congruencia{a}{4}{5} }$ y $\blue{\congruencia{a}{1}{3} }$.
Concluyendo así que el 1 es \underline{otro} posible valor para el MCD.\par
Posibles valores:
$$
  \cajaResultado{
    \text{MCD} \en \set{1, 9}
  }
$$

% Contribuciones
\begin{aportes}
  \item \aporte{\dirRepo}{naD GarRaz \github}
  \item \aporte{https://github.com/JowinTeran}{Ale Teran \github}
\end{aportes}
