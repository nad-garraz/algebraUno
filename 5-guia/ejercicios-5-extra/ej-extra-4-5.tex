\begin{enunciado}{\ejExtra}
  Determinar para cada $n \en \naturales$ el resto de dividir a $8^{3^n-2}$ por 20.
\end{enunciado}

Quiero encontrar $r_{20}(8^{3^n-2})$ entonces analizo congruencia:
$$
  \congruencia{8^{3^n-2}}{X}{20}
  \equivalente
  \llave{l}{
    \congruencia{8^{3^n-2}}{3^{3^n-2}}{5} \llamada1 \\
    \congruencia{8^{3^n-2}}{0}{4} \to \paratodo n \en \naturales
  }
$$

Laburo con $\llamada1$, 5 es primo y $5 \noDivide 3$, puedo usar PTF:
$$
  \congruencia{8^{3^n-2}}{ 3^{3^n-2} \congruente 3^{r_4(3^n-2)}  }{5} \\
$$
A ver esta última expresión:
$$
  \blue{X} = r_4(3^n - 2)
  \Sii{def}
  \congruencia{\blue{X}}{3^n-2}{4}
  \sii
  \congruencia{\blue{X}}{(-1)^n + 2}{4}
  \Sii{\red{!}}
  \llave{rl}{
    \congruencia{\blue{X}}{3}{4} & \text{si $n$ par}    \\
    \congruencia{\blue{X}}{1}{4} & \text{si $n$ impar}
  }
$$
Volviendo a $\llamada1$ con los resultados calculados de $\blue{X}$:
$$
  \llamada3
  \llave{ll}{
    \congruencia{8^{3^n-2}}{0}{4} & \paratodo n \en \naturales              \\
    \congruencia{8^{3^n-2}}{2}{5} & \paratodo n \en \naturales  \text{ par}
  }
  \quad \ytext \quad
  \llamada4
  \llave{ll}{
    \congruencia{8^{3^n-2}}{0}{4} & \paratodo n \en \naturales               \\
    \congruencia{8^{3^n-2}}{3}{5} & \paratodo n \en \naturales \text{ impar}
  }
$$
Lo que resta por hacer es resolver los sistemas. Empiezo por el sistema con $n$ par $\llamada3$:
$$
  \congruencia{n}{0}{2}
  \entonces
  \llave{l}{
    \congruencia{8^{3^n-2}}{0}{4}
    \Sii{def}
    8^{3^n-2} = 4\green{j}
    \Sii{reemplazo}
    \congruencia{4\green{j}}{2}{5}
    \sii
    \congruencia{\green{j}}{3}{5}
    \Sii{def}
    \green{j} = \green{5k + 3}\\
    \entonces
    8^{3^n-2} = 4\green{(5k+3)}
    \Sii{def}
    \cajaResultado{
      \congruencia{8^{3^n-2}}{12}{20}
      \sii
      \congruencia{n}{0}{2}
    }.
  }
$$
Con $n$ impar $\llamada4$:
$$
  \congruencia{n}{1}{2}
  \entonces
  \llave{l}{
    \congruencia{8^{3^n-2}}{0}{4}
    \Sii{def}
    8^{3^n-2} = 4\green{j}
    \Sii{reemplazo}
    \congruencia{4\green{j}}{3}{5}
    \sii
    \congruencia{\green{j}}{2}{5}
    \Sii{def}
    \green{j} = \green{5k + 2}\\
    \entonces
    8^{3^n-2} = 4\green{(5k+2)}
    \sii
    \cajaResultado{
      \congruencia{8^{3^n-2}}{8}{20}
      \sii
      \congruencia{n}{1}{2}
    }.
  }
$$

Se concluye que:
$$
  \cajaResultado{
    r_{20}(8^{3^n-2}) = 12
    \paratodo n \en \naturales \text{ par }
    \ytext
    r_{20}(8^{3^n-2}) = 8
    \paratodo n \en \naturales \text{ impar }
  }
$$

\begin{aportes}
  \item \aporte{\dirRepo}{naD GarRaz \github}
\end{aportes}
