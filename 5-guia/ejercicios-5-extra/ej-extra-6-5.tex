\begin{enunciado}{\ejExtra}
  Hallar el resto de la división de $12^{2^n}$ por 7 para cada $n \en \naturales$
\end{enunciado}

Arrancamos con un descontracturante masajeo del enunciado:
$$
  \congruencia{12^{2^n}}{5^{2^n}}{7}
$$
Busco entonces:
$$
  r_7(12^{2^n}) = r_7(5^{2^n})
$$

Ataco con PTF, 7 es primo y $7 \noDivide 5$:
$$
  r_7(5^{2^n}) = \orange{X}
  \Sii{def}
  \congruencia{\orange{X}}{5^{2^n}}{7}
  \Sii{PTF}
  \congruencia{\orange{X}}{5^{r_6(2^n)}}{7} \llamada1
$$
Estudio $r_6(2^n)$:
$$
  r_6(2^n) = \blue{Y}
  \Sii{def}
  \congruencia{2^n}{\blue{Y}}{6}
  \taa{\red{!}}\equivalente
  \llave{l}{
      \congruencia{2^n}{\blue{Y}}{3}
    \Sii{\red{!!}}
    \llave{rl}{
      \congruencia{\blue{Y}}{2}{3} &\text{ si $n$ impar}\\
      \congruencia{\blue{Y}}{1}{3} &\text{ si $n$ par}
    }
    \\
    \congruencia{2^n \conga2 0}{\blue{Y}}{2}
  }
$$
En \red{!!} pensalo como $2^{2k + 1} = 4^k \cdot 2 \conga3 2$.
Es así que quedan dos sistemas:
$$
  \text{para $n$ impar }
  \llave{l}{
    \congruencia{\blue{Y}}{2}{3}\\
    \congruencia{\blue{Y}}{0}{2}
  }
  \Sii{\href{\chinito}{TCR}}
  \congruencia{\blue{Y}}{2}{6}
  \quad
  \ytext
  \quad
  \text{para $n$ par }
  \llave{l}{
    \congruencia{\blue{Y}}{1}{3}\\
    \congruencia{\blue{Y}}{0}{2}
  }
  \Sii{\href{\chinito}{TCR}}
  \congruencia{\blue{Y}}{4}{6}
$$
Volviendo a $\llamada1$ sé que los posibles valores que puede
tomar el exponente $\blue{Y} = r_6(2^n)$ son 2 o 4. Es decir que:
$$
  \llave{rl}{
    \congruencia{\orange{X}}{5^2 \conga7 4}{7} & \text{ si $n$ impar}  \\
    \congruencia{\orange{X}}{5^4 \conga7 2}{7} & \text{ si $n$ par}  \\
  }
$$
Finalmente:
$$
  \cajaResultado  {
    r_7(12^{2^n})  =
    \llave{ll}{
      2 & \text{ si } n \text{ par}                        \\
      4 & \text{ si } n \text{ impar}
    }
  }
$$

% Contribuciones
\begin{aportes}
  \item \aporte{\dirRepo}{naD GarRaz \github}
  \item \aporte{https://github.com/daniTadd}{Dani Tadd \github}
  \item \aporte{https://github.com/nick052}{Nico Alegre \github}
  \item \aporte{https://github.com/RamaEche}{Ramiro E. \github}
\end{aportes}

