\newcommand{\congPTF}[1]{\taa{(#1)}{\text{PTF}}{\congruente}}

\begin{enunciado}{\ejExtra}
  Hallar todos los primos $p \en \naturales$ tales que
  $$
    \congruencia{3^{p^2 + 3}}{-84}{p} \ytext \congruencia{(7p + 8)^{2024}}{4}{p}.
  $$
\end{enunciado}

A lo largo del ejercicio se va a usar fuerte el colorario del pequeño teorema de Fermat.

$$
  \text{si $p$ primo y $p\noDivide a$, con $a \en \enteros$} \entonces \congruencia{a^n}{a^{r_{p-1}}}{p}
$$

Arrancando:
$$
  \congruencia{3^{p^2 + 3}}{-84}{p}
$$
Estudio los dos casos. Uno cuando $p \noDivide 3$ y $p \divideA 3$:

\medskip

\textit{Caso $p \noDivide 3$ entonces puedo usar el PTF para simplificar un poco}:
$$
  \congruencia{3^{p^2 + 3}}{3^{r_{\magenta{(p-1)}}(p^2 + 3)}}{p}
  \Sii{$\llamada1$}
  \congruencia{3^{p^2 + 3}}{81}{p}
$$
Calculo ese resto con división de polinomios $\llamada1$:
$$
  \polyset{vars=p}
  \divPol{p^2 + 3}{p - 1}
$$
El problema ahora queda así:
$$
  \congruencia{3^{p^2 + 3} \conga{p} 81}{-84}{p}
  \Sii{\red{!}}
  \congruencia{\ub{165}{5\cdot3\cdot11}}{0}{p}
  \Sii{\red{!!}}[$p \noDivide 3$]
  \llave{c}{
    p = 5\\
    \otext \\
    p = 11
  }
$$

\bigskip

\textit{Caso $p \divideA 3$}. Pero como $p$ es primo, entonces $\purple{p} = \purple{3}$:
$$
  \congruencia{3^{p^2 + 3}}{-84}{\purple{3}}
  \sii
  \congruencia{0}{-84}{\purple{3}}
  \sii
  p = 3
$$

\bigskip

Tengo entonces 3 \textit{posibles} valores para $p$:
$$
  p \en \set{ 3,5,11}.
$$

Ahora viene el problema de ver si estos 3 valores verifican la segunda condición del enunciado:
$$
  \congruencia{(7\cdot p + 8)^{2024}}{4}{p}
$$

\textit{Caso $p = \purple{3}$}:
$$
  \congruencia{(7\cdot \purple{3} + 8)^{2024}}{4}{\purple{3}}
  \Sii{PTF}[\red{!!!}]
  \congruencia{2^{\red{0}}}{4}{\purple{3}}
  \sii
  \congruencia{1}{1}{\purple{3}}
$$
De donde se rescata que cuando
$\cajaResultado{
    p = 3,
  }$
está todo bien \faIcon[regular]{smile}.

\medskip

\textit{Caso $p = \green{5}$}:
$$
  \congruencia{(7\cdot \green{5} + 8)^{2024}}{4}{\green{5}}
  \Sii{PTF}[\red{!!!}]
  \congruencia{3^{\red{0}}}{4}{\green{5}}
  \Sii{\red{\faIcon{skull}}}
  \congruencia{1}{4}{\green{5}}
$$
De donde se rescata que cuando $ p = \green{5}$,
está todo mal \faIcon[regular]{sad-cry}.

\medskip

\textit{Caso $p = \yellow{11}$}:
$$
  \congruencia{(7\cdot \yellow{11} + 8)^{2024}}{4}{\yellow{11}}
  \Sii{PTF}[\red{!!!}]
  \congruencia{8^{\red{4}}}{4}{\yellow{11}}
  \sii
  \congruencia{4096}{4}{\yellow{11}}
  \Sii{\yellow{\faIcon{calculator}}}
  \congruencia{4}{4}{\yellow{11}}
$$
De donde se rescata que cuando
$\cajaResultado{
    p = 11,
  }$
está todo bien también \faIcon[regular]{smile}.

Por lo tanto los valores de $p$ que cumplen lo pedido son:
$$
  \cajaResultado{
    p = 3
    \ytext
    p = 11
  }
$$

% Contribuciones
\begin{aportes}
  \item \aporte{\dirRepo}{naD GarRaz \github}
\end{aportes}
