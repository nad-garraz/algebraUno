\begin{enunciado}{\ejExtra}

  Un coleccionista de obras de arte compró un lote compuesto por pinturas y dibujos.
  Cada pintura le costó 649 dólares y cada dibujo 132 dólares. Cuando el coleccionista llega
  a su casa no recuerda si gastó 9779 o 9780 dólares. Deducir cuánto le costó el lote y
  cuántas pinturas y dibujos compró.

\end{enunciado}

Del enunciado se deduce que el coleccionista no sabe si gastó:
$$
  \llave{c}{
    649P + 132D = 9779 \\
    \otext             \\
    649P + 132D = 9780 \\
  }
$$

Dos ecuaciones diofánticas que no pueden estar bien a la vez, porque el tipo gastó o 9779 o bien 9780,
seguramente alguna no tenga solución. \textit{Let's see:}

\medskip

El $(\ub{649}{11 \cdot 59}:\ub{132}{2^2 \cdot 3 \cdot 11}) = 11$
tiene que dividir al número independiente. En este caso $11 \noDivide 9780$ y $11 \divideA 9779$,
así que gastó un total de 9779 dólares.

\medskip

Lo que resta hacer es resolver la ecuación teniendo en cuenta que estamos trabajando con variables
que modelan algo físico por lo que $P \geq 0$ y $D \geq 0\llamada1$.

$$
  649P + 132D = 9779 \Sii{comprimizar} 59P + 12D = 889,
$$

Para buscar la solución particular uso a \textit{Euclides}, dado que entre 2 números coprimos
siempre podemos escribir al número una como una combinación entera.

$$
  \llave{l}{
    59 = 4 \cdot 12 + 11 \\
    12 = 1 \cdot 11 + 1
  }
  \to
  1 = \green{12} - 1  \cdot \ub{11}{\blue{59} - 4 \cdot \green{12}} =
  (-1) \cdot \blue{59} + 5\cdot \green{12}.
$$
Por lo que se obtiene que:
$$
  1 = (-1) \cdot 59 + 5 \cdot 12
  \flecha{$\times 889$}
  \ub{889 = (-889) \cdot 59 + 4445 \cdot 12}{\textit{Combineta entera buscada} \Tilde}
  \flecha{solución}[particular] (P,D)_\text{part} = (-889,4445).
$$

La solución del homogéneo sale fácil. Sumo las soluciones y obtengo la solución general:
$$
  (P,D)_{\blue{k}} = \blue{k} \cdot (12,-59) + (-889,4445)\quad \text{con } \blue{k} \en \enteros.
$$
\textit{Observación totalmente innecesaria, pero está buena:}
Esa ecuación es una recta común y corriente. Si quiero puedo ahora encontrar algún punto más bonito,
para expresarla distinto, por ejemplo si $\blue{k} = 75 \entonces (P,D)_{\text{part}} = (11, 20)$,
lo cual me permite reescribir a la solución general como:
$$
  (P,D)_{\blue{h}} = \blue{h} \cdot (12,-59) + (11,20)\quad \text{con } \blue{h} \en \enteros.
$$
\textit{Fin de observación totalmente innecesaria, pero está buena.}

\medskip

La solución tiene que cumplir $\llamada1$ :
$$
  \llave{l}{
    P = 12 \blue{h} + 11 \geq 0
    \sisolosi
    \blue{h} \geq -\frac{11}{12}
    \Sii{$\blue{h} \en \enteros$}
    \blue{h} \geq 0 \\
    D = -59 \blue{h} + 20 \geq 0
    \sisolosi
    \blue{h} \leq \frac{20}{59}
    \Sii{$\blue{h} \en \enteros$}
    \blue{h} \leq 0 \\
  }
  \sii \blue{h} = 0, \text{  Entonces: } (P,D) = (11,20) \Tilde
$$

El coleccionista compró \textit{once} pinturas y \textit{veinte} dibujos.

% Contribuciones
\begin{aportes}
  \item \aporte{https://github.com/nad-garraz}{Nad Garraz \github}
\end{aportes}
