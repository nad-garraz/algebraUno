\begin{enunciado}{\ejercicio}
	Hallar, cuando existan, todos los enteros $a$ que satisfacen simultáneamente:
	\begin{enumerate}[label=\roman*)]
		\begin{multicols}{3}
			\item\label{ej11-itemi}
			$
				\llave{l}{
					\congruencia{3a}{4}{5} \\
					\congruencia{5a}{4}{6} \\
					\congruencia{6a}{2}{7}
				}
			$

			\item\label{ej11-itemii}
			$
				\llave{l}{
					\congruencia{3a}{1}{10} \\
					\congruencia{5a}{3}{6} \\
					\congruencia{9a}{1}{14}
				}
			$

			\item\label{ej11-itemiii}
			$
				\llave{l}{
					\congruencia{15a}{10}{35} \\
					\congruencia{21a}{15}{8} \\
					\congruencia{18a}{24}{30}
				}
			$
		\end{multicols}
	\end{enumerate}
\end{enunciado}

Ejercicio para practica resolución de sistemas de ecuaciones de congruencia.
Acomodar, coprimizar, \href{\chinito}{Teorema Chino del resto} y resolver.

\bigskip

\begin{enumerate}[label=\roman*)]
	\item
	      Este no tiene mucha rosca, no hay nada que coprimizar, los divisores son coprimos 2 a 2. Debería ser cuestión de simplificar el
	      sistema y luego resolver despejando:
	      $$
		      \llave{l}{
			      \congruencia{3a}{4}{5} \\
			      \congruencia{5a}{4}{6} \\
			      \congruencia{6a}{2}{7}
		      }
		      \Sii{\red{!}}
		      \llave{l}{
			      \congruencia{a}{3}{5} \llamada1 \\
			      \congruencia{a}{2}{6} \llamada2\\
			      \congruencia{a}{5}{7} \llamada3
		      }
	      $$

	      Todos los divisores son coprimos 2 a 2, así que por el \href{\chinito}{THC} debería poder encontrar una solución:

	      $$
		      \llamada1
		      a = 5\magenta{k} + 3
		      \Entonces{meto en}[$\llamada2$]
		      \congruencia{5\magenta{k}+3 \conga6 -\magenta{k}+3}{2}{6}
		      \sii
		      \congruencia{\magenta{k}}{1}{6}
		      \Entonces{en}[$\llamada1$]
		      a = 5(6\green{j} + 1) + 3 = 30\green{j} + 8
	      $$
	      Hasta el momento:
	      $$
		      a = 30\green{j} + 8
		      \Entonces{meto en}[$\llamada3$]
		      \congruencia{30\green{j} + 8}{5}{7}
		      \sii
		      \congruencia{\green{j}}{2}{7}
		      \entonces
		      a = 30 \cdot (7\yellow{h} + 2) + 8 = 210\yellow{h} + 68
	      $$
	      Por lo tanto la solución al sistema \ref{ej11-itemi}:
	      $$
		      \cajaResultado{
			      \congruencia{a}{68}{210}
		      }
	      $$

	\item
	      Acá los divisores \underline{no} son coprimos, habría que \textit{quebrar} y estudiar la \textit{compatibilidad} de las ecuaciones.

	      \medskip

	      Primero simplifico un poco el sistema:
	      $$
		      \llave{l}{
			      \congruencia{3a}{1}{10} \\
			      \congruencia{5a}{3}{6} \\
			      \congruencia{9a}{1}{14}
		      }
		      \Sii{\red{!}}
		      \llave{lcl}{
			      \congruencia{a}{7}{10}
			      &\equivalente&
			      \llave{l}{
				      \congruencia{a}{1}{2} \\
				      \congruencia{a}{2}{5}
			      }
			      \\
			      \\
			      \congruencia{a}{3}{6}
			      &\equivalente&
			      \llave{l}{
				      \congruencia{a}{1}{2} \\
				      \congruencia{a}{0}{3}
			      }
			      \\
			      \\
			      \congruencia{a}{11}{14}
			      &\equivalente&
			      \llave{l}{
				      \congruencia{a}{1}{2} \\
				      \congruencia{a}{4}{7}
			      }
		      }
	      $$
	      La papa está en el divisor 2, pero como aparece con el mismo resto en todas las ecuaciones no tenemos problemas de \textit{compatibilidad}.
	      $$
		      \llave{l}{
			      \congruencia{3a}{1}{10} \\
			      \congruencia{5a}{3}{6} \\
			      \congruencia{9a}{1}{14}
		      }
                      \taa{\red{!}}\equivalente
		      \llave{l}{
			      \congruencia{a}{2}{5}   \llamada1\\
			      \congruencia{a}{0}{3}   \llamada2\\
			      \congruencia{a}{11}{14} \llamada3
		      }
	      $$
	      Tengo todos los divisores coprimos. Resolver este último sistema me va a dar las soluciones del problema original.

	      Ahora hay que resolver:
	      $$
		      \llamada1
		      a = 5\magenta{k} + 2
		      \Entonces{meto en}[$\llamada2$]
		      \congruencia{5 \magenta{k} + 2 \conga3 - \magenta{k} + 2}{0}{3}
		      \sii
		      \congruencia{\magenta{k}}{2}{3}
		      \Entonces{en}[$\llamada1$]
		      a = 5(3\green{j} + 2) + 2 = 15\green{j} + 12
	      $$
	      Hasta el momento:
	      $$
		      a = 15\green{j} + 12
		      \Entonces{meto en}[$\llamada3$]
		      \congruencia{15 \green{j} + 12}{11}{14}
		      \sii
		      \congruencia{\green{j}}{13}{14}
		      \entonces
		      a = 15 \cdot (14\yellow{h} + 13) + 12 = 210 \yellow{h} + 207
	      $$
	      Por lo tanto la solución al sistema \ref{ej11-itemii}:
	      $$
		      \cajaResultado{
			      \congruencia{a}{207}{210}
		      }
	      $$

	\item
	      Acá los divisores \underline{no} son coprimos, parecido al anterior, pero no tanto.

	      \medskip

	      Primero simplifico un poco el sistema, siempre coprimizo si se puede:
	      $$
		      \llave{l}{
			      \congruencia{15a}{10}{35} \\
			      \congruencia{21a}{15}{8} \\
			      \congruencia{18a}{24}{30}
		      }
		      \Sii{\red{!!}}[coprimizar]
		      \llave{l}{
			      \congruencia{3a}{2}{7} \\
			      \congruencia{5a}{7}{8} \\
			      \congruencia{3a}{4}{5}
		      }
		      \sii
		      \llave{l}{
			      \congruencia{a}{3}{7}     \llamada1 \\
			      \congruencia{a}{3}{8}     \llamada2 \\
			      \congruencia{a}{3}{5}     \llamada3
		      }
	      $$

	      Tengo todos los divisores coprimos. Resolver este último sistema me va a dar las soluciones del problema original.

	      Ahora hay que resolver:
	      $$
		      \llamada1
		      a = 7\magenta{k} + 3
		      \Entonces{meto en}[$\llamada2$]
		      \congruencia{7\magenta{k} + 3}{3}{8}
		      \sii
		      \congruencia{\magenta{k}}{0}{8}
		      \Entonces{en}[$\llamada1$]
		      a = 7(8\green{j} + 0) + 3 = 56\green{j} + 3
	      $$
	      Hasta el momento:
	      $$
		      a = 56\green{j} + 3
		      \Entonces{meto en}[$\llamada3$]
		      \congruencia{56 \green{j} + 3}{3}{5}
		      \sii
		      \congruencia{\green{j}}{0}{5}
		      \entonces
		      a = 56 \cdot (5\yellow{h} + 0) + 3 = 280 \yellow{h} + 3
	      $$
	      Por lo tanto la solución al sistema \ref{ej11-itemiii}:
	      $$
		      \cajaResultado{
			      \congruencia{a}{3}{280}
		      }
	      $$
\end{enumerate}

\begin{aportes}
	\item \aporte{\dirRepo}{naD GarRaz \github}
\end{aportes}
