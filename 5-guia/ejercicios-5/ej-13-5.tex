\begin{enunciado}{\ejercicio}
  En un depósito se almacenan latas de gaseosa. El viernes por la noche, un empleado realizó un control
  de inventario y observó que:
  \begin{itemize}
    \item Al poner las latas de cajas de 12 unidades sobraban 4.
    \item Al poner las latas de cajas de 63 unidades sobraban 43.
    \item Había por lo menos 12.600 latas y no más de 13.000, pero no tomó nota de la cantidad eXacta.
  \end{itemize}
  ¿Cuántas latas de gaseosa había en el depósito el viernes por la noche?
\end{enunciado}
Ejercicio en el que hay que armar el sistemita de ecuaciones para poder resolver.

\medskip

Voy a ponerle el poco original nombre \underline{$X$} al total de latas. Del enunciado salen las condiciones:
$$
  \llave{l}{
    \congruencia{X}{4}{12} \\
    \congruencia{X}{43}{63}\\
    12.600 \leq X \leq 13.000 \llamada1
  }
$$

Los divisores no son coprimos habrá que quebrar:
$$
  \llave{l}{
    \congruencia{X}{4}{12}
    \equivalente
    \llave{l}{
      \congruencia{X}{1}{\purple3} \\
      \congruencia{X}{0}{4}
    }\\
    \congruencia{X}{43}{63}
    \equivalente
    \llave{l}{
      \congruencia{X}{1}{\purple{3}} \\
      \congruencia{X}{1}{7}
    }
  }
$$
El sistema es compatible. Me quedo con la equación que tiene el divisor potencia de 3 más grande.
De no hacerlo obtendría soluciones de más.

Resuelvo el siguiente sistema:
$$
  \llave{l}{
    \congruencia{X}{4}{12}\\
    \congruencia{X}{43}{63}
  }
  \equivalente
  \llave{l}{
    \congruencia{X}{0}{4}\\
    \congruencia{X}{43}{63}
  }
$$
Este sistema tiene divisores coprimos y por el \href{\chinito}{TCHR} puedo encontrar una solución:
$$
  \begin{array}{c}
    X = 4 \blue{k}                        \\
    \congruencia{4k}{43}{63}
    \Sii{\red{!}}
    \congruencia{k}{58}{63}
    \Sii{def} \blue{k} = 63\green{j} + 58 \\
    X =
    4 (63\green{j} + 58) =
    252\green{j} + 232
  \end{array}
$$

Ahora hay que usar la condición $\llamada1$:
$$
  12.600 \leq 252\green{j} + 232 \leq 13.000
  \sii
  49.07 \dots \leq \green{j} \leq 50.\bar6
  \sii \green{j} = \green{50}
$$

Se puede concluir entonces que la cantidad de latas es:
$$
  \cajaResultado{
    X = 252 \cdot \green{50} + 232 = 12832
  }
$$

\begin{aportes}
  \item \aporte{\dirRepo}{naD GarRaz \github}
\end{aportes}
