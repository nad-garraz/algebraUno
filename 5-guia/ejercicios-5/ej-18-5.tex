\begin{enunciado}{\ejercicio}
  Sea $a \en \enteros$ coprimo con 561. Probar que $\congruencia{a^{560}}{1}{561}$
\end{enunciado}

Partiendo de la hipótesis:
$$
  (a:561)=1
  \sisolosi
  (a:3\cdot11\cdot17)=1
  \entonces
  \llave{l}{
    (a:3)=1  \\
    (a:11)=1 \\
    (a:17)=1
  }
$$

Es decir que $a \noDivide 3, 11, 17$. Además, como 3, 11 y 17 son primos, por PTF, tenemos que

$$
  \llave{l}{
    \congruencia{a^{2}}{1}{3}  \\
    \congruencia{a^{10}}{1}{11} \\
    \congruencia{a^{16}}{1}{17}
  }
  \Sii{$a \cop 3, 11, 17$}
  \llave{l}{
    \congruencia{(a^{2})^{280}}{1^{280}}{3}  \\
    \congruencia{(a^{10})^{56}}{1^{56}}{11} \\
    \congruencia{(a^{16})^{35}}{1^{35}}{17}
  }
  \sisolosi
  \llave{l}{
    \congruencia{a^{560}}{1}{3}  \\
    \congruencia{a^{560}}{1}{11} \\
    \congruencia{a^{560}}{1}{17}
  }
$$

Por útimo, utilizando que 3, 11 y 17 son coprimos dos a dos y haciendo TCR, obtenemos

$$
  \llave{l}{
    \congruencia{a^{560}}{1}{3}  \\
    \congruencia{a^{560}}{1}{11} \\
    \congruencia{a^{560}}{1}{17}
  }
  \sisolosi
  \congruencia{a^{560}}{1}{3\cdot11\cdot17}
  \sisolosi
  \cajaResultado{\congruencia{a^{560}}{1}{561}} \Tilde
$$

\begin{aportes}
  \item \aporte{https://github.com/Nunezca}{Nunezca \github}
\end{aportes}
