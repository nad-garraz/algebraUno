\begin{enunciado}{\ejercicio}
  Resolver en $\enteros$ los siguientes sistemas lineales
  de ecuaciones de congruencia:
  \begin{enumerate}[label=\roman*)]
    \begin{multicols}{2}
      \item
      $
        \llave{l}{
          \congruencia{2^{2013}X}{6}{13}\\
          \congruencia{5^{2013}X}{4}{7}\\
          \congruencia{7^{2013}X}{2}{5}
        }
      $
      \item
      $
        \llave{l}{
          \congruencia{10^{49}X}{17}{39}\\
          \congruencia{5X}{7}{9}
        }
      $
    \end{multicols}
  \end{enumerate}
\end{enunciado}

\begin{enumerate}[label=\roman*)]
  \item Todos divisores primos y coprimos dos a dos:
        $$
          \llave{lclcl}{
            \congruencia{2^{2013}X}{6}{13} & \Sii{PTF} & \congruencia{2^9 X}{6}{13} & \Sii{\red{!}} & \congruencia{X}{9}{13} \\
            \congruencia{5^{2013} X}{4}{7} & \Sii{PTF} & \congruencia{5^3 X}{4}{7}  & \Sii{\red{!}} & \congruencia{X}{3}{7}\\
            \congruencia{7^{2013}X}{2}{5} & \sii & \congruencia{2X}{2}{5} & \Sii{\red{!}} & \congruencia{X}{1}{5}
          }
        $$

        Listo hay que resolver esa goma de sistema:
        $$
          \llave{l}{
            \congruencia{X}{9}{13}\\
            \congruencia{X}{3}{7}\\
            \congruencia{X}{1}{5}
          }
        $$

        \parrafoDestacado{
          \it
          Este no es mi método favorito de resolución, pero me llegó esta versión que está bueno para aportar
          un poco de divsersidad resolutiva.
        }
        Dado que tengo a los divisores primos por \href{\chinito}{T\faIcon{cannabis}R} voy a tener una solución de la pinta:
        $$
          x_0 = x_1 + x_2 + x_3
        $$
        donde, $x_1, x_2 \ytext x_3$ son soluciones particulares de los sistemas:
        $$
          \llave{l}{
            \congruencia{x_1}{9}{13}\\
            \congruencia{x_1}{0}{7}\\
            \congruencia{x_1}{0}{5}
          }
          ,~~
          \llave{l}{
            \congruencia{x_2}{0}{13}\\
            \congruencia{x_2}{3}{7}\\
            \congruencia{x_2}{0}{5}
          }
          \ytext
          \llave{l}{
            \congruencia{x_3}{0}{13}\\
            \congruencia{x_3}{0}{7}\\
            \congruencia{x_3}{1}{5}
          }
        $$
        respectivamente.

        \parrafoDestacado[{\faIcon[regular]{eye}\faIcon{ruler}}]{
          \it
          \ul{Ojímetro:}
          Hay que mirar las ecuaciones con \ul{resto cero} que se cumplen para productos entre los divisores.

          Funciona directo para el primer y tercer sistema. Para el segundo hay que ir viendo los multiplos enteros de $13 \cdot 5$ y
          sale.
        }
        Las soluciones particulares, $x_i$ y la solución general $x_0$ quedarán como:
        $$
          \llave{rcl}{
            x_1 & = & 35\\
            x_2 & = & 325\\
            x_3 & = & 91
          }
          \equivalente
          \cajaResultado{
            \congruencia{x_0}{451}{455}
          }
        $$

        por lo tanto

  \item Se procede similar al ejercicio anterior:
        $$
          \llave{l}{
            \congruencia{10^{49}X}{17}{39}
            \equivalente
            \llave{l}{
              \congruencia{1^{49}X}{2}{3}
              \sii
              \congruencia{X}{2}{3}\\
              \congruencia{10^{49}X}{4}{13}
              \Sii{PTF}
              \congruencia{10X}{4}{13}
              \Sii{$\times 4$}[$4 \cop 13$]
              \congruencia{X}{3}{13}
            }
            \\
            \\
            \congruencia{5X}{7}{9}
            \sii
            \congruencia{X}{5}{9}
            \equivalente
            \llave{l}{
              \congruencia{X}{2}{3}
            }
          }
        $$
        El sistema es compatible. Tomo la ecuación que tiene el denominador con la mayor potencia de 3, si no obtendría más soluciones de las que
        tiene el sistema original.

        El sistema a resolver:
        $$
          \llave{l}{
            \congruencia{X}{3}{13} \\
            \congruencia{X}{5}{9}
          }
        $$
        Resolver eso es hacer:
        $$
          X \igual{def} 13\cdot \blue{k} + 3
          \flecha{meto en}[segunda ecuación]
          13 \cdot \blue{k} + 3 \conga9
          \congruencia{4 \cdot \blue{k} + 3}{5}{9}
          \Sii{\red{!}}
          \congruencia{\blue{k}}{5}{9}
          \sii \blue{k} \igual{def} \blue{k} = 9\magenta{j} + 5
        $$
        Reemplazo el valor de $\blue{k}$ para encontrar la $X$:
        $$
          X = 13 \cdot ( 9\magenta{j} + 5) + 3 = 117 \magenta{j} + 68
          \sii
          \cajaResultado{
            \congruencia{X}{68}{117}
          }
        $$

\end{enumerate}

\begin{aportes}[2]
  \item \aporte{\dirRepo}{naD GarRaz \github}
  \item \aporte{https://github.com/ivdou}{Ivan Doumerc \github}
\end{aportes}
