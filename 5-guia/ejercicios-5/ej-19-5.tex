\begin{enunciado}{\ejercicio}
  Resolver en $\enteros$ los siguientes sistemas lineales
  de ecuaciones de congruencia:
  \begin{enumerate}[label=\roman*)]
    \begin{multicols}{2}
      \item
      $
        \llave{l}{
          \congruencia{2^{2013}X}{6}{13}\\
          \congruencia{5^{2013}X}{4}{7}\\
          \congruencia{7^{2013}X}{2}{5}
        }
      $
      \item
      $
        \llave{l}{
          \congruencia{10^{49}X}{17}{39}\\
          \congruencia{5X}{7}{9}
        }
      $
    \end{multicols}
  \end{enumerate}
\end{enunciado}

\begin{enumerate}[label=\roman*)]
  \item Todos divisores primos y coprimos dos a dos:
        $$
          \llave{lclcl}{
            \congruencia{2^{2013}X}{6}{13} & \Sii{PTF} & \congruencia{2^9 X}{6}{13} & \Sii{\red{!}} & \congruencia{X}{9}{13} \\
            \congruencia{5^{2013} X}{4}{7} & \Sii{PTF} & \congruencia{5^3 X}{4}{7}  & \Sii{\red{!}} & \congruencia{X}{3}{7}\\
            \congruencia{7^{2013}X}{2}{5} & \sii & \congruencia{2X}{2}{5} & \Sii{\red{!}} & \congruencia{X}{1}{5}
          }
        $$

        Listo hay que resolver esa goma de sistema:
        $$
          \llave{l}{
            \congruencia{X}{9}{13}\\
            \congruencia{X}{3}{7}\\
            \congruencia{X}{1}{5}
          }
        $$

        Ehm, paja resolverlo.

        \hacer

  \item
        $$
          \llave{l}{
            \congruencia{10^{49}X}{17}{39}
            \equivalente
            \llave{l}{
              \congruencia{1^{49}X}{2}{3}
              \sii
              \congruencia{X}{2}{3}\\
              \congruencia{10^{49}X}{4}{13}
              \Sii{PTF}
              \congruencia{10X}{4}{13}
              \Sii{$\times 4$}[$4 \cop 13$]
              \congruencia{X}{3}{13}
            }
            \\
            \\
            \congruencia{5X}{7}{9}
            \sii
            \congruencia{X}{5}{9}
            \equivalente
            \llave{l}{
              \congruencia{X}{2}{3}
            }
          }
        $$
        El sistema es compatible. Tomo la ecuación que tiene el denominador con la mayor potencia de 3, si no obtendría más soluciones de las que
        tiene el sistema original.

        El sistema a resolver:
        $$
          \llave{l}{
            \congruencia{X}{3}{13} \\
            \congruencia{X}{5}{9}
          }
        $$
        Ehm, paja resolverlo.

        \hacer
\end{enumerate}

\begin{aportes}
  \item \aporte{\dirRepo}{naD GarRaz \github}
\end{aportes}
