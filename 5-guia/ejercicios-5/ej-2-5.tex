\begin{enunciado}{\ejercicio}
  Determinar todos los $(a,b)$ que simultáneamente $4 \divideA a,\, 8 \divideA b \ytext 33a + 9b = 120$.
\end{enunciado}

Para que la ecuación tenga solución necesito que el MCD entre 33 y 9 divida al 120, es decir:
$$
  (33:9) \divideA 120 \entonces 33a + 9b = 120
$$
y dado que $(33:9) = 3
  \ytext
  3\divideA 120$ sé que puedo encontrar dicha solución. Pero tengo más restricciones sobre
los valores de $a \ytext b$.

$$
  \llave{l}{
    4 \divideA a \Sii{def} a = 4k_1 \\
    8 \divideA b \Sii{def} b = 8k_2
  } \llamada1
$$
Pongo esa info en la ecuación:
$$
  \flecha{$\llamada1$}
  33a + 9b = 120
  \to
  132 k_1 + 72 k_2 = 120
$$
Siempre que puedo tengo que coprimizar:
$$
  132 k_1 + 72 k_2 = 120
  \Sii{coprimizo}
  11 k_1 + 6 k_2 = 10
$$
Busco \textit{solución particular} con Euclides, busco escribir al número 1 como combinación entera de \blue{11} y \blue{6}:
$$
  \llave{rcl}{
    11 & = & 6 \cdot 1 + 5 \\
    6  & = & 5 \cdot 1 + 1
  }
  \entonces
  1 =  \blue{11} \cdot (-1) + \blue{6} \cdot 2 \llamada2
$$
Obtengo así la \textit{solución particular}:
$$
  \Entonces{$\llamada2 \times 10$}
  10 = \blue{11}\cdot(-10)   + \blue{6} \cdot 20
  \entonces
  (k_1, k_2) = (-10, 20)
$$
La solución del homogéneo queda:
$$
  11 k_{1_h} + 6 k_{2_h} = 0 \entonces (k_{1_h}, k_{2_h}) = (-6, 11)
$$
Por lo que la solución general:
$$
  (k_{1_g}, k_{2_g}) = (k_{1_p}, k_{2_p}) + \purple{k} \cdot (k_{1_h}, k_{2_h}) =
  (-10, 20) + \purple{k} \cdot (-6, 11) =
  (-10 - 6\purple{k}, 20 + 11\purple{k})
$$
Pero me pidieron los pares $(a,b)$ con $a \divideA 4 \ytext b \divideA 8$:
$$
  \Entonces{$\llamada1$}
  \cajaResultado{
    (a,b) = (-40 - 24k, 160 + 88k)
  }
$$

\begin{aportes}
  \item \aporte{\dirRepo}{naD GarRaz \github}
\end{aportes}
