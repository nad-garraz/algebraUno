\def\sumLocal{\sumatoria{i=1}{1759}}
\begin{enunciado}{\ejercicio}
  %Macro local
  % fin macro local
  Hallar el resto de la división de:
  \begin{enumerate}[label=\roman*)]
    \item $43 \cdot 7^{135} + 24^{78} + 11^{222}$ por 70
    \item $\sumLocal i^{42}$ por 56
  \end{enumerate}
\end{enunciado}

\begin{enumerate}[label=\roman*)]
  \item \hacer

  \item Calcular el resto pedido equivale a resolver la ecuación de equivalencia:
        $$
          \congruencia{X}{\sumLocal i^{42}}{56}
        $$
        que será aún más simple si quiebro en la forma:
        $$
          \congruencia{X}{\sumLocal i^{42}}{56}
          \leftrightsquigarrow
          \llave{l}{
            \congruencia{X}{\sumLocal i^{42}}{7} \llamada1 \\
            \congruencia{X}{\sumLocal i^{42}}{8} \llamada2
          }
        $$
        Primero estudio $\llamada1$.\par
        Acomodo la sumatoria, voy a abrirla y separar
        los términos \textit{convenientemente}:
        $$
          \sumLocal i^{42} = 1^{42} + 2^{42} + 3^{42}  + \cdots 1759^{42}
        $$
        Le calculo el módulo 7 a todos los términos y obtengo:
        $$
          \begin{array}{c}
            \scriptstyle
            \sumLocal i^{42} \conga7
            1^{42} + 2^{42} + 3^{42} + 4^{42} + 5^{42} + 6^{42} + 0^{42} +
            1^{42} + 2^{42} + 3^{42} + 4^{42} + 5^{42} + 6^{42} + 0^{42} +
            1^{42} + 2^{42} +3^{42} + 4^{42} + 5^{42} \dots \\
          \end{array}
        $$
        La sumatoria tiene un total de 1759 términos, que se puede agrupar en
        $1759 = 251\cdot \blue{7}  + \purple{2}$.

        $$
          \begin{array}{c}
            \sumLocal i^{42} \conga7 251 \cdot \bigl(
            \blue{1^{42} + 2^{42} + 3^{42} + 4^{42} + 5^{42} + 6^{42} + 0^{42}}\bigr) +
            \bigl(\purple{1^{42} + 2^{42}}\bigr)
          \end{array}
        $$
        $$
          \begin{array}{c}
            \Sii{7 primo y $7 \noDivide i$}[PTF en cada término] \\
            \sumLocal i^{42} \conga7 251 \cdot
            \bigl( 1 + 1 + 1 + 1 + 1 + 1 + 0\bigr) + \bigl(1 + 1 \bigr) =
            251 \cdot  6 + 2 \conga7 3
          \end{array}
        $$
        Encontramos entonces que $\llamada1$:
        $$
          \congruencia{X}{3}{7}
        $$

        Ahora se labura la expresión $\llamada2$. Es la misma idea de antes, pero cuidado que como 8 no es primo, no
        se puede usar el PTF. Haciendo lo mismo de antes, abriendo la sumatoria y aplicando el módulo 8 a cada término:
        $$
          \begin{array}{c}
            \scriptstyle
            \sumLocal i^{42} \conga8
            1^{42} + 2^{42} + 3^{42} + 4^{42} + 5^{42} + 6^{42} + 7^{42} + 0^{42} +
            1^{42} + 2^{42} + 3^{42} + 4^{42} + 5^{42} + 6^{42} + 7^{42} + 0^{42} +
            1^{42} + 2^{42} +3^{42} + 4^{42} + 5^{42} \dots \\
          \end{array}
        $$

        La sumatoria tiene un total de 1759 términos, que se puede agrupar en
        $1759 = 219\cdot \blue{8}  + \purple{7}$.
        $$
          \sumLocal i^{42} \conga8 219 \cdot \bigl(
          \blue{1^{42} + 2^{42} + 3^{42} + 4^{42} + 5^{42} + 6^{42} + 7^{42} +0^{42}}\bigr) +
          \bigl(\purple{1^{42} + 2^{42} + 3^{42} + 4^{42} + 5^{42} + 6^{42} + 7^{42}}\bigr)
        $$
        Para analizar los términos a mano, se puede jugar con el exponente, buscando que el cálculo quede simple:
        $$
          \llave{l}{
            2^{42} = (2^3)^{14}\conga8 0                  \\
            4^{42} = (2^3)^{14} \cdot (2^3)^{14}\conga8 0 \\
            6^{42} = (2^3)^{14} \cdot 3^{42} \conga8 0
          } \quad
          \llave{l}{
            1^{42} = 1                             \\
            3^{42} = (3^2)^{21} \conga8 1^{21} = 1 \\
            5^{42} = (5^2)^{21} \conga8 1^{21} = 1 \\
            7^{42} = (7^2)^{21} \conga8 1^{21} = 1
          }
        $$
        {\tiny Y por alguna razón \textit{matemágica}, la cual no podría importarme menos, los pares dieron 0 y los impares 1. \red{\faIcon{ghost}}}
        $$
          \sumLocal i^{42} \conga8 219 \cdot \bigl(
          \blue{1 + 0 + 1 + 0 + 1 + 0 + 1 +0}\bigr) +
          \bigl(\purple{1 + 0 + 1 + 0 + 1 + 0 + 1}\bigr) =
          219 \cdot 4 + 4 = 880 \conga8 0
        $$
        Encontramos entonces que $\llamada2$:
        $$
          \congruencia{X}{0}{8}
        $$

        Para resolver el ejercicio solo falta resolver el sistema que queda de juntar los resultados de $\llamada1$ y $\llamada2$:
        $$
          \llave{l}{
            \congruencia{X}{3}{7} \\
            \congruencia{X}{0}{8}
          }
        $$
        tiene solución por \href{\chinito}{TCH}, dado que 8 y 7 son coprimos. La solución da
        $\congruencia{X}{24}{56}$, por lo tanto el \textit{resto pedido}:
        $$
          r_{56}\parentesis{\sumLocal i^{42}} = 24
        $$
\end{enumerate}

\begin{aportes}
  \item \aporte{https://github.com/nad-garraz}{Nad Garraz \github}
\end{aportes}
