\begin{enunciado}{\ejercicio}
  Hallar el resto de la división de $2^{2^n}$ por 13 para cada $n \en \naturales$.
\end{enunciado}

Enunciado corto y al pie, eso vaticina \textit{infierno}. Bueno, no, estoy exagerando un poco.

$$
  r_{13}(2^{2^n}) = \violet{X}
  \Sii{def}
  \congruencia{2^{2^n}}{\violet{X}}{13}
  \Sii{$(13:2) = 1$}[PTF]
  \congruencia{2^{\magenta{r_{12}(2^n)}}}{\violet{X}}{13} \llamada1
$$
La papa está en reconocer qué es esa expresión del \magenta{exponente}. \textit{Let's estudiate it}:
{\small
$$
  \begin{array}{c}
    \magenta{r_{12}(2^n)}
    \Sii{def}
    \congruencia{2^n}{\magenta{Y}}{12}
    \equivalente
    \llamada2
    \llave{l}{
      \congruencia{2^n}{\magenta{Y}}{4}
      \Sii{\red{!!}}
      \llave{rcl}{
    \text{si } n = 1    & \entonces & \congruencia{2^1 = 2}{\magenta{Y}}{4}                                                            \\
    \text{si } n \geq 2 & \entonces & \congruencia{2^n \conga4 2^{2m + m'} \igual{\red{!}} 4^m \cdot 2^{m'} \conga4 0}{\magenta{Y}}{4}
    }                                                                                                                                  \\
      \congruencia{2^n}{\magenta{Y}}{3}
      \Sii{PTF}[$3\cop 2$]
      \congruencia{2^{r_2(n)}}{\magenta{Y}}{3}
      \Sii{\red{!}}
      \llave{rcl}{
    \text{si $n$ par}   & \entonces & \congruencia{2^0 = 1}{\magenta{Y}}{3}                                                            \\
                        & \otext    &                                                                                                  \\
    \text{si $n$ impar} & \entonces & \congruencia{2^1 = 2}{\magenta{Y}}{3}                                                            \\
      }
    }
  \end{array}
$$
}
En donde pongo el exponente $2m + m'$ es una forma de descomponer un número mayor a 2 para mostrar que $4 \divideA 2^{2m + m'}$ siempre.
Podrías usar una tabla de restos también, \textit{you are free to choose}.

Por lo tanto para el sistema de $\llamada2$ saco 3 sistemas:
$$
  \text{Si } n = 1
  \Entonces{$\llamada3$}
  \llave{l}{
    \congruencia{\magenta{Y}}{2}{4}\\
    \congruencia{\magenta{Y}}{2}{3}
  }
  ,
  \quad
  \text{si } \congruencia{n}{0}{2}
  \Entonces{$\llamada4$}
  \llave{l}{
    \congruencia{\magenta{Y}}{0}{4}\\
    \congruencia{\magenta{Y}}{1}{3}
  }
  \ytext
  \
  \text{si } \congruencia{n}{1}{2}
  \Entonces{$\llamada5$}
  \llave{l}{
    \congruencia{\magenta{Y}}{0}{4}\\
    \congruencia{\magenta{Y}}{2}{3}
  }
$$
Para cada uno de esos casos quiero encontrar $\magenta{Y}$ así puedo después volver al principio del ejercicio
para calcular lo que piden. Pero en el sistema $\llamada3$ es más fácil ponerlo directamente en el $\llamada1$,
porque es un solo valor. Igual porque soy un tipazo lo resuelvo de forma mecánica por si no lo ves:
$$
  \flecha{$\llamada3$}
  \magenta{Y} = 4\blue{k} + 2
  \flecha{meto en}[mod $(3)$]
  \congruencia{4\blue{k} + 2 \conga3 \blue{k} + 2}{2}{3}
  \sii
  \congruencia{\blue{k}}{0}{3}
  \Sii{def}
  \blue{k} = 3\yellow{j}
  \Entonces{\red{!!}}
  \congruencia{\magenta{Y}}{\green{2}}{12}
$$
Reemplazando ese resultado en $\llamada1$:
$$
  \cajaResultado{
    r_{13}(2^{\green{2}}) = 4 \quad \text{para $n = 1$}
  }
$$
Como ya habrás notado, mucho más barato era ir a $\llamada1$ y poner $n = 1$.

\bigskip

Ahora voy a resolver el sistema $\llamada4$:
$$
  \flecha{$\llamada4$}
  \magenta{Y} = 4\blue{k}
  \flecha{meto en}[mod $(3)$]
  \congruencia{4\blue{k} \conga3 \blue{k}}{1}{3}
  \Sii{def}
  \blue{k} = 3\yellow{j} + 1
  \Entonces{\red{!!}}
  \congruencia{\magenta{Y}}{\green{4}}{12}
$$
Reemplazando ese resultado en $\llamada1$:
$$
  \cajaResultado{
    r_{13}(2^{\green{4}}) = 3 \quad \text{para $\congruencia{n}{0}{2}$}
  }
$$

\bigskip

Por último voy a resolver el sistema $\llamada5$:
$$
  \flecha{$\llamada5$}
  \magenta{Y} = 4\blue{k}
  \flecha{meto en}[mod $(3)$]
  \congruencia{4\blue{k} \conga3 \blue{k}}{2}{3}
  \Sii{def}
  \blue{k} = 3\yellow{j} + 2
  \Entonces{\red{!!}}
  \congruencia{\magenta{Y}}{\green{8}}{12}
$$
Reemplazando ese resultado en $\llamada1$:
$$
  \cajaResultado{
    r_{13}(2^{\green{8}}) =
    r_{13}(2^{\green{4}} \cdot 2^{\green{4}}) =
    9 \quad \text{para $\congruencia{n}{1}{2}$}
  }
$$

\bigskip

Concluyendo que si quiero calcular $r_{13}(2^{2^n})\ \paratodo n \en \naturales$:
$$
  \cajaResultado{
    r_{13}(2^{2^n}) =
    \llave{rcl}{
      4 & \text{si} & n = 1 \\
      3 & \text{si} & \congruencia{n}{0}{2} \\
      9 & \text{si} & \congruencia{n}{1}{2} \ytext n \geq 2 \\
    }
  }
$$

\begin{aportes}
  \item \aporte{\dirRepo}{naD GarRaz \github}
  \item \aporte{https://github.com/olivportero}{Olivia Portero \github}
\end{aportes}
