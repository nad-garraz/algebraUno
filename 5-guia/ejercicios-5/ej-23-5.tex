\ejercicio
Hallar todos los divisores positivos de $5^{140} = 25^{70}$ que sean congruentes
a 2 módulo 9 y 3 módulo 11.

\separadorCorto

Quiero que ocurra algo así:
$\llave{l}{
		\congruencia{25^{70}}{0}{d} \to \congruencia{5^{140}}{0}{d}  \\
		\congruencia{d}{2}{9} \\
		\congruencia{d}{3}{11}
	}$.
De la primera ecuación queda que el divisor $d = 5^\alpha$ con $\alpha$ compatible
con las otras ecuaciones.
$\to
	\llave{l}{
		\congruencia{5^\alpha}{2}{9} \\
		\congruencia{5^\alpha}{3}{11}
	}$\\

$\to \text{Busco periodicidad en los restos de las exponenciales $5^{\text{¿}\alpha?} \congruente 1$: }\\
	\flecha{Busco}[$5^{a} \conga{d} 1$]
	\llave{l}{
		\congruencia{5^\alpha}{2}{9} \\
		\congruencia{5^3}{-1}{9}
		\sii
		\congruencia{5^6}{1}{9}
		\sii
		5^{6k+r_6(\alpha)} = (\ob{5^6}{\conga9 1})^k 5^{r_6(\alpha)}.\\
        \text{Busco, posibles valores para $r_6(\alpha)$: }
		\begin{array}{|l|l|l|l|l|l|l|}
			\hline
			r_6(\alpha)   & 0 & 1 & 2 & 3 & 4 & 5       \\ \hline
			r_9(5^\alpha) & 1 & 5 & 7 & 8 & 4 & \red{2} \\ \hline
		\end{array}\\
		\flecha{por lo}[tanto] \text{para que }
		\congruencia{5^\alpha}{2}{9}
		\sii
		\boxed{\congruencia{\alpha}{5}{6}}\Tilde \\

		\separadorCorto

		\congruencia{5^\alpha}{3}{11}
		\flecha{fermateo en búsqueda de }[periodicidad 11 es primo, $11\noDivide 5$]
		\congruencia{\green{5^{10}}}{1}{11}\\
        \text{El PTF no me asegura que no haya un $\alpha < 10$ que también cumpla $\congruencia{5^\alpha}{1}{11}$}\\
		\begin{array}{|l|l|l|l|l|l|l|l|l|l|l|}
			\hline
			r_{10}(\alpha)   & 0 & 1 & 2       & 3 & 4 & 5        \\ \hline
			r_{11}(5^\alpha) & 1 & 5 & \red{3} & 4 & 9 & \cyan{1} \\ \hline
		\end{array}\\
		\flecha{por lo tanto hay}[periodicidad de 5] \text{Se obtiene enteonces: }\\
		\congruencia{5^\alpha}{3}{11}
		\sii
		\boxed{\congruencia{\alpha}{2}{5}} \Tilde
	}$\\
El sistema
$\llave{l}{
		\congruencia{\alpha}{5}{6} \\
		\congruencia{\alpha}{2}{5}
	}$
 6 y 5 son coprimos, se resuelve para $\congruencia{\alpha}{17}{30}$ y además $0<\alpha \leq 140$ lo que se
cumple para $\alpha = 30k + 17 =
	\llaves{lcr}{
		17  &  \text{ si } & k = 0\\
		 47 &  \text{ si } & k = 1 \\
		77 &  \text{ si } & k = 2 \\
		107 &  \text{ si } & k = 3 \\
		137 &  \text{ si } & k = 4 \\
	} \to$
    \boxed{\divsetP{25^{70}}{ 5^{17}, 5^{47}, 5^{77}, 5^{107}, 5^{137} }}
