\begin{enunciado}{\ejercicio}
  Hallar todos los $p \en \naturales$ que satisfacen:
  \begin{multicols}{2}
    \begin{enumerate}[label=\alph*)]
      \item  $2p \divideA 38^{2p^2 -p -1} + 3p + 171$
      \item  $3p \divideA 5^{p-1} + 3^{p^2+2} + 833$
    \end{enumerate}
  \end{multicols}
\end{enunciado}

\begin{enumerate}[label=\alph*)]
  \item Para poder usar PTF tengo que tener un primo en el divisor, quiebro:
        $$
          \begin{array}{c}
            \congruencia{38^{2p^2 -p -1} + 3p + 171}{\blue{0}}{2p}
            \leftrightsquigarrow
            \llave{l}{
            \congruencia{p}{1}{2} \llamada1 \\
              \congruencia{38^{2p^2 -p -1} + 3p + 171}{\blue{0}}{p} \llamada2
            }
          \end{array}
        $$

        $$
          \congruencia{38^{2p^2 -p -1} + 3p + 171}{\blue{0}}{p}
          \sii
          \llave{l}{
            \Sii{si $p \divideA 38$}[$\entonces p = 19$]
            \congruencia{ 38^{2p^2 -p -1} + 3p + 171 \conga{19} 0}{\blue{0}}{p} \llamada3 \\
            \Sii{PTF}[$p \noDivide 38$ \red{!!}]
            \congruencia{38^{\red{0}} + 0 + 171 = \ub{172}{2^2 \cdot 43}}{\blue{0}}{p} \llamada4
          }
        $$

        Cálculo para hacer el PTF:
        $$
          \polyset{vars=p}
          \divPol{2p^2 - p - 1}{p-1}
        $$

        Después de hacer todo eso, sacamos de $\llamada1$ que $p$ es impar. De $\llamada3$ obtenemos que un posible valor sería $p = 19$ y
                luego del caso $p \noDivide 38$ sale que debe ocurrir que $172 \conga{p} 0$, y dado que $p$ tiene que ser impar, $p = 43$.

        \item \hacer
\end{enumerate}

% Contribuciones
\begin{aportes}
  %% iconos : \github, \instagram, \tiktok, \linkedin
  %\aporte{url}{nombre icono}
  \item \aporte{https://github.com/nad-garraz}{Nad Garraz \github}
\end{aportes}
