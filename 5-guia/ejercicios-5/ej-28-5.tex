\begin{enunciado}{\ejercicio}
    \begin{enumerate}[label = \roman*)]
        \item Probar que $\{\bar{2}^n : n \in \naturales\} = \enteros/11\enteros - \{\bar{0}\}$
        \item Hallar $\bar{a} \in \enteros/7\enteros$ tal que $\{\bar{a}^n : n \in \naturales\} = \enteros/7\enteros - \{\bar{0}\}$
    \end{enumerate}
\end{enunciado}

\begin{enumerate}[label = \roman*)]
    \item    
    Bueno, en definitiva lo que queremos es que elevando 2 a potencias podamos construir todo $\enteros/11\enteros$,
    estamos elevando 2 a potencias, y luego tomando modulo $11$, es decir, en algun momento se van a repetir \href{https://es.wikipedia.org/wiki/Principio_del_palomar}{(Ppio del palomar)}. 
    Vamos a ver casos individuales que es la manera mas facil de probarlo porque $11$ no es un numero grande. 
    $$
    \begin{array}{c}
     \bar{2}^1  = \red{\bar{2}} \\
     \bar{2}^2  = \bar{4} \\
     \bar{2}^3  = \bar{8} \\
     \bar{2}^4  = \bar{5} \\
     \bar{2}^5  = \bar{10} \\
     \bar{2}^6  = \bar{9} \\
     \bar{2}^7  = \bar{7} \\
     \bar{2}^8  = \bar{3} \\
     \bar{2}^9  = \bar{6} \\
     \bar{2}^{10} = \bar{1} \\
     \bar{2}^{11} = \red{\bar{2}} \\
    \end{array}
    $$
    Vemos que a partir de $\bar{2}^{11}$ se va a volver a repetir todo el ciclo de vuelta.
    Obtuvimos todos los elementos de $\enteros/11\enteros$, sin incluir el 0 (igual esto no podria haber pasado pues $11 \noDivide 2^k \paratodo k \in \naturales$). 
    \item \hacer

\end{enumerate}

\begin{aportes}
    \item \aporte{https://github.com/sigfripro}{sigfripro \github}
\end{aportes}
