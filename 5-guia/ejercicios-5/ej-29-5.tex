\begin{enunciado}{\ejercicio}
Sea $p$ un primo. Probar que en $\enteros/p\enteros$ vale que $(\bar{a} + \bar{b})^p = \bar{a}^p + \bar{b}^p, \paratodo \bar{a}, \bar{b} \in \enteros/p\enteros$
(sug: ver Ej. 25 Practica 4). Vale lo mismo en $\enteros/m\enteros$ si $m$ no es primo?
\end{enunciado}

Expandimos con Newton la expresion: 

\begin{align}
(\bar{a} + \bar{b})^p = \sum_{k=0}^p \binom{p}{k} \bar{a}^{k} \cdot \bar{b}^{p-k} = \\
= \bar{a}^p + \red{\sum_{k=1}^{p - 1} \blue{\binom{p}{k}} \bar{a}^{k} \cdot \bar{b}^{p-k}} + \bar{b}^p
\end{align}

En el ejercicio 25 de la practica 4 se probó que $p \divideA \blue{\binom{p}{k}}, 0 < k < p$, por lo tanto:
$$
\red{\sum_{k=1}^{p - 1} \blue{\binom{p}{k}} \bar{a}^{k} \cdot \bar{b}^{p-k}} = \bar{0}
$$
Entonces 
$$
 \bar{a}^p + \red{\sum_{k=1}^{p - 1} \blue{\binom{p}{k}} \bar{a}^{k} \cdot \bar{b}^{p-k}} + \bar{b}^p = \bar{a}^p + \bar{b}^p
$$
Como se queria probar.
Ahora veamos si lo mismo se cumple para $\enteros/m\enteros, m$ compuesto. Con un contraejemplo basta para desmotrar que no se cumple. 
Consideremos $\enteros/6\enteros$, elijamos $a = \bar{2}$, $b = \bar{4}$. 
$$
(\bar{2} + \bar{4})^6 = \bar{0} \text{ pero } \bar{2}^6 + \bar{4}^6 = \bar{4} + \bar{4} = \bar{2} \neq \bar{0}
$$

\begin{aportes}
    \item \aporte{https://github.com/sigfripro}{sigfripro \github}
\end{aportes}