\ejercicio
Si se sabe que cada unidad de un cierto producto $A$ cuesta $39$ pesos y que cada unidad de un cierto
producto $B$ cuesta 48 pesos, ¿cuántas unidades de cada producto se pueden comprar gastando exactamente
135 pesos?

\separadorCorto

$
	\llave{l}{
		A \geq 0 \y B \geq 0 \text{. Dado que son productos.}\\
		(A:B) = 3 \entonces 39A + 28B = 135
		\flecha{coprimizar}
		13A + 16B = 45\\
		\text{ A ojo } \to (A,B) = (1,2)
	}
$
