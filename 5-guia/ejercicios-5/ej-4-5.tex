\begin{enunciado}{\ejercicio}
  Hallar, cuando existan, todas las soluciones de las siguientes ecuaciones de congruencia:
  \begin{multicols}{4}
    \begin{enumerate}[label=\roman*)]
      \item $\congruencia{17X}{3}{11}$
      \item $\congruencia{56X}{28}{35}$
      \item $\congruencia{56X}{2}{884}$
      \item $\congruencia{78X}{30}{12126}$
    \end{enumerate}
  \end{multicols}
\end{enunciado}

\begin{itemize}
  \item Algunos cálculos salen a ojo. Recordar que una ecuación de congruencia se puede pensar expresar como una diofántica:
        $$
          \congruencia{aX}{r}{b}
          \equivalente
          a \cdot X + b \cdot Y = r
        $$

  \item Coprimizar siempre que se pueda.
\end{itemize}

\begin{enumerate}[label=\roman*)]
  \item $$
          \congruencia{17X}{3}{11}
          \sisolosi
          \congruencia{6X}{3}{11}
          \Sii{$\times 2$}[$(\Leftarrow) 2 \cop 11$]
          \cajaResultado{
            \congruencia{X}{6}{11}
          }
        $$

  \item
        $$
          \congruencia{56X}{28}{35}
          \sisolosi
          \congruencia{21X}{28}{35}
          \Sii{coprimizo}[$(21:35) = 7$]
          \congruencia{3X}{4}{5}
          \Sii{$\times 7$}[$(\Leftarrow) 7\cop 5$]
          \cajaResultado{
            \congruencia{X}{3}{5}
          }
        $$

  \item Empiezo coprimizando:
        $$
          \congruencia{56X}{2}{884}
          \sisolosi
          \congruencia{28X}{1}{442}
        $$

        Esa ecuación tiene solución, si y solo si $(28:442) \divideA 1$, peeeeero no es el caso \textit{cause} $28 \not\perp 442$ así que:
        $$
          \cajaResultado{
            X = \vacio
          }
        $$

  \item Empiezo coprimizando:
        $$
          \congruencia{78X}{30}{12126}
          \Sii{coprimizar}[(78:12126) = 6]
          \congruencia{13X}{5}{2021},\,
        $$
        Dado que $\ob{(13:2021)}{1} \divideA 5$ hay solución \faIcon[regular]{smile}.
\medskip

        Busco solución particular con Euclides. Escribo al 5 como combinación entera de \blue{13} y \blue{2021}:
        $$
          \llave{c}{
            \blue{2021} =  \blue{13} \cdot 155 + 6                                                                          \\
            \blue{13} =  6 \cdot 2 + 1
          }
          \Entonces{1 como combinación}[de \blue{13} y \blue{2021}]
          1 = \blue{13} \cdot 311 + \blue{2021} \cdot (-2)
          1 = 13 \cdot 311 + 2021 \cdot (-2) \llamada1
        $$

        Multiplico $\llamada1$ por 5 para obtener la expresión que buscaba:
        $$
          \llamada1
          \Sii{$\times 5$}[$(\Leftarrow) 5 \cop 2021$]
          5 =  13\cdot 1555 + 2021 \cdot (-10)                                                                            \\
        $$
        De donde obtengo la \textit{solución particular}:
        $$
          13\cdot \ub{1555}{\text{Solución} \\ \text{particular}} = 2021 \cdot 10  + 5
          \Entonces{Solución general, $X$}[\red{!!}]
          \cajaResultado{
            \congruencia{X}{1555}{2021}
          }
        $$
        Si no ves el paso \red{!!}, hacé el procedimiento para resolver la diofántica, $13X + 2021Y = 5$ que es equivalente
        a $\congruencia{13X}{5}{2021}$.
\end{enumerate}

\begin{aportes}
  \item \aporte{\dirRepo}{naD GarRaz \github}
\end{aportes}
