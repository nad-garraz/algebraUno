\begin{enunciado}{\ejercicio}
  Hallar todos los $(a,b) \en \enteros^2$ tales que $\congruencia{b}{2a}{5}$ y $28a + 10b = 26$.
\end{enunciado}

Este es parecido al \refEjercicio{ej:2}. Voy a despejar de una ecuación y meter en la otra:

Despejo:
$$ \congruencia{b}{2a}{5}
  \Sii{def}
  \magenta{b} = 5\blue{k} + 2a \llamada1
$$

Meto ahora en la otra ecuación:
$$
  28a + 10\magenta{b} = 26
  \Sii{$\llamada1$}
  48a + 50\blue{k} = 26
$$

¿Esta última ecuación tiene solución? Sí, dado que: $(48:50) = 2$ y $2 \divideA 26$.

Coprimizo:
$$
  24a + 25k = 13
$$

A ojo veo que {\tiny(si no se ve a ojo, se puede hacer Euclides)}:
$$
  (a,k) = q\cdot (-25, 24) + (-13, 13)
  \sii
  \llave{rcl}{
    a & = &  -13 + (-25)q\\
    k & = &  13 + 24 q
  }
$$

\textit{Let's corroborate:} Uso esos valores para comprobar que se cumplen las ecuaciones del enunciado:
$$
  b = 5\cdot \ub{(13 + 24q)}{k} + 2\cdot \ub{(-13 + (-25)q)}{a} =
  39 + 70q
  \Entonces{módulo}[5]
  b   =  \congruencia{39 + 70 q}{4}{5}
  \sii
  \cajaResultado{
    \congruencia{b}{4}{5}
  }
$$

Por otro lado:
$$
  2a = \congruencia{-26-50q}{-1}{5} \congruente 4\ (5)
  \sii
  \cajaResultado{
    \congruencia{2a}{4}{5}
  }
$$

Concluyendo que efectivamente:
$$
  \congruencia{b}{2a}{5}
$$

\begin{aportes}
  \item \aporte{\dirRepo}{naD GarRaz \github}
  \item \aporte{https://github.com/MPoncini}{M Poncini \github}
\end{aportes}
