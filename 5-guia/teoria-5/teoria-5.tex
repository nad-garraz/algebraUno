\def\MCD{(a:b)}

\textit{Diofánticas:}
\begin{itemize}

  \item Sea $aX + bY = c \text{ con } a,\, b,\, c \en \enteros,\, a \distinto 0 \ytext b \distinto 0$ y sea
        $$\cajaResultado{
            S = \set{ (x,y) \en \enteros^2 : aX + bY = c } \entonces
            S \distinto \vacio \sisolosi (a:b) \divideA c}
        $$
        \red{¡Coprimizar siempre que se pueda!: } Las soluciones de $S$ son las mismas que las de $S$ coprimizado.
        $$
          \cajaResultado{
            aX + bY = c
            \Sii{$a' = \frac{a}{(a:b)} \ytext  b' = \frac{b}{(a:b)}$}[$ c' = \frac{c}{(a:b)}$]
            a'X + b'Y = c'
          }
        $$

  \item Las solución general del sistema S coprimizado :
        $$
          \cajaResultado{
            S = \bigg\{
            (x,y) \en \enteros^2 : (x,y) =
            \ub{(x_0, y_0)}{\text{Solución particular}} + k \cdot \ob{(-b', a')}{\text{Solución homogéneo}}
            \text{ con } , k \en \enteros
            \bigg\}
          }
        $$
\end{itemize}\bigskip

\textit{Ecuaciones de congruencia: }
\begin{itemize}
  \item $\congruencia{aX}{c}{b}$  con  $ a,\, b \distinto 0$\par
        \red{¡Coprimizar siempre que se pueda!:} Las soluciones de la ecuación original son las mismas que las de la ecuación coprimizada.
        $$
          \boxed{
            \congruencia{aX}{c}{b}
            \Sii{$a' = \frac{a}{(a:b)} \ytext  b' = \frac{b}{(a:b)}$}[$ c' = \frac{c}{(a:b)}$]
            \congruencia{a'X}{c'}{b'}
          }
        $$

  \item Ojo con el "$\sisolosi$": Si vas a multiplicar la ecuación por algún número $\purple{d}$ y se te ocurre poner un $\sisolosi$
        conectando la operación justificá así:
        $$
          \boxed{
            \congruencia{aX}{c}{b}
            \Sii{}[$ \purple{d} \cop b$]
            \congruencia{\purple{d}aX}{\purple{d}c}{b}
          }
        $$
        Porque si $d \nocop b$ \red{no vale la vuelta $(\Leftarrow)$} en el "$\sisolosi$", y la cagás.

  \item Si te ponés a hacer cuentas en $\congruencia{aX}{c}{b}$
        \textit{sin que} $a \cop b$,
        \underline{la vas a cagar}. Yo te avisé \faIcon{hands-wash}.
\end{itemize}\bigskip

\textit{Sistemas de ecuaciones de congruencia: \href{\superIdol}{Teorema chino del resto}}
\begin{itemize}
  \item
        Sean $m_1,\dots m_n \en \enteros$ \underline{\red{coprimos dos a dos}}, es decir que $\paratodo i \distinto j$, se tiene $m_i \cop m_j$,
        entonces, dados $c_1,\dots, c_n \en \enteros$ cualesquiera, el sistema de ecuaciones de congruencia
        $$
          \llave{c}{
            \congruencia{X}{c_1}{m_1}\\
            \congruencia{X}{c_2}{m_2}\\
            \vdots\\
            \congruencia{X}{c_n}{m_n}
          } \leftrightsquigarrow
          \congruencia{X}{\magenta{x_0}}{m_1\cdot m_2 \cdots m_n},
        $$
        tiene solución y esa solución, $\magenta{x_0}$ cumple $0 \leq \magenta{x_0} < m_1\cdot m_2 \cdots m_n$.
\end{itemize}\bigskip

\hypertarget{teoria-5:PTF}{\textit{Pequeño teorema de Fermat}}
\begin{itemize}
  \item Sea $p$ primo, y sea $a \en \enteros$. Entonces:
        \begin{enumerate}[label=\arabic*)]
          \item $ \congruencia{a^p}{a}{p} $
          \item $ p \noDivide a \entonces \congruencia{a^{p-1}}{1}{p} $
        \end{enumerate}

  \item Sea $p$ primo, entonces $ \paratodo a \en \enteros$ tal que $ p \noDivide a$ se tiene:
        $$
          \boxed{
            \congruencia{a^n}{a^{r_{\red{p-1}}(n)}}{p} ,\, \paratodo n \en \naturales
          }
        $$
        Amigate con ésta porque se usa mucho. Marco el \red{$p-1$} en rojo, porque por alguna razón uno se olvida.

  \item Sea $a \en \enteros$ y $p > 0$ primo tal que $\ob{(a:p) = 1}{p \noDivide a}$, y sea
        $\purple{d} \en \naturales$ con $\purple{d} \leq p - 1$
        el mínimo tal que:
        $$
          \boxed{
            \congruencia{a^{\purple{d}}}{1}{p} \entonces \purple{d} \divideA (p-1)
          }
        $$
        Atento a esto que en algún que otro ejercicio uno encuentra un valor usando PTF, pero eso no quiere
        decir que no haya \underline{otro valor menor}! Que habrá que encontrar con otro método.\par

        \textit{Nota:} Cuando $p$ es primo y $a$ un entero cualquiera, será obvio o no, pero: \boxed{p \noDivide a \sii p \cop a}.
        Se usan indistintamente.
\end{itemize}
