\begin{enunciado}{\ejExtra}\fechaEjercicio{2do parcial 4/7/25}
  \begin{enumerate}[label=(\alph*)]
    \item Sea $\omega \en G_7,\, \omega \distinto 1$. Probar que
          $(\omega^{16} - \omega^5 + \omega^{29} - \bar{\omega} - \omega^{-18} + \omega^{11})^5$ es imaginario puro.
    \item Hallar todos los $\omega \en G_{60}$ que verifican $1 + \omega^2 = - \omega^4(1 + \omega^2)$.
  \end{enumerate}
\end{enunciado}

\begin{enumerate}[label=(\alph*)]
  \item Esa expresión está para ser acomodada usando las propiedades que tienen los números de los grupos $G_n$,
        en particular para $\omega \en G_7$:
        $$
          \llave{rcl}{
            \omega^7 & = & 1\\
            \omega^n & = & \omega^{r_7(n)}\\
            \conj{\omega^n} & = & \omega^{-n}\\
            \sumatoria{i = 0}{6} \omega^k & = & 0
          }
        $$
        Usando esas propiedades queda algo así:
        $$
          \begin{array}{rcl}
            (\omega^{16} - \omega^5 + \omega^{29} - \bar{\omega} - \omega^{-18} + \omega^{11})^5
             & \igual{\red{!!}} &
            (\omega^2 - \omega^5 + \omega - \omega^6 - \omega^3 + \omega^4)^5 \\
             & =                &
            (\omega + \omega^2 - \omega^3 + \omega^4 - \omega^5 - \omega^6)^5 \\
             & =                &
            \blue{z}^5                                                        \\
          \end{array}
        $$
        Ahí esta la expresión un poco más bonita y reescribí a la base como $\blue{z}$. Para probar que un número es imaginario puro,
        puedo calcular que su parte real valga 0:
        $$
          \textstyle
          \text{Si } \blue{z} \en \complejos,~ \re(\blue{z}) = \frac{\blue{z} + \blue{\bar{z}}}{2} = 0
          \sii
          \blue{z} \text{ es imaginario puro o cero}.
        $$
        Calculo usando las propiedades:
        $$
          \begin{array}{rcl}
            \blue{z} + \blue{\bar{z}}
             & =                &
            \omega + \omega^2 - \omega^3 + \omega^4 - \omega^5 - \omega^6 + \conj{\omega + \omega^2 - \omega^3 + \omega^4 - \omega^5 - \omega^6}                                    \\
             & =                &
            \omega + \omega^2 - \omega^3 + \omega^4 - \omega^5 - \omega^6 + \conj{\omega} + \conj{\omega^2} - \conj{\omega^3} + \conj{\omega^4} - \conj{\omega^5} - \conj{\omega^6} \\
             & \igual{\red{!!}} &
            \omega + \omega^2 - \omega^3 + \omega^4 - \omega^5 - \omega^6 + \omega^6 + \omega^5 - \omega^4 + \omega^3 - \omega^2 - \omega                                           \\
             & \igual{\red{!}}  &
            0
          \end{array}
        $$
        Es así que $\re(\blue{z}) = 0$ por lo tanto $(\re(\blue{z}))^5 = 0$ y $\blue{z}^5$ es imaginario puro.

  \item Acomodo el enunciado:
        $$
          1 + \omega^2 = - \omega^4(1 + \omega^2)
          \sii
          (1 + \omega^2) \cdot (1 + \omega^4) = 0
          \sii
          \llave{l}{
            \omega^2 = -1  \sii \omega \en \set{G_4 - G_2}\\
            \omega^4 = -1 \sii \omega \en \set{G_8 - G_4}
          }
          \llamada1
        $$
        Ese último resultado puede ser un poco oscuro, calculemos los resultados a manopla:
        $$
          \textstyle
          \omega^4 = -1
          \sii
          r^4e^{i 4\theta} = e^{i\pi}
          \sii
          \llave{rcl}{
            r^4 = 1 &\sii& r = 1 \\
            4\theta = \pi + \blue{2k\pi} &\sii & \theta = \frac{2k + 1}{4}\pi \text{ con } k \en [0,3]
          }
          \sii
          \omega
          \en
          \set{
            e^{i\frac{\pi}{4}},
            e^{i\frac{3\pi}{4}},
            e^{i\frac{5\pi}{4}},
            e^{i\frac{7\pi}{4}}
          }\llamada2
        $$
        Esté más fácil lo hago de una:
        $$
          \omega^2 = -1
          \sii
          \omega
          \en
          \set{
            e^{i\frac{\pi}{2}},
            e^{i\frac{3\pi}{2}}
          } \llamada3
        $$
        La unión de $\llamada2$ y $\llamada3$ son las soluciones de $\llamada1$:
        $$
          \set{
            e^{i\frac{\pi}{2}},
            e^{i\frac{3\pi}{2}},
            e^{i\frac{\pi}{4}},
            e^{i\frac{3\pi}{4}},
            e^{i\frac{5\pi}{4}},
            e^{i\frac{7\pi}{4}}
          }
        $$
        Esto de la inclusión o intersección de un conjunto $G_n$ en otro $G_m$ lo podés
        (\hyperlink{teoria6:gruposGn}{mirar acá \click}), por lo tanto los $\omega$
        que están en $G_{60}$ son los que cumplen:
        $$
          \omega
          \en
          \set{
            e^{i\frac{\pi}{2}},
            e^{i\frac{3\pi}{2}},
            e^{i\frac{\pi}{4}},
            e^{i\frac{3\pi}{4}},
            e^{i\frac{5\pi}{4}},
            e^{i\frac{7\pi}{4}}
          }
          \quad \land \quad
          \omega^{60} = 1
          \Sii{\red{!!}}
          \cajaResultado{
            \omega \en
            \set{
              e^{i\frac{\pi}{2}},
              e^{i\frac{3\pi}{2}}
            }
          }
        $$
        El exponente tiene que ser un múltiplo para de $\pi$. Fin

        \bigskip

        \textit{Nota que nadie pidió pero que pintó poner:}

        \parrafoDestacado{
          $\llamada1$
          Fijate que esas ecuaciones son extremadamente parecidas a $G_2$ y $G_4$, solo cambia que del otro lado hay un signo negativo:
          $$
            \omega^2 = -1
            \sii
            (re^{i \theta})^2 = e^{i\pi}
            \sii
            \llave{rcl}{
              2\theta & = &  \pi + \blue{2k\pi}
              \sii
              \theta = \frac{2k + 1}{2}\pi \text{ con } k \en [0,1]
              \\
              r  & = & 1
            }
          $$
          $$
            \omega^4 = -1
            \sii
            (re^{i \theta})^4 = e^{i\pi}
            \sii
            \llave{rcl}{
              4\theta & = &  \pi + \blue{2k\pi}
              \sii
              \theta = \frac{2k + 1}{4}\pi \text{ con } k \en [0,3]
              \\
              r  & = & 1
            }
          $$
          ¿Se ve el parecido? Las soluciones son lo mismo, pero están rotadas, para $G_n$ en un ángulo $\magenta{\frac{\pi}{n}}$ en sentido antihorario.
          $$
            \begin{array}{rcl}
              (\ub{
                e^{i \frac{2k}{n}\pi}
              }{
              \omega \en G_n \text{ de }                                            \\
                \text{toda la vida}
              }
              )^n = 1
               & \Sii{\red{!}} &
              (e^{i \frac{2k}{n}\pi})^n \cdot e^{i\pi} = -1                         \\
               & \Sii{\red{!}} &
              (e^{i \frac{2k}{n}\pi})^n \cdot (e^{i\magenta{\frac{\pi}{n}}})^n = -1 \\
               & \Sii{\red{!}} &
              \big(
              \ub{e^{i (\frac{2k}{n}\pi + \frac{\pi}{n})}
              }{
              \omega's \text{ rotadas }                                             \\
                \text{en } \frac{\pi}{n}
              }
              \big)^n = -1
            \end{array}
          $$
          Si la acción de \textit{rotar} te marea, pensá que estás sumándole algo al \textit{argumento}, entonces
          estás moviéndolo al rededor del origen. Mirá los gráficos así ves lo que pasa.
        }
        $$
          \begin{tikzpicture}[baseline=0,scale = 2.3, every node/.style={font=\tiny}]
            \draw[ultra thin,->,gray] (-1.5,0) -- (1.7,0) node[below] {Re};
            \draw[ultra thin,->,gray] (0,-1.5) -- (0,1.7) node[right] {Im};
            \draw[ultra thin] (0,0) circle (1);
            \node at (-1,1.5)  {\blue{$G_4$} y \magenta{$G_4$ rotada}};
            \foreach \x in {0,...,4} {
                \filldraw (\x*360/4:1) circle (0.5pt);
                \filldraw (\x*360/4 + 180/4:1) circle (0.5pt);
                \ifnum\x<4
                  \draw[Cerulean] ({\x*360/4}:1) -- ({(\x+1)*360/4}:1);
                  \draw[magenta] ({\x*360/4 + 180/4}:1) -- ({(\x+1)*360/4 + 180/4}:1);
                  \node at ({\x*360/4}:1.3)  { $e^{i\frac{{2 \cdot \x }}{4}\pi}$};
                  \node at ({\x*360/4 + 180/4}:1.3)  { $e^{i\frac{{2 \cdot \x + 1}}{4}\pi}$};
                \fi
              }
            \draw[dotted] (0,0) -- (45:1);
            \draw[-Latex, thin, orange] (1,0) arc  (0:45:1)node [midway, right]{$\frac{\pi}{4}$};
          \end{tikzpicture}
          \quad
          \begin{tikzpicture}[baseline=0,scale = 2.3, every node/.style={font=\tiny}]
            \draw[ultra thin,->,gray] (-1.5,0) -- (1.7,0) node[below] {Re};
            \draw[ultra thin,->,gray] (0,-1.5) -- (0,1.7) node[right] {Im};
            \draw[ultra thin] (0,0) circle (1);
            \node at (-1,1.5)  {\blue{$G_2$} y \magenta{$G_2$ rotada}};
            \foreach \x in {0,...,2} {
                \filldraw (\x*360/2:1) circle (0.5pt);
                \filldraw (\x*360/2 + 180/2:1) circle (0.5pt);
                \ifnum\x<2
                  \draw[Cerulean] ({\x*360/2}:1) -- ({(\x+1)*360/2}:1);
                  \draw[magenta] ({\x*360/2 + 180/2}:1) -- ({(\x+1)*360/2 + 180/2}:1);
                  \node at ({\x*360/2}:1.3)  { $e^{i\frac{{2 \cdot \x }}{2}\pi}$};
                  \node at ({\x*360/2 + 180/2}:1.3)  { $e^{i\frac{{2 \cdot \x + 1}}{2}\pi}$};
                \fi
              }
            \draw[dotted] (0,0) -- (90:1);
            \draw[-Latex, thin, orange] (1,0) arc  (0:90:1) node [midway, right]{$\frac{\pi}{2}$};
          \end{tikzpicture}
        $$
        \textit{Fin nota que nadie pidió pero que pintó poner}
\end{enumerate}

\begin{aportes}
  \item \aporte{\dirRepo}{naD GarRaz \github}
\end{aportes}
