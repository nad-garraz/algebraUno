\begin{enunciado}{\ejExtra}
  Sea $z = \frac{\sqrt{3}}{2} - \frac{1}{2}i$. Hallar todos los $n \en \naturales$ que cumplen simultáneamente
  las siguientes condiciones:
  \begin{itemize}
    \item $8 \divideA 3n + |z^3|$
    \item $\arg(z^{7n+6}) = \arg(i)$
  \end{itemize}
\end{enunciado}

En forma exponencial es más fácil laburar esas cosas (para mí):
$$
  \llave{l}{
    |z| = 1 \\
    \theta_z = \frac{11}{6}\pi
  }
  \to z = |z|e^{\theta_z i} =
  e^{i\frac{11}{6}\pi}
  \entonces
  z^3 = e^{i\frac{11}{2}\pi} = -1
  \sii |z^3| = 1
$$

\textit{Primera condición: }
$$
  8 \divideA 3n + |z^3| = 3n + 1
  \Sii{def}
  \congruencia{3n + 1}{0}{8}
  \sii
  \congruencia{3n}{7}{8}
  \Sii{$\times 3$}[$(\Leftarrow) 3 \cop 8$]
  {\congruencia{9n}{21}{8}}
  \sii
  \congruencia{n}{5}{8}  \Tilde
$$

\textit{Segunda condición: }
$$
  \arg(z^{7n+6}) = \arg(i)
  \sii
  \parentesis{e^{i\frac{11}{6}\pi}}^{7n+6} = e^{i\frac{\pi}{2}}
  \sii
  e^{i\frac{77}{6}\pi + 11 \pi} =
  e^{i\frac{\pi}{2}}
  \sii
  \frac{77}{6}n\pi + 11 \pi = \frac{\pi}{2} + \magenta{2k\pi}
$$
Hay 2 números como para armar una ecuación de congruencias. Despejo $n$:
$$
  \frac{77}{6}n + 11 = \frac{1}{2} + 2k
  \sii
  77n = -63 + 12k
  \Sii{def}
  \congruencia{77n}{-63}{12}
  \Sii{\red{!}}
  \congruencia{5n}{-3}{12}
  \Sii{\red{!} \quad $\times 5$}[$(\Leftarrow) 5 \cop 12$]
  \congruencia{n}{9}{12}
$$
Junto la info que saqué de cada una de las condiciones del enunciado y
armo sistemita:
$$
  \llave{l}{
    \congruencia{n}{9}{12} \\
    \congruencia{n}{5}{8}
  }
  \Sii{quiero}[\textit{divisores coprimos}]
  \llave{l}{
    \congruencia{n}{1}{4} \blue{\Tilde}                 \\
    \congruencia{n}{0}{3}                               \\
    \congruencia{n}{1}{4} \magenta{\Tilde}\blue{\Tilde} \\
    \congruencia{n}{1}{2} \magenta{\Tilde}
  }
  \Sii{me quedo con el}[de mayor divisor]
  \llave{l}{
    \congruencia{n}{0}{3} \llamada1 \\
    \congruencia{n}{5}{8} \llamada2 \\
  }
$$
Ahora sí, tengo el sistema con \red{\textit{divisores coprimos}},
por \href{\chinito}{T\faIcon{cannabis}R} tengo solución, hay que resolver el sistema:

De $\llamada1$:
$$
  n = 3k \llamada3
$$
Reemplazo en $\llamada2$ y hago cuentitas:
$$
  \congruencia{3k}{5}{8}
  \Sii{$\times 3$}[$(\Leftarrow) 3 \cop 5$]
  \congruencia{k}{7}{8}
  \sii
  k = \yellow{8j + 7}
$$
Meto $k$ en $\llamada3$ y finito:
$$
  n = 3 (\yellow{8j + 7}) = 24j + 21
  \sii
  \cajaResultado{\congruencia{n}{21}{24}}
$$

\begin{aportes}
  \item \aporte{\dirRepo}{naD GarRaz \github}
\end{aportes}
