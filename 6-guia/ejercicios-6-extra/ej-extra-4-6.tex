\begin{enunciado}{\ejExtra}
  Sea $w = e^{\frac{\pi}{18} i}$. Hallar todos los $n\en \naturales$ que cumplen simultáneamente:
  $$
    \sumatoria{k=0}{5n+1}w^{3k} = 0 \qquad
    \sumatoria{k=0}{4n+6}w^{4k} = 0.
  $$
  Expresar la solución como una única ecuación de congruencia.
\end{enunciado}

Por enunciado tengo que:
$$
  w = e^{\frac{1}{18} \pi i}
  \sii
  \llave{l}{
    w^3 = e^{\frac{1}{6} \pi i} \distinto  1 \\
    w^4 = e^{\frac{2}{9} \pi i} \distinto  1
  },
$$
puedo usar la serie geométrica.

$$
  \sumatoria{k=0}{5n+1}w^{3k} =
  \sumatoria{k=0}{5n+1}(w^3)^k =
  \frac{(w^3)^{5n+2} - 1}{w^3 - 1} = 0
  \sii
  (w^3)^{5n+2} = 1.
$$
Queda una ecuación para encontrar $w$:
$$
  \begin{array}{c}
    (w^3)^{5n+2} = 1
    \Sii{laburo}[exponente]
    \frac{15n + 6}{18}\pi = 2k\pi
    \sii
    5n + 2 = 12k
    \Sii{def}
    \congruencia{5n}{10}{12}\llamada1
  \end{array}
$$
Y tenemos una ecuación. Calculo la otra sumatoria:
$$
  \sumatoria{k=0}{4n+6}w^{4k} =
  \sumatoria{k=0}{4n+6}(w^4)^k =
  \frac{(w^4)^{4n+7} - 1}{w^4 - 1} = 0
  \sii
  (w^4)^{4n+7} = 1 \\
$$
Igual que antes, busco los $w$ que satisfacen:
$$
  \begin{array}{c}
    (w^4)^{4n+7} = 1
    \Sii{laburo}[exponente]
    \frac{16n + 28}{18}\pi = 2k\pi
    \sii
    4n + 7 = 9k
    \Sii{def}
    \congruencia{4n}{2}{9}\llamada2
  \end{array}
$$
Con la segunda ecuación armo sistema y \href{\chinito}{TCH}:
$$
  \begin{array}{c}
    \llamada1 \\
    \llamada2
  \end{array}
  \llave{l}{
    \congruencia{n}{2}{12} \\
    \congruencia{n}{5}{9}
  }
  \leftrightsquigarrow
  \llave{l}{
    \congruencia{n}{2}{3} \\
    \congruencia{n}{2}{4} \\
    \congruencia{n}{2}{3}
  }
  \taa{\red{!}}{\leftrightsquigarrow}
  \llave{l}{
    \congruencia{n}{2}{4} \\
    \congruencia{n}{5}{9}
  }
  \flecha{$9 \cop 4$ hay}[solución por \href{\chinito}{TCR}]
  \cajaResultado{\congruencia{n}{14}{36}}
$$

% Contribuciones
\begin{aportes}
  \item \aporte{\dirRepo}{naD GarRaz \github}
  \item \aporte{https://github.com/JowinTeran}{Ale Teran \github}
\end{aportes}
