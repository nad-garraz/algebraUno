\begin{enunciado}{\ejExtra}
	Sea $\omega \en G_{10}$ tal que $\omega^5 \distinto 1$. Encuentre la parte real de
	$$
		\omega + \omega^{-7} + \conj{\omega}^6 + \omega^8 + \sumatoria{k=5}{98}\omega^{5k}.
	$$
\end{enunciado}
\hyperlink{teoria6:propiedadesGn}{Algunas resultados de este tema acá {\tiny $\ot$ click} }


Dado que $\omega \en G_{10}$ ocurre:
$$
	\omega^5 = \left(e^{i \frac{2k\pi}{10}}\right)^5
	\sii
	\omega^5 =
	\llave{cl}{
		1 & \text{si $k$ es par}\\
		-1 & \text{si $k$ es impar}
	}
	\Entonces{\red{enunciado}}
	\omega^5 = -1
$$
Con ese \textit{resultadillo} ahora podemos reescribir el enunciado como:
$$
	\begin{array}{rcl}
		\omega + \omega^{-7} + \conj{\omega}^6 + \omega^8 + \sumatoria{k = 5}{98}\omega^{5k}
		 & \igual{$\llamada1$}[\red{!!}] &
		\omega + \omega^3 + \omega^4 + \omega^8 + \red{0}                                     \\
		 & \igual{\red{!}}               & \omega + \omega^3 + \omega^4 + (-1) \cdot \omega^3 \\
		 & \igual{\red{!}}               & \omega + \omega^4                                  \\
		 & \igual{\red{!}}               & \omega + (-1) \cdot \conj{\omega}                  \\
		 & \igual{$\llamada2$}           & \omega - \conj{\omega} = i \cdot 2\im(\omega)
	\end{array}
$$

En $\llamada1$ la sumatoria es una suma onda $1 - 1 + 1 - 1 + \cdots$ donde
se cancela todo.

En $\llamada2$ hago $\omega^4 = \frac{1}{\omega} \cdot \omega^5$ y un poco de acomodar.

Es así que si la expresión es igual a un número imaginario puro se concluye:
$$
	\cajaResultado{
		\re(\omega + \omega^{-7} + \conj{\omega}^6 + \omega^8 + \sumatoria{k = 5}{98}\omega^{5k}) =  0.
	}
$$

\begin{aportes}
	\item \aporte{\dirRepo}{naD GarRaz \github}
\end{aportes}
