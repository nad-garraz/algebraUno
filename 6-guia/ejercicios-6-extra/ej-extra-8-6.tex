\begin{enunciado}{\ejExtra}
  Sea $\omega = e^{\frac{2\pi i}{33}}$. Encuentre todos los $n \en \naturales$ tales que
  $\conj{\omega}^6 \en G_{2n+1}$ y
  $$
    \sumatoria{j=0}{n+4} \omega^{11j} = 0.
  $$
\end{enunciado}

\begin{center}
  $\to$ Mirá \hyperlink{teoria6:propiedadesGn}{estas propiedades} y  \hyperlink{teoria6:gruposGn}{estos gráficos} para sacar intuición de lo que viene
\end{center}

Del enunciado y propiedades de estos números de $G_n$:
$$
  \omega = e^{i\frac{2\pi }{33}}
  \Sii{\red{!}}
  \conj{\omega}^6 = e^{i\frac{18\pi }{11}}
$$

Para que ese número espantoso  $\conj{\omega}^6 \en G_{2n+1}$ debe ocurrir que \red{!!}:
$$
  11 \divideA 2n+1
  \Sii{def}
  \congruencia{2n+1}{0}{11}
  \sii
  \congruencia{n}{5}{11} \llamada1
$$

Por otro lado para que la sumatoria del enunciado de cero, sabiendo que $\omega \distinto 1$, uso suma geométrica:
$$
  \sumatoria{j=0}{n+4} \omega^{11j}
  =
  \frac{\left(\omega^{11} \right)^{n + 5} - 1}{\omega - 1} = 0
  \sii
  \left(\omega^{11} \right)^{n + 5} = 1
  \sii
  \left(e^{\frac{2\pi i}{33}}\right)^{11n + 55} \igual{\red{!}} e^{2\blue{h}\pi}
  \Sii{\red{!}}
  \frac{22\pi}{33}n + \frac{110\pi}{33}n = 2\blue{h}\pi
$$
Esa última ecuación da {\tiny salacadula chalchicomula} \magic:
$$
  \congruencia{n}{1}{3} \llamada2
$$

Por lo tanto $n \en \naturales$ debe cumplir $\llamada1$ y $\llamada2$:
$$
  \llave{l}{
    \congruencia{n}{5}{11}\\
    \congruencia{n}{1}{3}
  }
  \Sii{\href{\chinito}{TCH}}
  \congruencia{n}{16}{33}
$$

Y si no mandé mucha fruta los $n \en \naturales$ que cumplen serían:
$$
  \cajaResultado{
    \congruencia{n}{16}{33}
    \quad \text{con} \quad
    n > 0
  }
$$

\begin{aportes}
  \item \aporte{\dirRepo}{naD GarRaz \github}
\end{aportes}
