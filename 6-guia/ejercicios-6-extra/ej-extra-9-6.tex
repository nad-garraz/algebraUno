\begin{enunciado}{\ejExtra}\fechaEjercicio{2do parcial 7/4/25}
  \begin{enumerate}[label=(\alph*)]
    \item Sea $w \en G_7, ~ w \distinto 1$, probar que
          $(w^{16} - w^5 + w^{29} - \bar{w} - w^{-18} + w^{11})^3$
          es imaginario puro

    \item Hallar los $w \en G_{36}$ que verifican $1 + w^2 = - w^4(1 + w^2)$
  \end{enumerate}
\end{enunciado}

\begin{enumerate}[label=(\alph*)]
  \item Esa expresión está para ser acomodada usando las propiedades que tienen los números de los grupos $G_n$,
        en particular para $w \en G_7$:
        $$
          \llave{rcl}{
            w^7 & = & 1\\
            w^n & = & w^{r_7(n)}\\
            \conj{w^n} & = & w^{-n}\\
            \sumatoria{i = 0}{6} w^k & = & 0
          }
        $$
        Usando esas propiedades queda algo así:
        $$
          \begin{array}{rcl}
            (w^{16} - w^5 + w^{29} - \bar{w} - w^{-18} + w^{11})^3
             & \igual{\red{!!}} &
            (w^2 - w^5 + w - w^6 - w^3 + w^4)^3 \\
             & =                &
            (w + w^2 - w^3 + w^4 - w^5 - w^6)^3 \\
             & =                &
            \blue{z}^3                          \\
          \end{array}
        $$
        Ahí esta la expresión un poco más bonita y reescribí a la base como $\blue{z}$. Para probar que un número es imaginario puro,
        puedo calcular que su parte real valga 0:
        $$
          \textstyle
          \text{Si } \blue{z} \en \complejos,~ \re(\blue{z}) = \frac{\blue{z} + \blue{\bar{z}}}{2} = 0
          \sii
          \blue{z} \text{ es imaginario puro o cero}.
        $$
        Calculo usando las propiedades:
        $$
          \begin{array}{rcl}
            \blue{z} + \blue{\bar{z}}
             & =                &
            w + w^2 - w^3 + w^4 - w^5 - w^6 + \conj{w + w^2 - w^3 + w^4 - w^5 - w^6}                                    \\
             & =                &
            w + w^2 - w^3 + w^4 - w^5 - w^6 + \conj{w} + \conj{w^2} - \conj{w^3} + \conj{w^4} - \conj{w^5} - \conj{w^6} \\
             & \igual{\red{!!}} &
            w + w^2 - w^3 + w^4 - w^5 - w^6 + w^6 + w^5 - w^4 + w^3 - w^2 - w                                           \\
             & \igual{\red{!}}  &
            0
          \end{array}
        $$
        Es así que $\re(\blue{z}) = 0$ por lo tanto $(\re(\blue{z}))^3 = 0$ y $\blue{z}^3$ es imaginario puro.

  \item
        Acomodo el enunciado:
        $$
          1 + \omega^2 = - \omega^4(1 + \omega^2)
          \sii
          (1 + \omega^2) \cdot (1 + \omega^4) = 0
          \sii
          \llave{l}{
            \omega^2 = -1  \sii \omega \en \set{G_4 - G_2}\\
            \omega^4 = -1 \sii \omega \en \set{G_8 - G_4}
          }
          \llamada1
        $$
        Ese último resultado puede ser un poco oscuro, calculemos los resultados a manopla:
        $$
          \textstyle
          \omega^4 = -1
          \sii
          r^4e^{i 4\theta} = e^{i\pi}
          \sii
          \llave{rcl}{
            r^4 = 1 &\sii& r = 1 \\
            4\theta = \pi + \blue{2k\pi} &\sii & \theta = \frac{2k + 1}{4}\pi \text{ con } k \en [0,3]
          }
          \sii
          \omega
          \en
          \set{
            e^{i\frac{\pi}{4}},
            e^{i\frac{3\pi}{4}},
            e^{i\frac{5\pi}{4}},
            e^{i\frac{7\pi}{4}}
          }\llamada2
        $$
        Esté más fácil lo hago de una:
        $$
          \omega^2 = -1
          \sii
          \omega
          \en
          \set{
            e^{i\frac{\pi}{2}},
            e^{i\frac{3\pi}{2}}
          } \llamada3
        $$
        La unión de $\llamada2$ y $\llamada3$ son las soluciones de $\llamada1$:
        $$
          \set{
            e^{i\frac{\pi}{2}},
            e^{i\frac{3\pi}{2}},
            e^{i\frac{\pi}{4}},
            e^{i\frac{3\pi}{4}},
            e^{i\frac{5\pi}{4}},
            e^{i\frac{7\pi}{4}}
          }
        $$
        Esto de la inclusión o intersección de un conjunto $G_n$ en otro $G_m$ lo podés
        (\hyperlink{teoria6:gruposGn}{mirar acá \click}), por lo tanto los $\omega$
        que están en $G_{36}$ son los que cumplen:
        $$
          \omega
          \en
          \set{
            e^{i\frac{\pi}{2}},
            e^{i\frac{3\pi}{2}},
            e^{i\frac{\pi}{4}},
            e^{i\frac{3\pi}{4}},
            e^{i\frac{5\pi}{4}},
            e^{i\frac{7\pi}{4}}
          }
          \quad \land \quad
          \omega^{36} = 1
          \Sii{\red{!!}}
          \cajaResultado{
            \omega \en
            \set{
              e^{i\frac{\pi}{2}},
              e^{i\frac{3\pi}{2}}
            }
          }
        $$
        El exponente tiene que ser un múltiplo para de $\pi$. Fin
\end{enumerate}

\begin{aportes}
  \item \aporte{\dirRepo}{naD GarRaz \github}
\end{aportes}
