\begin{enunciado}{\ejercicio}
  Para los siguientes $z \en \complejos$, hallar $\re(z), \im(z), |z|, \re(z^{-1})$ e $\im(i \cdot z)$
  \begin{multicols}{2}
    \begin{enumerate}[label=\roman*)]
      \item $z = 5 i (1+i)^4$
      \item $z = (\sqrt{2} + \sqrt{3} i)^2 (\conj{1 - 3i})$
      \item $z=i^{17} + \frac{1}{2} i (1 - i)^3$
      \item $z = \left(\frac{1}{\sqrt{2}} + \frac{1}{\sqrt{2}}i\right)^{10}$
      \item $z = \left(-\frac{1}{2} + \frac{\sqrt{3}}{2}i\right)^{-1}$.
    \end{enumerate}
  \end{multicols}
\end{enunciado}

\begin{enumerate}[label=\roman*)]

  \item  Cuando hay muchos productos, me gusta pasar todo a notación exponencial y jugar desde ahí:
        $$
          z =
          \magenta{5} \cdot \blue{i} \cdot \purple{(1+i)^4}
          \igual{\red{!!}}
          \magenta{5} \cdot \blue{e^{i\frac{\pi}{2}}} \cdot \purple{(\sqrt{2} \cdot e^{i \frac{\pi}{4}})^4}
          =
          5 \cdot e^{i\frac{\pi}{2}} \cdot (4 \cdot  e^{i \pi})
          =
          20 \cdot e^{i \frac{3}{2}\pi}
          \igual{\red{i}}
          -20i
        $$

        Por lo tanto si:
        $$
          z \cdot z^{-1} = 1
          \Entonces{$z = -20i$}
          z^{-1} = \frac{1}{20}i
        $$
        Finalmente:
        $$
          \re(z) = 0 ,\,
          \im(z) = -20 ,\,
          |z| = 20 ,\,
          \re(z^{-1}) = 0,\,
          \im(i \cdot z) = 20
        $$

  \item  
          A ojo, o casi, veo que los valores de los argumentos de los factores son feos. Recordar que hay muy pocos ángulos que tienen resultados
                agradables, los de la tablita, \hyperlink{teoria-6:tablita}{tablita de ángulos agradables}.
        $$
        z =
                \blue{(\sqrt{2} + \sqrt{3} i)^2} \cdot \purple{(\conj{1 - 3i})}
        \igual{\red{!!}}
                \magenta{\sqrt{7} e^{}<++>} \cdot \blue{e^{i\frac{\pi}{2}}} \cdot \purple{(\sqrt{2} \cdot e^{i \frac{\pi}{4}})^4}
        =
        5 \cdot e^{i\frac{\pi}{2}} \cdot (4 \cdot  e^{i \pi})
        =
        20 \cdot e^{i \frac{3}{2}\pi}
        \igual{\red{i}}
        -20i
        $$

          Por lo tanto si:
        $$
        z \cdot z^{-1} = 1
        \Entonces{$z = -20i$}
        z^{-1} = \frac{1}{20}i
        $$
          Finalmente:
        $$
        \re(z) = 0 ,\,
        \im(z) = -20 ,\,
        |z| = 20 ,\,
        \re(z^{-1}) = 0,\,
        \im(i \cdot z) = 20
        $$
\end{enumerate}

