\begin{enunciado}{\ejercicio}
  Para los siguientes $z \en \complejos$, hallar $\re(z), \im(z), |z|, \re(z^{-1})$ e $\im(i \cdot z)$
  \begin{multicols}{2}
    \begin{enumerate}[label=\roman*)]
      \item $z = 5 i (1+i)^4$
      \item $z = (\sqrt{2} + \sqrt{3} i)^2 (\conj{1 - 3i})$
      \item $z=i^{17} + \frac{1}{2} i (1 - i)^3$
      \item $z = \left(\frac{1}{\sqrt{2}} + \frac{1}{\sqrt{2}}i\right)^{10}$
      \item $z = \left(-\frac{1}{2} + \frac{\sqrt{3}}{2}i\right)^{-1}$.
    \end{enumerate}
  \end{multicols}
\end{enunciado}
Cosas para tener en cuenta sobre notación y algunos resultados:

En notación binomial:
$$
  z = a + ib = \re(z) + i \cdot \im(z)
  \flecha{donde}
  \llave{l}{
    a = \re(z) \en \reales                                                   \\
    b = \im(z) \en \reales                                                   \\
    z^{-1} = \frac{a -ib}{a^2 + b^2} = \frac{\conj{z}}{|z|^2} \\
    \magenta{\im(i \cdot z)} = \im(i\cdot a - b) = a = \magenta{\re(z)}
  }
$$
Y en notación exponencial:
$$
  z = r \cdot e^{i\theta} \flecha{donde}[$r>0$]
  \llave{l}{
    r \cdot \cos(\theta) = \re(z) \en \reales                                                                                           \\
    r \cdot \sin(\theta) = \im(z) \en \reales                                                                                           \\
    z^{-1} = \frac{1}{r} \cdot e^{-i\theta} = \frac{1}{r} \cdot \frac{\conj{z}}{r} = \frac{\conj{z}}{r^2} \\
    r\cdot e^{i \theta} = r \cdot (\cos(\theta) + i \sin(\theta))
  }
$$
\begin{enumerate}[label=\roman*)]

  \item  Cuando hay muchos productos, me gusta pasar todo a notación exponencial y jugar desde ahí:
        $$
          z =
          \magenta{5} \cdot \blue{i} \cdot \purple{(1+i)^4}
          \igual{\red{!!}}
          \magenta{5} \cdot \blue{e^{i\frac{\pi}{2}}} \cdot \purple{(\sqrt{2} \cdot e^{i \frac{\pi}{4}})^4}
          =
          5 \cdot e^{i\frac{\pi}{2}} \cdot (4 \cdot  e^{i \pi})
          =
          20 \cdot e^{i \frac{3}{2}\pi}
          \igual{\red{i}}
          -20i
        $$

        Por lo tanto si:
        $$
          z \cdot z^{-1} = 1
          \Entonces{$z = -20i$}
          z^{-1} = \frac{1}{20}i
        $$
        Finalmente:
        $$
          \llave{l}{
            \re(z) = 0      \\
            \im(z) = -20    \\
            |z| = 20        \\
            \re(z^{-1}) = 0 \\
            \im(i \cdot z) = \re(z) = 0
          }
        $$

  \item
        A ojo, o casi, veo que los valores de los argumentos de los factores son feos. Recordar que hay muy pocos ángulos que tienen resultados
        agradables, los de la tablita, \hyperlink{teoria-6:tablita}{tablita de ángulos agradables}.

        Dado que el exponente más alto es 2, se puede distribuir sin morir en el intento:
        $$
          z =
          \blue{(\sqrt{2} + \sqrt{3} i)^2} \cdot \purple{(\conj{1 - 3i})}
          \igual{\red{!}}
          \blue{(-1 + 2\sqrt{6}i)} \cdot \purple{(1+ 3i)}
          \igual{\red{!!}}
          -1 - 6\sqrt{6} + i(2\sqrt{6} - 3)
        $$
        Donde en \red{!!} es distribuir y luego sacar factor común en los términos con $i$, nada extraño.

        Después de hacer las cuentas pertinentes:
        $$
          \llave{l}{
            \re(z) = -(1+6\sqrt{6})      \\
            \im(z) = 2\sqrt{6} - 3    \\
            |z| = 5\sqrt{10}         \\
            \re(z^{-1}) = -\frac{(1+6\sqrt{6})}{250} \\
            \im(i \cdot z) = \re(z) = -(1+6\sqrt{6})
          }
        $$

  \item Atento a que $i^4 \igual{$\llamada1$} 1$:
        $$
          z = i^{17} + \frac{1}{2} i (1 - i)^3
          =
          i \cdot (i^4)^4 + \frac{1}{2} \cdot e^{i \frac{\pi}{2}} \cdot ( \sqrt{2} \cdot e^{i \frac{7}{4}\pi})^3
          \igual{$\llamada1$}
          i + \sqrt{2} \cdot e^{i ( \frac{1}{2} + \frac{21}{4}) \pi}
          =
          i + \sqrt{2} \cdot e^{i\frac{23}{4} \pi}
          \igual{\red{!!}}
          i + \sqrt{2} \cdot e^{i\frac{7}{4} \pi}
        $$
        En \red{!!} usé la periodicidad de la función exponencial, con el exponente complejo es $2\pi$-periódica.
        $$
          z = i + \sqrt{2}(\frac{1}{\sqrt{2}} - \frac{1}{\sqrt{2}}i) = 1
        $$
        Finalmente:
        $$
          \llave{l}{
            \re(z) = 1      \\
            \im(z) = 0      \\
            |z| = 1         \\
            \re(z^{-1}) = 1 \\
            \im(i \cdot z) = \re(z) = 1
          }
        $$
        $$
          \re(z) = 1 ,\,
          \im(z) = 0 ,\,
          |z| = 1 ,\,
          \re(z^{-1}) = 1,\,
          \im(i \cdot z) = i
        $$

  \item Fácil con exponenciales:
        $$
          z
          =
          \left(\frac{1}{\sqrt{2}} + \frac{1}{\sqrt{2}}i\right)^{10}
          \igual{\red{!}}
          (e^{i \frac{\pi}{4}})^{10}
          =
          e^{i\frac{5}{2}\pi}
          \igual{\red{!}}
          e^{i\frac{\pi}{2}}
          =
          i
        $$
        Finalmente:
        $$
          \re(z) = 0 ,\,
          \im(z) = 1 ,\,
          |z| = 1 ,\,
          \re(z^{-1}) \igual{\red{!}} \re(-i) = 0,\,
          \im(i \cdot z) = -1
        $$

  \item Fácil con exponenciales:
        $$
          z
          =(-\frac{1}{2} + \frac{\sqrt{3}}{2}i)^{-1}
          \igual{\red{!}}
          (e^{i \frac{4}{3}\pi})^{-1}
          =
          e^{-i\frac{4}{3}\pi}
          \igual{\red{!}}
          -\frac{1}{2} - \frac{\sqrt{3}}{2}i
        $$
        Finalmente:
        $$
          \llave{l}{
            \re(z) = -\frac{1}{2}                       \\
            \im(z) = -\frac{\sqrt{3}}{2} \\
            |z| = 1                                     \\
            \re(z^{-1}) = -\frac{1}{2}                  \\
            \im(i \cdot z) = -\frac{1}{2}
          }
        $$
\end{enumerate}

% Contribuciones
\begin{aportes}
  %% iconos : \github, \instagram, \tiktok, \linkedin
  %\aporte{url}{nombre icono}
  \item \aporte{\dirRepo}{naD GarRaz \github}
\end{aportes}
