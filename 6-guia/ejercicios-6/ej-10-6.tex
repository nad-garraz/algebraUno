\begin{enunciado}{\ejercicio}
  Hallar todos los $n \en \naturales$ para los cuales la ecuación $z^n + i\conj z^2 = 0$,
  tenga exactamente 6 soluciones y resolver en ese caso.
\end{enunciado}

$\flecha{acomodo la }[ecuación]
  z^n = -i\conj z^2
  \flecha{$r = |z|$, expreso todo}[en notación exponencial]
  \llaves{l}{
    z^n = r^n e^{n  \theta i} \\
    \conj z^2 = r^2 e^{-2\theta i} \\
    -i = e^{\frac{3}{2}\pi}
  }\checkmark\\
  \flecha{reescribo ecuación con}[notación exponencial]
  r^n e^{n \theta i} = r^2 e^{(\frac{3}{2}\pi - 2\theta)i}
  \sisolosi
  \llaves{l}{
    n \theta =\frac{3}{2}\pi - 2\theta  + \magenta{2k\pi}\quad (k \en \enteros)\\
    r^n = r^2 \to r^2 (r^{n-2} - 1) = 0
  }$\\

\textit{La ecuación de $r$: }\\
$r = 0$ aporta una solución trivial para cualquier $n \en \naturales$.\\
$r = 1$ es un comodín que me deja usar cualquier $n$ para jugar con la ecuación de $\theta$.\\
$n = 2$ es un valor que daría una solución para cada $r \en \reales_{\geq 0}$. \underline{No sirve} porque necesito solo 6 soluciones.\\

\textit{La ecuación de $\theta$: }\\
$\flecha{$r = 1$}[$n$ libre] (n + 2)\theta = (\frac{3}{2} + 2k) \pi
  \flecha{$n+2 \distinto 0$}[$\paratodo n \en \naturales$]
  \theta = \frac{1}{n+2}(\frac{3}{2} + 2k) \pi
  \flecha{$n = 3$\red{ Cómo justificar esto elegantemente?}}[5 porciones de $2k\pi$]
  \theta = \frac{3 + 4k}{10} \pi
$\\
\textit{Las 6 soluciones para $n = 3$: }\\
$z^n + i\conj z^2 = 0 \sisolosi
  \llave{l}{
    n = 3                          \\
    z = 0,\, \text{ cuando } r = 0 \\
    \text{ o }                     \\
    z_k = e^{\theta_k i} \text{ con } \theta_k = \frac{3 + 4k}{10} \pi ,\, k\en [0,4]
  }$
