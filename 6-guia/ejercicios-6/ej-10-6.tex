\begin{enunciado}{\ejercicio}
  Hallar todos los $n \en \naturales$ para los cuales la ecuación $z^n + i\conj z^2 = 0$,
  tenga exactamente 6 soluciones y resolver en ese caso.
\end{enunciado}

Pasar todo a notación exponencial:
$$
  z^n + i \conj z^2 = 0
  \sii
  z^n = -i\conj z^2
  \sii
  \llaves{l}{
    z^n = r^n e^{n  \theta i} \\
    \conj z^2 = r^2 e^{-2\theta i} \\
    -i = e^{\frac{3}{2}\pi}
  }
  \sii
  r^n e^{n \theta i} = r^2 e^{(\frac{3}{2}\pi - 2\theta)i}
$$
Esa ecuación se resuelve como siempre igualando los módulos y los argumentos, \textit{sin olvidar la periodicidad} de éste último:
$$
  \llave{rcl}{
    r^n      & = & r^2 \sii r^2 (r^{n-2} - 1) = 0 \llamada1                                                       \\
    n \theta & = & \frac{3}{2}\pi - 2\theta  + \magenta{2k\pi} \sii (n+2)\theta = (\frac{3}{2} + 2k)\pi \llamada2
  }
$$
\textit{La ecuación de $r \llamada1$: }\par
Analizo para cuales valores de $r$ y de $n$ se cumple la ecuación:
\begin{enumerate}[label={\tiny\faIcon{calculator}}]
  \item $r = 0$ Aporta una solución trivial para cualquier $n \en \naturales$ en la ecuación $z^n + i \conj z^2 = 0$. Pero solo habría una solución $z = 0$
        necesito encontrar otras 5.

  \item $\blue{r = 1}$ serviría. Quiere decir que voy a poder encontrar solución en $\llamada1$ que me deja usar cualquier $n$ para jugar con la ecuación de $\theta \llamada2$.

  \item $n = 2$ no sirve. Si bien cumple $\llamada1$ es un valor que daría una solución para cada $r \en \reales_{\geq 0}$. Pero tengo que tener solo 6 soluciones.
\end{enumerate}

\textit{La ecuación de $\theta \llamada2$: }\par
Por lo analizado antes, juego con $\blue{r = 1}$, eso no impone de momento ninguna condición sobre $n$:
$$
  (n + 2)\theta = (\frac{3}{2} + 2k) \pi
  \Sii{\red{!!}}[$\paratodo n \en \naturales_{\distinto 2}$]
  \theta = \frac{1}{n+2}(\frac{3}{2} + 2k) \pi
$$
Y ahora surge la pregunta: ¿Qué onda esto? Necesitamos 6 soluciones según el enunciado, pero a no olvidar que ya tenemos una solución proporcionada
por el $r = 0$. Así que ahora laburo el $\theta$ para que me de 5 soluciones y así tener 6 en total. Pido entonces $n=3$, para partir en 5 y obtener
de esta forma 5 valores para $\theta_k \en [0, 2\pi)$:
$$
  \theta_k =
  \frac{1}{5} \cdot \frac{3 + 4k}{2} \pi
  \sii
  \theta_k =
  \frac{3 + 4k}{10} \pi
  \quad \text{ con } k \en \set{0,1,2,3, 4}
$$
Finalmente para que la ecuación falopa esa tenga únicamente 6 soluciones, necesito que $\red{n = 3}$:
$$
  z^n + i\conj z^2 = 0
  \Sii{$\red{n = 3}$ para tener}[solo 6 soluciones]
  \llave{ll}{
    z = 0                                      & \text{ con } r = 0 \\
    z_{k=0} = e^{i \frac{3}{10}\pi } & \text{ con } r = 1 \\
    z_{k=1} = e^{i \frac{7}{10}\pi } & \text{ con } r = 1 \\
    z_{k=2} = e^{i \frac{11}{10}\pi } & \text{ con } r = 1 \\
    z_{k=3} = e^{i \frac{15}{10}\pi } & \text{ con } r = 1 \\
    z_{k=4} = e^{i \frac{19}{10}\pi } & \text{ con } r = 1
  }
$$

% Contribuciones
\begin{aportes}
  %% iconos : \github, \instagram, \tiktok, \linkedin
  %\aporte{url}{nombre icono}
  \item \aporte{https://github.com/nad-garraz}{naD GarRaz \github}
\end{aportes}
