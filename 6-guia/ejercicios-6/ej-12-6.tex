\begin{enunciado}{\ejercicio}
  \begin{enumerate}[label=\alph*)]
    \item Sea $w \en G_{36}$, $w^4 \distinto 1.$ Calcular $\sumatoria{k=7}{60}w^{4k}$
    \item Sea $w \en G_{11}$, $w \distinto 1$. Calcular $\re\parentesis{\sumatoria{k=0}{60}w^k}$.
  \end{enumerate}
\end{enunciado}

\parrafoDestacado{
  En este tipo de ejercicios es común hacer la sumas esas usando la serie geométrica $\sum q^n$, acordate
  que tenés que separar el caso cuando el $q = 1$.
}
\begin{enumerate}[label=\alph*)]
  \item\label{ej12-6:item-a} Sea $w \en G_{36}$, $w^4 \distinto 1.$ Calcular $\sumatoria{k=7}{60}w^{4k}$.

        Sé que si $w \en G_{36}$ tiene la pinta:
        $$
          w = e^{i \frac{2\pi}{36}k} \text{ para algún } k \en \enteros
        $$
        por lo tanto es fácil ver que es un número muy loquito que tiene las siguientes propiedades:
        $$
          \llave{l}{
            w^{36} = 1 \llamada1\\
            \sumatoria{k = 0}{35} w^k = 0
          }
        $$
        Por enunciado me dicen que $w^4 \distinto 1 \entonces w \distinto \pm 1$. Esto está bueno, porque ahora puedo usar la
        fórmula geométrica sin pensar que estoy metiendo un 0 en el denominador. Si no tendría que considerar casos particulares para la suma.
        Acomodo un poco la sumatoria, para que aparezcan todos los términos para armar la serie geométrica como la formulita. Le
        agrego los términos que faltan entre 0 y 7:
        $$
          \begin{array}{c}
            \sumatoria{k = 7}{60}w^{4k} =
            \ub{\sumatoria{k = 7}{60}w^{4k} + \magenta{\sumatoria{k = 0}{6}w^{4k}}}{\sumatoria{k=0}{60}w^{4k}}
            - \magenta{\sumatoria{k = 0}{6}w^{4k}} =
            \sumatoria{k = 0}{60}w^{4k} - \sumatoria{k=0}{6}w^{4k} = \\
            \frac{(w^4)^{61} - 1}{w^4 - 1} - \frac{(w^4)^7 - 1}{w^4 - 1} =
            \frac{(w^4)^{61} - (w^4)^7 }{w^4 - 1}
            \Sii{$61 = 9\cdot6 + 7 $}[$\llamada1$]
            \frac{ (w^{36})^6  \cdot (w^4)^7 - (w^4)^7 }{w^4 - 1} = 0
          \end{array}
        $$
        Se concluye que:
        $$
          \cajaResultado{\sumatoria{k=7}{60} w^{4k} = 0}
        $$

  \item Sea $w \en G_{11}$,
        $w \distinto 1$.
        Calcular $\re\parentesis{\sumatoria{k=0}{60}w^k}$.

        \separadorCorto
        Sé que si $w \en G_{11} \entonces
          \llave{l}{
            w^{11} = 1                  \\
            \sumatoria{k=0}{10} w^k = 0 \\
            11 \text{ es impar} \entonces -1 \not \en G_{11}
          }$\par
        Como $w \distinto 1$ no calculo caso particular para la suma.
        Me piden la parte real voy a usar:
        $$
          \re(z) \igual{$\llamada1$} \frac{z + \conj z}{2}
        $$
        Probé hacer la suma geométrica como en el ítem \ref{ej12-6:item-a} anterior, pero no llegué a nada. Voy a hacer eso que me encanta de
        abrir la sumatoria y usar que $61 = 5 \cdot 11 +6$, porque hay 61 términos en total.

        $$
          \sumatoria{k=0}{60}w^k =
          w^0 + \dots + w^{60}
          \ua{
            \igual{\red{!!!}}
          }{
            \ \\ \text{\tiny agrupé usando: } \\w \en G^{11} \entonces w^k = w^{r_{11}(k)}
          }
          \blue{5} \cdot \ob{(w^0 + w^1 + \dots + w^9 + w^{10})}{=0} +
          w^{55} + w^{56} + w^{57} + w^{58} + w^{59} + w^{60} = \llamada2
        $$
        Si bien no sé cuál es número $w$, sé que una sumatoria de 11 términos de potencias consecutivas de $w^n$ da 0, porque $w \en G_{11}$.
        $$
          \llamada2 =
          w^0 + w^1 + w^2 + w^3 + w^4 + w^5
          \sii
          \sumatoria{k=0}{60}w^k \igual{$\llamada3$}
          w^0 + w^1 + w^2 + w^3 + w^4 + w^5
        $$
        La expresión que quedo es más manejable. Recuerdo que estoy buscando la parte real de esa \textit{gaver}:
        $$
          \re\parentesis{\sumatoria{k=0}{60}w^k}
          \igual{$\llamada1$}[$\llamada3$]
          \frac{w^0 + w^1 + w^2 + w^3 + w^4 + w^5 + \conj w^0 + \conj w^1 + \conj w^2 + \conj w^3 + \conj w^4 + \conj w^5}{2}
          \igual{$\llamada4$}
        $$
        Nuevamente esta es una propiedad muy útil de los números $G_{n}$:
        $$
          w \en G_{11} \entonces \conj{w}^k \igual{$\llamada5$} w^{r_{11}(-k)}
        $$
        $$
          \igual{$\llamada4$}[$\llamada5$]
          \frac{w^0 +
            \ob{
              w^1 + w^2 + w^3 + w^4 + w^5 + w^0 + w^{10} + w^9 + w^8 + w^7 + w^6
            }{
              \sumatoria{k = 0}{10} w^k}
          }{2} =
          \frac{w^0 + \ob{\scriptstyle\sumatoria{k=0}{10} w^k}{\igual{\red{!}} 0}}{2} = \frac{w^0}{2} = \frac{1}{2}
        $$
        Finalmente:
        $$
          \cajaResultado{
            \re\parentesis{\sumatoria{k=0}{60}w^k} = \frac{1}{2}
          }
        $$
\end{enumerate}

\begin{aportes}
  \item \aporte{\dirRepo}{naD GarRaz \github}
\end{aportes}
