\ejercicio
\begin{enumerate}[label=\roman*)]
	\item Sea $w \en G_{36}$, $w^4 \distinto 1.$ Calcular $\sumatoria{k=7}{60}w^{4k}$

	      \separadorCorto
	      Sé que si $w \en G_{36} \entonces
		      \llave{l}{
			      w^{36} = 1 \\
			      \sumatoria{k=0}{35} w^k = 0
		      }$\\
	      Como $w^4 \distinto 1$ sé que $w \distinto \pm1$. Si no tendría que considerar casos particulares para la suma.\\

	      Si
	      $\sumatoria{k=7}{60}w^{4k} =
		      \ub{\sumatoria{k=7}{60}w^{4k} + \magenta{\sumatoria{k=0}{6}w^{4k}}}{\sumatoria{k=0}{60}w^{4k}}
		      - \magenta{\sumatoria{k=0}{6}w^{4k}} =
		      \sumatoria{k=0}{60}w^{4k} - \sumatoria{k=0}{6}w^{4k} =
		      \frac{(w^4)^{61} - 1}{w^4 - 1} - \frac{(w^4)^7 - 1}{w^4 - 1} =
		      \frac{(w^4)^{61} - (w^4)^7 }{w^4 - 1}\\
		      \flecha{$61 = 9\cdot6 + 7 $}[$w^36 = 1$]
		      \frac{((\ob{ \scriptstyle w^{36}}{=1})^6  \cdot (w^4)^7 - (w^4)^7 }{w^4 - 1}
		      \to$
	      \boxed{\sumatoria{k=7}{60}w^{4k} =0}

	\item Sea $w \en G_{11}$, $w \distinto 1.$ Calcular $\re\parentesis{\sumatoria{k=0}{60}w^k}$.

	      \separadorCorto
	      Sé que si $w \en G_{11} \entonces
		      \llave{l}{
			      w^{11} = 1 \\
			      \sumatoria{k=0}{10} w^k = 0\\
			      11 \text{ es impar} \entonces -1 \not\en G_{11}
		      }$\\
	      Como $w \distinto 1$ no calculo caso particular para la suma.
	      Me piden la parte real $ \flecha{uso} \re(z) = \frac{z + \conj z}{2}$.\\

	      Probé hacer la suma de Gauss como en el anterior, pero no llegué a nada, abro sumatoria y uso que $61 = 5 \cdot 11 +6$, porque hay 61 sumandos.\\

	      $\sumatoria{k=0}{60}w^k =
		      w^0 + \dots + w^{60} =
		      5 \cdot \ub{\ob{(w^0 + w^1 + \dots + w^9 + w^{10})}{=0}}{\text{\tiny agrupé usando: }w \en G^{11} \entonces w^k = w^{r_{11}(k)}} +
		      w^{55} + w^{56} + w^{57} + w^{58} + w^{59} + w^{60}=\\
		      = w^0 + w^1 + w^2 + w^3 + w^4 + w^5 \llamada{1}
	      $\\

	      También voy a usar que si $w \en G_{11} \entonces \conj w^k = w^{r_{11}(-k)}$\\
          $\re{\sumatoria{k=0}{60}w^k} = \frac{\sumatoria{k=0}{60}w^k +\sumatoria{k=0}{60}\conj w^k}{2} \stackrel{\llamada1}=
              \frac{w^0 + w^1 + w^2 + w^3 + w^4 + w^5 + \conj w^0 + \conj w^1 + \conj w^2 + \conj w^3 + \conj w^4 + \conj w^5}{2} =\\
              = \frac{w^0}{2} + \frac{\ob{w^1 + w^2 + w^3 + w^4 + w^5 + w^0 + w^{10} + w^9 + w^8 + w^7 + w^6}{\sumatoria{k=0}{10} w^k}}{2} =
              \frac{\ob{\scriptstyle w^0}{1}}{2} + \frac{\ob{\scriptstyle\sumatoria{k=0}{10} w^k}{= 0}}{2} = \frac{1}{2}
	      $
\end{enumerate}
