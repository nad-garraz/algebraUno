\begin{enunciado}{\ejercicio}
  Hallar todos los $n \en \naturales$ tales que
  \begin{enumerate}[label=\alph*)]
    \item $(\sqrt3 -i)^n = 2^{n-1}(-1 + \sqrt3 i)$
    \item $(-\sqrt3 + i)^n \cdot \parentesis{\frac{1}{2} + \frac{\sqrt3}{2}i}$ es un número real negativo.
    \item $\text{arg}((-1+i)^{2n}) = \frac{\pi}{2}$ y $\text{arg}((1-\sqrt3 i)^{n-1}) = \frac{2}{3}\pi$
  \end{enumerate}
\end{enunciado}

\begin{enumerate}[label=\roman*)]

  \item Para resolver las ecuaciones en números complejos con exponentes, en general, es más
        fácil resolver en notación exponencial.
        El miembro izquierdo queda:
        $$
          (\sqrt{3} - i)^n =
          \big( 2 \cdot e^{i \frac{11}{6}\pi} \big)^n =
          2^n \cdot e^{i \frac{11}{6}\pi n}
        $$
        El miembro derecho queda:
        $$
          2^{n-1}(-1 + \sqrt{3} i) =
          2^{n-1} \cdot (2 \cdot e^{\frac{2}{3}})=
          2^n\cdot e^{i\frac{2}{3}\pi}
        $$
        Ahora la igualdad de los números se dará cuando sus módulos y argumentos sean iguales:
        $$
          2^n \cdot e^{i \frac{11}{6}\pi n} = 2^n\cdot e^{i\frac{2}{3}\pi}
          \sii
          \llave{l}{
            2^n = 2^n  \Tilde \\
            \frac{11}{6}\pi n = \frac{2}{3}\pi + \magenta{2k \pi}
            \sii
            11 n = 4 + 12k \llamada1
          }
        $$
        En $\llamada1$ quedó una ecuación para despejar $n$ que es un número entero:
        $$
          \llamada1
          11 n = 4 + 12k
          \Sii{def}
          \congruencia{11 n}{4}{12}
          \sii
          \congruencia{-n}{4}{12}
          \sii
          \congruencia{n}{-4}{12}
          \sii
          \congruencia{n}{8}{12}
        $$
        Finalmente los valores de $n$ buscados para que la ecuación se cumpla son:
        $$
          \congruencia{n}{8}{12}
        $$

  \item
        Un número real negativo tendrá un arg$(z) = \pi$\\
        $\ub{(-\sqrt3 + i)^n}{2^ne^{i\frac{5}{6}\pi n}} \cdot \ub{\parentesis{\frac{1}{2} + \frac{\sqrt3}{2}i}}{e^{\frac{\pi}{3}i}} =
          2^ne^{i(\frac{5}{6}n + \frac{1}{3}) \pi} \to \theta = (\frac{5}{6}n + \frac{1}{3}) \pi $\\
        $\flecha{$\theta = \pi + 2k\pi$}
          \cancel\pi \frac{5}{6}n + \frac{\cancel\pi}{3} = \cancel\pi + 2k\cancel\pi
          \flecha{acomodo}[congruencia]
          \congruencia{5n}{4}{12}
          \flecha{multiplico}[por 5]
          \boxed{\congruencia{n}{8}{12}} $

  \item $\text{arg}((-1+i)^{2n}) = \frac{\pi}{2}$ y $\text{arg}((1-\sqrt3 i)^{n-1}) = \frac{2}{3}\pi$

\end{enumerate}
