\begin{enunciado}{\ejercicio}
  Hallara en cada caso las raíces $n$-ésimas de $z \en \complejos$:
  \begin{multicols}{2}
    \begin{enumerate}[label=\roman*)]
      \item $z =  8,\ n = 6$
      \item $z = -4,\ n = 3$
      \item $z = -1 + i,\ n = 7$
      \item $z = (2-2i)^{12},\ n = 6$
    \end{enumerate}
  \end{multicols}
\end{enunciado}

\underline{Ejercicio importante}. La raíz $n$-ésima de $z$ es el número que multiplicado por sí mismo $n$ veces me da $z$:
$$
  w^n = z,
$$
es decir que quiero encontrar $w$. Siempre va a haber tantas soluciones como $n$.

\begin{enumerate}[label=\roman*)]
  \item Dado un número \textit{genérico} $w = r\cdot e^{\theta i}$, lo visto con la info del enunciado:
        $$
          w^6 = w = \big( r\cdot e^{\theta i} \big)^6 = r^6 \cdot e^{6\theta i} \llamada1
        $$
        Ahora hago lo mismo con el otro número $z = 8$:
        $$
          z = 8 \cdot e^{0 i} = 8 \llamada2
        $$
        Una vez con todo escrito en forma exponencial, es igualar prestar atención a la periodicidad del argumento y listo:
        $$
          \begin{array}{c}
            w^6 = z
            \Sii{$\llamada1$}[$\llamada2$]
            r^6 \cdot e^{6\theta i} = 8
            \Sii{\red{!}}
            \llave{l}{
            r^6 = 8 \sii r = \sqrt[6]{8} = \sqrt{2} \\
              6\theta \igual{\red{!}} 0 + 2k\pi \sii \theta_k = \frac{1}{3}k \pi
            }
          \end{array}
        $$
        Con eso concluímos que las raíces son de la forma:
        $$
          w_k = \sqrt{2} \cdot e^{i\frac{1}{3}k \pi} \text{ con } k \en [0,5]
        $$

  \item Mismo procedimiento, te tiro una pista: Los números negativos tienen argumento $\pi$, así que en notación exponencial:
        $$
          -4 = 4\cdot e^{\pi i}
        $$

  \item En notación exponencial $z$, que está en segundo cuadrante:
        $$
          z = -1 + i  = \sqrt{2} \cdot e^{\frac{3}{4} \pi i}
        $$

  \item En notación exponencial $z$, se calcula primero con la base :
        $$
          z = (2 - 2i)^{12} =
          \big(2\sqrt{2} \cdot e^{\frac{7}{4} \pi i}\big)^{12} =
          2^{18}\cdot e^{21 \pi i}
          \igual{\red{!!}}
          2^{18}\cdot e^{\pi i} \llamada1
        $$
        \textit{Máquina de hacer chorizos: ¿Cuál número $w \en \complejos$ multiplicado 6 veces por si mismo me da $\llamada1$?}
        $$
          \begin{array}{c}
              w^6 \igual{?} 2^{18}\cdot e^{\pi i}
            \sii
            (|w|e^{i\theta})^6 = 2^{18}\cdot e^{\pi i}
            \sii
            |w|^6 \cdot e^{i 6\theta} = 2^{18} \cdot e^{\pi i}
            \sii
            \llave{l}{
            |w|^6 = 2^{18} \sii |w| = 8 \\
              6\theta = \pi  \blue{ + 2k\pi} \sii \theta_k = (\frac{1}{6} + \frac{1}{3}k)\pi
            }
          \end{array}
        $$
        Como $ 0 \leq \theta < 2 \pi$:
        $$
          w_k = 8 \cdot e^{i\theta_k} \text{ con } k \en \enteros_{[0,5]}
        $$

\end{enumerate}

\begin{aportes}
  \item \aporte{\dirRepo}{naD GarRaz \github}
\end{aportes}
