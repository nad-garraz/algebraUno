\begin{enunciado}{\ejercicio}
  Hallar todos los $z \en \complejos$ tales que $3z^5 + 2|z|^5 + 32 = 0$
\end{enunciado}

Para que se cumpla la igualdad entre 2 números complejos, \textit{las partes reales y imaginarias}
deben ser iguales:
$$
  3 z^5 + 2|z|^5 +32 = 0
  \sii
  \ub{3z^5}{\en \complejos} = \ub{-2|z|^5- 32}{\en \reales}
  \Sii{\red{!}}
  \llaves{l}{
    \re(3z^5) =  -2|z|^5 - 32\\
    \im(3z^5) =  0
  }
$$

\textit{De la ecuación de la parte imaginaria: }
{\tiny(Es útil recordar que $z = \re(z) + i\im(z) \entonces \im(z) = \frac{z - \conj{z}}{2i}$)}
$$
  \im(3z^5) = 3 \cdot \frac{z^5 - \conj z^5}{2i} = 0
  \sisolosi
  z^5 = \conj z^5
  \sisolosi
  |z|^5 e^{5 \theta i} = |z|^5 e^{-5 \theta i}
  \Sii{\red{!}}[$2k\pi$]
  \llave{l}{
    5 \theta = -5 \theta + \magenta{2k\pi} \\
    \Sii{\red{!}}[$\llamada1$] \theta_k = \frac{1}{5}k\pi \text{ con } k \en \enteros
  }
$$

\textit{De la ecuación de la parte real: }
{\tiny(Es útil recordar que si $z = \re(z) + i\im(z)$, entonces se puede expresar $\re(z) = \frac{z + \conj{z}}{2}$)}
$$
  \begin{array}{l}
    \re(3z^5) = 3 \cdot \frac{z^5 + \conj z^5}{2} =
    3 \cdot \frac{|z|^5 e^{5\theta i} + |z|^5 e^{-5\theta i}}{2} =
    3|z|^5 \cos(5\theta) =  -2|z|^5 - 32 \sii                                         \\
    \sii
    |z|^5(3\cos(5\theta) + 2) = -2^5
    \flecha{evaluando}[en $\theta_k \llamada1$]
    |z|^5(3\cos(k\pi) + 2) = -2^5
    \llave{cl}{
    \flecha{$k$}[par]   & 0 < |z|^5(3 + 2) \distinto -2^5 \quad \text{\faIcon{skull}} \\
    \flecha{$k$}[impar] & |z|^5(-3 + 2) = -2^5 \sii |z| = 2
    }
  \end{array}
$$
Finalmente teniendo en cuenta que $k$ tiene que ser impar, y que el $\arg(z) \en [0, 2\pi)$:
$$
  z_k = 2 e^{\theta_k i} \quad \text{ con  $ \theta_k = \frac{1}{5}k\pi \ytext k \en \set{1,3,5,7,9}$}
$$

% Contribuciones
\begin{aportes}
  %% iconos : \github, \instagram, \tiktok, \linkedin
  %\aporte{url}{nombre icono}
  \item \aporte{https://github.com/nad-garraz}{naD GarRaz \github}
\end{aportes}
