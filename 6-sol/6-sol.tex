\documentclass[12pt,a4paper, spanish]{article}
% Sacar draft para que aparezcan las imagenes.
% Opciones: 12pt, 10pt, 11pt, landscape, twocolumn, fleqn, leqno...
% Opciones de clase: article, report, letter, beamer...

% Paquetes:
% =========
\usepackage[headheight=110pt, top = 2cm, bottom = 2cm, left=1cm, right=1cm]{geometry} %modifico márgenes
\usepackage[T1]{fontenc} % tildes
\usepackage[utf8]{inputenc} % Para poder escribir con tildes en el editor.
\usepackage[english]{babel} % Para cortar las palabras en silabas, creo.
\usepackage[ddmmyyyy]{datetime}
\usepackage{amsmath} % Soporte de mathmatics
\usepackage{amssymb} % fuentes de mathmatics
\usepackage{array} % Para tablas y eso
\usepackage{caption} % Configuracion de figuras y tablas
\usepackage[dvipsnames]{xcolor} % Para colorear el texto: black, blue, brown, cyan, darkgray, gray, green, lightgray, lime, magenta, olive, orange, pink, purple, red, teal, violet, white, yellow.
\usepackage{graphicx} % Necesario para poner imagenes
\usepackage{enumitem} % Cambiar labels y más flexibilidad para el enumerate
\usepackage{tikz} % para graficar
\usepackage{cancel}
\usepackage{titlesec} % para editar titulos y hacer secciones con formato a medida
\usepackage{ulem}
\usepackage{centernot} % tacha cosas
\usepackage{bbding} % símbolos de donde uso FiveStar
\usepackage{skull} % símbolos de donde uso Skull
% \usepackage{lipsum}
\usepackage{soul} % para tachar en mathmode -> \hbox{\sout{$x+1$}}

% para hacer los graficos tipo grafos
\usetikzlibrary{shapes,arrows.meta, chains, matrix, calc, trees, positioning, fit}
\usetikzlibrary{external,angles,quotes}

\begin{document}
% Definiciones y nuevos comandos:def
% =============
% Conjuntos
\DeclareMathOperator{\partes}{\mathcal P}
\DeclareMathOperator{\relacion}{\,\mathcal{R}\,}
\DeclareMathOperator{\norelacion}{\,\cancel{\relacion}\,}
\DeclareMathOperator{\universo}{\mathcal U}
\DeclareMathOperator{\reales}{\mathbb R}
\DeclareMathOperator{\naturales}{\mathbb N}
\DeclareMathOperator{\enteros}{\mathbb Z}
\DeclareMathOperator{\racionales}{\mathbb Q}
\DeclareMathOperator{\irracionales}{\mathbb I}
\DeclareMathOperator{\complejos}{\mathbb C}


\DeclareMathOperator{\K}{\mathbb K} % cuerpo K
\DeclareMathOperator{\i}{\text{i}}
\DeclareMathOperator{\vacio}{\varnothing}
\DeclareMathOperator{\union}{\cup}
\DeclareMathOperator{\inter}{\cap}
\DeclareMathOperator{\diferencia}{\ \setminus \ }
\DeclareMathOperator{\y}{\land}
\def\o{\lor}
\def\neg{\sim}

\def\entonces{\Rightarrow}
\def\noEntonces{\centernot\Rightarrow}

\def\sisolosi{\iff} % largo
\def\sii{\Leftrightarrow} % corto

\def\clase{\overline}
\def\ord{\text{ord}}


\def\existe{\,\exists\,}
\def\noexiste{\,\nexists\,}
\def\paratodo{\ \, \forall}
\def\distinto{\neq}
\def\en{\in}
\def\talque{\;/\;}

% =====
\def\qvq{\text{ quiero ver que }}

%funciones
\DeclareMathOperator{\dom}{Dom}
\DeclareMathOperator{\cod}{Cod}
\def\F{\mathcal F}
\def\comp{\circ}
\def\inv{^{-1}}
\def\infinito{\infty}

% Llaves, paréntesis, contenedores
\newcommand{\llave}[2]{ \left\{ \begin{array}{#1} #2 \end{array}\right. }
\newcommand{\llaveInv}[2]{ \left\} \begin{array}{#1} #2 \end{array}\right. }
\newcommand{\llaves}[2]{ \left\{ \begin{array}{#1} #2 \end{array} \right\} }
\newcommand{\matriz}[2]{\left( \begin{array}{#1} #2 \end{array} \right)}
\newcommand{\deter}[2]{\left| \begin{array}{#1} #2 \end{array} \right|}
\newcommand{\lista}[2][(1)]{\begin{enumerate}[\bf #1]\setlength\itemsep{-0.6ex} #2 \end{enumerate}}
\newcommand{\listal}[2][-0.6ex]{\begin{enumerate}[\bf(a)]\setlength\itemsep{#1} #2 \end{enumerate}}

% naturales
\newcommand{\sumatoria}[2]{\sum\limits_{#1}^{#2}}
\newcommand{\productoria}[2]{\prod\limits_{#1}^{#2}}
\newcommand{\kmasuno}[1]{\underbrace{#1}_{k+1\text{-ésimo}}}
\newcommand{\HI}[1]{\underbrace{#1}_{\text{HI}}}

% % enteros
\def\divideA{\, | \,}
\def\noDivide{\centernot\divideA}
\def\congruente{\, \equiv \,}
\newcommand{\congruencia}[3]{#1 \equiv #2 \;(#3)}
\newcommand{\noCongruencia}[3]{#1 \not\equiv #2 \;(#3)}
\newcommand{\conga}[1]{\stackrel{(#1)}{\congruente}}
\newcommand{\divset}[2]{\mathcal{D}(#1) = \set{#2}}
\newcommand{\divsetP}[2]{\mathcal{D_+}(#1) = \set{#2}}
\newcommand{\ub}[2]{ \underbrace{\textstyle #1}_{\mathclap{#2}} }
\newcommand{\ob}[2]{ \overbrace{\textstyle #1}^{\mathclap{#2}} }
\def\cop{\, \perp \, }

% complejos
\DeclareMathOperator{\re}{Re}
\DeclareMathOperator{\im}{Im}
%\DeclareMathOperator{\arg}{arg}
\def\conj{\overline}

% Polinomios
\DeclareMathOperator{\cp}{cp}
\DeclareMathOperator{\gr}{gr}
\DeclareMathOperator{\mult}{mult}
\newcommand{\divPol}[2]{\polylongdiv[style=D]{#1}{#2}}
\newcommand{\mcd}[2]{\polylonggcd{#1}{#2}}


% =====
% Miscelanea
% =====
\def\ot{\leftarrow}
\newcommand{\estabien}{{\color{blue} Consultado, está bien. \checkmark}}
\newcommand{\hacer}{
  {\color{red!80!black}{\Large \faIcon{radiation} Falta hacerlo!}}\par
  {\color{black!70!white}
    \small Si querés mandarlo: Telegram $\to$ \href{https://t.me/+1znt2GV1i8cwMTNh}{\small\faIcon{telegram}},
    o  mejor aún si querés subirlo en \LaTeX $\to$ \href{https://github.com/nad-garraz/algebraUno}{\small \faIcon{github}}.
  }\par
}

\newcommand{\Hacer}{{\color{black!30!red}\Large Hacer!}}
\def\Tilde{\quad\checkmark}
\def\ytext{\text{ y }}
\def\otext{\text{ o }}

% Estrellita para hacer llamadas de atención, viene en divertidos colores
% para coleccionar.
\newcommand{\llamada}[1]{
  {\small{\textcolor{
          \ifcase \numexpr#1 mod 6\relax
            cyan\or magenta\or OliveGreen\or YellowOrange\or Cerulean\or Violet\or Purple\or
          \fi
        }
        {\text{{\small\FiveStar}}^{#1}}%
      }%
    }
}


% separadores
\def\separador{\rule{\linewidth}{0.4pt}\par}
\def\separadorCorto{\rule{0.5\linewidth}{0.4pt}\par\addvspace{10pt}}


% Colores
\newcommand{\red}[1]{\textcolor{red}{#1}}
\newcommand{\green}[1]{\textcolor{OliveGreen}{#1}}
\newcommand{\blue}[1]{\textcolor{Cerulean}{#1}}
\newcommand{\cyan}[1]{\textcolor{cyan}{#1}}
\newcommand{\yellow}[1]{\textcolor{YellowOrange}{#1}}
\newcommand{\magenta}[1]{\textcolor{magenta}{#1}}
\newcommand{\purple}[1]{\textcolor{purple}{#1}}

% Conjuntos entre llaves y paréntesis
% te ahorrás escribir los \left y \right, así dejando el código más legible.
\newcommand{\set}[1] { \left\{ #1 \right\} }
\newcommand{\parentesis}[1]{ \left( #1 \right) }

% Stackrel text. Es para ahorrarse ecribir el \text
\newcommand{\stacktext}[2]{ \stackrel{\text{#1}}{#2} }

% Dado que muchas veces ponemos cosas sobre un signo '='
%  acá está el comando para escribir \igual{arriba}[abajo] con texto!
\NewDocumentCommand{\igual}{m o}{%
  \IfNoValueTF{#2}{%
    \overset{\mathclap{\text{#1}}}=
  }{
    \overset{\mathclap{\text{#1}}}{\underset{\mathclap{\text{#2}}}=}
  }
}


%=======================================================
% Comandos con flechas extensibles.
%=======================================================
% *Flechita* extensible con texto {arriba} y [abajo] 
\NewDocumentCommand{\flecha}{m o}{%
  \IfNoValueTF{#2}{%
    \xrightarrow[]{\text{#1}}
  }{
    \xrightarrow[\text{#2}]{\text{#1}}
  }
}
% *Si solo si* extensible con texto {arriba} y [abajo] 
\NewDocumentCommand{\Sii}{m o}{%
  \IfNoValueTF{#2}{%
    \xLeftrightarrow[]{\text{#1}}
  }{
    \xLeftrightarrow[\text{#2}]{\text{#1}}
  }
}

%=======================================================
% fin comandos con flechas extensibles.
%=======================================================


% como el stackrel pero también se puede poner algo debajo
\newcommand{\taa}[3]{ % [t]exto [a]rriba y [a]bajo
  \overset{\mathclap{#1}}{\underset{\mathclap{#2}}{#3}}
}

%Update time
\def\update{
  actualizado: \today
}


%=======================================================
% sección ejercicio con su respectivo formato y contador
%=======================================================
\newcounter{ejercicio}[section] % contador que se resetea en cada sección
\renewcommand{\theejercicio}{\arabic{ejercicio}} % el contador es un número arabic
\newcommand{\ejercicio}{%
  \stepcounter{ejercicio}% incremento en uno
  \titleformat{\section}[runin]{\bfseries}{\theejercicio}{1em}{}%
  \section*{\theejercicio.}\labelEjercicio{ej:\theejercicio}
}

% Label y refencia para ejercicio hay alguna forma más elegante de hacer esto?
\newcommand{\labelEjercicio}[1]{
  \addtocounter{ejercicio}{-1} % counter - 1
  \refstepcounter{ejercicio} % referencia al anterior y luego + 1
  \label{#1}}
\newcommand{\refEjercicio}[1]{{ \bf\ref{#1}.}}

\def\fueguito{{\color{orange}{\faIcon{fire}}}}
\newcounter{ejExtra}[section] % contador que se resetea en cada sección
\renewcommand{\theejExtra}{\arabic{ejExtra}} % el contador es un número arabic
\newcommand{\ejExtra}{%
  \stepcounter{ejExtra}% incremento en uno
  \titleformat{\section}[runin]{\bfseries}{\theejExtra}{1em}{}%
  % Es como una sección. Le pongo un ícono, luego el número del ejercicio con la etiqueta para poder
  % linkearlo en el índice u otro lugar.
  % con \ref{ejExtra:{numero del ejercicio}} es que salto al ejercicio.
  \section*{\fueguito\theejExtra.}\labelEjExtra{ejExtra:\theejExtra}
}

% Label y refencia para ejercicio hay alguna forma más elegante de hacer esto?
\newcommand{\labelEjExtra}[1]{
  \addtocounter{ejExtra}{-1} % counter - 1
  \refstepcounter{ejExtra} % referencia al anterior y luego + 1
  \label{#1} % etiqueta para cada ejercicio extra
  \unskip
}
% Con esto llamos al ejercicio extra
\newcommand{\refEjExtra}[1]{
  {\fueguito\bf\ref{#1}.}
}

%=======================================================
% fin sección ejercicio con su respectivo formato y contador
%=======================================================

\newenvironment{enunciado}[1]{ % Toma un parametro obligatorio: \ejExtra o \ejercicio 
  \begin{minipage}{\textwidth}
          \par
    \vspace{5pt}
    \separador % linea sobre el enunciado
    #1
    }% contenido
    {\vspace{5pt}
    \par
    \separadorCorto % linea debajo del enunciado
  \end{minipage}
}

 % idem con las definiciones

\pagestyle{empty} % Para que no muestre el número en pie de página

% Info para armar título.
\title{Práctica 6 de álgebra 1} % título
\author{D. Garraz} % autor
\date{last update: \today} % Cambiar de ser necesario

\maketitle  % para que aprezca el título en el documento

\section{Definiciones y fórmulas útiles}
\textit{\underline{Raíces de un número complejo: }}
\begin{itemize}
	\item Sean $z, w \en \complejos -\set{0}$, $z = re^{\theta i}$ y $w = se^{\varphi i }$ con $r,\, s \en \reales_{>0}$
	      y $\theta,\, \varphi \en \reales$.\\ Entonces $z = w \sisolosi
		      \llave{l}{
			      r = s\\
			      \theta = \varphi + 2 k\pi,\ \text{para algún } k \en \enteros
		      }$
	\item raíces $n$-esimas: $w^n = z
		      \to
		      \llave{l}{
		      s^n = r \\
		      \varphi \cdot n = \theta + 2 k \pi \quad\to \text{para algún $k\en \enteros$}\\
		      \text{$n$ raíces distintas} \to w_k=se^{\varphi_k i}, \text{ donde } s = \sqrt{r} \text{ y }
		      \varphi_k = \frac{\theta}{n} + \frac{2k\pi}{n} = \frac{\theta + 2k\pi}{n}


		      }$
\end{itemize}
\begin{itemize}
	\item $G_n = \set{w \en \complejos / w^n = 1} = \set{e^{\frac{2k \pi}{n} i}\ :\ 0\leq k \leq n-1}$

	\item $(G_n, \cdot)$ es un grupo abeliano, o conmutativo.
	      \begin{itemize}
		      \item $\paratodo w, z \en G_n, w z = z  w \text{ y } z m \en G_n$.

		      \item $1 \en G_n,\ w \cdot 1 = 1 \cdot w = w \qquad \paratodo w \en G_n$.

		      \item $w \en G_n \entonces \existe w^{-1} \en G_n,\ w \cdot w^{-1} = w^{-1}\cdot w = 1$
		            \begin{itemize}
			            \item $\conj w \en G_n,\ w \cdot \conj w = |w|^2 = 1 \entonces \conj w = w^{-1}$
		            \end{itemize}
	      \end{itemize}
	\item \textit{Propiedades: $w \en G_n$}
	      \begin{itemize}
		      \item $m \en \enteros$ y $n \divideA m \entonces w^m = 1$.

		      \item $\congruencia{m}{m'}{n} \entonces w^m = w^{m'}\quad (w^m = w^{r_n(m)})$

		      \item $n \divideA m \sisolosi G_n \subseteq G_m$

		      \item $G_n \inter G_m = G_{(n:m)}$

		      \item Si $(G, \cdot)$ es un grupo y $\#G = n$ decimos que $G$ siempre es cíclico si
		            $\existe w\en G / G = \set{1,w, w^2,\dots, w^{n-1}}$\\
		            \begin{itemize}
			            \item \textit{Observación: } $G_n$ es un grupo cíclico, ej, $w_1 = e^\frac{2\pi i}{n} \to (w_1)^k = w_k$\\
			                  $\to$ las potencias de $w_1$ generan todo $G_n = \set{1, w_1, w_1^2,\dots,w_1^{n-1}}$
		            \end{itemize}

		      \item $w$ es raíz $n-$ésima primitiva de 1 si:
		            $G_n = \set{1,w,w^2,\dots,w^{n-1}} =
			            \set{w^k\ :\ 0\leq k \leq n-1}$\\
		            Ejemplo: $i, -i$ son primitivas de $G_4 = \set{1,i,-1,-i} = \set{i^k\ :\ 0 \leq k \leq 3}$, pero 1 y -1 no son raíces primitivas de $G_4$.
	      \end{itemize}
	\item \textit{Definición: }
	      Sea $w$ una raíz primitiva de orden $n$ (el orden de
	      $w \en G_n,\, \text{ord}(w) = \text{min}\set{k \en \naturales / w^k = 1}$)
	      \begin{itemize}
		      \item $w^m = 1 \sisolosi n \divideA m$
		      \item \textit{Observación: } Si $w \en G_n \entonces \ord(w) \divideA n$
	      \end{itemize}
	\item La suma de las raíces $n$-ésimas de 1 da:
	      $\sumatoria{k=0}{n-1}w_1^k = \frac{w_1^n -1}{w_1 -1} = 0$ pues $w_1 \distinto 1$
	\item El producto de las raíces $n$-ésimas de 1 da:
	      $\productoria{k=0}{n-1} w_1^k = w_1^{0+1+\dots + n-1} =
		      w_1^{\frac{n(n-1)}{2}} =
		      \llave{rl}{
			      1 & \text{si $n$ es impar}\\
			      -1 & \text{si $n$ es par}
		      }$
	\item Sea $w \en G_n$ primitiva. Entonces
	      \begin{itemize}
		      \item $w^k \text{ es primitiva } \sisolosi k \cop n $
		      \item $w_k = e^{\frac{2k\pi}{n}i}$ es primitiva $\sisolosi k \cop n$
		      \item En particular para $n = p$ primo: $w_k$ es primitiva para $1\leq k < p$ o sea si
		            $w \en G_p$ y $w \distinto 1$, entonces $w$ es primitiva
	      \end{itemize}
        \item $w$ es raíz primitiva de $G_n$ y $k \divideA n \entonces w^k$ es primitiva de $G_\frac{n}{k}$
\end{itemize}

\subsubsection*{Ejercicios dados en clase:}
\ejercicio

\ejercicio

\newpage

%=========================
% Ejercicios guia
%=========================

\section*{Ejercicios de la guía:}
\setcounter{ejercicio}{0} % Reset the custom counter

%1
\ejercicio

\setcounter{ejercicio}{6}

%7
\ejercicio
Hallar todos los $n \en \naturales$ tales que
\begin{enumerate}[label=\roman*)]
	\begin{minipage}{0.7\textwidth}
		\item $(\sqrt3 -i)^n = 2^{n-1}(-1 + \sqrt3 i)$ \\
		\separadorCorto
		$(\sqrt3 -i)^n = 2^n e^{i\frac{11}{12}\pi n} = 2^{n+1}\cdot 2e^{i \frac{2}{3} \pi}\\
			\to
			\llave{l}{
				2^n = 2^n\\
				\frac{11}{12}\pi n = \frac{2}{3}\pi + 2k \pi \to 11n = 8+8k \flecha{8(k+1)} \boxed{\congruencia{n}{0}{8}}
			}$
	\end{minipage}

	\item $(-\sqrt3 + i)^n \cdot \parentesis{\frac{1}{2} + \frac{\sqrt3}{2}i}$ es un número real negativo.\\
	      \separadorCorto
	      Un número real negativo tendrá un arg$(z) = \pi$\\
	      $\ub{(-\sqrt3 + i)^n}{2^ne^{i\frac{5}{6}\pi n}} \cdot \ub{\parentesis{\frac{1}{2} + \frac{\sqrt3}{2}i}}{e^{\frac{\pi}{3}i}} =
		      2^ne^{i(\frac{5}{6}n + \frac{1}{3}) \pi} \to \theta = (\frac{5}{6}n + \frac{1}{3}) \pi $\\
	      $\flecha{$\theta = \pi + 2k\pi$}
		      \cancel\pi \frac{5}{6}n + \frac{\cancel\pi}{3} = \cancel\pi + 2k\cancel\pi
		      \flecha{acomodo}[congruencia]
		      \congruencia{5n}{4}{12}
		      \flecha{multiplico}[por 5]
		      \boxed{\congruencia{n}{8}{12}} $

	\item $\text{arg}((-1+i)^{2n}) = \frac{\pi}{2}$ y $\text{arg}((1-\sqrt3 i)^{n-1}) = \frac{2}{3}\pi$
	      \separadorCorto
\end{enumerate}

\setcounter{ejercicio}{11}
\ejercicio
\begin{enumerate}[label=\roman*)] 
  \item Sea $w \en G_{36}$, $w^4 \distinto 1.$ Calcular $\sumatoria{k=7}{60}w^{4k}$\\ 
    \separadorCorto
      Si 
      $\sumatoria{k=7}{60}w^{4k} =
      \sumatoria{k=0}{60}w^{4k} - \sumatoria{k=0}{6}w^{4k} =
      \frac{(w^4)^{61} - 1}{w^4 - 1} - \frac{(w^4)^7 - 1}{w^4 - 1} =
      \frac{(w^4)^{61} - (w^4)^7 }{w^4 - 1}\\ 
      \flecha{$61 = 9\cdot6 + 7 $}[$w^36 = 1$]
      \frac{( (w^{36})^6 \cdot w^7 - (w^4)^7 }{w^4 - 1}
      \flecha{$w^{36} = 1$}$ \boxed{\sumatoria{k=7}{60}w^{4k} =0}

    \item Sea $w \en G_{11}$, $w \distinto 1.$ Calcular Re$\parentesis{\sumatoria{k=0}{60}w^k}$.\\ 
      \separadorCorto

 \end{enumerate}

\end{document}
