%==================
% Macro Local:
\def\c{\textbf{c}}
%==================

\begin{enunciado}{\ejExtra}
  \begin{enumerate}[label=\alph*)]
    \item Hallar todos los posibles $\c \en \reales,\, \c > 0$ tales que:
          $$
            f = X^6 - 4X^5 - X^4 + 4X^3 + 4X^2 + 48X + \c
          $$ tenga una raíz de argumento $\frac{3\pi}{2}$

    \item Para cada valor de $\c$ hallado, factorizar $f$ en $\racionales[X],\, \reales[X]$ y $\complejos[X]$, sabiendo que tiene al menos una raíz doble.
  \end{enumerate}

\end{enunciado}

\begin{enumerate}[label=\alph*)]
  \item
        Si la raíz $\alpha = r e^{i\frac{3\pi}{2}} = r(-i) \entonces f(r(-i)) = 0$

        Voy a usar que:
        $$
          \llamada1
          \llave{rcr}{
            (-i)^2 & = & -1 \\
            (-i)^3 & = & i  \\
            (-i)^4 & = & 1  \\
            (-i)^5 & = & -i \\
            (-i)^6 & = & -1
          }
        $$

        Evalúo $f(r(-i))$:

        $$
          \begin{array}{rcl}
            f(r(-i)) & =                   & (r(-i))^6 - 4(r(-i))^5 - (r(-i))^4 + 4 r^3i + 4(r(-i))^2 + 48(r(-i)) + \c \\
                     & \igual{$\llamada1$} & -r^6 + 4r^5i - r^4 + 4 r^3i - 4r^2 - 48ri + \c = 0
          \end{array}
        $$

        Esta expresión va a ser 0 cuando su parte imaginaria y su parte real sean ambas 0:
        $$
          \begin{array}{c}
            f(r(-i))  =   -r^6 + 4r^5i - r^4 - 4 r^3i - 4r^2 - 48ri + \c = 0 \\
            \sisolosi                                                        \\
            \llave{l}{
              \im(f(r(-i))): 4r(r^4 +  r^2 - 12) = 0
              \Sii{$r \en \reales$}[$r > 0$]
            r = \sqrt{3} \llamada2                                           \\
              \re(f(r(-i))) : -r^6  - r^4  - 4r^2  + \c = 0
              \Sii{$\llamada2$}[$r = \sqrt{3}$]
              \cajaResultado{\c = 48}
            }
          \end{array}
        $$

        Por lo tanto con ese $\c = 48$:
        $$
          f = X^6 - 4X^5 - X^4 + 4X^3 + 4X^2 + 48X + 48
        $$
        y las raíces que tiene este polinomio son:
        $$
          \begin{array}{rcrcr}
            \alpha_1 & = & \sqrt{3} \cdot e^{i\frac{3}{2}\pi}  & = & -\sqrt{3}i \\
            \alpha_2 & = & \sqrt{3} \cdot e^{-i\frac{3}{2}\pi} & = & \sqrt{3}i
          \end{array}
        $$
        Apareció el conjugado de la raíz dado que que $f \en \racionales[X]$
  \item

        Debe ocurrir que $(X - \sqrt{3}i)(X + \sqrt{3}i) = X^2 + 3 \divideA f$
        $$
          \divPol{X^6 - 4X^5 - X^4 + 4X^3 + 4X^2 + 48X + 48}{X^2 + 3}
        $$

        Hasta el momento $f$ queda:
        $$
          f = (X^2 + 3)\ub{(X^4 - 4X^3 - 4X^2 + 16X + 16)}{g}
        $$
        como $f$ tiene al menos una raíz doble la busco en las raíces de la derivada de $g$:\par
        $$
          \begin{array}{rcl}
            g' & = & (4X^3 - 12X^2 - 8 X + 16)' \\
               & = & 4(X^3 - 3X^2 - 2X + 4) = 0 \\
          \end{array}
        $$
        Con el lema de Gauss se que las posibles raíces de $g'$ están en:
        $$
          \set{\pm1,\pm2,\pm4}
        $$
        Probando encuentro que  $g'(1) = 0$, pero $q(\magenta{1}) \distinto 0 \entonces f(1) \distinto 0$. Si $X=1$ no es raíz de g,
        continúo bajándole el grado a $g'$ para buscar otras raíces:
        $$
          \divPol{X^3 - 3X^2 - 2X + 4}{X - 1}
        $$
        Con este resultado se puede escribir a $g'$ como:
        $$
          g' = 4(X-1)(X^2 - 2X -4)
        $$
        De la parte cuadrática salen 2 raíces de $g'$:
        $$
          \begin{array}{c}
            \alpha_{1,2} = 1 \pm \sqrt{5} \\
            X^2 - 2X -4 = (X - (1 + \sqrt{5}))(X - (1 - \sqrt{5}))
          \end{array}
        $$
        {\tiny(para mostrar que son raíces dobles y no triples, por ejemplo, debería comprobar que $\alpha_{1,2}$ no son raíces de $g''=4(3X^2-6X-2)$, pero no tengo ganas,
            \href{\dontWorryAboutAThing}{elijo creer que no lo son}).}

        Compruebo que sean también raíces de $g$:
        $$
          \divPol{X^4 - 4X^3 - 4X^2 + 16X + 16}{X^2 - 2X -4}
        $$
        Dado que el resto dio 0 $\alpha_{1,2}$ son raíces de $g$ y como son raíces de $g'$ entonces son raíces dobles de $g$, y de $f$.

        Notar que viendo el cociente de esa última división quizás podría haber visto el caso de factoreo a ojo, pero bueh, no pasó.

        \textit{Factorizaciones: }
        \begin{center}
          \cajaResultado{
            \begin{array}{rcl}
              \racionales[X] & \to & f = (X^2 + 3) \cdot (X^2 - 2X - 4)^2                                                               \\
              \reales[X]     & \to & f = (X^2 + 3) \cdot (X - (1 + \sqrt{5}))^2  \cdot (X - (1 - \sqrt{5}))^2                           \\
              \complejos[X]  & \to & f =  (X-\sqrt{3}i) \cdot (X+\sqrt{3}i) \cdot  (X - (1 + \sqrt{5}))^2  \cdot (X - (1 - \sqrt{5}))^2
            \end{array}
          }
        \end{center}
\end{enumerate}
