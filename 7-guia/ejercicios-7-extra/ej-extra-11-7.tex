\begin{enunciado}{\ejExtra}
  Hallar todos los $k \en \racionales$ para los cuales el polinomio $f = X^6 + kX^3 + 25 \en \racionales[X]$
  tiene al menos una raíz compleja múltiple. Para cada uno de los valores de $k$ hallados, factorizar $f$
  en $\complejos[X], \reales[X] \ytext \racionales[X]$.
\end{enunciado}
Antes de empezar este ejercicio, estaría bueno que hagas un minuto de silencio por los que rindieron este examen...

\bigskip
\textit{1 minuto después}
\bigskip

Si $f$ tiene raíces múltiples, busco raíces en su derivada, $f'$:
$$
  f'(r)= 0 \sii f' = 6r^5 + 3kr^2 = 0 \sii 3r^2 \cdot (2r^3 + k) = 0 \sii k = -2r^3
$$
Entonces para el valor de una raíz $r$, tengo lo que tiene que valer $k$. Como tengo raíces múltiples, meto a $r$ y el valor
de $k$ encontrado en $f$:
$$
  f(r) = 0
  \Sii{$k=-2r^3$}
  r^6 -2r^6+ 25 = 0
  \sii
  \fcolorbox{orange}{white}{$r^6 = 25$}
$$

Ese último resultado es $G_6$ con módulo $\sqrt[3]{5}$:
$$
  r_q = \sqrt[3]{5} \cdot e^{i \frac{1}{3}q \pi} \quad\text{ con } \quad q \en [0,5]
$$

Estos valores son las raíces de $f$, pero hay que ver para cuál valor de $k$:
$$
  k = -2(r_q)^3
  \sii
  \llave{rcccr}{
    \text{si $q$ par}   & \entonces & -2 \big( \sqrt[3]{5} \cdot e^{i \frac{1}{3}q \pi} \big)^3 & = & -10 \\
    \text{si $q$ impar} & \entonces & -2 \big( \sqrt[3]{5} \cdot e^{i \frac{1}{3}q \pi} \big)^3 & = & 10
  }
$$

Por lo tanto hay 2 valores posibles para $k$:
$$
  k \en \set{-10, 10}
$$
Tengo entonces que factorizar 2 polinomios $f$:
$$
  \boxed{
    f_{-10} = X^6 \red{- 10}X^3 + 25
    \ytext
    f_{10} = X^6 \red{+ 10}X^3 + 25
  }
$$
Esto va a ser útil:
\begin{center}
  \blue{$(X - z)(X- \conj{z}) = X^2 - 2\re(z)X + |z|^2$}:
\end{center}

\bigskip

\textit{Factorizo $f_{-10}$:}\par

El valor $k = -10$ tiene las raíces con $q$ par:
$$
  \set{\sqrt[3]{5}, \sqrt[3]{5} \cdot e^{i\frac{2}{3}\pi}, \sqrt[3]{5} \cdot e^{i\frac{4}{3}\pi} } =
  \set{\sqrt[3]{5}, \sqrt[3]{5} \cdot (- \frac{1}{2} + i\frac{\sqrt{3}}{2}), \sqrt[3]{5} \cdot (- \frac{1}{2} - i\frac{\sqrt{3}}{2}) }.
$$
Que cosa horrible esto:
$$
  \bigg(X - \sqrt[3]{5} \cdot \Big(- \frac{1}{2} + i\frac{\sqrt{3}}{2}\Big)\bigg)
  \cdot
  \bigg(X - \sqrt[3]{5} \cdot \Big(- \frac{1}{2} - i\frac{\sqrt{3}}{2}\Big)\bigg)
  =
  \blue{X^2  + \sqrt[3]{5} \cdot X + (\sqrt[3]{5})^2}
$$

\begin{center}
  \fcolorbox{orange}{white}{
    $
      \begin{array}{rcl}
        f_{-10} & =               & X^6 - 10X^3 + 25                                                                            \\
                & \igual{\red{!}} & (X^3 - 5)^2  \en \racionales[X]                                                             \\
                & \igual{\red{!}} & \Big((X - \sqrt[3]{5}) \cdot (\blue{X^2  + \sqrt[3]{5} \cdot X + (\sqrt[3]{5})^2)}\Big)^2 =
        \Big(X - \sqrt[3]{5}\Big)^2 \cdot \Big(\blue{X^2  + \sqrt[3]{5} \cdot X + (\sqrt[3]{5})^2)}\Big)^2 \en \reales[X]       \\
                & \igual{\red{!}} &
        \Big(X - \sqrt[3]{5}\Big)^2 \cdot
        \Big(X - \sqrt[3]{5}  \big(- \frac{1}{2} - i\frac{\sqrt{3}}{2}\big)\Big)^2 \cdot
        \Big(X - \sqrt[3]{5}  \big(- \frac{1}{2} + i\frac{\sqrt{3}}{2}\big)\Big)^2 \en \complejos[X]
      \end{array}
    $
  }
\end{center}

\bigskip

\textit{Factorizo $f_{10}$:}

El valor $k = 10$ tiene las raíces con $q$ impar:
$$
  \set{\sqrt[3]{5}, \sqrt[3]{5} \cdot e^{i\frac{1}{3}\pi}, \sqrt[3]{5} \cdot e^{i\frac{5}{3}\pi} } =
  \set{\sqrt[3]{5}, \sqrt[3]{5} \cdot ( \frac{1}{2} + i\frac{\sqrt{3}}{2}), \sqrt[3]{5} \cdot ( \frac{1}{2} - i\frac{\sqrt{3}}{2}) }.
$$
Esto es igual de horrible, pero solo hay que cambiar un par de signos a lo de antes:
$$
  \bigg(X - \sqrt[3]{5} \cdot \Big( \frac{1}{2} + i\frac{\sqrt{3}}{2}\Big)\bigg)
  \cdot
  \bigg(X - \sqrt[3]{5} \cdot \Big( \frac{1}{2} - i\frac{\sqrt{3}}{2}\Big)\bigg)
  =
  \blue{X^2  - \sqrt[3]{5} \cdot X + (\sqrt[3]{5})^2}
$$
\fcolorbox{orange}{white}{
  $
    \begin{array}{rcl}
      f_{10} & =               & X^6 + 10X^3 + 25                                                                            \\
             & \igual{\red{!}} & (X^3 + 5)^2  \en \racionales[X]                                                             \\
             & \igual{\red{!}} & \Big((X + \sqrt[3]{5}) \cdot (\blue{X^2  - \sqrt[3]{5} \cdot X + (\sqrt[3]{5})^2)}\Big)^2 =
      \Big(X + \sqrt[3]{5}\Big)^2 \cdot \Big(\blue{X^2  - \sqrt[3]{5} \cdot X + (\sqrt[3]{5})^2)}\Big)^2 \en \reales[X]      \\
             & \igual{\red{!}} &
      \Big(X + \sqrt[3]{5}\Big)^2 \cdot
      \Big(X - \sqrt[3]{5}  \big( \frac{1}{2} - i\frac{\sqrt{3}}{2}\big)\Big)^2 \cdot
      \Big(X - \sqrt[3]{5}  \big( \frac{1}{2} + i\frac{\sqrt{3}}{2}\big)\Big)^2 \en \complejos[X]
    \end{array}
  $
}

% Contribuciones
\begin{aportes}
  %% iconos : \github, \instagram, \tiktok, \linkedin
  %\aporte{url}{nombre icono}
  \item \aporte{https://github.com/nad-garraz}{Nad Garraz \github}
\end{aportes}
