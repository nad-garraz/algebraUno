\begin{enunciado}{\ejExtra}
  Sea $f \en \racionales[X]$ el polinomio mónico de grado mínimo que satisface simultáneamente
  \begin{enumerate}[label={\tiny $\blacksquare$}]
    \item $X^2 + 2X + 5$ divide a $(f:f')$,
    \item $X^2 - 4X + 1$ divide a $(f:f'')$,
    \item $f'(2-\sqrt{3}) = 0$.
  \end{enumerate}
  Hallar la factorización de $f$ en $\complejos[X]$, en $\reales[X]$ y en $\racionales[X]$.
\end{enunciado}

Acomodo un poco el enunciado:
$$
  \begin{array}{l}
    X^2 + 2X + 5
    \igual{\red{!}}
    (X - (-1 + 2i)) \cdot (X - (-1 - 2i))                  \\
    X^2 -4 X + 1
    \igual{\red{!}}
    (X - (2 + \sqrt{3})) \cdot (X - (\blue{2 - \sqrt{3}})) \\
  \end{array}
$$
Tengo que $\blue{2 - \sqrt{3}}$ por lo menos es triple ya que por la segunda y tercera condición:
$$
  \begin{array}{l}
    (X - (\blue{2 - \sqrt{3}}) \divideA f                                        , \\
    (X - (\blue{2 - \sqrt{3}}) \divideA f' \qquad \text{ por tercerca condición}   \\
    \ytext                                                                         \\
    (X - (\blue{2 - \sqrt{3}}) \divideA f'' \qquad \text{ por segunda condición}   \\
  \end{array}
$$

Las raíces complejas $-1 + 2i$ y $-1 - 2i$ son por lo menos dobles.

\bigskip

Para que se cumpla la segunda condición las 2 raíces irracionales van a tener que tener la misma multiplicidad.

\medskip

Factorización en $\racionales[X]$:
$$
  \cajaResultado{
    f =
    (X^2 + 2X + 5)^2
    \cdot
    (X^2 - 4X + 1)^3
  }
$$

Factorización en $\reales[X]$:
$$
  \cajaResultado{
    f =
    (X^2 + 2X + 5)^2
    \cdot
    (X - (\blue{2 - \sqrt{3}}))^3
    \cdot
    (X - (2 + \sqrt{3}))^3
  }
$$

Factorización en $\complejos[X]$:
$$
  \cajaResultado{
    f =
    (X - (-1 + 4i))^2 \cdot (X - (-1 - 4i))^2                  \\
    \cdot
    (X - (\blue{2 - \sqrt{3}}))^3
    \cdot
    (X - (2 + \sqrt{3}))^3
  }
$$
\begin{aportes}
  \item \aporte{\dirRepo}{naD GarRaz \github}
  \item \aporte{https://github.com/olivportero}{Olivia Portero \github}
\end{aportes}
