\begin{enunciado}{\ejExtra}
  Factorizar el polinomio
  $
 P = X^6 - X^5 - 13X^4 + 14X^3 + 35X^2 -49X + 49
  $
  como producto de irreducibles en $\complejos[X], \reales[X] \ytext \racionales[X]$ sabiendo que $\sqrt7$ es una
  raíz múltiple.

\end{enunciado}

\begin{center}
  Un polinomio con coeficientes racionales, y una raíz irracional $\alpha = \sqrt7$,
  tendrá también al \textit{conjugado irracional}
  \footnote{Estoy usando la misma notación para \textit{conjugado racional} y
    \textit{conjugado complejo}. ¿Está bien? No sé, no me importa mientras se entienda.}
  , $\conj{\alpha} = -\sqrt{7}$\par

  Si agregamos la información de que  $\sqrt7$ es \textit{por lo menos} raíz doble, obtenemos que:\par
\end{center}
$$
  \llave{l}{
    \sqrt7 \text{ es raíz de } f
    \entonces
    -\sqrt7 \text{ es raíz de } f
    \entonces
    (X^2 - 7) \divideA f\\
    \sqrt7 \text{ es raíz doble de } f
    \entonces
    -\sqrt7 \text{ es raíz doble de } f
    \entonces
    (X^2 - 7)^2 = X^4 - 14X^2 + 49 \divideA f \Tilde\\
  }
$$

$$
  \divPol{X^6-X^5-13X^4+14X^3+35X^2-49X+49}{X^4-14X^2+49}
$$

Todo hermoso. Nos queda un polinomio de grado 2 para laburar en la factorización:
$$
  f = (X^4-14X^2+49) \cdot (X^2 - X + 1),
$$
se fusila con la resolvente:
$$
  \flecha{se escribe así para}[no ofender a nadie]
  \llave{l}{
    \alpha_{+,-} = \frac{1 \pm \blue{w}}{2}\\
    \blue{w}^2 = -3
  }
  \to
  f = (X^4-14X^2+49) \cdot  \textstyle (X - (\frac{1}{2} + i\frac{\sqrt3 }{2})) (X - (\frac{1}{2} -i \frac{\sqrt3 }{2}))
$$
Finalmente las factorizaciones en sus 3 deliciosos sabores:
$$
  \cajaResultado{
    \llave{rcl}{
      \racionales[X] & \to & f = (X^2 + 7)^2  (X^2 - X + 1) \\
      \reales[X] & \to & f = (X + \sqrt7)^2(X - \sqrt7)^2  (X^2 - X + 1)\\
      \complejos[X] & \to & f = (X + \sqrt7)^2(X - \sqrt7)^2  (X - (\frac{1}{2} + i\frac{\sqrt3 }{2})) (X - (\frac{1}{2} - i\frac{\sqrt3 }{2}))
    }}
$$

\begin{aportes}
  \item \aporte{\dirRepo}{naD GarRaz \github}
\end{aportes}
