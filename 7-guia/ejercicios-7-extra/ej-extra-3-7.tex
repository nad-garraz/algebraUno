\begin{enunciado}{\ejExtra}
  Hallar \textbf{todos} los polinomios \textbf{mónicos} $f \en \racionales[X]$
  de grado mínimo que cumplan simultáneamente las siguientes condiciones:
  \begin{enumerate}[label=\roman*)]
    \item $1 - \sqrt{2}$ es raíz de $f$;
    \item $X(X-2)^2 \divideA (f:f')$;
    \item $(f:X^3 - 1) \distinto 1$;
    \item $f(-1) = 27$;
  \end{enumerate}

\end{enunciado}

\begin{enumerate}[label=\roman*)]
  \item Como $f \en \racionales[X]$ si $\alpha_1 = 1 - \sqrt{2}$ es raíz entonces  $\alpha_2 = 1 + \sqrt{2}$
        para que no haya coeficientes irracionales en el polinomio.\\
        $$
          (X - (1 - \sqrt{2}))
          \cdot
          (X - (1 + \sqrt{2}))
          =
          X^2 - 2X - 1
        $$
        Por lo tanto $X^2 - 2X - 1$ será un factor de $f \en \racionales[X]$.

  \item Si
        $X(X-2)^2 \divideA (f:f')
          \entonces
          \llave{l}{
            \alpha_3 = 0
            \text{ raíz simple de $f'$}
            \entonces
            \text{ raíz doble de $f$} \\
            \alpha_4 = 2
            \text{ raíz simple de $f'$}
            \entonces
            \text{ raíz doble de $f$}
          }
        $
        Por lo tanto $X^2(X-2)^3$ serán factores de $f$.

  \item Si $(f:X^3 - 1) \distinto 1$ quiere decir que por lo menos alguna de las 3 raíces de:\\
        $
          X^3 - 1 =
          (X-1)\cdot (X - (-\frac{1}{2} + \frac{\sqrt{3}}{2}))\cdot (X - (-\frac{1}{2} - \frac{\sqrt{3}}{2}))$
        tiene que aparecer en la factorización de $f$. Pero
        parecido al item i) si tengo una raíz compleja, también necesito el conjugado complejo, para que no me queden
        coeficientes de $f$ en complejos,\\
        $ X^3 - 1 = (X-1) \cdot (X^2 + X +1)$, me quedaría con el \textit{factor de menor grado} si eso no \textit{rompe} otras condiciones.\\

        Por lo tanto $(X-1)$ o $(X^2 + X +1)$ aparecerá en la factorización de $f$.

  \item $f(-1) = 27$. Hasta el momento:\\
        $$
          \llave{l}{
            \magenta{f_1} = (X^2 - 2X - 1)\cdot (X-2)^3 \cdot X^2 \cdot (X^2 + X + 1) \to \magenta{f_1}(-1) = 2 \cdot (-27) \cdot 1 \cdot 1 = -54 \\
            \blue{f_2} = (X^2 - 2X - 1)\cdot (X-2)^3 \cdot X^2 \cdot (X-1) \to \blue{f_2}(-1) =  2 \cdot (-27) \cdot 1 \cdot (-2) = 108
          },
        $$
        ninguno cumple la condición iv).
\end{enumerate}
Para encontrar \textit{un} polinomio que cumpla lo pedido tomaría el $\blue{f_2}$ que tiene \red{menor grado} de los dos y lo multiplicaría por:
$$
  f = \blue{f_2} \cdot \yellow{(X - a)} \quad \text{con} \quad a \en \racionales
$$

de manera que:
$$
  f(-1) = \blue{f_2}(-1) \cdot \yellow{(X - a)}  = 108 \cdot (-1 - a) = 27 \sii a = -\frac{5}{4}
$$
$$
  f =  (X^2 - 2X - 1)\cdot (X-2)^3 \cdot X^2 \cdot (X-1) \cdot (X + \frac{5}{4})
$$
así cumpliendo todas las condiciones.

% Contribuciones
\begin{aportes}
	%% iconos : \github, \instagram, \tiktok, \linkedin
	%\aporte{url}{nombre icono}
	\item \aporte{https://github.com/nad-garraz}{Nad Garraz \github}
	\item \aporte{https://github.com/JowinTeran}{Ale Teran \github}
\end{aportes}
