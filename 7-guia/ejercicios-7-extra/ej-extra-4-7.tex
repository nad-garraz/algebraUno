\begin{enunciado}{\ejExtra}
  Factorizar como producto de polinomios irreducibles en
  $\racionales[X], \reales[X], \complejos[X]$ al polinomio

  $$
    f= X^5 + 2X^4 - 7X^3 - 7X^2 + 10X -15
  $$
  sabiendo $(f:X^4 - X^3 +6X^2  -5X +5) \distinto 1$

\end{enunciado}

Si el $(f:X^4 - X^3 +6X^2  -5X +5) \distinto 1$, esto nos da información
sobre \textit{raíces comunes} entre $f$ y $X^4 - X^3 +6X^2  -5X +5$. Puedo hacer el algoritmo de Euclides para encontrar el MCD, con esa
o esas raíces. El último resto no nulo hecho \textbf{mónico} será el MCD.

\medskip

{\tiny
$$
      \mcd{X^5 + 2X^4 - 7X^3 - 7X^2 + 10X -15}{X^4 - X^3 +6X^2  -5X +5}
$$
}
Por lo que:

$$
  (f:X^4 - X^3 +6X^2  -5X +5) = X^2 - X + 1.
$$

Las raíces del MCD son $\alpha_{1,2} = \frac{1 \pm \magenta{w}}{2}$ con $\magenta{w}^2 = 3i$.

$$
  X^2 - X + 1 = (X - (\frac{1}{2}  - \frac{\sqrt{3}}{2} ))(X - (\frac{1}{2}  + \frac{\sqrt{3}}{2} ))\Tilde
$$
Por definición de lo que es el MCD sabemos que $X^2 - X + 1 \divideA f$, haciendo la división
bajamos el grado y seguimos buscando las raíces.

\medskip

$$
  \divPol{X^5 + 2X^4 - 7X^3 - 7X^2 + 10X -15}{X^2 - X + 1}
$$
\medskip

Obtuvimos que:
$$
  f = (X^2 - X + 1) \cdot (\magenta{X^3 + 3X^2 -5X - 15}) + 0.
$$

Hermoso resultado, donde la hermosura se mide en su simpleza para ser factorizado.
\underline{Sin} usar calculadora ni Guass ni ninguna cosa extraña podemos expresar a $f$ como:

$$
  f \igual{\red{!!!}} (X^2 - X + 1) \cdot \ub{\magenta{(X-\sqrt{5}) \cdot (X + \sqrt{5}) \cdot (X + 3)}}{X^3 + 3X^2 -5X - 15}
$$

\underline{Si todavía no viste como fue la factorización en \red!}
te recomiendo que sigas mirando sin \textit{spoiler de calculadora o del pesado o pesada sabelotodo}
que quizás tenés al lado y que no te deja tiempo para pensar. Son puros casos de factoreo que \textit{deberían verse a ojo}.

Ahora factorizamos en irreducibles, que son polinomios mónicos que  solo se dividen por
sí mismos y por 1, \textit{los primos en el mundo de polinomios}. Para tener una mejor explicación
\hyperlink{teoria-7:irreducibles}{clickeá acá! Y vas a la teoría del apunte.}

\textit{Factorizaciones: }
$$
  \cajaResultado{
    \begin{array}{rcl}
      \racionales[X] & \to & f =  (X^2 - 5) \cdot
      (X^2 - X + 1) \cdot
      (X+3)                                       \\
      \reales[X]     & \to & f = (X-\sqrt5) \cdot
      (X + \sqrt5)  \cdot
      (X^2 - X + 1) \cdot
      (X+3)                                       \\
      \complejos[X]  & \to & f =
      (X+3 \cdot
      (X - \sqrt5) \cdot
      (X + \sqrt5) \cdot
      (X - (\frac{1}{2}  - \frac{\sqrt{3}}{2} i)) \cdot
      (X - (\frac{1}{2}  + \frac{\sqrt{3}}{2} i))
    \end{array}
  }
$$

\begin{aportes}
  \item \aporte{\dirRepo}{naD GarRaz \github}
  \item \aporte{https://github.com/JowinTeran}{Ale Teran \github}
\end{aportes}
