\begin{enunciado}{\ejExtra}
  Sea $(f_n)_{(n\geq 1)}$ la sucesión de poliniomios en $\reales[X]$ definida como:\par
  $$
    f_1 = X^5 + 3X^4 + 5X^3 + 11X^2 - 20 \ytext f_{n+1} = (X + 2)^2 f'_n + 3 f_n, \text{ para cada } n \en \naturales.
  $$\par
  Probar que $-2$ es raíz doble de $f_n$ para todo $n \en \naturales$.
\end{enunciado}

No caer en la \red{\textit{trampilla}$^{\text{\faIcon{skull}}}$} de olvidar que para que una raíz de
$f$ sea doble, i.e.  $\mult(-2;f) \red{\igual{!}} 2$ debe ocurrir lo \textit{"obvio"},
$f(-2) = f'(-2) = 0$ y también que $\red{f^{''}(-2) \distinto 0}$.
Si olvidamos esto último solo probaríamos que la $\mult(-1;f) \geq 2$ y tendríamos el ejercicio mal \red{\faIcon{skull}}.

\textit{Por inducción en $n$: }
$$
  p(n):  \text{$-2$ es raíz doble de } f_n,\, \paratodo n \en \naturales
$$

\textit{Caso base:}
$$
  p(\blue{1}) :  \text{$-2$ es raíz doble de } f_{\blue{1}}
$$
Derivar y evaluar:
$$
  \llave{lcl}{
    f_{\blue{1}} = X^5 + 3X^4 + 5X^3 + 11X^2 - 20 &
    \flecha{evaluar}[en $-2$]            &
    f_{\blue{1}}(-2) = 0                            \\

    f'_{\blue{1}} = 5X^4 + 12X^3 + 15X^2 + 22X    &
  \flecha{evaluar}[en $-2$]            &
  f'_{\blue{1}}(-2) = 0                           \\

  f^{''}_{\blue{1}} = 20X^3 + 36X^2 + 30X + 22  &
  \flecha{evaluar}[en $-2$]            &
  f^{''}_{\blue{1}}(-2) = -54 \distinto 0
  }
  $$

    Por lo tanto $\mult(-2; f_1) = 2 \entonces -2$ es raíz doble de $f_{\blue{1}} \entonces p(\blue{1})$ resultó ser verdadera.

    \medskip

    \textit{Paso inductivo:}

    Asumo que para algún $\blue{k} \en \naturales$
  $$
  p(\blue{k}) :  \ub{
    \text{$-2$ es raíz doble de } f_{\blue{k}}
  }{
    \text{\purple{hipótesis inductiva}}
  }
  $$
    es verdadera. Entonces quiero probar que:
  $$
  p(\blue{k+1}) : \text{$-2$ es raíz doble de } f_{\blue{k+1}}
  $$
    también lo sea.

    Sé que:
  $$f_{\blue{k}}
  \Sii{cumple}[que]
  \llave{l}{
    f_{\blue{k}}(-2) = 0 \llamada1  \\
    f'_{\blue{k}}(-2) = 0 \llamada2 \\
  f''_{\blue{k}}(-2) \distinto 0 \llamada3
  }
  $$

    Laburo con $f_{\blue{k+1}}$:
  $$
  \llave{l}{
    f_{\blue{k+1}} \igual{def} (X + 2)^2 f'_{\blue{k}} + 3 \cdot f_{\blue{k}} \\
  f'_{\blue{k+1}}  = 2(X+2)f'_{\blue{k}} + (X+2)^2 f^{''}_{\blue{k}} + 3\cdot f'_{\blue{k}}\\
  f^{''}_{\blue{k+1}} =
  2f'_{\blue{k}} + (2X+4)f^{''}_{\blue{k}} + 2(X+2) f^{''}_{\blue{k}} + (X+2)^2 f^{'''}_{\blue{k}} + 3 \cdot f^{''}_{\blue{k}}
  }
  $$

    \medskip

    \textit{Evaluar en $-2$:}
  $$
  \begin{array}{lll}
    f_{\blue{k+1}}(-2) \igual{?} 0
     & \sii &
    f_{\blue{k+1}}(-2) = \cancel{(-2 + 2)}^2 f'_{\blue{k}}(-2) + 3f_{\blue{k}}(-2) =
    0^2 f'_{\blue{k}}(-2) + 3 f_{\blue{k}}(-2)=
    3 f_{\blue{k}}(-2) \igual{$\llamada1$}
    0 \Tilde \vspace{10pt}                                      \\
    f'_{\blue{k+1}}(-2) \igual{?} 0
     & \sii &
    f_{\blue{k+1}}'(-2) =
    2\cancel{(-2 + 2)}f'_{\blue{k}}(-2) + \cancel{(-2 + 2)}^2 f^{''} + f'_{\blue{k}}(-2) =
    f'_{\blue{k}}(-2) \igual{$\llamada2$} 0 \Tilde\vspace{10pt} \\
    f^{''}_{\blue{k+1}}(-2) \taa{?}{}\distinto 0
     & \sii &
    \llave{l}{
      {f^{''}_{\blue{k+1}}(-2) = 2f'_{\blue{k}}(-2) +
    2\cancel{(-2+2)}f^{''}_{\blue{k}}(-2) +
    2\cancel{(-2+2)} f^{''}_{\blue{k}}(-2) } +                  \\
    + \cancel{(-2+2)}^2 f^{'''}_{\blue{k}}(-2) +
    f^{''}_{\blue{k}}(-2) = 2\ob{f'_{\blue{k}}(-2)}{= 0 \llamada2} + \ob{f^{''}_{\blue{k}}(-2)}{\distinto 0 \llamada3}
    \distinto 0 \Tilde
    }
  \end{array}
$$

Por lo tanto $\mult(-2; f_{\blue{k+1}}) = 2 \entonces -2$ es raíz doble de $f_{\blue{k+1}} \entonces p(\blue{k+1})$ es verdadera también.

\bigskip

Como $p(\blue{1}),\, p(\blue{k})$ y $q(\blue{k+1})$ resultaron verdaderas, por principio de inducción $p(n)$ también lo es $\paratodo n \en \naturales$.

% Contribuciones
\begin{aportes}
  \item \aporte{\dirRepo}{naD GarRaz \github}
  \item \aporte{https://github.com/daniTadd}{Dani Tadd \github}
  \item \aporte{\neverGonnaGiveYouUp}{autor original anónimo \youtube}
\end{aportes}
