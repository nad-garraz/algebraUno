\begin{enunciado}{\ejExtra}
  \begin{enumerate}[label=\alph*)]
    \item
          Determinar todos los $f \en \reales[X]$ mónicos de grado mínimo tales que cumplan:
          \begin{itemize}
            \item $f$ contiene entre sus raíces al menos una raíz cúbica de la unidad,
            \item $X^2 + 1 \divideA (f:f')$,
            \item $f$ tiene al menos 2 raíces enteras,
            \item $f(1) = -12$,
          \end{itemize}

    \item Con el polinomio $f$ hallado expresar factorización en irreducibles en $\racionales[X], \reales[X] \ytext \complejos[X]$.
  \end{enumerate}
\end{enunciado}

\begin{enumerate}[label=\alph*)]
  \item
        Arrancando con la primera condición, tenemos al menos a una de las $w$ tales que:\par
        $
          w^3 = 1
          \sii
          \llave{l}{
            w_1 = 1                   \llamada1                    \\
            w_2 = -\frac{1}{2} + \frac{\sqrt{3}}{2}i \\
            w_3 = -\frac{1}{2} - \frac{\sqrt{3}}{2}i
          }.$
        \par\medskip

        Si no te acordás como calcular las raíces, mirá el ejercicio \ref{ej:12}, donde se resuelve algo casi idéntico.

        Como el polinomio \textit{debe ser de grado mínimo} y tiene coeficientes en $\reales$ hay que elegir con \underline{cuidado}. Lo mejor es ver el resto
        de las condiciones para no hacer \textit{cagadas}. {\color{lightgray}\tiny(¡spoiler alert: Elegí el $1$ si sos picante!)}
        \par\bigskip

        De la segunda condición sacamos que:
        $$
          X^2 + 1 = (X - i)\cdot(X+i) \divideA (f:f')
          \sii
          \llaves{c}{
            X^2 + 1 \divideA f
            \sii
            \llave{c}{
              (X - i) \divideA f \\
              \ytext             \\
              (X + i) \divideA f
            }  \\
            \ytext              \\
            X^2 + 1 \divideA f'
            \sii
            \llave{c}{
              (X - i) \divideA f' \\
              \ytext              \\
              (X + i) \divideA f'
            } \\
          }
          \sii
          \llave{c}{
            (X - i)^2 \divideA f \\
            \ytext               \\
            (X + i)^2 \divideA f \\
          }
        $$

        Si no entendés el porqué de eso mirate \hyperlink{teoria-7:raicesMultiples}{esto de las notas teóricas}, para tener contexto.
        Básicamente si $\alpha$ es una raíz de $f$ y también de $f'$, entonces es una raíz \textit{por lo menos} doble de $f$.\par\bigskip

        En el tercer punto, nos dicen que tiene al menos 2 raíces en $\enteros$. ¿Una de esas podría ser el 1 que obtuvimos como raíz de $G_3$?,
        Dejame que lo piense.

        En el último punto tenemos que cumplir que al evaluar en nuestro polinomio $f$ en 1, eso nos dé $-12$. Y es acá donde nos damos cuenta de que
        no podemos elegir a $1\llamada1$ para que sea raíz de $f$\red{!!}
        Y dado que $f \en \reales[X]$ tenemos que elegir entonces \underline{ambas}
        $\llave{l}{
            w_2 = -\frac{1}{2} + \frac{\sqrt{3}}{2}i \\
            w_3 = -\frac{1}{2} - \frac{\sqrt{3}}{2}i
          }.$
        Propongo:
        $$
          \begin{array}{rcl}
            f & =                   &
            (X - (-\frac{1}{2} - \frac{\sqrt{3}}{2}i) )
            (X - (-\frac{1}{2} + \frac{\sqrt{3}}{2}i) )
            (X - i)^2
            (X + i)^2
            (X - \magenta{a})
            (X - \magenta{b})         \\
              & \igual{$\llamada2$} &
            (X^2 + X + 1)
            (X^2 + 1)^2
            (X - \magenta{a})
            (X - \magenta{b})
            ,
          \end{array}
        $$
        con $\magenta{a}$ y $\magenta{b}$ a determinar, de manera tal de cumplir las últimas dos condiciones: \textit{ambas} enteras y $f(1) = -12$.
        $$
          f(1) = -12
          \Sii{$\llamada2$}
          12 \cdot (1 - \magenta{a})(1 - \magenta{b}) = -12
          \sisolosi
          (1 - \magenta{a})(1 - \magenta{b}) = -1
          \Sii{$\magenta{a} \ytext \magenta{b}$}[$\en \enteros$]
          \magenta{a} = 2 \ytext \magenta{b} = 0.
        $$
        Esas serían las candidatas a raíces enteras, obteniendo así un único polinomio
        $$
          f =
          (X^2 + X + 1)
          (X^2 + 1)^2
          (X - 2)
          (X - 0) =
          X
          (X - 2)
          (X^2 + X + 1)
          (X^2 + 1)^2
        $$
        mónico y de grado mínimo que cumple las condiciones pedidas.

  \item La definición de polinomio irreducible \hyperlink{teoria-7:irreducibles}{está acá}.\par
        $$
          \boxed{
            \begin{array}{rcl}
              \racionales[X] & \to & f = X (X - 2) (X^2 + X + 1) (X^2 + 1)^2                                                             \\
              \reales[X]     & \to & f = X (X - 2) (X^2 + X + 1) (X^2 + 1)^2                                                             \\
              \complejos[X]  & \to & f = X (X-2) (X - (-\frac{1}{2} - \frac{\sqrt{3}}{2}i) ) (X - (-\frac{1}{2} + \frac{\sqrt{3}}{2}i) )
              (X - i)^2 (X + i)^2
            \end{array}
          }
        $$
        Notar que en $\racionales$ y en $\reales$ las factorizaciones son iguales, dado que no hay raíces irracionales.
\end{enumerate}

% Contribuciones
\begin{aportes}
  %% iconos : \github, \instagram, \tiktok, \linkedin
  %\aporte{url}{nombre icono}
  \item \aporte{\dirRepo}{naD GarRaz \github}
  \item \aporte{https://github.com/daniTadd}{Dani Tadd \github}
\end{aportes}
