\begin{enunciado}{\ejercicio}
  Calcular el grado y el coeficiente principal de los siguientes polinomios en $\racionales[X]$:
  \begin{enumerate}[label=\roman*)]
    \item $(4X^6 - 2X^5 + 3X^2 - 2X + 7)^{77}$,
    \item $(-3X^7 + 5X^3 + X^2 - X + 5)^4 - (6X^4 + 2X^3 + X - 2)^7$,
    \item $(-3X^5 + X^4 - X + 5)^4 - 81X^{20} + 19X^{19}$,
  \end{enumerate}

\end{enunciado}

\begin{enumerate}[label=\roman*)]
  \item \textit{coeficiente principal: } $4^{77}$\par
        \textit{grado: } $6\cdot 77$\par

  \item \textit{coeficiente principal: } $(-3)^4 - 6^7 = -279.855$\par
        \textit{grado: } $28$\par

  \item \textit{coeficiente principal: }
        $(\ub{-3X^5 + X^4 - X + 5}{f})^4 + \ub{-81X^{20} + 19X^{19}}{g}$\par
        Cuando sumo me queda: $\cp{f^4} - \cp{g} = (-3)^4 - 81 = 0 \entonces gr(f^4 + g) < 20 $\par
        $\to$ Calculo el $\cp{f^4+g}$ con $\gr(f^4 + g ) = 19$.\par
        \textit{Laburo a $f$: }\par
        $
          \llave{l}{
          \flecha{para usar}[fórmula de $f\cdot g$]
          (-3X^5 + X^4 - X + 5)^4 = (\blue{-3}X^5 + \magenta{1}X^4 - X + 5)^2 \cdot (-3X^5 + X^4 - X + 5)^2        \\
          f^2 \cdot f^2 = \sumatoria{k=0}{20} \parentesis{ \sumatoria{i+j=k}{ }a_i \cdot b_j }X^k
          \text{ con $a_i$ y $b_i$ los coeficientes de $f^2$ y el otro $f^2$ respectivamente }\llamada2            \\
          \sumatoria{k=0}{20}\parentesis{\sumatoria{i+j=k}{ }a_i \cdot b_j}X^k
          \flecha{me interesa solo}[el término con $k = 19$]
          \sumatoria{i+j=19}{ } a_ib_j X^{19}
          \igual{$\llamada1$} a_9 \cdot b_{10}  + a_{10} \cdot b_9
          \igual{$\llamada2$} 2\cdot a_9 \cdot  b_{10}                                                             \\
          \llave{l}{
          \flecha{$b_{10}$ sale a}[ojímetro] b_{10} = (\blue{-3})^2 = 9 \\
          \flecha{$a_9$ no tan fácil, volver}[a usar $\sum f\cdot g$ en $k=9$] f\cdot f =
          \sumatoria{k=0}{10}\parentesis{\sumatoria{i+j=k}{ }c_i \cdot d_j}X^k
          \flecha{$k = 9$}
          \sumatoria{i+j = 9}{ }c_i \cdot d_j X^9
          \igual{$\llamada3$}
          \magenta{c_4} \cdot \blue{d_5} + \blue{c_5} \cdot \magenta{d_4}
          \igual{$\llamada2$}
          2\cdot \magenta{c_4} \cdot \blue{d_5}
          } \\
          \llaves{l}{
            \flecha{$\blue{d_5}$ sale a}[ojímetro] \blue{d_5} = \blue{-3} \\
            \flecha{$\magenta{c_4}$ sale a}[ojímetro] \magenta{c_4} = 1
          }\to a_9 = -6
          \to
          \llaves{l}{
            \cp{f^4} = 2\cdot(-6) \cdot (9) = -108 \\
            \cp{g} = 19 \\
          }
          \to \boxed{\cp{f^4+g} = -89}
          \Tilde
          } $

        $\llamada1$: Sabemos que el $\gr(f^4) = 20 \entonces \gr(f^2) = 10$. Viendo las posibles combinaciones al multiplicar 2 polinomios
        de manera tal que los exponentes de las $X$ sumen 19, es decir $X^i\cdot X^j = X^{19}$ con $i,j \leq 10$
        solo puede ocurrir \textit{cuando los exponentes}
        $\llaves{c}{
            i = 10,\, j = 9\\
            \o\\
            i = 9,\, j = 10\\
          }$\\

        $\llamada2$: porque estoy multiplicando el mismo polinomio, $a_i = b_i$. Pero lo dejo distinto para hacerlos \textit{visualmente} más genérico.\\

        $\llamada3$: Idem $\llamada1$ para el polinomio $f$

        \textit{grado: 19 }\\
\end{enumerate}
