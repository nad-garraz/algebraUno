\begin{enunciado}{\ejercicio}
  Hallar la forma binomial de cada una de las raíces complejas del polinomio $f(X) = X^6 + X^3 - 2$.
\end{enunciado}

Primera raíz a \textit{ojímetro}:
$$
  f(\alpha_1 = 1) = 0
  \sisolosi
  f(X) = q(X) \cdot (X - 1)
  \sisolosi
  (X - 1) \divideA f
  \sisolosi
  r_{_{(X-1)}}(f) = 0
$$
Busco $q(X)$ con algoritmo de división.
{\footnotesize
$$
  \polylongdiv[style=D]{X^6 + X^3 - 2 }{X - 1}
$$
}

El cociente $q(X)$ se puede factorizar en grupos como:
$$
  q(X) =
  X^5 + X^4 + X^3 + 2X^2 + 2X + 2
  \igual{\red{!}}
  (X^2+X+1) \cdot (X^3 + 2).
$$
Entonces las 5 raíces que me faltan para tener las 6 que debe tener $f \en \complejos[X]$ salen de esos dos polinomios.
Salen fácil las del polinomio de grado 2:
$$
  X^2 + X +1 = 0
  \sii
  \cajaResultado{
    \begin{array}{l}
      \alpha_2 = -\frac{1}{2} + \frac{\sqrt{3}}{2} \\
      \alpha_3 = -\frac{1}{2} - \frac{\sqrt{3}}{2}
    \end{array}
  }
$$

Resuelvo la ecuación $ X^3 + 2 = 0$ usando la notación exponencial del número complejo:
$$
  X = re^{i\theta}
$$
Reemplazo y máquina de hacer chorizos:
$$
  \llaves{l}{
    r^3 = 2 \to r = \sqrt[3]{2}\\
    3\theta = \pi + \magenta{2k\pi} \to \theta = \frac{\pi}{3} + \frac{2k\pi}{3} \text{ con } k = 0,\, 1,\, 2.
  } \to
  \cajaResultado{
    \begin{array}{l}
      \alpha_4 = \sqrt[3]{2} e^{i \frac{\pi}{3}} = \sqrt[3]{2} (\frac{1}{2} + i \frac{\sqrt{3}}{2}) \\
      \alpha_5 = \sqrt[3]{2} e^{i \pi}  = -\sqrt[3]{2}                                              \\
      \alpha_6 = \sqrt[3]{2} e^{i \frac{5\pi}{3}} = \sqrt[3]{2} (\frac{1}{2} - i \frac{\sqrt{3}}{2})
    \end{array}
  }
$$

\begin{aportes}
  \item \aporte{\dirRepo}{naD GarRaz \github}
\end{aportes}
