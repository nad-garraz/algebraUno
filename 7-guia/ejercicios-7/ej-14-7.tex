\begin{enunciado}{\ejercicio}
  \begin{enumerate}[label=\roman*)]
    \item Probar que si $w = e^{\frac{2\pi}{5}i} \en G_5$, entonces $X^2 + X -1 = [X - (w + w^{-1})] \cdot [X - (w^2 + w^{-2})]$.

    \item Calcula, justificando cuidadosamente, el valor exacto de $\cos(\frac{2\pi}{5})$.
  \end{enumerate}
\end{enunciado}
\begin{enumerate}[label=\roman*)]
  \item\label{ej14:item}
        Voy a usar que si $w \en G_5
          \entonces
          \llave{l}{
            \sumatoria{j=0}{4} w^j = 0 \quad (w \distinto 1) \llamada2\\
            w^k = w^{r_5(k)} \llamada{1}
          }$\\

        $ X^2 + X -1 =
          [X - (w + w^{-1})] \cdot [X - (w^2 + w^{-2})] =\\
          X^2 - (w^2 + w^{-2}) X - (w + w^{-1}) X + \ub{(w + w^{-1})(w^2 + w^{-2})}{\llamada1}=\\
          = X^2 - X(\ub{w^2 + w^{-2} + w + w^{-1}}{\llamada1}) + \ub{ w + w^2 + w^3 + w^4 }{\llamada2}=\\
          = X^2 - X(\ub{w + w^2 + w^3 + w^4}{\llamada2}) + \magenta{-1} + \ub{ \magenta{1} + w + w^2 + w^3 + w^4}{ = 0} =
          {X^2 - X(\magenta{-1} + \ub{ \magenta{1} + w + w^2 + w^3 + w^4}{ = 0}) - 1 =}\\
          = X^2 + X - 1 \Tilde
        $

  \item Calculando las raíces a mano de
        $$
          X^2 + X - 1
          \to
          \llave{c}{
            \frac{-1 + \sqrt{5}}{2}\\
            \text{y}\\
            \frac{-1 - \sqrt{5}}{2}
          }
        $$

        Pero del resultado del inciso \ref{ej14:item} tengo que :\\
        $
          w = e^{i\frac{2\pi}{5}}
          \flecha{sé que una raíz dada}[la factorización es]
          w + w^{-1} =
          w + \conj w =
          2\re(w) =
          2\cdot \ub{\cos(\frac{2\pi}{5})}{\cos{\theta} \geq 0,\, \theta \en [0,2\pi]} = \frac{-1 + \sqrt{5}}{2}\\
          \to
          \boxed{\cos(\frac{2\pi}{5}) = \frac{-1 + \sqrt{5}}{4}} \Tilde
        $
\end{enumerate}

\begin{aportes}
  \item \aporte{\dirRepo}{naD GarRaz \github}
\end{aportes}
