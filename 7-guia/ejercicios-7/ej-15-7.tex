\begin{enunciado}{\ejercicio}
  \begin{enumerate}[label=\roman*)]
    \item Sean $f,g \en \complejos[X]$ y sea $a\en \complejos$. Probar que $a$ es raíz de $f$ y $g$ si y sólo sí $a$ es raíz de
          $(f:g)$.

    \item Hallar todas las raíces complejas de $X^4 + 3X - 2$ sabiendo
          que tiene una raíz en común con ${X^4 + 3X^3 -3X +1}$.
  \end{enumerate}
\end{enunciado}

\begin{enumerate}[label=\roman*)]
  \item Hay que probar la doble implicación:
        \begin{itemize}
          \item[($\red{\Rightarrow}$)]
                $$
                  \text{Si $a$ es raíz de $f$ y $g$}
                  \entonces
                  \llave{l}{
                    (X - a) | f\\
                    (X - a) | g
                  }
                  \entonces
                  (X - a) \text{ es una raíz común }
                  \entonces
                  (X - a) \divideA (f:g)
                $$
          \item[($\red{\Leftarrow}$)]
                El máximo común divisor tiene los monomios de las factorizaciones comunes elevados al menor exponente.
                Así que por definición:
                $$
                  (f:g) = (X - a)^m
                $$
                entonces $X-a$ está en la factorización de $f$ y $g$.
        \end{itemize}
        No sé siento que la demo esa es muy circular. Salió pedorra.

  \item  Usando lo que se demuestra en el ítem anterior, si dos polinomios $f$ y $g$ tienen
        raíces en común, entonces esas raíces tienen que ser raíces del $(f:g)$:

        Si
        $$
          f = X^4 + 3X - 2
          \ytext
          g = X^4 + 3X^3 -3X +1,
        $$
        busco el $(f:g)$:
        $$
          \mcd{X^4 + 3X - 2}{X^4 + 3X^3 -3X +1}
        $$
        Obtuve que:
        $$
          (f:g) = X^2 + X - 1
        $$
        Las raíces:
        $$
          \cajaResultado{
            \llave{l}{
              \alpha_1 = \frac{1 + \sqrt{5}}{2} \\
              \alpha_2 = \frac{1 - \sqrt{5}}{2}
            }
          }
        $$
        \parrafoDestacado[\red{\atencion}]{
          Por lo tanto esas raíces son comunes a $f$ y a $g$.
        }
        Luego puedo escribir
        $$
          X^4 + 3X - 2 = (X^2 + X - 1) \cdot (X^2 - X + 2)
        $$
        Y las raíces de $X^2 - X + 2$:
        $$
          \cajaResultado{
            \llave{l}{
              \alpha_3 = \frac{1}{2} - i\frac{\sqrt{7}}{2} \\
              \alpha_4 = \frac{1}{2} + i\frac{\sqrt{7}}{2}
            }
          }
        $$
\end{enumerate}

\begin{aportes}
  \item \aporte{\dirRepo}{naD GarRaz \github}
\end{aportes}
