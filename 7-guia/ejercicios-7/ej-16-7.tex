\begin{enunciado}{\ejercicio}
  Determinar la multiplicidad de $a$ como raíz de $f$  en los casos
  \begin{enumerate}[label=\roman*)]
    \item $f = X^5 - 2X^3 + X$,\quad $a=1$,
    \item $f = X^6 - 3X^4 + 4$,\quad $a=i$,
    \item $f = (X-2)^2(X^2-4) + (X-2)^3(X-1)$, \quad $a=2$,
    \item $f = (X-2)^2(X^2-4) - 4(X-2)^3$, \quad $a=2$.
  \end{enumerate}
\end{enunciado}

En este ejercicio hay que hacer todo tipo de \textit{casos de factoreo}:

\begin{enumerate}[label=\roman*)]
  \item $f = X^5 - 2X^3 + X$,\quad $a=1$,

        $$
          \begin{array}{rcl}
            f & = & X^5 - 2X^3 + X                \\
              & = & X(X^4-2X^2 + 1)               \\
              & = & X(X^2 - 1)^2                  \\
              & = & X (X - 1)^{\red{2}} (X + 1)^2
          \end{array}
        $$

        $$
          \cajaResultado{\text{ La multiplicidad de $a=1$ como raíz es 2.}}
        $$

  \item $f = X^6 - 3X^4 + 4$,\quad $a=i$.

        Si $a=i$ es raíz, entonces $-i$ también lo es en un polinomio $\reales[X]$:
        $$
          \divPol{X^6 - 3X^4 + 4}{X^2 + 1}
        $$
        $$
          \begin{array}{rcl}
            f & =               & (X^2 + 1)(X^4 - 4X^2 + 4)                             \\
              & \igual{\red{!}} & (X^2 + 1)(X^2-2)^2                                    \\
              & \igual{\red{!}} & (X^2 + 1)(X - \sqrt2)^2 (X + \sqrt2)^2                \\
              & =               & (X - i)^{\red{1}}(X + i)(X - \sqrt2)^2 (X + \sqrt2)^2
          \end{array}
        $$
        $$
          \cajaResultado{
            \text{ La multiplicidad de $a=i$ como raíz de $f$ es 1.}
          }
        $$

  \item
        $$
          f = (X-2)^2 (X^2 - 4) + (X-2)^3 (X-1), \quad a = 2,
        $$

        $$
          f  =                (X-2)^3 \cdot ( (X+2) + (X-1))
          =  2 \cdot (X-2)^3 (X + \frac{1}{2})
        $$
        $$
          \cajaResultado{
            \text{La multiplicidad de $a=2$ como raíz de $f$ es 3.}
          }
        $$

  \item $f = (X-2)^2(X^2-4) - 4(X-2)^3$, \quad $a=2$,

        $$
          \begin{array}{rcl}
            f & = & (X-2)^2(X^2-4) - 4(X-2)^3    \\
              & = & (X-2)^2(X-2)(X+2) - 4(X-2)^3 \\
              & = & (X-2)^3(X+2 - 4)             \\
              & = & (X-2)^4
          \end{array}
        $$

        $$
          \cajaResultado{\text{La multiplicidad de $a=2$ como raíz de $f$ es 4.}}
        $$

\end{enumerate}

\begin{aportes}
  \item \aporte{\dirRepo}{naD GarRaz \github}
  \item \aporte{https://github.com/MPoncini}{M Poncini \github}
\end{aportes}
