\begin{enunciado}{\ejercicio}
  Determinar todos los $a \en \reales$ para los cuales $f = X^{2n+1} -(2n+1) X + a$ tiene al menos una raíz múltiple en $\complejos$.
\end{enunciado}

Si $r$ es raíz múltiple de $f$ \ul{debe} ocurrir que:
$$
  \llave{rcl}{
    f(r)   & = & 0\\
    f'(r)  & = & 0
  }
$$
Derivo $f$ y acomodo:
$$
  f' = (2n+1) \cdot (X^{2n} - 1)
  \Entonces{evalúo}[$r$]
  f'(r) = (2n+1) \cdot (r^{2n} - 1) = 0
  \sii
  (r^{2n} - 1) = 0 \quad \llamada1
$$
Volviendo a $f$, si evalúo en $r$
$$
  \begin{array}{rcl}
    f(r) =  0
    \sii
    r^{2n+1} - (2n+1)r + a = 0
     & \sii          &
    r \cdot (\ob{r^{2n} - 1}{\igual{$\llamada1$} 0} - 2n) + a = 0 \\
     & \Sii{\red{!}} &
    a = 2n \cdot r
  \end{array}
$$
Por lo tanto \ul{$r \en \reales$}  $\llamada2$.

\medskip

Volviendo a $\llamada1$ y con el resultado de que las raíces $r \en \reales$:
$$
  (r^{2n} - 1) = 0
  \Sii{\red{!}}
  (r^n - 1)(r^n + 1) = 0
  \Sii{$\llamada2$}
  r = \pm 1
$$
Por lo tanto los valores de $a \en \reales$ para que el polinomio tenga por lo menos una raíz múltiple:
$$
  \cajaResultado{
    a = \pm 2n
  }
$$

\begin{aportes}
  \item \aporte{\dirRepo}{naD GarRaz \github}
\end{aportes}

