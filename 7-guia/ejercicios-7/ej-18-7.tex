\begin{enunciado}{\ejercicio}
  Determinar todos los $a \en \reales$ para los cuales $f = X^{2n+1} -(2n+1) X + a$ tiene al menos una raíz múltiple en $\complejos$.
\end{enunciado}

Si $\alpha$ es raíz múltiple de $f$ \ul{debe} ocurrir que:
$$
  \llave{rcl}{
    f(\alpha)   & = & 0\\
    f'(\alpha)  & = & 0
  }
$$
Derivo $f$ y acomodo:
$$
  f' = (2n+1) \cdot (X^{2n} - 1)
  \Entonces{evalúo}[$\alpha$]
  f'(\alpha) = (2n+1) \cdot (\alpha^{2n} - 1) = 0
  \sii
  (\alpha^{2n} - 1) = 0 \quad \llamada1
$$
Volviendo a $f$, si evalúo en $\alpha$
$$
  \begin{array}{rcl}
    f(\alpha) =  0
    \sii
    \alpha^{2n+1} - (2n+1)\alpha + a = 0
     & \sii          &
    \alpha \cdot (\ob{\alpha^{2n} - 1}{\igual{$\llamada1$} 0} - 2n) + a = 0 \\
     & \Sii{\red{!}} &
    a = 2n \cdot \alpha
  \end{array}
$$
Por lo tanto \ul{$\alpha \en \reales$}  $\llamada2$.

\medskip

Volviendo a $\llamada1$ y con el resultado de que las raíces $\alpha \en \reales$:
$$
  (\alpha^{2n} - 1) = 0
  \Sii{\red{!}}
  (\alpha^n - 1)(\alpha^n + 1) = 0
  \Sii{$\llamada2$}
  \alpha = \pm 1
$$
Por lo tanto los valores de $a \en \reales$ para que el polinomio tenga por lo menos una raíz múltiple:
$$
  \cajaResultado{
    a = \pm 2n
  }
$$

\begin{aportes}
  \item \aporte{\dirRepo}{naD GarRaz \github}
\end{aportes}

