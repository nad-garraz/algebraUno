\begin{enunciado}{\ejercicio}
	Sea $f = X^{20} + 8X^{10} + 2a$. Determinar todos los valores de
	$a \en \complejos$ para los cuales $f$ admite una raíz múltiple en
	$\complejos$. Para cada valor hallado determinar cuántas raíces
	distintas tiene $f$ y la multiplicidad de cada una de ellas.
\end{enunciado}
Ataco como en el ejercicio anterior. La idea no es calcular todas las raíces.

Si $f$ tiene raíces múltiples $r_k$:
$$
	r_k
	\sii
	f(r_k) = f'(r_k) =  0,
$$
por lo tanto  tanto comienzo buscando las raíces de $f'$ para sacarme ese $a$ de en medio.
$$
	f' = 20X^{19} + 80 X^9 =
	20 X^9 (X^{10} + 4)
$$
Evaluando en $r_k$:
$$
	f'(r_k) =
	20 (r_k)^9 \cdot ((r_k)^{10} + 4)=
	0
	\sii
	\llave{ccl}{
		r_k & = & 0 \\
		(r_k)^{10}  & = & -4 \ \llamada1
	}
$$
Hay de momento 11 raíces de $f'$. Me interesa saber si son raíces de $f$,

\textit{Cuando} $r_k = 0$:
$$
	f(0) = 2a
	\entonces
	f(0) = 0
	\sii
	\cajaResultado{
		a = 0
	}
$$
Acomodo $f$ y después reemplazo por los otros valores que anulan $f'$:
$$
	f =
	X^{20} + 8X^{10} + 2a =
	(X^{10})^2 + 8X^{10} + 2a
$$
\textit{Cuando} $(r_k)^{10} \igual{$\llamada1$} -4$:
$$
	\begin{array}{rcl}
		f(r_k) = 0
		 & \sii              &
		((r_k)^{10})^2 + 8(r_k)^{10}) + 2a = 0 \\
		 & \Sii{$\llamada1$} &
		(-4)^2 + 8(-4) + 2a = 0                          \\
		 & \sii              &
		-16 + 2a = 0                                     \\
		 & \sii              &
		\cajaResultado{
			a = 8
		}
	\end{array}
$$
Ahora tengo que ver cuántas raíces y sus multiplicidades para $a = 0$ y $a = 8$.

Si $a = 0 \entonces f = X^{10}(X^{10} + 8)$
$$
	f(r_k) = 0
	\sii
	(r_k)^{10}((r_k)^{10} + 8)
	\sii
	\llave{rcl}{
		X & = & 0 \\
		& \otext &\\
		X^{10} & = & -8,
	}
$$
donde se ve que con $a = 0$ hay 11 raíces distintas en total, las multiplicidades:
$$
	\cajaResultado{\mult(0;f) = 10}
	\ytext
	\cajaResultado{\mult(\sqrt[10]{8}e^{i\frac{2k+1}{10}\pi});f) = 1 \quad k \en \enteros_{[0-9]}}.
$$

\parrafoDestacado[\red{\atencion}]{
	No había necesidad de calcular las raíces, pero dado que la solución, es
	casi lo mismo que $G_{10}$ ya fue.
	Pero alcanzaría con decir que es un polinomio complejo de grado 10, entonces hay 10
	soluciones y las multiplicidades se ven en los exponentes de los polinomios irreducibles en la factorización.
}

\bigskip

Si $a = 8 \entonces f = (X^{20} + 8X^{10} + 16) \igual{\red{!!}} (X^{10} + 4)^2$
$$
	f(r_k) = 0
	\sii
	\big((r_k)^{10} + 4\big)^2
	\sii
	X^{10} = -4,
$$
donde se ve que con $a = 8$ hay un total de 10 raíces distintas, las multiplicidades:
$$
	\cajaResultado{\mult(\sqrt[5]{2}e^{i\frac{2k+1}{10}\pi});f) = 2 \quad k \en \enteros_{[0-9]}}.
$$

\begin{aportes}
	\item \aporte{\dirRepo}{naD GarRaz \github}
\end{aportes}
