\begin{enunciado}{\ejercicio}
  Calcular el coeficiente de $X^{20}$ de los siguientes polinomios
  \begin{enumerate}[label=\roman*)]
    \item $(X^{18} + X^{16} + 1)(X^5 + X^4 + X^3 + X^2 + X + 1)$ en $\racionales[X]$ y en $(\enteros/2\enteros)[X]$
    \item $(X - 3i)^{133}$ en $\complejos[X]$
    \item $(X - 1)^4(X + 5)^{19} + X^{33} - 5X^{20} + 7$ en $\racionales[X]$
    \item $X^{10}(X^5 + 4)^7$ en $(\enteros/5\enteros)[X]$
  \end{enumerate}
\end{enunciado}

\begin{enumerate}[label=\roman*)]
  \item Expandimos el polinomio y nos fijamos:
        {
        \footnotesize
        $$
          \textstyle
          (X^{18} + X^{16} + 1)(X^5 + X^4 + X^3 + X^2 + X + 1) =
          X^{23} + X^{22} + 2X^{21} + \red{2}X^{20} + 2X^{19} + 3X^{18} + X^{17} + X^{16} + X^5 + X^4 + X^3 + X^2 + X + 1
        $$
        }
        Para $\racionales[X]$ el coeficiente es $\red{2}$ y para $(\enteros/2\enteros)[X]$ es $0$ pues $\congruencia{2}{0}{2}$

  \item Consideramos el binomio de newton:
        $$
          \textstyle
          (a + b)^n = \sumatoria{k=0}{n} \binom{n}{k} a^k \cdot b^{n-k}
        $$
        Aplicamos el binomio a el ejercicio:
        $$
          \textstyle
          (X + (-3i))^{133} = \sumatoria{k=0}{133} \binom{133}{k} X^k \cdot (-3i)^{133 - k}
        $$
        Ahora queremos ver el coeficiente de $X^{20}$, es decir cuando en la sumatoria $k = 20$, se ve que el coeficiente
        que acompaña a la $X$ es
        $\cajaResultado{
            \binom{133}{20} \cdot (-3)^{113} \cdot i
          }$

  \item Consideremos el binomio de newton para los primeros dos términos:
        $$
          \textstyle
          (X - 1)^4(X + 5)^{19} =
          \sumatoria{k = 0}{4} \binom{4}{k} X^k \cdot (-1)^{4 - k}
          \cdot
          \sumatoria{j=0}{19} \binom{19}{j} X^j \cdot 5^{19 - j}
        $$
        Nos interesa cuando el coeficiente de la $X$ sea $20$, es decir las combinaciones de $k$ y $j$ tal que $k + j = 20$
        Esas posibles combinaciones son $(j,k) = (19,1),(18,2),(17,3),(16,4)$.
        $$
          \arraycolsep=1.1pt\def\arraystretch{1.2}
          \begin{array}{ccl}
            \text{Caso } (19,1) & : & \binom{19}{19}X^{19} \cdot 5^{19-19} \cdot \binom{4}{1}X^1 \cdot (-1)^{4-1} = -4X^{20}     \\
            \text{Caso } (18,2) & : & \binom{19}{18}X^{18} \cdot 5^{19-18} \cdot \binom{4}{2}X^2 \cdot (-1)^{4-2} = 570X^{20}    \\
            \text{Caso } (17,3) & : & \binom{19}{17}X^{17} \cdot 5^{19-17} \cdot \binom{4}{3}X^3 \cdot (-1)^{4-3} = -17100X^{20} \\
            \text{Caso } (16,4) & : & \binom{19}{16}X^{16} \cdot 5^{19-16} \cdot \binom{4}{4}X^4 \cdot (-1)^{4-4} = 121125X^{20}
          \end{array}
        $$

        Luego comparando el resto de coeficientes de $X^{20}$ del polinomio tenemos que los coeficientes suman:
        $$
          -4 + 570 - 17100 + 121125 -5 = \cajaResultado{104586}
        $$

  \item Del polinomio nos interesa solo la parte de la derecha cuando el coeficiente de $X$ sea $10$,
        así al multiplicarse por el $X^{10}$ el grado se hace $20$.

        Hacemos la expansion binomial:
        $$
          \textstyle
          (X^5 + 4)^7 = \sumatoria{k=0}{7} \binom{7}{k} (X^5)^k \cdot 4^{7 - k}
        $$
        Queremos $k = 2$, ahí el coeficiente de $X^{10}$ sería $\binom{7}{2} \cdot 4^5 = 21 \cdot 1024 = 21504 \conga{5} \cajaResultado{4}$

        Finalmente ese coeficiente que acompaña a $X^{10}$ se multiplica por el otro $X^{10}$, siendo el coeficiente de $X^{20}$.

\end{enumerate}

\begin{aportes}
  \item \aporte{https://github.com/sigfripro}{sigfripro \github}
\end{aportes}
