\ejercicio

\begin{enumerate}[label=\roman*)]
	\item Probar que para todo $a \en \complejos$,
	      el polinomio
	      $f = X^6 - 2X^5 + (1+a)X^3 + (1+a)X^2 - 2X + 1$
	      es divisible por $(X-1)^2$.

	\item Determinar todos los $a \en \complejos$
	      para los cuales
	      $f$ es divisible por $(X-1)^3$.
\end{enumerate}

\separadorCorto

\begin{enumerate}[label=\roman*)]
	\item
	      $(X-1)^2 \divideA f \paratodo a \en \complejos
		      \sii
	      $
	      1 es \textit{por lo menos} raíz doble de $f$
	      $
		      \sii f(1) = f'(1) = 0. \\
	      \llave{ll}{
		      f = X^6 - 2X^5 + (1+a)X^3 + (1+a)X^2 - 2X + 1 & \flecha{evalúo}[$X=1$] f(1) = 0 \ \paratodo a \en \complejos\\
		      f' = 6X^5 - 10 X^4 + 4(1+a)X^3 - 6aX^2 + 2(1+a)X - 2 & \flecha{evalúo}[$X=1$] f'(1) = 0 \ \paratodo a \en \complejos
	      }\\
          $

	      Calculando $f(1) \ytext f'(1)$ se comprueba. \Tilde

	\item
	      $(X-1)^3 \divideA f
		      \sii
		      f^{''}(1) = 0\\
		      \entonces
		      f^{''} = 30X^4 - 40 X^3 + 12(1+a)X^2 - 12aX + 2(1+a) 
              \flecha{evalúo}[$X=1$]
		      f^{''}(1) = 2a\\
              \entonces
              f^{''}(1) = 0
		      \sii
		      a = 0
	      $\\

	      \boxed{(X-1)^3 \divideA f \sisolosi a = 0} \Tilde\\

	      Observar que si $a \distinto 0$, 1 es una raíz \textit{doble} de $f$ de otra forma
          es una raíz \textit{por lo menos} triple.

\end{enumerate}
