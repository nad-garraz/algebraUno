\begin{enunciado}{\ejercicio}
  \begin{enumerate}[label=\roman*)]
    \item Probar que para todo $a \en \complejos$,
          el polinomio
          $f = X^6 - 2X^5 + (1+a)X^4 -2aX^3 + (1+a)X^2 - 2X + 1$
          es divisible por $(X-1)^2$.

    \item Determinar todos los $a \en \complejos$
          para los cuales
          $f$ es divisible por $(X-1)^3$.
  \end{enumerate}
\end{enunciado}

\begin{enumerate}[label=\roman*)]
  \item
        $(X-1)^2 \divideA f \paratodo a \en \complejos \sii 1 \textit{ es por lo menos raíz doble de } f \sii f(1) = f'(1) = 0$.

        $$
          \begin{array}{rcl}
            f = X^6 - 2X^5 + (1+a)X^4 -2aX^3 + (1+a)X^2 - 2X + 1 & \flecha{evalúo}[$X=1$] f(1) = 0 \ \paratodo a \en \complejos  \\
            f' = 6X^5 - 10 X^4 + 4(1+a)X^3 -6aX^2 + 2(1+a)X - 2  & \flecha{evalúo}[$X=1$] f'(1) = 0 \ \paratodo a \en \complejos
          \end{array}
        $$
        Calculando $f(1) \ytext f'(1)$ se comprueba lo pedido.

  \item
        $$
          (X-1)^3 \divideA f
          \sii
          f^{''}(1) = 0
        $$
        Parecido a antes vuelvo a derivar y evalúo:
        $$
          f^{''} = 30X^4 - 40 X^3 + 12(1+a)X^2 - 12aX + 2(1+a)
          \flecha{evalúo}[$X=1$]
          f^{''}(1) = 4 + 2a
          \entonces
          f^{''}(1) = 0
          \sii
          a = -2
        $$
        Por lo tanto:
        $$
          \cajaResultado{
            (X-1)^3 \divideA f \sisolosi a = -2
          }
        $$
        Observar que si $a \distinto -2$, 1 es una raíz \textit{doble} de $f$ de otra forma es una raíz \textit{por lo menos} triple.
\end{enumerate}

\begin{aportes}
  \item \aporte{\dirRepo}{naD GarRaz \github}
  \item \aporte{https://github.com/olivportero}{Olivia Portero \github}
\end{aportes}
