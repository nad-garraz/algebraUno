\begin{enunciado}{\ejercicio}
  Sea $(f_n)_{n\en \naturales}$ la sucesión de polinomios en $\complejos[X]$ definida por
  $$
    f_1 = X^4 + 2X^2 + 1 \ytext f_{n+1} = (X - i)(f_n + f'n), \quad \paratodo n \en \naturales.
  $$
  Probar que $i$ es raíz \textit{doble} de $f_n$ para todo $n \en \naturales$.
\end{enunciado}

Sale por inducción. Quiero probar la siguiente proposición:
$$
  p(n) : i \text{ es una raíz \textit{doble} de } f_n \paratodo n \en \naturales.
$$

\bigskip

\textit{Caso Base:}
$$
  p(\blue{1}) : i \text{ es una raíz \textit{doble} de } f_{\blue{1}}.
$$
Especializo a $f_{\blue{1}}$, $f'_{\blue{1}}$ y a \red{¡} $f''_{\blue{1}}$ \red{!} en $i$:
$$
  f_{\blue{1}} (i) = i^4 + 2i^2 + 1 = 0
  ,\quad
  f'_{\blue{1}} (i) = 4i^3 + 4i = 0
  \ytext
  f''_{\blue{1}} (i) = 12i^2 \distinto 0
$$
Por lo cual $p(\blue{1})$ resulta verdadera.

\medskip

\textit{Paso inductivo:}
Asumo como verdadero para algún $\blue{k} \en \naturales$ que la proposición:
$$
  p(\blue{k}) : \ub{ i \text{ es una raíz \textit{doble} de } f_{\blue{k}},
  }{
    \purple{\text{hipótesis inductiva}}
  }
$$
es verdadera. Entonces quiero probar que la proposición:
$$
  p(\blue{k+1}) : i \text{ es una raíz \textit{doble} de } f_{\blue{k+1}}
$$
también lo sea.

La \purple{hipótesis inductiva} dice que:
$$
  f_{\blue{k}}(i) = 0,
  \quad
  f'_{\blue{k}}(i) = 0,
  \ytext
  f''_{\blue{k}}(i) \taa{\red{!}}{}\distinto 0,
$$
Usando la definición de la función, especializo, evalúo o como quieras decirle a
$f_{\blue{k+1}}$, $f'_{\blue{k + 1}}$ y a \red{¡} $f''_{\blue{k + 1}}$ \red{!} en $i$:
$$
  \llave{rcl}{
    f_{\blue{k + 1}}(i)
    & = &
    (i - i) \big(f_{\blue{k}}(i) + f'_{\blue{k}}(i)\big) = 0,\\
  f'_{\blue{k + 1}}(i)
  & = &
  f_{\blue{k}}(i) + f'_{\blue{k}}(i) + (i - i) \big(f'_{\blue{k}}(i) + f''_{\blue{k}}(i)\big)
  \igual{\purple{HI}} 0\\
  f^{''}_{\blue{k + 1}}(i)
  & = &
  2 \cdot \big(f'_{\blue{k}}(i) + f^{''}_{\blue{k}}(i)\big) + (i - i) \big(f^{''}_{\blue{k}}(i) + f^{'''}_{\blue{k}}(i)\big)
  \taa{\text{\purple{HI}}}{}\distinto 0
  }
$$
Y así resultó la proposición $p(\blue{k+1})$ verdadera.

\medskip

Dado que $p(1),\, p(k) \ytext p(k+1)$ resultaron todas verdaderas por principio de inducción también lo es $p(n) \paratodo n \en \naturales$.

\begin{aportes}
  \item \aporte{\dirRepo}{naD GarRaz \github}
\end{aportes}
