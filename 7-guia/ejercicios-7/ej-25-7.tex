\begin{enunciado}{\ejercicio}
  Sea $(f_n)_{n\en \naturales}$ la sucesión de polinomios en $Q[X]$ definida por
  $$
    f_1 = X^3 + 2X
    \ytext
    f_{n+1} = Xf_n^2 + X^2f'_n,\, \paratodo n \en \naturales.
  $$
  Probar que para todo $n \en \naturales$ vale que:
  \begin{enumerate}[label=\roman*)]
    \item $\gr(f_n) = 2^{n+1} - 1$.
    \item 0 es raíz de multiplicidad $n$ de $f_n$.
  \end{enumerate}
\end{enunciado}

\begin{enumerate}[label=\roman*)]
  \item Inducción:

        Quiero probar la proposición:
        $$
          p(n) : gr(f_n) = 2^{n + 1} - 1 \ \paratodo n \en \naturales
        $$

        \textit{Caso base:}
        $$
          p(\blue{1}) : gr(f_{\blue{1}}) = 2^{\blue{1} + 1} - 1 = 3 = \gr(X^3 + 2X)
        $$
        Por lo tanto $p(\blue{1})$ es verdadera.

        \textit{Paso inductivo:}

        Asumo para un $\blue{k} \en \naturales$ que:
        $$
          p(\blue{k}) : \ub{
            \gr(f_{\blue{k}}) = 2^{\blue{k} + 1} - 1
          }{
            \text{\purple{hipótesis inductiva}}
          }
        $$
        es una proposición verdadera. Entonces quiero probar que:
        $$
          p(\blue{k + 1}) : \gr(f_{\blue{k+1}}) = 2^{\blue{k+1} + 1} - 1  = 2^{\blue{k} + 2} -1
        $$
        también lo sea.

        $$
          f_{\blue{k+1}}
          \igual{def}
          Xf_{\blue{k}}^2 + X^2f'_{\blue{k}}
        $$
        ¿Qué grado tiene esa cosa?
        $$
          \gr(Xf_{\blue{k}}^2 + X^2 \cdot f'_{\blue{k}})
          \igual{\red{!}}
          1 + \gr(\ub{f_{\blue{k}}^2}{2\gr(f_{\blue{k}})} + \ub{X \cdot f'_{\blue{k}}}{\gr(f_{\blue{k}})})
          =
          1 + 2 \cdot \gr(f_{\blue{k}})
          \igual{\purple{HI}}
          1 + 2 \cdot (\purple{2^{k+1} - 1})
          =
          2^{\blue{k+2}} - 1
        $$
        Si no viste el \red{!}, saqué factor común $X$, así es que aparece el 1 que se suma al grado. También tuve en
        cuenta que $\gr(f') = \gr(f) - 1 \ \paratodo f \en f[X]$

        Por lo tanto $p(k+1)$ resultó verdadera.

        \medskip

        Como $p(1), p(k) \ytext p(k+1)$ resultaron todas verdaderas por principio de inducción en también lo es $p(n) \paratodo n \en \naturales$.

  \item Lindo ejercicio, \cyan{intentá hacerlo antes de mirar la resolución}.

        Sale también por inducción.

        Quiero probar la proposición:
        $$
          p(n) : \text{0 es raíz de multiplicidad $n$ de $f_n \ \paratodo n \en \naturales$}
        $$
        O de forma equivalente:
        $$
          p(n) :  X^n \divideA f_n \ \paratodo n \en \naturales
        $$

        \textit{Caso base:}
        $$
          p(\blue{1}) : X^{\blue{1}} \divideA f_{\blue{1}}
        $$
        Se cumple enseguida que:
        $$
          X \divideA X^3 + 2X
          \sii
          X \divideA X \cdot (X^2 + 2)
        $$
        Por lo tanto $p(\blue{1})$ es verdadera.

        \textit{Paso inductivo:}

        Asumo para un $\blue{k} \en \naturales$ que:
        $$
          p(\blue{k}) : \ub{
            X^{\blue{k}} \divideA f_{\blue{k}}
          }{
            \text{\purple{hipótesis inductiva}}
          }
        $$
        es una proposición verdadera. Entonces quiero probar que:
        $$
          p(\blue{k + 1}) : X^{\blue{k+1}} \divideA f_{\blue{k+1}}
        $$
        también lo sea.
        $$
          X^{\blue{k+1}} \divideA f_{\blue{k+1}}
          \sii
          X^{\blue{k+1}} \divideA Xf_{\blue{k}}^2 + X^2f'_{\blue{k}}
        $$
        Exploto la \purple{hipótesis inductiva} para construir el paso $\blue{k+1}$:
        $$
          \begin{array}{c}
            \Entonces{\purple{HI}}
            X^k \divideA f_k
            \Sii{\red{!}}
            X^k \divideA f_k \cdot f_k
            \Sii{\red{!}}
            X \cdot X^k \divideA X\cdot f_k^2
            \sii
            X^{k+1} \divideA X \cdot f_k^2  \quad \llamada1 \\
            \text{y parecido con la derivada}               \\
            \Entonces{\red{!!}}	      X^{k-1} \divideA f'_k
            \Sii{\red{!}}
            X^2 \cdot X^{k-1} \divideA X^2 \cdot f'_k
            \sii
            X^{k+1} \divideA X^2 \cdot f'_k \quad \llamada2
          \end{array}
        $$
        Donde en \red{!!} usé la \purple{hipótesis inductiva}, si $f_k$ tiene como raíz al 0 con multiplidad $k$ entonces la derivada tiene que
        tener multiplicidad $k-1$ en 0.
        Con esos hermosos, tiernos y sabrosos resultados de $\llamada1 \ytext \llamada2$:
        $$
          \llave{l}{
            X^{k+1} \divideA X \cdot f_k^2  \\
            X^{k+1} \divideA X^2 \cdot f'_k
          }
          \sii
          X^{\blue{k+1}} \divideA X \cdot f_{\blue{k}}^2 +  X^2 \cdot f'_{\blue{k}} \\
        $$

        \medskip

        Como $p(1), p(k) \ytext p(k+1)$ resultaron todas verdaderas por principio de inducción también lo es $p(n) \paratodo n \en \naturales$.

\end{enumerate}

\begin{aportes}
  \item \aporte{\dirRepo}{naD GarRaz \github}
\end{aportes}
