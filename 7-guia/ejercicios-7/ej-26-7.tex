\begin{enunciado}{\ejercicio}
  Sea $\alpha \en \complejos$ raíz de multiplicidad 3 de $f \en \complejos[X]$. Probar que el resto de dividir a $f'$ por
  $(X - \alpha)^3$ es $a(X - \alpha)^2$, con $a \en \complejos$, $a \distinto 0$.
\end{enunciado}

Sé que cuando derivo el algoritmo de división es:
$$
  f = D \cdot Q + R
$$
En este caso $D = (X - \alpha)^3$
$$
  (X - \alpha)^3 \divideA f
  \sii
  f = \magenta{q} \cdot  (X - \alpha)^3 + 0
  \quad \text{con }
  q \en \complejos[X]
$$
Derivo esa expresión de $f$:
$$
  f' = q' \cdot (X - \alpha)^3 + 3q \cdot (X-\alpha)^2
$$
El dato dice que $\alpha$ es una raíz triple de $f$, por lo tanto si derivo $f$, $(X - \alpha)^3$:
$$
  f' = \ub{q'}{Q} \cdot \ub{(X - \alpha)^3}{D}  + \ub{3q \cdot (X - \alpha)^2}{R}
$$
No me acuerdo si es \textit{teorema del resto} o algo así que dice que especializar a $f$ en algún valor te da lo que vale el resto,
lo cual es \textit{razonable cuando hacés Ruffini}, pero ahora que se está dividiendo por una potencia de 3, es más raro.
pero esta es la primera vez que aparece en
En fin, especializo en $\alpha$, recordando que es raíz triple de $f$, por lo tanto me va a dar cero:
$$
  f'(\alpha) = R(\alpha) =
  \ub{3\magenta{q}(\alpha)}{\distinto 0} \cdot (\alpha - \alpha)^2 = 0
$$
Por lo tanto $R$ cumple condición de resto y además es de la forma $\ua{a}{\distinto 0 \en \complejos} \cdot (X - \alpha)^2$

\begin{aportes}
  \item \aporte{\dirRepo}{naD GarRaz \github}
\end{aportes}
