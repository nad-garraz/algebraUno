\begin{enunciado}{\ejercicio}
  Hallar, cuando existan, todos los $f \en \complejos[X]$ tales que:
  \begin{multicols}{2}
    \begin{enumerate}[label=\roman*)]
      \item $f^2 = Xf + X + 1$,
      \item $f^2 - Xf = - X^2 + 1$,
      \item $(X+1)f^2 = X^6 + Xf$,
      \item $f \distinto 0 \ytext f^3 = \gr(f) \cdot X^2f$.
    \end{enumerate}
  \end{multicols}
\end{enunciado}

\begin{enumerate}[label=\roman*)]
  \item
        La ecuación se tiene que cumplir para todo valor de $X$, así que no es cuestión de buscar algún valor para el que la igualdad se cumpla.
        Acomodo la ecuación:
        $$
          f^2 = Xf + X + 1
          \Sii{\red{!}}
          (f-1) \cdot (f + 1) = X \cdot (f + 1)
          \Sii{\red{!}}
          (f+1) \cdot (f - (X + 1)) = \ua{0}{g}
        $$
        Eso último es una igualación de polinomios, donde el polinomio del miembro derecho es $g=0$. Entonces, para que
        se cumpla esa igualdad para todo valor de $X$, el miembro izquierdo \ul{también tiene que ser 0 para todo valor de $X$}.
        Eso ocurre cuando:
        $$
          f = -1 \otext f = X + 1
        $$

  \item Mirando la ecuación se puede calcular el grado que debería tener $f$:
        \begin{enumerate}[label=\tiny\faIcon{calculator}$_{\arabic*}$)]
          \item ¿Puede ser $\gr(f) = 0$?
                $$
                  f = k \entonces k^2 - X \cdot k = -X^2 + 1,
                $$
                No cierra el tema del grado. Para que un polinomio sea igual a otro, estos deben tener igual grado.
          \item ¿Puede ser $\gr(f) = 1$?
                $$
                  f = bX + c \entonces b^2X + 2bcX + c^2 - bX^2 + Xc = -X^2 + 1,
                $$
                en el miembro izquierdo se cancelan los términos cuadráticos, por lo que nuevamente no voy a poder tener un polinomios iguales
                en ambos miembros de la ecuación.

          \item ¿Puede ser $\gr(f) = 2$?
                $$
                  f = aX^2 + bX + c \Entonces{\red{!}} (aX^2 + b^2X + c)^2  - X(aX^2 + bX + c) = -X^2 + 1,
                $$
                Acá nos queda el miembro izquierdo con $\gr(4)$ y el izquierdo con $\gr(2)$, así que no hay $f$, \textit{bla, bla, bla}.

          \item ¿Puede ser $\gr(f) \geq 3?$. Diría que no por razones muy interesantes.

        \end{enumerate}

  \item En este caso se puede ver que el miembro izquierdo va a tener siempre grado impar. Para que el miembro derecho tenga grado impar,
        necesito que $f$ sea algo que cancele el $X^6$
        $$
          \gr((X+1)\cdot f^2) = \gr(X\cdot (X^5 + f))
          \sii
          \gr(X+1) + \gr(f^2) = \gr(X) \cdot \gr(X^5 + f)
          \Sii{\red{!}}
          \ub{
            2 \cdot \gr(f)
          }{
            \text{par}
          }
          \igual{$\llamada1$}
          \gr(X^5 + f)
        $$
        Analizamos la última ecuación para distintos grados:
        $$
          \begin{array}{rccl}
            \text{si} & \gr(f) < 5 & \entonces & \gr(X^5 + f) = 5 \quad  \llamada2    \\
            \text{si} & \gr(f) = 5 & \entonces & \gr(X^5 + f) \leq 5 \quad \llamada3  \\
            \text{si} & \gr(f) > 5 & \entonces & \gr(X^5 + f) = \gr(f) \quad\llamada4
          \end{array}
        $$
        Entonces no tenemos un valor para el grado de $f$ en el que haya un balance en la ecuación $\llamada1$, porque:
        \begin{enumerate}
          \item[$\llamada2$] El miembro derecho de $\llamada1$ tendría un valor par
                así que descartado.

          \item[$\llamada3$] El miembro derecho de $\llamada1$ tendría un grado igual a 10 así que descartado.

          \item[$\llamada4$] El miembro derecho de $\llamada1$ tendría un grado del doble que el polinomio del miembro izquierdo.
        \end{enumerate}

  \item Si $f \distinto 0$:
        $$
          f^3 = \gr(f) \cdot X^2f
          \Sii{\red{!}}
          f \cdot (f^2 - \gr(f)X^2) = 0
        $$
        Como por enunciado $f \distinto 0$, para que el miembro izquierdo sea 0, necesitamos que:
        $$
          (f^2 - \gr(f)X^2) = 0 \sii \gr(f^2) = \gr\big( \gr(f) \cdot X^2 \big) = 2 \sii 2\gr(f) = 2 \sii \gr(f) = 1
        $$
        Entonces $\gr(f) = 1 \entonces f = aX$, evalúo en la ecuación del enunciado para averiguar el valor de $a$:
        $$
          a^3 \cdot X^3 = 1 \cdot X^2\cdot aX = aX^3
          \sii
          a\cdot(a^2 - 1)X^3 = 0
          \Sii{si $f\distinto 0 $}[$\entonces a\distinto 0$]
          \llave{l}{
            a = 1  \\
            \ytext \\
            a = -1
          }
        $$
        Por lo tanto los polinomios $f$ que cumplen son:
        $$
          f = -X
          \ytext
          f = X
        $$
\end{enumerate}

% Contribuciones
\begin{aportes}
  %% iconos : \github, \instagram, \tiktok, \linkedin
  %\aporte{url}{nombre icono}
  \item \aporte{\dirRepo}{naD GarRaz \github}
  \item \aporte{https://github.com/RamaEche}{Ramiro E. \github}
\end{aportes}
