\begin{enunciado}{\ejercicio}
  Sea $n \en \naturales$.
  Probar que $\sumatoria{k=0}{n}X^k \en \complejos[X]$ tiene todas sus raíces complejas simples.
\end{enunciado}

Sale por inducción:
$$
  p(n):  \sumatoria{k=0}{n}X^k \en \complejos[X] \text{ tiene todas sus raíces complejas simples } \paratodo n \en \naturales.
$$

\textit{Caso base:}
$$
  p(\blue{1}):  \sumatoria{k=0}{\blue{1}}X^k = 1 + X^{\blue{1}} \text{ tiene todas sus raíces complejas simples. }
$$
En este caso $f(X) = 1 + X$:
$$
  f(-1) = 0
  \ytext
  f'(-1) = 1 \distinto 0
$$
Por lo tanto $p(\blue{1})$ es verdadera.

\textit{Paso inductivo:}
Asumo que para algún $\blue{k} \en \naturales$ la proposición:
$$
  p(\blue{h}):
  \ub{
    \sumatoria{k=0}{\blue{h}}X^k = 1 + X + X^2 + \cdots + X^{\blue{h}} \text{ tiene todas sus raíces complejas simples. }
  } {
    \text{\purple{hipótesis inductiva}}
  }
$$
es verdadera. Entonces quiero ver que:
$$
  p(\blue{h+1}):  \sumatoria{k=0}{\blue{h+1}}X^k = 1 + X + X^2 + \cdots + X^h + X^{\blue{h+1}} \text{ tiene todas sus raíces complejas simples. }
$$
también lo sea.

Primero veo que la \purple{hipótesis inductiva} es algo así:
$$
  f_{\blue{h}}(\alpha_{\blue{h}})
  \igual{$\oa{\ }{\alpha_{\blue{h}} \distinto 0}$}[\red{!!}] 0
  \ytext
  f'_{\blue{h}}(X) =
  \sumatoria{k=\red{1}}{\blue{h}} kX^{k-1} = 1 + 2X + 3X^2 + \cdots + \blue{h}X^{\blue{h}-1}
  \text{ donde }
  f'_{\blue{h}}(\alpha_{\blue{h}}) \distinto 0
  \paratodo \alpha_{\blue{h}} \text{ raíz de } f.
$$

Ahora laburo el polinomio $(\blue{h+1})-$ésimo, con $\alpha$  raíz del mismo:

$$
  \llave{l}{
    f_{\blue{h+1}}(\alpha_{\blue{h+1}})
    \igual{$\oa{\ }{\alpha_{\blue{h+1}} \distinto 0}$}[\red{!!}] 0\\
    f'_{\blue{h+1}}(X) =
  \sumatoria{k=\magenta{1}}{\blue{h+1}}kX^{k-1} =
  \ub{
  1 + 2X + \cdots + \blue{h}X^{\blue{h-1}}
  }
  {
  f'_{\blue{h}}(X)
  }
  + (\blue{h+1})X^{\blue{h}}
  =
  f'_{\blue{h}}(X) + (\blue{h+1})X^{\blue{h}}
  \\
  \flecha{evalúo}[en $\alpha_{_{\blue{h+1}}}$]
  f'_{\blue{h+1}}(\alpha_{_{\blue{h+1}}})
  \igual{\purple{HI}}
  \ub{
  f'_{\blue{h}}(\alpha_{_{\blue{h+1}}})
  }{
  \distinto 0
  }
  + (\blue{h+1})
  \ub{
    \alpha_{_{\blue{h+1}}}^{\blue{h}}
  }{
    \distinto 0
  }
  \distinto 0
  \\
  }
$$
Por lo tanto $p(k+1)$ resultó verdadera.

Como $p(1), p{h}\ytext p(h+1)$ resultaron verdaderas por principio de inducción en $n$ también lo es $p(n) \paratodo n \en \naturales$.

\begin{aportes}
  \item \aporte{\dirRepo}{naD GarRaz \github}
\end{aportes}
