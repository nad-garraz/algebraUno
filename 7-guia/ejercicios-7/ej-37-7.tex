\begin{enunciado}{\ejercicio}
  Sean $a, b, c \en \complejos$ las raíces de $2X^3 - 3X^2 + 4X + 1$. Determinar
  \begin{enumerate}[label=\roman*)]
    \begin{multicols}{3}
      \item $a + b + c$,
      \item $ab + ac + bc$,
      \item $abc$ .
    \end{multicols}
  \end{enumerate}
\end{enunciado}

Si $a, b, c$ son las 3 raíces del polinomio $P = 2X^3 - 3X^2 + 4X + 1$. El polinomio $P$ en \textit{forma factorizada}:
$$
  2(X-a)(X-b)(X-c) \quad \text{con } a,\, b,\, c \en \complejos.
$$
Distribuyendo todo se recupera la \textit{forma polinómica} de $P$:
$$
  \begin{array}{rcl}
    \red{2}(X-a)(X-b)(X-c) & = & \red{2}(X^2 - aX - bX + ab)(X - c)                                       \\
                           & = & \red{2}(X^3 - aX^2 - bX^2 + abX -cX^2 +acX +bcX - abc)                   \\
                           & = & \red{2}X^3 - \red{2}(a + b + c)X^2 + \red{2}(ab + ac + bc)X - \red{2}abc
  \end{array}
$$
Si comparamos esta expansión genérica con el polinomio, vemos que justamente lo que nos piden en el enunciado
son los coeficientes de $P$, \ul{expresados en función de sus raíces}.
{\tiny
\parrafoDestacado{
  Esta forma de relacionar raices con coeficientes se puede generalizar y se conoce como \href{https://en.wikipedia.org/wiki/Vieta\%27s_formulas}{Fórmulas de Viete}.
}
}

Para resolver el ejercicio se plantea la igualdad de los polinomios:
{\small
$$
  \red{2}X^3 - \red{2}(a + b + c)X^2 + \red{2}(ab + ac + bc)X - \red{2}abc
  =
  2X^3 - 3X^2 + 4X + 1
$$
Dos polinomios son iguales si y solo si todos sus coeficientes son iguales:
$$
  \llave{rcl}{
    -\red{2}(a+b+c) = -3 & \sii & \cajaResultado{\textstyle a + b + c = \frac{3}{2}} \\
    \red{2}(ab + ac + bc) = 4 & \sii & \cajaResultado{\textstyle ab + ac + bc = 2} \\
    -\red{2}abc = 1 & \sii & \cajaResultado{\textstyle abc = -\frac{1}{2}}
  }
$$
}

\begin{aportes}
  \item \aporte{https://github.com/sigfripro}{sigfripro \github}
\end{aportes}

