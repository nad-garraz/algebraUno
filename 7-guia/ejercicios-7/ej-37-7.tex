\begin{enunciado}{\ejercicio}
  Sean $a, b, c \en \complejos$ las raíces de $2X^3 - 3X^2 + 4X + 1$. Determinar
  \begin{enumerate}[label=\roman*)]
    \begin{multicols}{3}
      \item $a + b + c$,
      \item $ab + ac + bc$,
      \item $abc$ .
    \end{multicols}
  \end{enumerate}
\end{enunciado}

$a, b, c$ son las 3 raices del polinomio, luego podemos plantear al polinomio como 
$2(X-a)(X-b)(X-c)$. 

\begin{align*}
\red{2}(X-a)(X-b)(X-c) = \red{2}(X^2 - aX - bX + ab)(X - c) = \\
= \red{2}(X^3 - aX^2 - bX^2 + abX -cX^2 +acX +bcX - abc) = \\
= \red{2}(X^3 -(a + b + c)X^2 + (ab + ac + bc)X - abc)
\end{align*}
Si comparamos esta expansion generica con el polinomio, vemos que justamente lo que nos piden son los coeficientes
del polinomio original, esta forma de relacionar raices con coeficientes se puede generalizar y se conoce como \href{https://en.wikipedia.org/wiki/Vieta\%27s_formulas}{Formulas de Viete}. 
Vemos tambien que toda la expresion se multiplica por $2$ porq es el coeficiente principal, asi que a cada coeficiente tenemos que dividirlo por $2$
para obtener lo pedido, Entonces:
\begin{enumerate}[label=\roman*)]
  \item $-2(a+b+c) = -3 \iff (a+b+c) = \frac{3}{2}$
  \item $2(ab + ac + bc) = 4 \iff (ab + ac + bc) = 2$
  \item $-2(abc) = 1 \iff (abc) = -\frac{1}{2}$

\end{enumerate}

\begin{aportes}
 \item \aporte{https://github.com/sigfripro}{sigfripro \github}
\end{aportes}

