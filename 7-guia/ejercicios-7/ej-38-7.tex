\begin{enunciado}{\ejercicio}
  \begin{enumerate}[label=\roman*)]
    \item Hallar todas las raíces en $\complejos$ del polinomio
          $f = X^4 - (i + 4)X^3 + (8 + 4i)X^2 - (2i + 24)X + 12$
          sabiendo que tiene al menos una raíz real.

    \item Hallar todas las raíces en $\complejos$ del polinomio
          $f = X^6 - 3X^4 -  (2 + 8i)X^3 + 24iX + 16i$,
          sabiendo que tiene al menos una raíz entera.
  \end{enumerate}
\end{enunciado}

\begin{enumerate}[label=(\roman*)]
  \item Separo en parte real e imaginaria para que sea más manejable:
        {\small
        $$
          f = X^4 - (i + 4)X^3 + (8 + 4i)X^2 - (2i + 24)X + 12 =
          \ub{
            X^4 - 4X^3 + 8X^2 -24X + 12
          }{
            \re(f)
          }
          + i \ub{
            (-X^3 + 4X^2 -2X)
          }{
            \im(f)
          }
        $$
        }
        Ataco primero $\im(f)$ que está más fácil:
        $$
          \im(f(r)) = -r^3 + 4r^2 -2r = 0
          \sii
          -r\cdot(r^2 - 4r + 2) = 0
          \Sii{\red{!}}
          r \en \set{0,\, 2 + \sqrt{2},\, 2 - \sqrt{2}}
        $$
        El 0 claramente no es raíz de la parte real $\re(f)$, por lo tanto las otras tienen que serlo, debido a que el polinomio no tiene
        coeficientes irracionales y por enunciado $f$ tiene \ul{al menos una raíz real}:
        $$
          \divPol{X^4 - 4X^3 + 8X^2 -24X + 12}{X^2 - 4X + 2}
        $$
        Por lo tanto:
        $$
          \re(f) =  (X^2 - 4X +2) \cdot (X^2 + 6) =
          \big(X - (2 + \sqrt{2})\big)
          \big(X - (2 - \sqrt{2})\big)
          \big(X - i\sqrt{6}\big)
          \big(X + i\sqrt{6}\big)
        $$
        Sé que las raíces complejas \ul{no} son raíces de la parte imaginaria así que no me sirven. Las únicas raíces que tienen en común
        $\re(f)$ y $\im(f)$, me forman el $(\re(f): \im(g)) = X^2 - 4X + 2$, datazo que nadie pidió!

        Ahora le bajo el grado al polinomio original $f$ y queda ahí medio cocinado:
        $$
          \begin{array}{rcl}
            f & = & X^4 - (i + 4)X^3 + (8 + 4i)X^2 - (2i + 24)X + 12 \\
              & = & (X^2 - iX + 6) \cdot (X^2 - 4X + 2)              \\
              & = &
            \cajaResultado{
              \big(X - 3i\big)
              \big(X + 2i\big)
              \big(X - (2 + \sqrt{2})\big)
              \big(X - (2 - \sqrt{2})\big)
            }
          \end{array}
        $$

  \item Si $r$ es una raíz y $r \en \enteros$:
        $$
          \begin{array}{rcl}
            f(r) = 0
             & \sisolosi      &
            r^6 - 3r^4 -  (2 + 8i)r^3 + 24ir + 16i = 0 \\
             & \Sii{\red{!!}} &
            \llave{rcl}{
            r^3 \cdot (r^3 -3r - 2) = 0                \\
              -8i \cdot (r^3 -3r - 2) = 0
            }                                          \\
             & \Sii{\red{!!}} &
            \llave{rcl}{
            r^3 \cdot (r+1) \cdot (r^2 -r - 2) = 0     \\
              -8i \cdot (r+1) \cdot (r^2 -r - 2) = 0
            }                                          \\
             & \Sii{\red{!!}} &
            \llave{rcl}{
            r^3 \cdot (r+1)^2 \cdot (r - 2) = 0        \\
              -8i \cdot (r+1)^2 \cdot (r - 2) = 0
            }
          \end{array}
        $$
        Ya encontré 3 raíces. Bajo el grado de $f$ y sigo buscando:
        $$
          \polyset{vars=X}
          \divPol{X^6 - 3X^4 +  (-2 - 8i)X^3 + 24iX + 16i}{X^3 -3X - 2}
        $$
        Solo resta buscar las raíces en el cociente calculado. Si $X = r e^{i \theta}$:
        $$
          \begin{array}{rcl}
            X^3 - 8i = 0
                   & \sii &
            r^3 e^{i3 \theta} = 8e^{i\frac{\pi}{2}}                                                                                     \\
                   & \sii &
            \llave{rcl}{
            r      & =    & 2                                                                                                           \\
            \theta & =    & \frac{\pi}{6} + \frac{\blue{2k \pi}}{3} \sii \theta \en \set{\frac{\pi}{6}; \frac{5\pi}{6}; \frac{3\pi}{2}}
            }
          \end{array}
        $$
        Las raíces del polinomio $f$:
        $$
          \textstyle
          \set{-1;\, 2;\, \sqrt{3} + i;\, -\sqrt{3} + i;, -2i }
        $$
\end{enumerate}

\begin{aportes}
  \item \aporte{\dirRepo}{naD GarRaz \github}
\end{aportes}
