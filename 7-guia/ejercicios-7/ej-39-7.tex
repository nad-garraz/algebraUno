\begin{enunciado}{\ejercicio}
  Sea $p$ un número primo. ¿Cuántos polinomios mónicos de grado 2 hay en $(\enteros/p\enteros)[X]$? ¿Cuántos de ellos
  son reducibles y cuántos irreducibles?
\end{enunciado}

La estructura de un polinomio mónico de grado $2$ es la siguiente:
$$
  X^2 + aX + b \quad a, b \en (\enteros/p\enteros)
$$

Tenemos que elegir un valor para $a$ y otro para $b$, entre $0$ y $p-1$, o sea
$p$ distintos, dos veces.
Así que la cantidad de polinomios mónicos de grado dos \ul{total}:
$$
  \cajaResultado{\green{p^2}}.
$$

Para buscar aquellos que sean \ul{reducibles}, es decir que puedan factorizarse en mónicos de $(\enteros/p\enteros)[X]$:
$$
  X^2 + aX + b
  =
  (X - p_1) \cdot (X - p_2)
$$
\textit{Caso donde $p_1 \distinto p_2$:}

Hay $\blue{p \cdot (p - 1)}$ opciones, pero como el orden de los factores no va a alterar el polinomio,
por ejemplo:
$$
  (X - 2)(X - 3) = (X - 3)(X - 2).
$$
\textit{¡Hay que dividir por $\blue{2}$ para no contar dos veces lo mismo!}.

\textit{Caso donde $p_1 = p_2$:}

Los elementos de la forma $(X - p_0)^2$ están contados solo una vez,
de estos hay exactamente $\magenta{p}$ elementos.

La cantidad de polinomios en $(\enteros/p\enteros)$ \ul{reducibles}:
$$
  \cajaResultado{\blue{\frac{p^2 - p}{2}} + \magenta{p} = \frac{p^2 + p}{2}}
$$

La cantidad de polinomios en $(\enteros/p\enteros)$ \ul{irreducibles}:
$$
  \cajaResultado{
    \green{p^2} - \frac{p^2 + p}{2} = \frac{p^2 - p}{2}
  }
$$

\begin{aportes}
  \item \aporte{https://github.com/sigfripro}{sigfripro \github}
  \item \aporte{\dirRepo}{naD GarRaz \github}
\end{aportes}
