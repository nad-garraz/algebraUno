\begin{enunciado}{\ejercicio\labelEjercicio{ej:cong}}
  % Etiqueta para referenciar el ejercicio

  Hallar el cociente y el resto de la división de $f$ por $g$ en los casos

  \begin{enumerate}[label=\roman*)]
    \item $f = 5X^4 + 2X^3 - X + 4$ y $g = X^2 + 2$ en $\racionales[X],\,\reales[X],\,\complejos[X]$,
    \item $f = 4X^4 + X^3 - 4$ y $g = 2X^2 + 1$ en $\racionales[X],\,\reales[X],\,\complejos[X]$ y $(\enteros/7\enteros)[X]$,
    \item $f = X^n - 1$ y $g = X - 1$ en $\racionales[X],\,\reales[X],\,\complejos[X]$ y $(\enteros/p\enteros)[X]$
  \end{enumerate}
\end{enunciado}

\begin{enumerate}[label=\roman*)]
  \item $$
          \divPol{5X^4 + 2X^3 - X + 4}{X^2 + 2}
        $$

        Resultado válido para $\racionales[X],\,\reales[X],\,\complejos[X]$

  \item $$
          \divPol{4X^4 + X^3 - 4}{2X^2 + 1}
        $$

        Resultado válido para $\racionales[X],\,\reales[X],\,\complejos[X]$

        En $\enteros/p\enteros\
          \to
          4X^4 + X^3 -4 = (2X^2 + 1) \cdot \ub{(2X^2 + 4X + 6)}{q[X]} + \ub{(3X + 4)}{r[X]}$

  \item\label{ej4:item-iii} Después de hacer un par iteraciones en la división asoma la idea de que:
        $$
          X^n - 1 = (X-1) \cdot \ub{\sumatoria{j=0}{n-1} X^j}{q[X]} + \ub{0}{r[X]},
          \qquad\qquad{\text{\tiny(que es la geométrica con $X \distinto 1$)}}
        $$

        \textit{Inducción: } Quiero probar que:
        $$
          p(n): X^n - 1 = (X-1) \cdot \sumatoria{j=0}{n-1} X^j  \ \paratodo n \en \naturales
        $$
        \textit{Caso base: }
        $$
          p(\magenta1):
          X^{\magenta1} - 1 = (X - 1) \ub{\sumatoria{j=0}{\magenta1 - 1} X^j}{X^0 = 1}
          \entonces
          p(\magenta1)
        $$
        Entonce, $p(\magenta{1})$ es Verdadero

        \textit{Paso inductivo.}

        Asumo que:
        $$
          p(k) : \ub{X^k - 1 = (X-1) \cdot \sumatoria{j=0}{k-1}X^j}{\purple{\text{hipótesis inductiva}}} \text{ es Verdadera}.
        $$
        Entonces quiero probar que:
        $$
          p(k+1): X^{k+1} - 1 = (X-1) \cdot \sumatoria{j=0}{k}X^j
        $$
        también lo sea. Arranco las cuentulis:
        $$
          \textstyle
          (X-1) \cdot \sumatoria{j=0}{k} X^j =
          (X-1) \cdot (\sumatoria{j=0}{\blue{k-1}} X^j + X^{\blue{k}} ) =
          (X-1) \cdot \sumatoria{j=0}{k-1} X^j  + (X - 1) \cdot X^k
          \igual{\purple{HI}}
          \purple{X^k - 1} + X^{k+1} - X^k = X^{k+1} - 1
        $$

        Dado que $p(1),\, p(k) $ y $p(k+1)$ resultaron verdaderas por el principio de inducción también
        será verdadera $p(n) \paratodo n \en \naturales$

\end{enumerate}

\begin{aportes}
  \item \aporte{\dirRepo}{naD GarRaz \github}
\end{aportes}
