\begin{enunciado}{\ejercicio}
  \begin{enumerate}[label=\roman*)]
    \item Hallar todos los polinomios de grado 2 irreducibles en $\enteros/2\enteros[X]$.
    \item Decidir cuáles de los siguientes polinomios son irreducibles en $\enteros/2\enteros[X]$:
          \begin{enumerate}[label=(\alph*)]
            \begin{multicols}{3}
              \item $f = X^4 + X + 1$,
              \item $f = X^4 + X^2 + 1$,
              \item $f = X^4 + X^3 + 1$.
            \end{multicols}
          \end{enumerate}
  \end{enumerate}
\end{enunciado}

\begin{enumerate}[label=\roman*)]
  \item Antes de nada, vemos que el enunciado no nos pide que los polinomios sean monicos, esto en este caso no va a ser un problema ya que 
  el coeficiente si no es congruente a $1 mod 2$ va a ser 0 y por lo tanto no seria un polinomio de grado $2$.
  
  Usando el ejericio anterior, deberia haber $3$ polinomios reducibles y $1$ irreducible. Como son pocos casos, vayamos uno por uno. 
  \begin{align*}
  (X - \bar{0})^2 &= X^2 \\
  (X - \bar{1})^2 &= X^2 + \bar{1} \\
  (X - \bar{0})(X - \bar{1}) &= X^2 - X
  \end{align*}
  Vemos que el unico que falta que podemos construir con elementos de $(\enteros/2\enteros)$ es $X^2 - X + \bar{1}$. Asi que 
  ese es el unico polinomio irreducible en $(\enteros/2\enteros)[X]$
  
  \item Son todos polinomios de grado $4$, para cada caso vemos que no hay un elemento directo que los anule, es decir 
  que los haga cero, sin embargo esto no significa que sean irreducibles, ya que podrian ser descomponibles por $2$ polinomios de grado $2$ indescomponibles. 
  
  Consideremos la siguiente expansion generica:
  $$
  (aX^2 + bX + c)(dX^2 + eX + f) = adX^4 + (ae + bd)X^3 + (af + be + cd)X^2 + (bf + ce)X + cf
  $$
  Luego armamos con ello un sistema de ecuaciones, en el que igualamos cada expresion con el respectivo coeficiente del polinomio 
  a descomponer, si el sistema tiene solución, con los valores encontrados de $a,b,c,d,e,f$ reconstruimos los polinomios que lo descompusieron,
  si el sistema no tiene solucion o se llega a un absurdo, entonces el polinomio original de grado $4$ es irreducible.
        \begin{enumerate}[label=(\alph*)]
          \item Planteamos el sistema que tiene las siguientes ecuaciones:
          \begin{align}
          ad = 1 \\
          ae + bd = 0 \\
          af + be + cd = 0 \\
          bf + ce = 1 \\
          cf = 1
          \end{align}

          De $(5)$ vemos que $c = f = 1$, de $(1)$ vemos que $a = d = 1$, ahora notamos que $bf + ce = 1
          \iff b \neq e$, pero $ae + bd = 0 \iff b = e$, por lo tanto llegamos a un absurdo, y el polinomio es irreducible. 
          \item Planteamos el sistema que tiene las siguientes ecuaciones:
          \begin{align}
          ad = 1 \\
          ae + bd = 0 \\
          af + be + cd = 1 \\
          bf + ce = 0 \\
          cf = 1
          \end{align}

          Vemos ya que $a = d = c = f = 1$, de $(8)$ vemos que $be = 1 \iff b = e = 1$. Con esta eleccion, todas las demas
          ecuaciones se cumplen, asi que este polinomio es descomponible. 
          $$
          X^4 + X^2 + 1 = (X^2 + X + 1)(X^2 + X + 1) = (X^2 + X + 1)^2
          $$
          
          \item Planteamos el sistema que tiene las siguientes ecuaciones:
          \begin{align}
          ad = 1 \\
          ae + bd = 1 \\
          af + be + cd = 0 \\
          bf + ce = 0 \\
          cf = 1
          \end{align}
          Vemos que $a = d = c = f = 1$, de $(12)$ tenemos que $e \neq b$, pero para $(14)$ necesitamos
          que $e = b$, por lo tanto llegamos a una contradiccion, entonces el polinomio es irreducible.
        \end{enumerate}
\end{enumerate}

\begin{aportes}
 \item \aporte{https://github.com/sigfripro}{sigfripro \github}
\end{aportes}
