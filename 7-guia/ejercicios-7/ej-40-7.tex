\begin{enunciado}{\ejercicio}
  \begin{enumerate}[label=\roman*)]
    \item Hallar todos los polinomios de grado 2 irreducibles en $\enteros/2\enteros[X]$.
    \item Decidir cuáles de los siguientes polinomios son irreducibles en $\enteros/2\enteros[X]$:
          \begin{enumerate}[label=(\alph*)]
            \begin{multicols}{3}
              \item $f = X^4 + X + 1$,
              \item $f = X^4 + X^2 + 1$,
              \item $f = X^4 + X^3 + 1$.
            \end{multicols}
          \end{enumerate}
  \end{enumerate}
\end{enunciado}

\begin{enumerate}[label=\roman*)]
  \item
        Usando resultados del ejercicio \refEjercicio{ej:39},
        debería haber un total de $3$ polinomios \ul{reducibles} y $1$ \ul{irreducible}.

        Son solo \blue{4 polinomios}, vayamos uno por uno:
        $$
          \begin{array}{rclcl}
            (X - 0)^2      & = & \blue{X^2}                                                                         \\
            (X - 1)^2      & = & X^2 - 2X + 1 & \igual{$\scriptscriptstyle \enteros/_{2\enteros}$} & \blue{X^2 + 1} \\
            (X - 0)(X - 1) & = & X^2 - X      & \igual{$\scriptscriptstyle \enteros/_{2\enteros}$} & \blue{X^2 + X}
          \end{array}
        $$
        Vemos que el único que falta que podemos construir con elementos de $(\enteros/2\enteros)$:
        $$
          \blue{X^2 + X + 1},
        $$
        el único polinomio irreducible en $(\enteros/2\enteros)[X]$.

  \item
        Este ejercicio sale usando el resultado del ítem anterior:

        \textit{Nota que puede ser de interés:}
        \parrafoDestacado[\red{\atencion}]{
          Por ejemplo $P \en \reales[X]$:
          $$
            P = X^4 + 3X^2 + 2 = (X^2 + 1) \cdot (X^2 + 2)
          $$
          es un polinomio que \ul{no tiene} raíces reales, pero se puede factorizar como producto de irreducibles,
          es decir otros polinomio que {\tiny obviamente} \ul{ tampoco tienen} raíces reales.
        }
        \textit{fin nota que puede ser de interés}

        Los polinomio no tienen raíces en $(\enteros/2\enteros)$, lo cual \textit{no} implica que sean \textit{irreducibles}. Porque
        podrían ser divisibles por un polinomio \textit{irreducible} de menor grado, o sorpresa \red{\surprise}, tenemos el que calculamos
        en el ítem anterior.

        \begin{enumerate}[label=(\alph*)]
          \item
                $$
                  \polyset{vars=X}
                  \divPol{X^4 + X + 1}{X^2 + X + 1}
                $$
                Ese resultado en $(\enteros/2\enteros)$:
                $$
                  X^4 + X + 1 \igual{\red{!}} (X^2 + X + 1)\cdot(X^2 + 1) + 1
                $$
                El resto es siempre 1, así que no se va a poder factorizar, $X^4 + X + 1$ es \ul{irreducible} en $(\enteros/2\enteros)$.

          \item
                $$
                  \polyset{vars=X}
                  \divPol{X^4 + X^2 + 1}{X^2 + X + 1}
                $$
                Ese resultado en $(\enteros/2\enteros)$:
                $$
                  X^4 + X^2 + 1 \igual{\red{!}} (X^2 + X + 1)\cdot(X^2 + X + 1)
                $$
                Este sí se puede factorizar, $X^4 + X^2 + 1$ es \ul{reducible} en $(\enteros/2\enteros)$.

          \item
                $$
                  \polyset{vars=X}
                  \divPol{X^4 + X^3 + 1}{X^2 + X + 1}
                $$
                Ese resultado en $(\enteros/2\enteros)$:
                $$
                  X^4 + X^3 + 1 \igual{\red{!}} (X^2 + X + 1)\cdot(X^2 + 1) + X
                $$
                Este no se puede factorizar, $X^4 + X^3 + 1$ es \ul{irreducible} en $(\enteros/2\enteros)$.
        \end{enumerate}
\end{enumerate}

\begin{aportes}
  \item \aporte{https://github.com/sigfripro}{sigfripro \github}
  \item \aporte{\dirRepo}{naD GarRaz \github}
\end{aportes}
