\ejercicio
Determinar todos los $a \en \complejos$ tales que\\
\begin{enumerate}[label=\roman*)]
  \item $X^3 + 2X^2 + 2X + 1$ sea divisible por $X^2 + aX + 1$,
  \item $X^4 - aX^3 + 2X^2 + X + 1$ sea divisible por $X^2 + X + 1$,
  \item El resto de la división de $X^5 - 3x^3 - x^2 - 2X + 1$ por $X^2 + aX + 1$ sea $-8X + 4$.
\end{enumerate}

\separadorCorto

\begin{enumerate}[label=\roman*)]
  \item
        Haciendo la division de $X^3 + 2X^2 + 2X + 1$ por $X^2 + aX + 1$, se tiene que:
        \[
          X^3 + 2X^2 + 2X + 1 = (X - a + 2)(X^2 + aX + 1) + \ub{(a^2 - 2a + 1)X + a - 1}{\text{resto}}
        \]
        Así, para que $X^3 + 2X^2 + 2X + 1$ sea divisible por $X^2 + aX + 1$ tiene que ocurrir que el resto sea 0.\\
        O sea,

        \begin{align*}
          X^2 + aX + 1 \divideA X^3 + 2X^2 + 2X + 1 \sisolosi & (a^2 - 2a + 1)X + a - 1 = 0 \\
          \sisolosi                                           & \llave{r}{
          a^2 - 2a + 1 = 0                                                                  \\
            a - 1 = 0
          }
        \end{align*}

        Analizo las ecuaciones:\\
        \begin{itemize}
          \item $a - 1 = 0 \sisolosi a = 1$
          \item $a^2 - 2a + 1 = 0
                  \flecha{$a = 1$}
                  1^2 - 2 \cdot 1 + 1 = 1 - 2 + 1 = 0$
        \end{itemize}

        Luego, el valor de $a \en \complejos$
        tal que $X^3 + 2X^2 + 2X + 1$ es divisible por $X^2 + aX + 1$ es \cajaResultado{a = 1}.

  \item \hacer

  \item Haciendo la division de:
        $$
          X^5 - 3X^3 - X^2 - 2X + 1 \text{ por } X^2 + aX + 1,
        $$
        se tiene que:
        $$
          \begin{array}{c}
            X^5 - 3X^3 - X^2 - 2X + 1 = q \cdot (X^2 + aX + 1) + \oa{r}{\text{resto}} \\
            \text{ con }
            \llave{rcl}{
            q           & = & (X^3 - aX^2 + (a^2 - 4)X - a^3 + 5a - 1)                \\
            \magenta{r} & = & (a^4 - 6a^2 + a + 2)X + a^3 - 5a + 2.
            }
          \end{array}
        $$
        Ahora viene la igualación de polinomios para encontrar ese valor de $a$:
        $$
          \begin{array}{rcl}
            \magenta{r} = -8X + 4
                                & \sisolosi &
            (a^4 - 6a^2 + a + 2)X + a^3 - 5a + 2 = -8X + 4          \\
                                & \sisolosi &
            \llave{r}{
            a^4 - 6a^2 + a + 2 = -8                                 \\
              a^3 - 5a + 2 = 4
            }                                                       \\
                                & \sisolosi &
            \llave{rcl}{
            a^4 - 6a^2 + a + 10 & =         & 0        \  \llamada1 \\
            a^3 - 5a - 2        & =         & 0 \  \llamada2
            }
          \end{array}
        $$
        Analizo las ecuaciones:
        \begin{itemize}
          \item[$\llamada2$] $a^3 - 5a - 2 = 0 \sii a(a^2 - 5) - 2 = 0$\\
                Veo que $a = -2$ es solución, por lo que divido $a^3 - 5a - 2$ por $a + 2$ con Ruffini:
                $$
                  \begin{array}{c|cccc}
                       & 1 & 0  & -5 & -2        \\
                    -2 &   & -2 & 4  & 2         \\
                    \hline
                       & 1 & -2 & -1 & \boxed{0}
                  \end{array}
                $$
                Por lo que:
                $$
                  a^3 - 5a - 2 = (a + 2)(a^2 - 2a - 1).
                $$
                Busco las raíces de $a^2 - 2a - 1$ con la fórmula resolvente:
                \begin{align*}
                  a_{+,-} & = \frac{2 \pm \sqrt{(-2)^2 - 4 \cdot (-1)}}{2} \\
                          & = \frac{2 \pm \sqrt{8}}{2}                     \\
                          & = 1 \pm \sqrt{2}
                \end{align*}
                Por lo que:
                $$
                  a^3 - 5a - 2 = (a + 2)(a - 1 + \sqrt{2})(a - 1 - \sqrt{2}) = 0
                  \sisolosi
                  \llave{l}{
                    a = -2\\
                    a = 1 + \sqrt{2}\\
                    a = 1 - \sqrt{2}
                  }
                $$

          \item[$\llamada1$] $a^4 - 6a^2 + a + 10 = 0$. Me fijo que valores de $a$ obtenidos antes verifican:

                \begin{itemize}
                  \item Si $a = -2 \entonces (-2)^4 - 6(-2)^2 - 2 + 10 = 16 - 24 - 2 + 10 = 0 \Tilde$

                  \item Si $a = 1 + \sqrt{2} \entonces (1 + \sqrt{2})^4 - 6(1 + \sqrt{2})^2 + 1 + \sqrt{2} + 10 = 10 + \sqrt{2} \distinto 0$

                  \item Si $a = 1 - \sqrt{2} \entonces (1 - \sqrt{2})^4 - 6(1 - \sqrt{2})^2 + 1 - \sqrt{2} + 10 = 10 - \sqrt{2} \distinto 0$
                \end{itemize}
        \end{itemize}
        Luego, el único valor de $a \en \complejos$
        tal que el resto de dividir a $X^5 - 3x^3 - x^2 - 2X + 1$ por $X^2 + aX + 1$ sea $-8X + 4$ es \cajaResultado{a = -2}
\end{enumerate}

\begin{aportes}
  \item \aporte{\neverGonnaGiveYouUp}{Autor original \youtube}
  \item \aporte{\dirRepo}{naD GarRaz \github}
\end{aportes}
