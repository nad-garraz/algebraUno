\ejercicio

\textit{\underline {Definición}:} Sea $K$ un cuerpo y sea $h \en \K[X]$ un polinomio no nulo. Dados $f,g \en \K[X]$,
se dice que $f$ es congruente a $g$ módulo $h$ si $h \divideA f-g$. En tal caso se escribe $\congruencia{f}{g}{h}$.

\begin{enumerate}[label=\roman*)]
	\item Probar que $\congruencia{}{}{h}$ es una relación de equivalencia en $\K[X]$.

	\item Probar que si
	      $\congruencia{f_1}{g_1}{h}$ y
	      $\congruencia{f_2}{g_2}{h}$
	      entonces
	      $\congruencia{f_1 + f_2}{g_1 + g_2}{h}$
	      $\congruencia{f_1 \cdot f_2}{ g_1 \cdot g_2}{h}$.

	\item Probar que si $\congruencia{f}{g}{h}$ entonces $\congruencia{f^n}{g^n}{h}$ para todo $n \en \naturales$.

	\item Probar que $r$ es el resto de la división de $f$ por $h$ si y solo si $\congruencia{f}{r}{h}$ y $r=0$ o $\gr(r) < \gr(h)$.
\end{enumerate}

\separadorCorto

\begin{enumerate}[label=\roman*)]
	\item uff... \quad
	      Para probar que esto es una relación de equivalencia pruebo que sea
	      \textit{reflexiva},
	      \textit{simétrica} y
	      \textit{transitiva},

	      \begin{itemize}
		      \item \textit{reflexiva}: Es $f$ congruente a $f$ módulo h?\\
		            $\congruencia{f}{f}{h}
			            \sisolosi
			            h \divideA f-f = 0
			            \sisolosi
			            h \divideA 0$ \Tilde

		      \item \textit{simétrica}: Si $\congruencia{f}{g}{h}  \stackrel{?}\sisolosi \congruencia{g}{f}{h}$ \\
		            $\congruencia{f}{g}{h}
			            \sisolosi
			            h \divideA f-g
			            \sisolosi
			            h \divideA -(g-f)
			            \sisolosi
			            h \divideA g-f
			            \sisolosi
			            \congruencia{g}{f}{h}$ \Tilde

		      \item \textit{transitiva}: Si
		            $\llave{l}{
				            \congruencia{f}{g}{h}\\
				            \congruencia{g}{p}{h}
			            }
			            \stackrel{?}\sisolosi
			            \congruencia{f}{p}{h}.\\
		            $

		            $\llave{l}{
				            h \divideA f - g\\
				            h \divideA g - p
			            }
			            \flecha{$F_1 + F_2$}[$\to F_2$]
			            \llave{l}{
				            h \divideA f - g\\
				            h \divideA f - p }
			            \to \congruencia{f}{p}{h} $ \Tilde
	      \end{itemize}
	      Cumple condiciones para ser una relación de equivalencias en $\K[X]$

	\item Si
	      $\llave{l}{
			      \congruencia{f_1}{g_1}{h}\\
			      \congruencia{f_2}{g_2}{h} \llamada1
		      }$\\

	      $\congruencia{f_1}{g_1}{h}
		      \sisolosi
		      h \divideA f_1 - g_1
		      \entonces
		      h \divideA f_2\cdot (f_1 - g_1)
		      \sisolosi
		      \congruencia{f_1\cdot f_2}{g_1\cdot f_2}{h}
		      \stackrel{\llamada1}\sisolosi
		      \congruencia{f_1\cdot f_2}{g_1\cdot g_2}{h}
	      $

	\item \textit{Inducción: } Quiero probar $p(n):$ Si $\congruencia{f}{g}{h}$ entonces $\congruencia{f^n}{g^n}{h}$ para todo $n \en \naturales$.\\
	      $\textit{Caso base: } p(1): \congruencia{f^1}{g^1}{h}\llamada2 \text{ Verdadera}$\Tilde\\

	      $\textit{Paso inductivo: } p(k):
		      \ub{\congruencia{f^k}{g^k}{h}}{HI} \text{ es verdadera}
		      \stackrel{?}\entonces
		      p(k+1): \congruencia{f^{k+1}}{g^{k+1}}{h} \text{ ¿También lo es?}
	      $\\

	      $\congruencia{f^k}{g^k}{h}
		      \sisolosi
		      h \divideA f^k - g^k
		      \entonces
		      h \divideA f \cdot (f^k - g^k)
		      \sisolosi
		      \congruencia{f^{k+1}}{f\cdot g^k}{h}
		      \stackrel{\llamada2}\sisolosi
		      \congruencia{f^{k+1}}{g^{k+1}}{h}
	      $\Tilde\\

	      Finalmente $p(1), p(k), p(k+1)$ resultaron verdaderas y por el principio de inducción
	      $p(n)$ es verdadera $\paratodo n \en \naturales$

	\item \hacer

\end{enumerate}
