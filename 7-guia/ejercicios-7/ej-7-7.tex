\ejercicio

Hallar el resto de la división de $f$ por $g$ para:

\begin{enumerate}[label=\roman*)]
	\item $f = X^{353} - X - 1$ y $g = X^{31} - 2$ en $\racionales[X],\,\reales[X],\,\complejos[X]$,

	\item $f = X^{1000} + X^{40} + X^{20}  + 1$ y $g = X^6 + 1$ en $\racionales[X],\,\reales[X],\,\complejos[X]$ y $(\enteros/p\enteros)[X]$

	\item $f = X^{200} - 3X^{101} + 2$, y $g = X^{100} - X + 1$ en $\racionales[X],\,\reales[X],\,\complejos[X]$,

	\item $f = X^{3016} + 2X^{1833} - X^{174} + X^{137} + 2X^4 - X^3 + 1$, y $g = X^4 + X^3 +X^2 + X + 1$ en $\racionales[X],\,\reales[X],\,\complejos[X]$
	      (Sugerencia ver \refEjercicio{ej:cong} \ref{item})).
\end{enumerate}

\separadorCorto

\begin{enumerate}[label=\roman*)]
	\item $g \divideA g \sisolosi \congruencia{X^{31} - 2}{ 0 }{X^{31} - 2} \sisolosi \congruencia{X^{31}}{2}{g}$

	      $f = X^{353} - X - 1 =
		      (\ub{X^{31}}{\conga g 2})^{11} X^{12} - X - 1 \conga g
		      2^{11} X^{12} -X -1
		      \to $
	      \boxed{r_g(f) = 2^{11} X^{12} -1 }

	\item
	      $g \divideA g
		      \sisolosi
		      \congruencia{X^6 + 1}{ 0 }{X^6 + 1}
		      \sisolosi
		      \congruencia{X^6}{-1}{g}$

	      $f =
		      X^{1000} + X^{40} + X^{20}  + 1 =
		      (X^6)^{166}X^4 + (X^6)^6 X^4 + (X^6)^3 X^2  + 1 \conga g
		      X^4 + X^4 \magenta{-} X^2  + 1 =
		      2X^4 - X^2 + 1\\
		      \to$
	      \boxed{r_g(f) = 2X^4 - X^2 + 1}\\
	      \red{¿Qué onda en $\enteros/p\enteros$?}

	\item
	      $g \divideA g
		      \sisolosi
		      \congruencia{X^{100} - X + 1}{0}{X^{100} - X + 1}
		      \sisolosi
		      \congruencia{X^{100}}{X - 1}{g}$

	      $f =
		      X^{200} - 3X^{101} + 2 =
		      (X^{100})^2 - 3 X^{100} X + 2 \conga g
		      (X - 1)^2 - 3 (X - 1) X + 2
		      \\
		      \to$
	      \boxed{r_g(f) = (X - 1)^2 - 3 (X - 1) X + 2}

	\item \textit{Usando la sugerencia}: Del ejercicio \refEjercicio{ej:cong} \ref{item}
	      sale que $X^n - 1 = (X-1) \cdot \sumatoria{k=0}{n-1} X^k$\\

	      $\flecha{$n = 5$}[para el $g$]
		      X^5 - 1 = (X-1)\ub{(X^4 + X^3 +X^2 + X + 1)}{g}
		      \sisolosi
		      \congruencia{X^5}{\ub{1}{r_g(X^5)}}{g} \Tilde
	      $\\

	      $f = (X^5)^{603} X + 2(X^5)^{366} X^3 - (X^5)^{34} X^4 + (X^5)^{27} X^2 + 2X^4 - X^3 + 1\\
		      \congruencia{f}{\ub{X + 2X^3 - X^4 + X^2 + 2X^4 - X^3 + 1}{= X^4 + X^3 +X^2 + X + 1 =g}}{g}
		      \sisolosi
		      \boxed{\congruencia{f}{0}{g}}$

\end{enumerate}
