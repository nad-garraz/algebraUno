\begin{enunciado}{\ejercicio}\label{guia7:ej8}
  Sea $\K$ un cuerpo. Sean $n \en \naturales$ y $a \en \K$.
  \begin{enumerate}[label=\roman*)]
    \item Probar que $X - a \divideA X^n - a^n$ en $K[X]$.
    \item Probar que si $n$ es impar entonces $X + a \divideA X^n + a^n$ en $\K[X]$.
    \item Probar que si $n$ es par entonces $X + a \divideA X^n - a^n$ en $\K[X]$.
  \end{enumerate}

  Calcular los cocientes en cada caso.
\end{enunciado}

\begin{enumerate}[label=\roman*)]
  \item Pruebo por inducción:
        $$
          p(n) : X - a \divideA X^n - a^n \quad \text{en} \K[X] \paratodo n \en \naturales.
        $$

        \textit{Caso base:}
        $$
          p(\blue{1}) : X - a \divideA X^{\blue{1}} - a^{\blue{1}}
        $$
        El caso  $p(\blue{1})$ es verdadero.

        \textit{Paso inductivo:}
        Asumo que
        $$
          p(\blue{k}) : \ub{X - a \divideA X^{\blue{k}} - a^{\blue{k}}}{\text{\purple{hipótesis inductiva}}}
        $$
        es verdadero. Entonces quiero probar que
        $$
          p(\blue{k+1}) : X - a \divideA X^{\blue{k+1}} - a^{\blue{k+1}}
        $$
        también lo sea.

        Arranco haciendo a mano la división a mano del caso \blue{$k+1$} en la
        primera iteración tengo que:
        $$
          \begin{array}{rlc}
            \textstyle
            X^{\blue{k+1}} - a^{\blue{k+1}} & = & X^k \cdot (X-a) + aX^k - a^{k+1}      \\
                                            & = & X^k \cdot (X-a) + a \cdot (X^k - a^k)
          \end{array}
          \Entonces{\purple{HI}}[$\div$ MAM $(X-a)$]
          \begin{array}{rcl}
            \frac{X^{\blue{k+1}} - a^{\blue{k+1}}}{X-a}  =  \frac{X^k \cdot (\cancel{X-a})}{\cancel{X-a}} +
            \frac{a \cdot (\cancel{\purple{X^k - a^k}})}{\cancel{\purple{X-a}}} \\
          \end{array}
        $$
        Ese último paso muestra que $X-a \divideA X^{\blue{k+1}} - a^{\blue{k+1}}$ entonces $p(\blue{k+1})$ también es verdadera.

        \medskip

        Dado que $p(1), p(k) \ytext p(k+1)$ resultaron verdaderas por criterio de inducción $p(n)$ también lo es $\paratodo n \en \naturales$.

  \item Por induccion nuevamente:
  
          $$
           p(n): X + a \divideA X^{2k + 1} + a^{2k + 1} \quad \text{en} \K[X] \paratodo n \in \naturales_0
          $$
          \textit{Caso base:}
          $$
           p(0) : X + a \divideA X^1 + a^1
          $$
          El caso $p(0)$ es verdadero. 

          \textit{Paso inductivo: } 
          Asumo que
          $$
           p(k) : \underbrace{X + a \divideA X^{2k + 1} + a^{2k + 1}}_{\text{hipotesis inductiva}}
          $$
          es verdadero. Luego quiero probar que 
          $$
           p(k + 1) : X + a \divideA X^{2(k+1) + 1} + a^{2(k+1) + 1} = X^{2k + 3} + a^{2k + 3}
          $$
          sea verdadero tambien.
          Acá la idea es similar al item anterior, hacemos una division a mano y luego agrupamos y aplicamos la 
          hipotesis inductiva. Vemos que para la hipotesis inductiva necesitamos de exponente $2k + 1$ pero tenemos
          $2k + 3$, entonces en vez de dividir por $X + a$, dividimos por $X^2 - a^2$. 
          
          $$
          \begin{array}{rcl}
            X^{2k + 3} + a^{2k + 3} & = & X^{2k + 1}\cdot(X^2 - a^2) + a^{2k+3} + X^{2k + 1}\cdot a^2 \\
                                    & = & X^{2k + 1}\cdot\blue{(X + a)}(X - a) + a^2 \cdot \blue{(X^{2k + 1} + a^{2k + 1})}
          \end{array}
          $$
          El cociente seria $X^{2k + 1}\cdot (x-a) + a^2$
          Aplicando la hipotesis inductiva se ve que el caso $p(k+1)$ es verdadero. Luego 
          que $p(k)$ es verdadera $\forall k \in \naturales_0$, como se queria probar.
           
      
          

  \item \hacer

\end{enumerate}

\begin{aportes}
  \item \aporte{\dirRepo}{naD GarRaz \github}
  \item \aporte{https://github.com/sigfripro}{sigfripro \github}
\end{aportes}
