\begin{enunciado}{\ejercicio}
  Calcular el máximo común divisor entre $f$ y $g$ en $\racionales[X]$ y escribirlo como combinación
  polinomial de $f$ y $g$ siendo:
  %Macro Local
  \def\f1{X^5 + X^3 - 6X^2 + 2X +2}
  \def\g1{X^4 - X^3 - X^2 + 1}
  \begin{enumerate}[label=\roman*)]
    \item $f = \f1$, $g = \g1$,

    \item $f = X^6 + X^4 + X^2  +1$, $g = X^3 + X$,

    \item $f = 2X^6 - 4X^5 + X^4 + 4X^3 - 6X^2 + 4X + 1$, $g = X^5 - 2X^4 + 2X^2 - 3X + 1$,
  \end{enumerate}
\end{enunciado}

\begin{enumerate}[label=\roman*)]
  \item Hacemos la división hermosa de polinomios:
        $$
          \divPol{X^5 + X^3 - 6X^2 + 2X +2}{X^4 - X^3 - X^2 + 1}
        $$

        Todo muy lindo. Según Euclides:
        $$
          (f : g) = (\ub{X^4 - X^3 - X^2 + 1}{g} : 3X^3 -55X^2 +X + 1)
        $$
        Escribo a $f$ en función de $g$:
        $$
          f = (X+1) \cdot (X^4 - X^3 - X^2 + 1) + 3X^3 -55X^2 +X + 1
        $$
        Otra vez:
        $$
          \divPol{X^4 - X^3 - X^2 + 1}{3X^3 -5X^2 +X + 1}
        $$
        y otra vez... ?:
        $$
          \divPol{3X^3 -5X^2 +X + 1}{- \frac{2}{9}X^2-\frac{5}{9}X+\frac{7}{9}}
        $$
        \textit{...da fuck is this?}
        $$
          \divPol{- \frac{2}{9}X^2-\frac{5}{9}X+\frac{7}{9}}{\frac{171}{4}X - \frac{171}{4}}
        $$

        Todo lindo:
        $$
          \mcd{X^5 + X^3 - 6X^2 + 2X +2}{X^4 - X^3 - X^2 + 1}
        $$

        El MCD será el \textit{\underline{último resto no nulo y mónico}}:
        $$
          \cajaResultado{
            (f : g) = X - 1
          }
        $$

  \item Este es más humano:
        $$
          \mcd{X^6 + X^4 + X^2  +1}{X^3 + X}
        $$

        El MCD será el último resto no nulo y mónico:
        $$
          \cajaResultado{ (f : g) = X^2+1}
        $$
        El MCD escrito como combinación polinomial de $f$ y $g$:
        $$
          \cajaResultado{ X^2 + 1 = f \cdot 1 + g \cdot (-X^3)}
        $$

  \item
        Euclides nuevamente:
        $$
          \mcd{2X^6 - 4X^5 + X^4 + 4X^3 - 6X^2 + 4X + 1}{X^5 - 2X^4 + 2X^2 - 3X + 1}
        $$
        El MCD será el último resto no nulo y \textit{mónico}:
        $$
          \cajaResultado{
            (f : g) = 1
          }
        $$
        El MCD escrito como combinación polinomial de $f$ y $g$:
        $$
          \cajaResultado{
            1 = \frac{1}{3} g \cdot(2X^2-4X+1) - \frac{1}{3} f \cdot (X-2)
          }
        $$

\end{enumerate}

\begin{aportes}
  \item \aporte{\dirRepo}{naD GarRaz \github}
\end{aportes}
