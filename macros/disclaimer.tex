\thispagestyle{empty}
\underline{Disclaimer:}\par
Dirigido para aquél que esté listo para leerlo, o no tanto. Va con onda.\par \bigskip

\vspace*{\fill}
\noindent\makebox[\textwidth]{
  \begin{minipage}{0.7\textwidth}
    \centering
    {\Large \red{¡Si usás este apunte vas a reprobar!}}\par\bigskip

    \textit{Not really.} Dependerá de como lo uses, puede ser un arma de doble filo. Ya sabés como se usa \magenta{esto \faIcon{toilet-paper}}.
    Depende de vos lo que hagas con él.\par Si estás trabado, antes de ver la solución que hizo otra persona:

    \begin{itemize}[label=\purple{\faIcon{toilet-paper}}]

            \item Mirar la solución ni bien te trabás, te \textit{condicionas pavlovianamente} a \textbf{no} pensar.
                    Necesitás darle tiempo al cerebro para llegar a la solución.

      \item Intentá un ejercicio similar, pero \textbf{más fácil}.

      \item ¿No sale el fácil? Intentá uno \textbf{\large aún más fácil}.


      \item Fijate si tenés un ejercicio similar hecho en clase. Y mirá ese, así no quemás el ejercicio de la guía.

      \item Tomate 2 minutos para formular una pregunta que realmente
            sea lo que \textbf{no} entendés. Decir \textit{`no me sale'} $\noexiste +$. Escribí
                    esa pregunta, vas a dormir mejor. 
    \end{itemize}\par \medskip

    Ahora sí mirá la solución.\par \medskip

    \textit{Si no te salen los ejercicios fáciles} de un tema en particular, no te van a salir los ejercicios
    más difíciles: \cyan{Sentido común}.\par\medskip

    ¡Los más fáciles van a salir! Son el alimento de nuestra confiaza.\par\medskip

    Si mirás miles de soluciones a parciales en el afán de tener un ejemplo hecho de todas las
    variantes, estás apelando demasiado a la suerte de que te toque uno igual, \textit{pero no estás aprendiendo nada}.
    Hacer un parcial bien lleva entre 3 y 4 horas. Así que si vos en 4 horas "hiciste" 3 o 4 parciales, \textit{algo raro debe haber}.
    A los parciales se va a \textbf{pensar} y eso hay que practicarlo desde el primer día.\par\bigskip

          Mirá los videos de las teóricas \href{\videosTeresa}{de Teresa que son buenísimos \faIcon{youtube}}.\par\medskip
          Videos de prácticas de pandemia, complemento extra: \href{\videosPracticas}{Prácticas Pandemia \faIcon{youtube}}.\par\bigskip

    Los ejercicios que se dan en clase suelen ser similares a los parciales,
    a veces más difíciles, repasalos siempre \href{\justDoIt}{Just Do IT \faIcon{jedi}\faIcon{hand-sparkles}\faIcon{jedi}!}
  \end{minipage}
}
\vspace*{\fill}

Eh, loco, fatalista, distópico, \href{\dontWorryAboutAThing}{relajá un toque te vas a\
  quedar (más) pelado... \faIcon{dove}\faIcon{dove}\faIcon{dove} \textit{va a salir todo bien!}}
