\underline{Disclaimer:}
Esto va a dirigido para aquél que esté listo para escucharlo, más bien leerlo.\par \Bigskip



\vspace*{\fill}
\noindent\makebox[\textwidth]{
	\begin{minipage}{0.7\textwidth}
		\centering
		{\Large \red{¡Si usás este apunte vas a reprobar!}}\par\bigskip

		\textit{Not really}, depende de como lo uses. Ya sabés como se usa \magenta{esto \faIcon{toilet-paper}}.
		Depende de vos lo que hagas con él. Si estás trabado antes de ver lo que hizo otra persona:

		\begin{itemize}[label=\purple{\faIcon{toilet-paper}}]

			\item No mires la solución inmediatamente, porque te \textit{condicionas pavlovianamente}.

			\item Intentá un ejercicio similar, pero \textbf{más fácil}.

			\item No sale el fácil, intentá uno \textbf{aún más fácil}.

			\item Repasar las teóricas. \href{\videosTeresa}{Ver videos de Teresa \faIcon{youtube}}.

			\item Ver algún ejercicio similar hecho en clase.

			\item Tomate 2 minutos para formular una pregunta que realmente
			      sea lo que \textbf{no} entendés. Decir \textit{no me sale} $\noexiste$ más. Y si encima escribís esa pregunta, vas a dormir mejor.
		\end{itemize}

		\textit{Si no te salen los ejercicios fáciles} de un tema en particular, no te van a salir los ejercicios
		más difíciles: Sentido común. Pero los más fáciles van a salir y eso te va a dar un \textit{confidence boost}.\par

		Si hacés miles de parciales en el afán de tener un ejemplo hecho de todas las
		variantes, estás apelando demasiado a la suerte. Un poco de originalidad de los profes y te la ponen.\par

		\href{\videosTeresa}{Los videos de Teresa son buenísimos \faIcon{youtube}}.\par

		Los ejercicios que se dan en clase suelen ser similares a los parciales,
		a veces más difíciles, repasalos siempre \href{\justDoIt}{Just Do IT \faIcon{jedi}\faIcon{hand-sparkles}\faIcon{jedi}!}
	\end{minipage}
}
\vspace*{\fill}

Eh, loco, fatalista, distópico, \href{\dontWorryAboutAThing}{relajá un toque te vas a quedar (más) pelado... \faIcon{dove}\faIcon{dove}\faIcon{dove}}
