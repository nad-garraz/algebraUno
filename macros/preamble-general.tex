\documentclass[11pt, a4paper, spanish, twoside]{article}
% Sacar draft para que aparezcan las imagenes.
% Opciones: 12pt, 10pt, 11pt, landscape, twocolumn, fleqn, leqno...
% Opciones de clase: article, report, letter, beamer...

% Paquetes:
% =========
\usepackage[headheight=110pt, top = 2cm, bottom = 2cm, left=1cm, right=1cm]{geometry} %modifico márgenes
\usepackage[T1]{fontenc} % tildes
\usepackage[utf8]{inputenc} % Para poder escribir con tildes en el editor.
\usepackage[english]{babel} % Para cortar las palabras en silabas, creo.
\usepackage[ddmmyy]{datetime}
\usepackage{amsmath} % Soporte de mathmatics
\usepackage{mathtools} % Más herramientas para matemáctica
\usepackage{amssymb} % fuentes de mathmatics
\usepackage{array} % Para tablas y eso
\usepackage[dvipsnames,table]{xcolor} % Para colorear el texto: black, blue, brown, cyan, darkgray, gray, green, lightgray, lime, magenta, olive, orange, pink, purple, red, teal, violet, white, yellow.
\usepackage{color} % Para colorear el texto: black, blue, brown, cyan, darkgray, gray, green, lightgray, lime, magenta, olive, orange, pink, purple, red, teal, violet, white, yellow.
\usepackage{enumitem} % Cambiar labels y más flexibilidad para el enumerate
\usepackage{multicol}
\usepackage{tikz} % para graficar
\usepackage{cancel} % cancelar fórmulas
\usepackage{titlesec} % para editar titulos y hacer secciones con formato a medida
\usepackage{ulem}
\usepackage{centernot} % tacha cosas
\usepackage{bbding} % símbolos de donde uso FiveStar
\usepackage{skull} % símbolos de donde uso Skull
\usepackage{soul} % Para tachar texto en text y math mode
\usepackage{polynom} % para división de polinomios y mcd
\usepackage{fontawesome5} % fuentes "extras"
\usepackage{simpleicons} % fuentes "extras"
\usepackage{venndiagram} % Para los diagramas de Venn
\usepackage{qrcode} % genera código qr
\usepackage{xspace}% para control de espacios en macros
\usepackage{aligned-overset}% para alinear simbolos con anotaciones sobre/abajo de ellos
%\usepackage{colortbl}
%\usepackage{listings} % Escribir código

%\usepackage{algorithm}
%\usepackage{algpseudocode}
%\usepackage{algorithmicx}

\usepackage{fancyhdr} % Encabezados y pie de páginas
% \usepackage{lipsum} % dummy text
% \usepackage{caption} % Configuracion de figuras y tablas

% para hacer los graficos tipo grafos
\usetikzlibrary{shapes,arrows.meta, chains, matrix, calc, trees, positioning, fit}
\usetikzlibrary{external,decorations.pathreplacing,angles,quotes}

% En general quiero que este paquete sea el último en importarse
\usepackage{hyperref} % para que haya links navegables en el PDF
\hypersetup{
  pdftitle={Apunté Único de Álgebra I},
  pdfauthor={Por los alumnos y exalumnos de Álgebra I},
  pdfkeywords={algebra 1, resueltos},
  colorlinks=true,
  pdfborder={0 0 0}, % sin border
  citecolor=cyan,
  % refcolor=magenta,
  linkcolor=blue!80!red,
  filecolor=green,
  urlcolor=red!10!purple,
}
\urlstyle{same}

\setlength{\parindent}{0pt} % Para que no haya indentación en las nuevas líneas.

%% Info SOCIAL
\def\dirRepo{https://github.com/nad-garraz/algebraUno}
\def\dirTelegram{https://t.me/+1znt2GV1i8cwMTNh}
\newcommand{\dirGuia}[1]{\dirRepo/blob/main/#1-guia/#1-sol.pdf}
